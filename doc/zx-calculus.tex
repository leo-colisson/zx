\documentclass[a4paper,doc2]{ltxdoc} % doc2 is needed to force the old version, or links get colored in a weird red way even with hidelinks. https://github.com/latex3/latex2e/issues/822

%%%%%%%%%%%%%%%%%%%%%%%%%%%%%%
%%% Packages
%%%%%%%%%%%%%%%%%%%%%%%%%%%%%%
%% Warning: if you compile and get:
%% ERROR: Argument of \tikz@lib@matrix@with@options has an extra }.
%% make sure to fix catcodes around it as | is given a different meaning in ltxdoc.

\usepackage{amsmath}
\usepackage[margin=3cm]{geometry}
\usepackage{calc}
\usepackage{braket}
\usepackage{tikz}
\usetikzlibrary{shadows,fit}
% \usetikzlibrary fails because file is not in current directory, lazy to setup TEXINPUTS
\makeatletter
  %%%%%%%%%%%%%%%%%%%%%%%%%%%%%%
%%% Version: 2023/10/03
%%% License: MIT
%%% Author: Léo Colisson
%%%%%%%%%%%%%%%%%%%%%%%%%%%%%%
%%% Understand why if a "fake" alias does not exists but was created before I get no error

%% TODO:
%% - Check why this displays weirdly:
% \begin{ZX}[myBold]
%   &                                       & \zxDivider[a=div]{} \ar['>, to=Gll] \ar['>, to=Gll2]               &                                 &                                   & \\
%   & \zxH{} \ar[to=div,<',wce] \ar[to=root,<.] &                                                                    & \zxMatrix[a=Gll]'*{G^\matrixUr} & \zxMatrix[a=Gll2]'/*{G^\matrixUr} & \\
%   \zxN{} \ar[rr] &                                       & [\zxwCol] \zxZ[a=root]{} \ar[B,rrr] \ar[to=Gll,.>] \ar[to=Gll2,.>] &                                 &                                   & 
% \end{ZX}
%

% A tikzlibrary[libraryName].code.tex is loaded automatically by tikz when using
% \usetikzlibrary{libraryName}. Therefore, you should just use
% \usetikzlibrary{zx} to load this library. We also provide a package to
% directly load this library using \usepackage{zx}.

%%% Some random notes for myself:
%%% - \tikz@pp@name{X} outputs the name prefix + X + the name suffix, where these prefix are local prefix assigned to all node names. While they are often empty, it might not be always the case.

\RequirePackage{amssymb} % For short minus
\RequirePackage{etoolbox} % \AfterEndPreamble...
\RequirePackage{ifthen} % For conditions
\RequirePackage{xparse} % For NewDocumentComments
\RequirePackage{bm} % For bold math fonts
\RequirePackage{xstring} % For \IfInteger

\usetikzlibrary{cd,backgrounds,positioning,shapes,calc,intersections,fit}
% Declare layers.
\pgfdeclarelayer{background} % Fox boxes using "fit" to group parts of the graph.
\pgfdeclarelayer{belownodelayer} % For nodes that always want to stay below any arrow...
\pgfdeclarelayer{edgelayer} % For nodes that always want to stay below any arrow...
\pgfdeclarelayer{nodelayer} % For nodes... in theory, this fails for now.
\pgfdeclarelayer{main}
\pgfdeclarelayer{abovenodelayer} % For nodes that want to stay above others.
\pgfdeclarelayer{box} % For boxes using "fit" to fake multi-column boxes
\pgfdeclarelayer{labellayer} % For labels... in theory, this fails for now. https://tex.stackexchange.com/questions/618823/node-on-layer-style-in-tikz-matrix-tikzcd
\pgfdeclarelayer{foreground} % For the user, if they want to put anything really above everything.

%%%%%%%%%%%%%%%%%%%%%%%%%%%%%%
%%%% User modifiable variables
%%%%%%%%%%%%%%%%%%%%%%%%%%%%%%

% Define colors, can be redefine by user
\definecolor{colorZxZ}{RGB}{204,255,204}
\definecolor{colorZxX}{RGB}{255,136,136}
\definecolor{colorZxH}{RGB}{255,255,0}
\definecolor{colorZxMatrix}{RGB}{200,200,200}

%%% Some wires (the one having an intermediate H, X, or S gate) may need some additional space for
%%% specific columns.
%%% Use these spaces like &[\zxHCol] or \\[\zxHRow] in that case
%% Defines the space to add for columns and rows containing a connection with Hadamard
% This is for "curved" wires
\newcommand{\zxHCol}{1mm}
\newcommand{\zxHRow}{1mm}
% This is for "flat" wires (usually takes more space)
\newcommand{\zxHColFlat}{1.5mm}
\newcommand{\zxHRowFlat}{1.5mm}
%% Defines the space to add for columns and rows containing a connection with small X/Z
\newcommand{\zxSCol}{1mm}
\newcommand{\zxSRow}{1mm}
\newcommand{\zxSColFlat}{1.5mm}
\newcommand{\zxSRowFlat}{1.5mm}
%% Defines the space to add for columns having both H and Spiders
\newcommand{\zxHSCol}{1mm}
\newcommand{\zxHSRow}{1mm}
\newcommand{\zxHSColFlat}{1mm}
\newcommand{\zxHSRowFlat}{1mm}
%% Wires only: when adding only wires with empty nodes, the space between columns can be too small. Useful not to shrink swap gates...
\newcommand{\zxWCol}{2mm}
\newcommand{\zxWRow}{2mm}
%% Same as zxWCol, but when a single empty node is connected to a non-empty node.
\newcommand{\zxwCol}{1mm}
\newcommand{\zxwRow}{1mm}
%% When vdots/dots are used in lines
\newcommand{\zxDotsCol}{3mm}
\newcommand{\zxDotsRow}{3mm}
%% Default separation between rows and columns
\newcommand{\zxDefaultColumnSep}{1.5mm}
\newcommand{\zxDefaultRowSep}{1.5mm}
% In circuit mode, we have mostly gates, and the default setting might be too small
\newcommand{\zxDefaultColumnSepCircuit}{4mm}
\newcommand{\zxDefaultRowSepCircuit}{4mm}
% To cancel the default tow separation
\newcommand{\zxZeroCol}{-\pgfmatrixcolumnsep}%\zxDefaultColumnSep}
\newcommand{\zxZeroRow}{-\pgfmatrixrowsep}%\zxDefaultRowSep}
%% When adding two lines with only empty nodes, it can create quite a large space. Use this to reduce it
%% Yes, we can add multiple arguments, space will add up (see documentation pgfmatrixnextcell)
\newcommand{\zxNCol}{-\zxDefaultColumnSep,.5mm}
\newcommand{\zxNRow}{-\zxDefaultRowSep,.5mm}


% Angles by default for s and o related arrows
\def\zxDefaultSoftAngleS{30}
\def\zxDefaultSoftAngleN{60}
\def\zxDefaultSoftAngleO{40}
\def\zxDefaultSoftAngleChevron{60}

% Angles by default for s and o related arrows
\def\zxDefaultSoftAngleS{30}
\def\zxDefaultSoftAngleN{60}
\def\zxDefaultSoftAngleO{40}
\def\zxDefaultSoftAngleChevron{60}

% Angles by default for s and o related arrows (in/out version)
\def\zxDefaultSoftAngleN{60}
\def\zxDefaultSoftAngleO{40}
\def\zxDefaultSoftAngleChevron{60}

\def\zxGroundScale{1.8}
\def\zxMeterScale{1}

% Styles for O,N,... (new bezier version)
\tikzset{
  /zx/args/-andL/.cd,
  defaultO/.style={-=.2,L=.4},
  defaultN/.style={-=.2,L=.8},
  defaultN-/.style={1-=.4,1L=0},
  defaultNN/.style={},
  defaultNIN/.style={1-=0,1L=.6},
  defaultChevron/.style={}, % < should be stronger than N- also in ui mode.
  defaultChevronDown/.style={}, % The ^ v versions (default NN and default NIN will be loaded before).
  defaultS/.style={-=.8,L=0},
  defaultS'/.style={-=.8,L=.2},
  default-S/.style={1-=.8,1L=0},
  defaultSIS/.style={1-=0,1L=.8},
  % Loads the default styles for the "use intersection" method.
  /zx/args/-andL/ui/.style={
    /zx/args/-andL/defaultO/.style={-=.2,L=.17},
    /zx/args/-andL/defaultN/.style={-=.2,L=.7},
    /zx/args/-andL/defaultN-/.style={1-=.5,1L=.2,2-=.2,2L=.5}, %% Todo: write a different style for <
    /zx/args/-andL/defaultNN/.style={},
    /zx/args/-andL/defaultChevron/.style={1-=.2,1L=0,2-=.2,2L=.8},
    /zx/args/-andL/defaultChevronDown/.style={1L=.2,1-=0,2L=.2,2-=.8},
    /zx/args/-andL/defaultNIN/.style={1-=0,1L=.66,2-=.55,2L=.16},
    /zx/args/-andL/defaultS/.style={-=.8,L=0},
    /zx/args/-andL/defaultS'/.style={-=.8,L=.3},
    /zx/args/-andL/default-S/.style={1-=.8,1L=0},
    /zx/args/-andL/defaultSIS/.style={1-=0,1L=.8,2-=.15,2L=.6},
  },
}


% Scale to use when scaling 3 dots
\def\zxScaleDots{.7}

% Size of the minimum width of boxes
% We need 4ex to fit most elements in height, at least with the default font (f_a is the heigher), and 1.83em to fit them in width (W is the wider).
% Since these two values are quite close, lets take the maximum (1.83em) to have a square shape
\def\zxBoxMinimumWidth{1.83em}
\def\zxBoxMinimumHeight{1.83em}


% 0.4pt is default in tikz. Also used to ensure it has not been modified document wise by other libraries
% (quantikz notably changes this parameter).
\newcommand{\zxDefaultLineWidth}{0.4pt}

% For phase in content: How to convert sign ("-" for minus, nothing for "+", anything else should be inserted directly),
% above fraction (no parens), below fraction (no parens), above fraction (with parens), below fraction (with parens)
\NewExpandableDocumentCommand{\zxConvertToFracInContent}{mmmmm}{%
  \ifthenelse{\equal{#1}{-}}{\zxMinus}{#1}\frac{#2}{#3}%
}

% For phase in label: How to convert sign ("-" for minus, nothing for "+", anything else should be inserted directly),
% above fraction (no parens), below fraction (no parens), above fraction (with parens), below fraction (with parens)
\NewExpandableDocumentCommand{\zxConvertToFracInLabel}{mmmmm}{%
  \ifthenelse{\equal{#1}{-}}{\zxMinus}{#1}#4/#5%
}

% Minus sign to use in \zxZ-{\alpha}
%\NewExpandableDocumentCommand{\zxMinusInShort}{}{-}
\NewExpandableDocumentCommand{\zxMinusInShort}{}{-}

% The library tries to find intersection between fake center east/... and the border of the node
% to draw the nodes directly on the border.
% To enable this function (can also be set locally), enable this macro:
% \def\zxEnableIntersectionsNodes{} %% To add the "name shape" on each node
% \def\zxEnableIntersectionsWires{} %% For wires only (can't work without \zxEnableIntersectionsNodes)
\newcommand{\zxEnableIntersections}{%
  \def\zxEnableIntersectionsWires{}%
  \def\zxEnableIntersectionsNodes{}%
}
% to disable intersections, undefined it:
\let\zxEnableIntersectionsNodes\undefined
\let\zxEnableIntersectionsWires\undefined
\newcommand{\zxDisableIntersections}{%
  \let\zxEnableIntersectionsNodes\undefined%
  \let\zxEnableIntersectionsWires\undefined%
}
% Now, if it cannot find a good intersection (either because "name path" is disabled, or
% because no intersection is found, or because intersections are disabled anyway) we need to
% have a fallback mechanism. Either we choose instead the "fake center" (or center if fake
% center does not exists), but then the wire will be hidden behind the node (it is a problem
% mostly if you use H nodes a lot, otherwise we recommend this solution):
\def\zxWireInsideIfNoIntersectionName{}
% If you prefer to start the wire on the surface of the node, do that instead... but it will take longer to compute. See also our ui style
% as the default styles are not meant to work in that setting.
% \let\zxWireInsideIfNoIntersectionName\undefined

%%%%%%%%%%%%%%%%%%%%%%%%%%%%%%
%%%% Global temporary variables/dimensions
%%%%%%%%%%%%%%%%%%%%%%%%%%%%%%
\newsavebox{\zx@tmp@box}%

%%%%%%%%%%%%%%%%%%%%%
%%%% Utils
%%%%%%%%%%%%%%%%%%%%%

% by default "execute at end node" accumulate by adding at the end of the existing value,
% it can be helpful to reset completely the value, or add it before
\tikzset{
  zx execute before end node/.code={%
    \def\zx@options{#1}%
    \edef\zx@cmd{%
      \noexpand\def\noexpand\tikz@atend@node{\expandonce{\zx@options}\expandonce{\tikz@atend@node}}%
    }%
    \zx@cmd%
  },
  zx set execute at begin node/.code={%
    \def\tikz@atbegin@node{#1}%
  },
  zx set execute at end node/.code={%
    \def\tikz@atend@node{#1}%
  },
}


%%%%%%%%%%%%%%%%%%%%%%%%%%%%%%
%%%% Adding anchors
%%%%%%%%%%%%%%%%%%%%%%%%%%%%%%

% We add anchors "fake center {north, south, east, west}" to the nodes. The wires will leave from these anchors (except in IO mode) depending on the direction. If the anchor does not exist, center is picked.
% https://tex.stackexchange.com/questions/14769/add-more-anchors-to-standard-tikz-nodes
\def\zx@pgfaddtoshape#1#2{%
  \begingroup
  \def\pgf@sm@shape@name{#1}%
  \let\anchor\pgf@sh@anchor
  #2%
  \endgroup
}
\def\zx@useanchor#1#2{\csname pgf@anchor@#1@#2\endcsname}
\zx@pgfaddtoshape{rounded rectangle}{%
  \anchor{fake center east}{%
    \zx@useanchor{rounded rectangle}{north east}%
    \pgf@yc=.5\pgf@y% final y = 0.5*this y + 0.5*other y.
    \zx@useanchor{rounded rectangle}{south east}%
    \pgf@y=.5\pgf@y%
    \advance\pgf@y by \pgf@yc%
  }%
  \anchor{fake center west}{%
    \zx@useanchor{rounded rectangle}{north west}%
    \pgf@yc=.5\pgf@y% final y = 0.5*this y + 0.5*other y.
    \zx@useanchor{rounded rectangle}{south west}%
    \pgf@y=.5\pgf@y%
    \advance\pgf@y by \pgf@yc%
  }%
  \anchor{fake center north}{%
    \zx@useanchor{rounded rectangle}{center}%
  }%
  \anchor{fake center south}{%
    \zx@useanchor{rounded rectangle}{center}%
  }%
}

\zx@pgfaddtoshape{coordinate}{%
  \anchor{fake center east}{%
    \zx@useanchor{coordinate}{center}%
  }%
  \anchor{fake center west}{%
    \zx@useanchor{coordinate}{center}%
  }%
  \anchor{fake center north}{%
    \zx@useanchor{coordinate}{center}%
  }%
  \anchor{fake center south}{%
    \zx@useanchor{coordinate}{center}%
  }%
}

\zx@pgfaddtoshape{circle}{%
  \anchor{fake center east}{%
    \zx@useanchor{circle}{center}%
  }%
  \anchor{fake center west}{%
    \zx@useanchor{circle}{center}%
  }%
  \anchor{fake center north}{%
    \zx@useanchor{circle}{center}%
  }%
  \anchor{fake center south}{%
    \zx@useanchor{circle}{center}%
  }%
}

\zx@pgfaddtoshape{rectangle}{
  \anchor{fake center east}{%
    \zx@useanchor{rectangle}{center}%
  }%
  \anchor{fake center west}{%
    \zx@useanchor{rectangle}{center}%
  }%
  \anchor{fake center north}{%
    \zx@useanchor{rectangle}{center}%
  }%
  \anchor{fake center south}{%
    \zx@useanchor{rectangle}{center}%
  }%
}


%%%%%%%%%%%%%%%%%%%%%%%%%%%%%%
%%%% For readability
%%%%%%%%%%%%%%%%%%%%%%%%%%%%%%

\NewExpandableDocumentCommand{\wire}{m}{
  \arrow[r,#1]
}

%%%%%%%%%%%%%%%%%%%%%%%%%%%%%% 
%%%% Tikz styles
%%%%%%%%%%%%%%%%%%%%%%%%%%%%%%

%%% For arguments later:
\tikzset{
  /zx/args/-andL/.cd,
  % - is "towards the point", or "x value",
  % L is "perpendicular to the line going towards the point" (L is for the perpendicular shape), or "y value".
  1-/.initial=.5, % random value, defaultO... will change that.
  2-/.initial=.5,
  -/.style={
    1-=#1,
    2-=#1,
  },
  1L/.initial=.5,
  2L/.initial=.5,
  L/.style={
    1L=#1,
    2L=#1,
  },
  symmetry-L/.code={% 1- <-> 1L, 2- <-> 2L
    \edef\zx@tmpone{\pgfkeysvalueof{/zx/args/-andL/1-}}%
    \edef\zx@tmptwo{\pgfkeysvalueof{/zx/args/-andL/2-}}%
    \pgfkeysalso{%
      1-/.evaluated=\pgfkeysvalueof{/zx/args/-andL/1L},
      2-/.evaluated=\pgfkeysvalueof{/zx/args/-andL/2L},
      1L/.evaluated=\zx@tmpone,
      2L/.evaluated=\zx@tmptwo,
    }%
  },
  symmetry/.code={% 1- <-> 2-, 1L <-> 2L
    \edef\zx@tmpone{\pgfkeysvalueof{/zx/args/-andL/1-}}%
    \edef\zx@tmptwo{\pgfkeysvalueof{/zx/args/-andL/1L}}%
    \pgfkeysalso{
      1-/.evaluated=\pgfkeysvalueof{/zx/args/-andL/2-},
      1L/.evaluated=\pgfkeysvalueof{/zx/args/-andL/2L},
      2-/.evaluated=\zx@tmpone,
      2L/.evaluated=\zx@tmptwo,
    }%
  },
  negate1L/.style={
    /zx/args/-andL/1L/.evaluated=-\pgfkeysvalueof{/zx/args/-andL/1L},
  },
  negate2L/.style={
    /zx/args/-andL/2L/.evaluated=-\pgfkeysvalueof{/zx/args/-andL/2L},
  },
  negateL/.style={
    negate1L,
    negate2L,
  },
  negate1-/.style={
    /zx/args/-andL/1-/.evaluated=-\pgfkeysvalueof{/zx/args/-andL/1-},
  },
  negate2-/.style={
    /zx/args/-andL/2-/.evaluated=-\pgfkeysvalueof{/zx/args/-andL/2-},
  },
  negate-/.style={
    negate1-,
    negate2-,
  },
  negateL/.style={
    negate1L,
    negate2L,
  },
  oneMinus1-/.style={
    /zx/args/-andL/2-/.evaluated=1-\pgfkeysvalueof{/zx/args/-andL/2-},
  },
  oneMinus2-/.style={
    /zx/args/-andL/2-/.evaluated=1-\pgfkeysvalueof{/zx/args/-andL/2-},
  },
  oneMinus1L/.style={
    /zx/args/-andL/1L/.evaluated=1-\pgfkeysvalueof{/zx/args/-andL/1L},
  },
  oneMinus2L/.style={
    /zx/args/-andL/2L/.evaluated=1-\pgfkeysvalueof{/zx/args/-andL/2L},
  },
  % Angle/length:
  1 angle and length/.code 2 args={%
    \pgfmathparse{#2*cos(#1)}%
    \pgfkeysalso{1-/.expand once=\pgfmathresult}%
    \pgfmathparse{#2*sin(#1)}%
    \pgfkeysalso{1L/.expand once=\pgfmathresult}%
  },
  2 angle and length/.code 2 args={%
    \pgfmathparse{#2*cos(#1)}%
    \pgfkeysalso{2-/.expand once=\pgfmathresult}%
    \pgfmathparse{#2*sin(#1)}%
    \pgfkeysalso{2L/.expand once=\pgfmathresult}%
  },
  1al/.style 2 args={
    1 angle and length={#1}{#2}
  },
  2al/.style 2 args={
    2 angle and length={#1}{#2}
  },
  al/.style 2 args={
    1 angle and length={#1}{#2},
    2 angle and length={#1}{#2},
  },
  1 angle/.style={
    1 angle and length={#1}{.6}
  },
  2 angle/.style={
    2 angle and length={#1}{.6}
  },
  1a/.style={
    1 angle={#1},
  },
  2a/.style={
    2 angle={#1},
  },
  angle/.style={
    1 angle={#1},
    2 angle={#1},
  },
  a/.style={
    1 angle and length={#1}{.6},
    2 angle and length={#1}{.6},
  },
}

%%% Useful debug functions
%% This function will be called when debugging the library.
\def\zx@message#1{}
% The ^^J adds a new line:
\def\zx@verbose@message#1{\message{#1^^J}} % For more important messages. Like to show the code is not looping forever
% To enable debugging:
% WARNING: if you use in code, do not forget the makeatletter!
%\def\zx@message#1{\message{#1^^J}}
% To display stroke of while trying to compute the intersections:
% WARNING: if you use in code, do not forget the makeatletter!
%\def\zx@draw@stroke@inter@debug{}


%% This function checks if the node has a "name path", if so it computes the intersection between
%% if not if \zxWireInsideIfNoIntersectionName is defined it defines the \tikzto* to the
%% new@center (argument #2), otherwise it does nothing.
% #1: Node to consider: \tikztostart or \tikztotarget
% #2: name of the new@center node corresponding to #1
% #3: name of the other new@center node
% #4: result will be stored in this macro... if there is something to change.
% #5: 1 if start, 2 if end
% Result will be moved back into #1
\def\zx@find@intersection@fakecenter#1#2#3#4#5{%
  % First, we check if the user wants to compute the intersection:
  \ifdefined\zxEnableIntersectionsWires%
     %% We compute the intersection
    \edef\tikz@intersect@path@a{zx@name@path@#1}%
    \ifcsname tikz@intersect@path@name@\tikz@intersect@path@a \endcsname%
      \zx@verbose@message{[ZX] I will start computing intersections}%
      \pgfintersectionofpaths{%
        \begin{pgfinterruptboundingbox}%
          %%%% /!\ This lines is the one I'm not sure how to write:
          \expandafter\pgfsetpath\csname tikz@intersect@path@name@\tikz@intersect@path@a\endcsname%
          \pgfgetpath\temppath%
          \ifdefined\zx@draw@stroke@inter@debug%
            \pgfusepath{stroke} % We draw it, useful to debug, and realize the shape is moved.
          \fi%
          \pgfsetpath\temppath%
        \end{pgfinterruptboundingbox}%
      }{% The first path is moved to the center... So we need to shift it also here.
        %%% First intersection: we use the line between the two nodes... but it's really ugly (e.g. in o it does not make a lot of sense)
        \begin{pgfinterruptboundingbox}%
          \ifdefined\zxIntersectionLineBetweenStartEnd%
            \pgfpointdiff{\pgfpointanchor{#1}{center}}{\tikz@scan@one@point\pgf@process(#2)}%
            \pgfpathmoveto{\pgfpoint{\pgf@x}{\pgf@y}}%
            \pgfpointdiff{\pgfpointanchor{#1}{center}}{\tikz@scan@one@point\pgf@process(#3)}%
            \pgfpathlineto{\pgfpoint{\pgf@x}{\pgf@y}}%
          \else%
            %% We use now the method used for bezier curve (at least it's what I see "experimentally")
            %% More precisely, we use the direction of the bezier curve to know where to start the wire.
            %% Parse once the control points for efficiency reasons:
            %% First parse the fake centers.
            \tikz@scan@one@point\pgf@process(#2)%
            \edef\zx@tmp@current@fake@x{\the\pgf@x}%
            \edef\zx@tmp@current@fake@y{\the\pgf@y}%
            \zx@message{The input fakes are #2 and #3.}%
            \tikz@scan@one@point\pgf@process(#3)%
            \edef\zx@tmp@other@fake@x{\the\pgf@x}%
            \edef\zx@tmp@other@fake@y{\the\pgf@y}%
            \zx@message{The parsed input fakes are (\zx@tmp@current@fake@x , \zx@tmp@current@fake@y) and (\zx@tmp@other@fake@x , \zx@tmp@other@fake@y).}%
            %% Reorder the controls to compute. #5 = 1 if "start fake center" is "current fake center".
            \ifnum#5=1 % <-- important space!!!
              \edef\zx@tmp@start@fake@x{\zx@tmp@current@fake@x}%
              \edef\zx@tmp@start@fake@y{\zx@tmp@current@fake@y}%
              \edef\zx@tmp@end@fake@x{\zx@tmp@other@fake@x}%
              \edef\zx@tmp@end@fake@y{\zx@tmp@other@fake@y}%
            \else%
              \edef\zx@tmp@end@fake@x{\zx@tmp@current@fake@x}%
              \edef\zx@tmp@end@fake@y{\zx@tmp@current@fake@y}%
              \edef\zx@tmp@start@fake@x{\zx@tmp@other@fake@x}%
              \edef\zx@tmp@start@fake@y{\zx@tmp@other@fake@y}%
            \fi%
            \zx@message{We computed: start fake (\zx@tmp@start@fake@x , \zx@tmp@start@fake@y) end fake (\zx@tmp@end@fake@x , \zx@tmp@end@fake@y )}%
            \pgfkeysalso{/zx/tmp/zx@getControlPoints={\zx@tmp@start@fake@x}{\zx@tmp@start@fake@y}{\zx@tmp@end@fake@x}{\zx@tmp@end@fake@y}}%
            % Now results are in controlOne, and controlTwo
            \ifnum#5=1 \relax%
              \def\zx@control{\controlOne}%
            \else%
              \def\zx@control{\controlTwo}%
            \fi%
            %%% Parse new control
            \tikz@scan@one@point\pgf@process(\zx@control)%
            \edef\zx@tmp@current@ctrl@x{\the\pgf@x}%
            \edef\zx@tmp@current@ctrl@y{\the\pgf@y}%
            \zx@message{[ZX] Current control is number #5, i.e. \zx@control (both are \controlOne and \controlTwo)}%
            % We store the center coordinates since it's used several times
            \pgfpointanchor{#1}{center}%
            \edef\zx@tmp@real@center@x{\the\pgf@x}%
            \edef\zx@tmp@real@center@y{\the\pgf@y}%
            %%% Go to  current fake center, except that real center is the new origin.
            \pgfpointdiff{\pgfpoint{\zx@tmp@real@center@x}{\zx@tmp@real@center@y}}{\pgfpoint{\zx@tmp@current@fake@x}{\zx@tmp@current@fake@y}}%
            \pgfpathmoveto{\pgfpoint{\pgf@x}{\pgf@y}}%
            %% What, now both axis are inverted??? If someone can explain me this...
            %% We scale the point to force it to cross the boundary.
            \pgfpathlineto{%
              \pgfpointadd{%
                \pgf@process{%
                  \pgfpointscale{10}{%
                    \pgf@process{%
                      \pgfpointdiff{%
                        \pgfpoint{\zx@tmp@current@fake@x}{\zx@tmp@current@fake@y}%
                      }{%
                        \pgfpoint{\zx@tmp@current@ctrl@x}{\zx@tmp@current@ctrl@y}%
                      }%
                    }%
                  }%
                }%
              }{%
                \pgfpointdiff{%
                  \pgfpoint{\zx@tmp@real@center@x}{\zx@tmp@real@center@y}%
                }{%
                  \pgfpoint{\zx@tmp@current@fake@x}{\zx@tmp@current@fake@y}%
                }%
              }%
            }%
          \fi%
          \pgfgetpath\temppath%
          \ifdefined\zx@draw@stroke@inter@debug%
            \pgfusepath{stroke}% We draw it, useful to debug, and realize the shape is moved.
          \fi%
          \pgfsetpath\temppath%
        \end{pgfinterruptboundingbox}%
      }%
      \pgfintersectionsolutions%
      \pgfpointintersectionsolution{1}%
      \ifnum\pgfintersectionsolutions=1%% The good number of solution has been found!
        \zx@message{[ZX] Cool, just found one solution!}
        %% Store the intersection (warning: the center of the shape is moved to the center!)
        \edef\zx@relinter@x{\the\pgf@x}%
        \edef\zx@relinter@y{\the\pgf@y}%
        %% Because the shape was moved to center, we shift it back by adding the coord of the shape:
        \pgfextractx\pgf@xa{\pgfpointanchor{#1}{center}}%
        \pgfextractx\pgf@xb{\pgfpointanchor{#1}{center}}%
        % WARNING! pgfmath removes the dimension (converted in pt). Make sure to put them back after
        \pgfmathsetmacro{\zx@inter@x}{\pgf@x+\zx@relinter@x}%
        \pgfmathsetmacro{\zx@inter@y}{\pgf@y+\zx@relinter@y}%
        \edef#4{\zx@inter@x pt,\zx@inter@y pt}%
      \else% The bad number of solution has been found
        \zx@message{[ZX] WARNING: expecting one solution, found \pgfintersectionsolutions}%
        \ifdefined\zxWireInsideIfNoIntersectionName%
          % No intersection was found, but still want to use fake center (or center).
          % ==> The wire will be inside the shape
          % Basically \tikztostart := tikztostart@new@center
          \edef#4{#2}%
        \fi% Otherwise, do nothing
      \fi%
    \else%
      \zx@message{ZX: no path named \tikz@intersect@path@a was found.}%
      \ifdefined\zxWireInsideIfNoIntersectionName%
        % No intersection was found, but still want to use fake center (or center).
        % ==> The wire will be inside the shape
        % Basically \tikztostart := tikztostart@new@center
        \edef#4{#2}%
      \fi% Otherwise, do nothing
    \fi%
  \else%
      \zx@message{[ZX]: intersection mechanism disabled}%
      \ifdefined\zxWireInsideIfNoIntersectionName%
        % No intersection was found, but still want to use fake center (or center).
        % ==> The wire will be inside the shape
        % Basically \tikztostart := tikztostart@new@center
        \edef#4{#2}%
      \fi% Otherwise, do nothing
  \fi%
}%

\def\zxRevertToNormalSpacing{%
  \thinmuskip=3.0mu\relax%
  \medmuskip=4.0mu plus 2.0mu minus 4.0mu\relax%
  \thickmuskip=5.0mu plus 5.0mu\relax%
}

% Styles. User should not modify "wires definition", but is free to change:
% - "/zx/default style nodes/" to change completely the node style
% - "/zx/user overlay nodes" to add stuff on current node style
% - "/zx/default style wires" to change the wire style
% - "/zx/user overlay wires/" to add stuff on wire style
% The user is not supposed to use node styles directly (use \zxZ{}, \zxZ{\alpha+\beta}, \zxFrac-{\pi}{4}...)
% but is free (and encouraged) to use the styles in "wires definition" like \ar[r,o'].
\tikzset{
  /zx/wires definition/.style={
    %%% Basic default properties
    draw,
    -,
    line width=\zxDefaultLineWidth,
    % This changes globally the thickness of boxes and wires, empty by default,
    % mostly useful to change gates and wires at the same time
    zx thickness wires style,
    %%% Useful shortcut (shorter lines means easy "align" of & symbols. Love M-x align in emacs btw.)
    ls/.style={looseness=##1},
    looseness wires only/.style={% Looseness used for wires only.
      looseness=1.2,
    },
    lw/.style={looseness wires only},
    bold/.style={very thick},
    non bold/.style={
      line width=\zxDefaultLineWidth,
      % We need to disable bold later since "bold wires" will automatically add bold at the end
      bold/.style={},
    },
    B/.style={bold},
    NB/.style={non bold},
    %% Adds a n above to denote the number of wires
    Bn/.style={BnArgs={##1}{}},
    Bn/.default={n},
    boldn/.style={BnArgs={##1}{}},
    boldn/.default={n},
    Bn'/.style={Bn'Args={##1}{}},
    Bn'/.default={n},
    boldn'/.style={Bn'Args={##1}{}},
    boldn'/.default={n},
    Bn./.style={Bn.Args={##1}{}},
    Bn./.default={n},
    boldn./.style={Bn.Args={##1}{}},
    boldn./.default={n},
    Bn-/.style={Bn-Args={##1}{}},
    Bn-/.default={n},
    boldn-/.style={Bn-Args={##1}{}},
    boldn-/.default={n},
    %% For wires left to right
    Bn'Args/.style 2 args={bold, "|" {rotate=65-90,anchor=center,##2}, "##1" {above,yshift=3pt,##2}},
    Bn'Args/.default={n}{},
    Bn.Args/.style 2 args={bold, "|" {rotate=65-90,anchor=center,##2}, "##1" {below,yshift=-3pt,##2}},
    Bn.Args/.default={n}{},
    %% for wires top/down (left,right)
    BnArgs/.style 2 args={bold, "|" {rotate=-65,anchor=center,##2}, "##1" {left,xshift=-2pt,##2}},
    BnArgs/.default={n}{},
    Bn-Args/.style 2 args={bold, "|" {rotate=-65,anchor=center,##2}, "##1" {right,xshift=2pt,##2}},
    Bn-Args/.default={n}{},
    % boldnPos/.style={bold, "/" {anchor=center}, "##1" {yshift=3pt}},
    % boldnPos/.default={n}{pos=50},
    % boldn.Pos/.style={bold, "/" {anchor=center}, "##1" {below,yshift=-3pt}},
    % boldn.Pos/.default={n}{pos=50},
    % Use this when you are drawing lines between none nodes only (like swap gates...) in IO mode
    between none/.style={
      looseness wires only,
      wire centered
    },
    bn/.style={
      between none
    },
    % To debug the bezier curve.
    edge above/.code={%
      \def\zxEdgesAbove{}%
    },%
    edge not above/.code={%
      \let\zxEdgesAbove\undefined%
    },
    control points visible/.code={%
      \def\zxControlPointsVisible{}%
    },%
    control points not visible/.code={%
      \let\zxControlPointsVisible\undefined%
    },
    % See explaination at the beginning where we define \def\zxEnableIntersectionsWires{}
    % Rouhgly, tries to start the wires on the border of the shape on the line between the
    % fake anchors.
    use intersections/.code={%
      \def\zxEnableIntersectionsWires{}%
    },
    dont use intersections/.code={%
      \let\zxEnableIntersectionsWires\undefined%
    },
    ui/.style={
      use intersections,
      /zx/args/-andL/ui
    },
    intersections mode between start end/.code={%
      \def\zxIntersectionLineBetweenStartEnd{}%
    },
    intersections mode bezier controls/.code={%
      \let\zxIntersectionLineBetweenStartEnd\undefined%
    },
    wires behind nodes/.code={%
      \def\zxWireInsideIfNoIntersectionName{}%
    },
    no wires behind nodes/.code={%
      \let\zxWireInsideIfNoIntersectionName\undefined%
    },
    intersection between start end/.code={
      \def\zxIntersectionLineBetweenStartEnd{}%
    },
    intersection bezier control/.code={
      \let\zxIntersectionLineBetweenStartEnd\undefined%
    },
    % ------------------------------
    % Practical stuff to draw lines easily:
    % Prefer to use these are they can be easily customized for each style and shorter to type.
    % Note that the letter is supposed to represent the shape of the link
    % dots/dashes are used to specify the position of the arrow.
    % Typically ' means top, . bottom, X- is right to X (or should arrive with angle 0),
    % -X is left to X (or should leave with angle zero). These shapes are usually designed to
    % work when the starting node is left most (or above of both nodes have the same column).
    % But they may work both way for some of them.
    % ------------------------------
    %%% Cup/Cap
    % Like a C shape (with without a perfect half circle). Useful maybe when perfect circles are too big.
    % If only tex was a functional language... https://tex.stackexchange.com/questions/618955
    C@generic/.style n args={8}{ % min/max, angle1, angle2, anchor, \x or \y, \y or \x, where to move,radius code (for circle should be "radius=\n3")
      to path={
        \pgfextra{%% <- we will use def... so need to "exit" a few seconds pgf
          \zxTryFirstSubnodeThenAnchorThenNode{\tikztostart}{true ##4}{##4}%
          \edef\StartPoint{\zx@result}%
          % % Test if tikztostart is a point or a node, and define StartPoint accordingly.
          % \ifPgfpointOrNode{\tikztostart}{%
          %   \def\StartPoint{\tikztostart}%
          % }{%
          %   % For some nodes, "north" might not point to north (notably after a rotation). So we can define some anchors "true north" that actually point
          %   % to the north.
          %   % Test if "true north" etc exists:
          %   \ifAnchorExistsNodeSpecific{\tikztostart}{true ##4}{%
          %     \def\StartPoint{\tikztostart.true ##4}%
          %   }{%
          %     \def\StartPoint{\tikztostart.##4}%
          %   }%  
          % }%
          % Test if tikztostart is a point or a node, and define StartPoint accordingly.
          \zxTryFirstSubnodeThenAnchorThenNode{\tikztotarget}{true ##4}{##4}%
          \edef\TargetPoint{\zx@result}%
          % \ifPgfpointOrNode{\tikztotarget}{%
          %   \def\TargetPoint{\tikztotarget}%
          % }{%
          %   % Test if "true north" etc exists:
          %   \ifAnchorExistsNodeSpecific{\tikztotarget}{true ##4}{%
          %     \def\TargetPoint{\tikztotarget.true ##4}%
          %   }{%
          %     \def\TargetPoint{\tikztotarget.##4}%
          %   }%  
          % }%
        }%
        (\StartPoint) % <- the path starts at StartPoint
        %%% Get x coordinate of left-most point
        let \p1=(\StartPoint),
        \p2=(\TargetPoint),
        \n1={##1(##51,##52)}, % coordinate of the most left part (when ##1=min and ##5=\x: ##52 goes to \x2)
        \n3={abs(##61-##62)/2} % Radius of circle
        in % Warning: no comma after last line before in
        %%%% We go on the left if needed (we check that we do move, otherwise we break the arrows tip if
        %%%% we stay on place
        %%%% First go to the left if needed
        \pgfextra{%
          %% We check if we are moving or not (required to preserve arrow tip direction)
          \pgfmathapproxequalto{##51}{\n1}%
        }%
        \ifpgfmathcomparison\else -- ##7\fi
        %%%% Version 1:
        \pgfextra{%
          \pgfmathparse{%
            ifthenelse(##61<##62, % if end angle < angle, draw clockwis
            "arc[start angle=##3,end angle=##2,##8]",%
            "arc[start angle=##2,end angle=##3,##8]"%
            )%
          }%
        }%
        \pgfmathresult%
        \pgfextra{%
          %% We check if we are moving or not (required to preserve arrow tip direction)
          \pgfmathapproxequalto{##52}{\n1}%
        }%
        \ifpgfmathcomparison\else -- (\TargetPoint)\fi
        \tikztonodes% All to path finishes with that to deal with future nodes I think
      }
    },
    % At the first version, styles were defined using in=... out=... looseness=... However
    % it gives sometimes bad results (like the curve goes forward at some points) when nodes are
    % too close or too far appart. However, it may still be useful, so now we define the old
    % styles, that you can use them using \ar[r,IO,C].
    % NB: for the newest styles, we add new anchors to the nodes depending on the direction.
    C/.style={C@generic={min}{90}{180+90}{west}{\x}{\y}{(\n1,\y1)}{y radius=\n3, x radius=##1*\n3}},
    C/.default=1,
    % Like C, but rotated
    C-/.style={C@generic={max}{90}{-90}{east}{\x}{\y}{(\n1,\y1)}{y radius=\n3, x radius=##1*\n3}},
    C-/.default=1,
    C'/.style={C@generic={max}{0}{180}{north}{\y}{\x}{(\x1,\n1)}{x radius=\n3, y radius=##1*\n3}},
    C'/.default=1,
    C./.style={C@generic={min}{0}{-180}{south}{\y}{\x}{(\x1,\n1)}{x radius=\n3, y radius=##1*\n3}},
    C./.default=1,
    %%% bezier{px1}{py1}{px2}{py2} creates a bezier curve where px1/py1 are the
    %%% coordinates of the first control (in "percentage" 0<px1<1, between start and end point) and px2/py2
    %%% are the coordinates in % of the second control point
    %%% bezier@general is used to construct also x/y versions, 5 and 6'th argument are \x or \y.
    bezier@general/.style n args={6}{
      possibly@on@edge@layer/.code={%% see "edges above" style. for debug mode only.
        \ifdefined\zxEdgesAbove\else%
          \pgfkeysalso{%
            on layer=edgelayer%
          }%
        \fi%
      },
      possibly@on@edge@layer,
      to path={%
        %%% First, check that inputs/outputs are points.
        \pgfextra{% <- Type latex code inside path
          %% Setup the startup node. This code can be changed by styles
          %% /zx/tmp/zx@setup@start@node will contain the anchor of the first fake center
          %% /zx/tmp/zx@setup@end@node will contain the anchor of the second fake center
          %% If none of them are defined, nothing happens and the start point will be a node.
          % Retrieve the content of the keys
          % But first check if they are defined
          \pgfkeysifdefined{/zx/tmp/zx@setup@start@node}{}{% Put an empty value
            \pgfkeysalso{/zx/tmp/zx@setup@start@node/.initial=,}%
          }%
          \pgfkeysifdefined{/zx/tmp/zx@setup@end@node}{}{% Put an empty value
            \pgfkeysalso{/zx/tmp/zx@setup@end@node/.initial=,}%
          }%
          \pgfkeysalso{
            /zx/tmp/zx@setup@start@node/.get=\zx@anchor@start@node,
            /zx/tmp/zx@setup@end@node/.get=\zx@anchor@end@node,
            /zx/tmp/zx@getControlPoints/.code n args={4}{%
              \zx@message{I will evaluate my control points:}%
              % Args: start point, x,y then end point x,y
              % Use ########1 ... (yes, LaTeX is a funny exponential language, and I love nested functions)
              % Because to path is considered as a function already.
              \def\x################1{%
                \ifnum################1=1 % <-- super important space...
                  ########1%
                \else%
                  ########3%
                \fi%
              }%
              \zx@message{[ZX] x is defined.}%
              \def\y################1{%
                \ifnum ################1 = 1 % <-- super important space...
                  ########2%
                \else%
                  ########4%
                \fi%
              }%
              \tikz@scan@one@point\pgf@process(\x1-##51*##1+##52*##1,\y1-##61*##2+##62*##2)%
              \edef\controlOne{\the\pgf@x,\the\pgf@y}%
              \tikz@scan@one@point\pgf@process(\x1-##51*##3+##52*##3,\y1-##61*##4+##62*##4)%
              \edef\controlTwo{\the\pgf@x,\the\pgf@y}%
              \zx@message{[ZX] First control point is: \controlOne, second is \controlTwo.}%
            },%
          }%
          %%% This will be the new \tikztostart and \tikztotarget... if someone defines them
          \let\new@tikztostart\undefined%
          \let\new@tikztotarget\undefined%
          % Check if a start anchor was defined...
          \ifthenelse{\equal{\zx@anchor@start@node}{}}{%
            % No start anchor was chosen... :c
            \ifthenelse{\equal{\zx@anchor@end@node}{}}{%
              % Nothing was defined... So I won't do anything! :C
            }{%
              % An end node was defined :D Let's configure its end anchor.
              \pgfkeysalso{%
                /zx/utils/change@tikz@fake@center@intersect={\tikztotarget}{\zx@anchor@end@node}{\tikztostart}{center}{\new@tikztotarget}{2},
              }%
            }%
          }{% A start node was chosen :D
            \ifthenelse{\equal{\zx@anchor@end@node}{}}{%
              % No end node was chosen... I'll take center to configure the start node,
              % and not configure the end node.
              \pgfkeysalso{
                /zx/utils/change@tikz@fake@center@intersect={\tikztostart}{\zx@anchor@start@node}{\tikztotarget}{center}{\new@tikztostart}{1},
              }%
            }{%
              % And an end node was chosen :D
              \zx@message{[ZX] Both anchors are set. Lets configure intersection.}%
              \pgfkeysalso{
                /zx/utils/change@tikz@fake@center@intersect={\tikztostart}{\zx@anchor@start@node}{\tikztotarget}{\zx@anchor@end@node}{\new@tikztostart}{1},
                /zx/utils/change@tikz@fake@center@intersect={\tikztotarget}{\zx@anchor@end@node}{\tikztostart}{\zx@anchor@start@node}{\new@tikztotarget}{2},
              }%
            }%
          }%
          \ifdefined\new@tikztostart%
            \edef\tikztostart{\new@tikztostart}%
          \fi%
          \ifdefined\new@tikztotarget%
            \edef\tikztotarget{\new@tikztotarget}%
          \fi%
          %%% Test if tikztostart is a point or a node, and define StartPoint accordingly.
          % \ifPgfpointOrNode{\tikztostart}{%
          %   \def\StartPoint{\tikztostart}%
          % }{%
          %   \def\StartPoint{\tikztostart.center}%
          % }%
          % % Test if tikztostart is a point or a node, and define StartPoint accordingly.
          % \ifPgfpointOrNode{\tikztotarget}{%
          %   \def\TargetPoint{\tikztotarget}%
          % }{%
          %   \def\TargetPoint{\tikztotarget.center}%
          % }%
          %%% Well, let's just use \tikztostart and \tikztotarget in any case, setting it to
          %%% center will put the wire below the node which is not really desired for H nodes
          %%% (position will be wrong) or when using arrows. And if center is really desired
          %%% it's easy to force it using start anchor and end anchor.
          \def\StartPoint{\tikztostart}%
          \def\TargetPoint{\tikztotarget}%
          %%% TODO: check that it does not change the shape of the wire since I'm not sure what \x1
          %%% is in that case%
        }%
        let
        \p1=(\StartPoint),
        \p2=(\TargetPoint),%% ##5 = \x or \y, so ##51 = \x1 or \y1.
        \p{control1}=(\x1-##51*##1+##52*##1, \y1-##61*##2+##62*##2),
        \p{control2}=(\x1-##51*##3+##52*##3, \y1-##61*##4+##62*##4) in
        (\StartPoint) .. controls (\p{control1}) and (\p{control2}) .. (\TargetPoint)
        \ifdefined\zxControlPointsVisible
          (\StartPoint) node[anchor=center,circle,fill=yellow,inner sep=.5pt,on layer=foreground]{}
          -- (\p{control1}) node[anchor=center,circle,fill=orange,inner sep=.5pt,on layer=foreground]{}
          (\p{control2}) node[anchor=center,circle,fill=purple,inner sep=.5pt,on layer=foreground]{}
          -- (\TargetPoint) node[anchor=center,circle,fill=brown,inner sep=.5pt,on layer=foreground]{}
        \fi
        \tikztonodes
      },
    },
    %% Not sure it will be very useful, compared to bezier, bezier x and bezier y directly.
    bezier 4/.style n args={4}{
      bezier@general={##1}{##2}{##3}{##4}{\x}{\y},
    },
    % Like bezier, all proportions are relative to x distance
    bezier 4 x/.style n args={4}{
      bezier@general={##1}{##2}{##3}{##4}{\x}{\x}
    },
    % Like bezier, all proportions are relative to y distance
    bezier 4 y/.style n args={4}{
      bezier@general={##1}{##2}{##3}{##4}{\y}{\y}
    },
    bezier/.code={%
      \pgfkeys{
        /zx/args/-andL/.cd,
        ##1
      }%
      \pgfkeysalso{
        bezier 4={\pgfkeysvalueof{/zx/args/-andL/1-}}{\pgfkeysvalueof{/zx/args/-andL/1L}}{\pgfkeysvalueof{/zx/args/-andL/2-}}{\pgfkeysvalueof{/zx/args/-andL/2L}}
      }%
    },
    bezier x/.code={%
      \pgfkeys{
        /zx/args/-andL/.cd,
        ##1
      }%
      \pgfkeysalso{
        bezier 4 x={\pgfkeysvalueof{/zx/args/-andL/1-}}{\pgfkeysvalueof{/zx/args/-andL/1L}}{\pgfkeysvalueof{/zx/args/-andL/2-}}{\pgfkeysvalueof{/zx/args/-andL/2L}}
      }%
    },
    bezier y/.code={%
      \pgfkeys{
        /zx/args/-andL/.cd,
        ##1
      }%
      \pgfkeysalso{
        bezier 4 y={\pgfkeysvalueof{/zx/args/-andL/1-}}{\pgfkeysvalueof{/zx/args/-andL/1L}}{\pgfkeysvalueof{/zx/args/-andL/2-}}{\pgfkeysvalueof{/zx/args/-andL/2L}}
      }%
    },
    %% /zx/utils/change@tikz@fake@center={\tikztostart}{fake center west}{\myoutput} will set \myoutput to be the coordinate of the fake center
    %% if it exists, and of center otherwise (and if tikztostart is already a coordinate, we just set myoutput to this coordinate):
    % Directions: configure start/end points to "fake center *" if it exists.
    % If the input is not a node, it's copied to ##3
    /zx/utils/change@tikz@fake@center/.code n args={3}{%
      % ##1: \tikztostart or \tikztotarget,
      % ##2: fake center north...
      % ##3 name of result. e.g. \tikztostart
      % This sets \zx@result to the appropriate node or to center if no exists:
      \zxTryFirstSubnodeThenAnchorThenNodeToCoordinate{##1}{##2}{center}%
      \edef##3{\zx@result}%
    },
    % /zx/utils/change@tikz@fake@center/.code n args={3}{%
    %   % ##1: \tikztostart or \tikztotarget,
    %   % ##2: fake center north...
    %   % ##3 name of result. e.g. \tikztostart
    %   \ifPgfpointOrNode{##1}{%
    %     %%% Sometimes, ##1 is not (1pt,2pt) but "tikz-1-2.north east". Since later we use
    %     %%% a parsing mechanism which assumes that we have a coordinate... this is a problem.
    %     \edef##3{##1}%
    %     % This is a great explaination of how parsing is done in tikz:
    %     % https://tex.stackexchange.com/a/516897/116348
    %   }{% This is a node, not a point:
    %     % "zx add anchor to node" might define per-node anchors, and they are stored in
    %     % a macro called like \pgf@sh@ma@tikz@f@3-2-2, so we need to read this first:
    %     % TODO: maybe use pp@name everywhere? (see comment beginning of this file)
    %     \begingroup% This group avoids pollution done by executing pgf@sh@ma@NODENAME
    %       \pgfutil@ifundefined{pgf@sh@ma@\tikz@pp@name{##1}}{}{%
    %         % We the per-node macro exists, we run it.
    %         \csname pgf@sh@ma@\tikz@pp@name{##1}\endcsname%
    %       }%
    %       \ifAnchorExists{##1}{##2}{%
    %         \pgfpointanchor{##1}{##2}%
    %       }{%
    %         \pgfpointanchor{##1}{center}%
    %       }%
    %       %%% Whoo, pgfpoint only redefine pgf@x/pgf@y, but in no way it stores it!!!
    %       %%% So can't do \pgfpoint{\pgf@x}{\pgf@y}...
    %       %%% I needed to create my own system... see \zx@parse@coordx.
    %       \xdef##3{\the\pgf@x,\the\pgf@y}%
    %     \endgroup%
    %   }%
    % },
    % Directions: configure start/end points to the intersection between
    % the "fake center *" if it exists and the "fake center *" of the destination.
    % If the input is not a node, nothing is done.
    /zx/utils/change@tikz@fake@center@intersect/.code n args={6}{%
      % ##1: \tikztostart or \tikztotarget
      % ##2: fake center north... for ##1
      % ##3: \tikztostart or \tikztotarget, the other one (needed to compute the intersection)
      % ##4 fake center north... for ##3
      % ##5: result will be moved to that macro
      % ##6: 1 if start, 2 if end.
      \ifPgfpointOrNode{##1}{}{% This is a node, not a point:
        %% Configure the zx@new@center@* which should point to "fake center" if it exists.
        \pgfkeysalso{%
          /zx/utils/change@tikz@fake@center={##1}{##2}{\zx@new@center},%
        }%
        \pgfkeysalso{%
          /zx/utils/change@tikz@fake@center={##3}{##4}{\zx@new@center@other},%
        }%
        \zx@message{ZX: coordinate of new center is \zx@new@center}%
        \zx@message{ZX: coordinate of new other center is \zx@new@center@other}%
        \zx@find@intersection@fakecenter{##1}{\zx@new@center}{\zx@new@center@other}{##5}{##6}%
      }%
    },
    start fake center east/.style={%
      /zx/tmp/zx@setup@start@node/.initial=fake center east,
    },
    start fake center west/.style={%
      /zx/tmp/zx@setup@start@node/.initial=fake center west,
    },
    start fake center north/.style={%
      /zx/tmp/zx@setup@start@node/.initial=fake center north,
    },
    start fake center south/.style={%
      /zx/tmp/zx@setup@start@node/.initial=fake center south,
    },
    start real center/.style={
      /zx/tmp/zx@setup@start@node/.initial=center,
    },
    start no fake center/.style={
      /zx/tmp/zx@setup@start@node/.initial=,
    },
    end fake center east/.style={%
      /zx/tmp/zx@setup@end@node/.initial=fake center east,
    },
    end fake center west/.style={%
      /zx/tmp/zx@setup@end@node/.initial=fake center west,
    },
    end fake center north/.style={%
      /zx/tmp/zx@setup@end@node/.initial=fake center north,
    },
    end fake center south/.style={%
      /zx/tmp/zx@setup@end@node/.initial=fake center south,
    },
    end real center/.style={
      /zx/tmp/zx@setup@end@node/.initial=center,
    },
    end no fake center/.style={
      /zx/tmp/zx@setup@end@node/.initial=,
    },
    left to right/.style={
      start fake center east,
      end fake center west
    },
    right to left/.style={
      start fake center west,
      end fake center east
    },
    up to down/.style={
      start fake center south,
      end fake center north
    },
    down to up/.style={
      start fake center north,
      end fake center south
    },
    no fake center/.style={
      start no fake center,
      end no fake center,
      start anchor=,
      end anchor=,
    },
    % Classical wires: we make them bend
    classical/.style={
      double distance=1pt,
      double,
      line cap=rect,
      % the .5pt is half of the double distance
      shorten <=.5pt+.5\pgflinewidth,
      shorten >=.5pt+.5\pgflinewidth,
      % to path={},
    },
    cl/.style={classical},
    connect -|/.style={
      on layer={edgelayer},
      to path={(\tikztostart) -| (\tikztotarget) \tikztonodes}
    },
    --|/.style={connect -|},
    connect |-/.style={
      on layer={edgelayer},
      to path={(\tikztostart) |- (\tikztotarget) \tikztonodes}
    },
    |--/.style={connect |-},
    % fake a wire being above by putting white stuff around
    3d above/.style={
      draw=white,
      double=black,
      double distance=\pgflinewidth,
      very thick,
    },
    %% To go to a subnode defined in a pic with from subnode=mysubnode.
    start subnode/.code={%
      \zxGetSubnodeOfNode{\tikzcd@ar@start}{##1}{\edef\tikzcd@ar@start{\zx@result}}{\PackageError{zx-calculus}{In "start subnode=##1", \zx@error}}%
    },
    % start subnode/.code={%
    %   \edef\zx@tmp{zx@alias@\zxCurrentDiagram @\tikzcd@ar@start}%
    %   % First, rewrite \tikzcd@ar@start so that it equals the name of the current node without alias
    %   \ifcsname \zx@tmp\endcsname% There exists an alias
    %     \edef\zx@tmp@alias{(alias \zx@tmp) }% Mostly for the error message
    %     \edef\tikzcd@ar@start{\csname zx@alias@\zxCurrentDiagram @\tikzcd@ar@start\endcsname}%
    %   \else%
    %     \edef\zx@tmp@alias{(no alias) }% Mostly for the error message
    %   \fi
    %   % First, we check if there is a node with this name on this node. If not, we will try with the parent node
    %   \@ifundefined{pgf@sh@ns@\tikzcd@ar@start ##1}{%
    %     % The current node has no such subnode name: let's start its parent.
    %     \@ifundefined{\tikzcd@ar@start -zxOriginalRow}{%
    %       % There is no original coordinate: error
    %       \PackageError{zx-calculus}{In "start subnode=##1", the subnode ##1 of node \tikzcd@ar@start \zx@tmp@alias does not exist.}{}%
    %     }{%
    %       % There is an original coordinate: we try to see if the original node admits a subnode.
    %       \edef\zx@tmp@originalRow{\csname \tikzcd@ar@start-zxOriginalRow\endcsname}%
    %       \edef\zx@tmp@originalCol{\csname \tikzcd@ar@start-zxOriginalCol\endcsname}%
    %       \@ifundefined{pgf@sh@ns@\zxCurrentDiagram-\zx@tmp@originalRow-\zx@tmp@originalCol ##1}{%
    %         \PackageError{zx-calculus}{In "start subnode=##1", the subnode ##1 of aliased node \tikzcd@ar@start{} \zx@tmp@alias does not exist (even in the parent node at coordinate \zx@tmp@originalRow - \zx@tmp@originalCol).}{}%
    %       }{%
    %         \edef\tikzcd@ar@start{\zxCurrentDiagram-\zx@tmp@originalRow-\zx@tmp@originalCol ##1}%%
    %       }%
    %     }%
    %   }{% There exists a subnode in the current name
    %     \edef\tikzcd@ar@start{\tikzcd@ar@start ##1}%%
    %   }%
    % },
    end subnode/.code={% The edef is needed, or it always go into the first loop and might give errors later
      \zxGetSubnodeOfNode{\tikzcd@ar@target}{##1}{\edef\tikzcd@ar@target{\zx@result}}{\PackageError{zx-calculus}{In "end subnode=##1", \zx@error}}%
    },
    % end subnode/.code={% The edef is needed, or it always go into the first loop and might give errors later
    %   \edef\zx@tmp{zx@alias@\zxCurrentDiagram @\tikzcd@ar@target}%
    %   % First, rewrite \tikzcd@ar@target so that it equals the name of the current node without alias
    %   \ifcsname \zx@tmp\endcsname% There exists an alias
    %     \edef\zx@tmp@alias{(alias \zx@tmp) }% Mostly for the error message
    %     \edef\tikzcd@ar@target{\csname zx@alias@\zxCurrentDiagram @\tikzcd@ar@target\endcsname}%
    %   \else%
    %     \edef\zx@tmp@alias{(no alias) }% Mostly for the error message
    %   \fi
    %   % First, we check if there is a node with this name on this node. If not, we will try with the parent node
    %   \@ifundefined{pgf@sh@ns@\tikzcd@ar@target ##1}{%
    %     % The current node has no such subnode name: let's start its parent.
    %     \@ifundefined{\tikzcd@ar@target -zxOriginalRow}{%
    %       % There is no original coordinate: error
    %       \PackageError{zx-calculus}{In "end subnode=##1", the subnode ##1 of node \tikzcd@ar@target{} \zx@tmp@alias does not exist.}{}%
    %     }{%
    %       % There is an original coordinate: we try to see if the original node admits a subnode.
    %       \edef\zx@tmp@originalRow{\csname \tikzcd@ar@target-zxOriginalRow\endcsname}%
    %       \edef\zx@tmp@originalCol{\csname \tikzcd@ar@target-zxOriginalCol\endcsname}%
    %       \@ifundefined{pgf@sh@ns@\zxCurrentDiagram-\zx@tmp@originalRow-\zx@tmp@originalCol ##1}{%
    %         \PackageError{zx-calculus}{In "end subnode=##1", the subnode ##1 of aliased node \tikzcd@ar@target{} \zx@tmp@alias does not exist (even in the parent node at coordinate \zx@tmp@originalRow - \zx@tmp@originalCol).}{}%
    %       }{%
    %         \edef\tikzcd@ar@target{\zxCurrentDiagram-\zx@tmp@originalRow-\zx@tmp@originalCol ##1}%%
    %       }%
    %     }%
    %   }{% There exists a subnode in the current name
    %     \edef\tikzcd@ar@target{\tikzcd@ar@target ##1}%%
    %   }%
    % },
    %% Forced versions are versions which do not nicely fallback to center if shape does not
    %% have "fake center XXX". This is used mostly for wires based on in/out (our function b.
    %% Not sure how to have a code which always fallback to center if shape does not.
    %% All shapes used in the default style should work with force left to right.
    force left to right/.style={
      on layer=edgelayer,
      start anchor=fake center east,
      end anchor=fake center west
    },
    force right to left/.style={
      on layer=edgelayer,
      start anchor=fake center west,
      end fake center east
    },
    force up to down/.style={
      on layer=edgelayer,
      start anchor=fake center south,
      end fake center north
    },
    force down to up/.style={
      on layer=edgelayer,
      start anchor=fake center north,
      end fake center south
    },
    % Similar to C, but with a softer angle. The '.- marker represents the portion of
    % the circle (hence the o) to keep (top, bottom,left/right).
    % Angle is customizable, for instance o'=50.
    %%% Actually, it may not be super useful to define it via bezier since it's used only
    %%% horizontally and vertically... Anyway.
    o'/.style={%
      left to right,
      bezier x={
        defaultO,
        ##1,
        %% 2- --> 1 - (2-)
        oneMinus2-,
      },
    },
    o./.style={%
      left to right,
      bezier x={%
        /zx/args/-andL/.cd,
        defaultO,
        ##1,
        %% 2- --> 1 - (2-)
        oneMinus2-,
        negate1L,
        negate2L,
      },%
    },
    -o/.style={%
      up to down,
      bezier y={%
        /zx/args/-andL/.cd,
        defaultO,
        ##1,
        /zx/args/-andL/2-/.evaluated=1-\pgfkeysvalueof{/zx/args/-andL/2-},
        symmetry-L,
      }
    },
    o-/.style={%
      up to down,
      bezier y={%
        /zx/args/-andL/.cd,
        defaultO,
        ##1,
        /zx/args/-andL/2-/.evaluated=1-\pgfkeysvalueof{/zx/args/-andL/2-},
        /zx/args/-andL/1L/.evaluated=-\pgfkeysvalueof{/zx/args/-andL/1L},
        /zx/args/-andL/2L/.evaluated=-\pgfkeysvalueof{/zx/args/-andL/2L},
        symmetry-L,
      },
    },
    % Similar to o, but can be used also for diagonal items. Combine it with
    % The dot versions are mode logic in "left to right", and the others for up to down (possible to change)
    % Can combine with "force left to right"...
    (/.style={
      bend right=##1,
    },
    (/.default=30,
    )/.style={
      bend left=##1,
    },
    )/.default=30,
    ('/.style={
      bend left=##1
    },
    ('/.default=30,
    (./.style={
      bend right=##1
    },
    (./.default=30,
    %%%% Links with a N-shape, i.e. like s shape, but symetric against the diagonal. Equivalently, it's a soft 's' shape with a much wider angle (>45).
    %% Nbase will be reused later by < > shapes as well. It's useful to make a distinction between N and
    %% Nbase when user will want to overwrite N without overwritting < >.
    Nbase/.style={
      left to right,
      bezier={
        /zx/args/-andL/.cd,
        defaultN,
        ##1,
        oneMinus2-,
        oneMinus2L,
      },
    },
    N/.style={Nbase={##1}},
    % With this model, we don't need differences between N' and N. so we can also call it N.
    % But keep other notations to allow easy style change (to use IO again for instance).
    N'/.style={N={##1}},
    N./.style={N={##1}},
    -N/.style={N={defaultN-,##1}},
    -N'/.style={-N={##1}},
    -N./.style={-N={##1}},
    N-/.style={-N'={symmetry,##1}},
    N'-/.style={N-={##1}},
    N.-/.style={N-={##1}},
    %% Up to down version
    NN/.style={N'={symmetry-L,defaultNN,##1},up to down},
    NN./.style={NN={##1}},
    .NN/.style={NN={##1}},
    NIN/.style={NN={defaultNIN,##1}},
    INN/.style={NIN={symmetry,##1}},
    NNI/.style={INN={##1}},
    %%% <' is basically like -N' (historically < was first), we just put an anchor on the east/....
    <'/.style={Nbase={defaultN,defaultN-,defaultChevron,symmetry,##1},end anchor=west},
    <./.style={<'={##1}},
    % Don't use <>^ alone char as style name, it could be useful later for other shortcuts
    % (replacement directions)
    '>/.style={Nbase={defaultN,defaultN-,defaultChevron,##1},start anchor=east},
    .>/.style={'>={##1}},
    ^./.style={
      Nbase={defaultN,defaultN-,defaultChevron,symmetry-L,defaultNN,defaultNIN,defaultChevronDown,symmetry,##1},
      end anchor=north
    },
    .^/.style={^.={##1}},
    'v/.style={
      Nbase={defaultN,defaultN-,defaultChevron,symmetry-L,defaultNN,defaultNIN,defaultChevronDown,##1},
      start anchor=south
    },
    v'/.style={'v={##1}},
    %%%% Links with a s-like shape.
    s/.style={
      left to right,
      bezier={%
        /zx/args/-andL/.cd,
        defaultS,
        ##1,
        oneMinus2-,
        oneMinus2L,
      }
    },
    % like S, but with anchor east and west
    S/.style={
      s={##1}, start anchor=east, end anchor=west
    },
    % -s'.- shapes are like s, but with a soften (customizable like o) angle.
    % The '. say if you are going up or down, and - forces a sharp angle (- is flat) on the side of the -
    s'/.style={s={defaultS',##1}},
    s./.style={s'={##1}},
    -s/.style={s'={default-S,##1}},
    -s'/.style={-s={##1}},  % To fit with s' notation
    -s./.style={-s={##1}},
    s-/.style={-s'={symmetry,##1}},
    s'-/.style={s-={##1}},
    s.-/.style={s-={##1}},
    % Anchor more marked
    S/.style={s={##1}, start anchor=east, end anchor=west,},
    -S/.style={-s'={##1},start anchor=east},
    -S'/.style={-S={##1}},
    -S./.style={-S={##1}},
    S-/.style={s'-={##1},end anchor=west},
    S'-/.style={S-={##1}},
    S.-/.style={S-={##1}},
    %%%% Doubling = up to down.
    % Links with a s-like shape... but read from top to bottom
    ss/.style={s={symmetry-L,##1},up to down},
    SS/.style={ss={##1}, start anchor=south, end anchor=north},
    % -s'.- shapes are like s, but with a soften (customizable like o) angle.
    % The '. say if you are going up or down, and 'I' forces a sharp angle (I is flat) on the side of the I
    ss./.style={s.={symmetry-L,##1},up to down},
    .ss/.style={ss.={##1}},
    sIs/.style={ss={defaultSIS,##1}},
    sIs./.style={ss.={defaultSIS,##1}},
    .sIs/.style={sIs.={##1}},
    Iss/.style={sIs={symmetry,##1}},
    ssI/.style={Iss={##1}},
    ss.I/.style={sIs.={symmetry,##1}},
    I.ss/.style={ss.I={##1}},
    SIS/.style={sIs={##1},start anchor=south},
    .SIS/.style={.sIs={##1},start anchor=south},
    ISS/.style={ssI={##1},end anchor=north},
    SS.I/.style={ss.I={##1},end anchor=north},
    I.SS/.style={SS.I={##1}},
    SSI/.style={ISS={##1}},
    % At the first version, styles were defined using in=... out=... looseness=... However
    % it gives sometimes bad results (like the curve goes forward at some points) when nodes are
    % too close or too far appart. However, it may still be useful, so now we define the old
    % styles, that you can use them using \ar[r,IO,C].
    IO/.style={
      C/.style={/tikz/in=180,/tikz/out=180,looseness=2},
      % Like C, but symetric
      C-/.style={/tikz/in=0,/tikz/out=0,looseness=2},
      C'/.style={/tikz/in=90,/tikz/out=90,looseness=2},
      C./.style={/tikz/in=-90,/tikz/out=-90,looseness=2},
      % Similar to C, but with a softer angle. The '.- marker represents the portion of
      % the circle (hence the o) to keep (top, bottom,left/right).
      % Angle is customizable, for instance o'=50.
      o'/.style={/tikz/out=####1,/tikz/in=180-####1},
      o'/.default=\zxDefaultSoftAngleO,
      o./.style={/tikz/out=-####1,/tikz/in=180+####1},
      o./.default=\zxDefaultSoftAngleO,
      -o/.style={/tikz/out=-90-####1,/tikz/in=90+####1},
      -o/.default=\zxDefaultSoftAngleO,
      o-/.style={/tikz/out=-90+####1,/tikz/in=90-####1},
      o-/.default=\zxDefaultSoftAngleO,
      % Similar to o, but can be used also for diagonal items.
      % Why ()? Visualize fixing the top part and moving the bottom part.
      (/.style={bend right=####1},
      (/.default=30,
      )/.style={bend left=####1},
      )/.default=30,
      ('/.style={bend left=####1},
      ('/.default=30,
      (./.style={bend right=####1},
      (./.default=30,
      <'/.style={out=####1,in=180,looseness=0.65},
      <'/.default=\zxDefaultSoftAngleChevron,
      <./.style={out=-####1,in=180,looseness=0.65},
      <./.default=\zxDefaultSoftAngleChevron,
      '>/.style={out=0,in=180-####1,looseness=0.65},
      '>/.default=\zxDefaultSoftAngleChevron,
      .>/.style={out=0,in=180+####1,looseness=0.65},
      .>/.default=\zxDefaultSoftAngleChevron,
      ^./.style={out=-90+####1,in=90,looseness=0.65},
      ^./.default=\zxDefaultSoftAngleChevron,
      .^/.style={out=-90-####1,in=90,looseness=0.65},
      .^/.default=\zxDefaultSoftAngleChevron,
      'v/.style={out=90+####1,in=-90,looseness=0.65},
      'v/.default=\zxDefaultSoftAngleChevron,
      v'/.style={out=90-####1,in=-90,looseness=0.65},
      v'/.default=\zxDefaultSoftAngleChevron,
      % Links with a s-like shape.
      s/.style={/tikz/out=0,/tikz/in=180,looseness=0.6},
      % -s'.- shapes are like s, but with a soften (customizable like o) angle.
      % The '. say if you are going up or down, and - forces a sharp angle (- is flat) on the side of the -
      s'/.style={/tikz/out=####1,/tikz/in=180+####1},
      s'/.default=\zxDefaultSoftAngleS,
      s./.style={/tikz/out=-####1,/tikz/in=180-####1},
      s./.default=\zxDefaultSoftAngleS,
      -s'/.style={/tikz/out=0,/tikz/in=180+####1},
      -s'/.default=\zxDefaultSoftAngleS,
      -s./.style={/tikz/out=0,/tikz/in=180-####1},
      -s./.default=\zxDefaultSoftAngleS,
      s'-/.style={/tikz/out=####1,/tikz/in=180},
      s'-/.default=\zxDefaultSoftAngleS,
      s.-/.style={/tikz/out=-####1,/tikz/in=180},
      s.-/.default=\zxDefaultSoftAngleS,
      % Links with a s-like shape... but read from top to bottom
      ss/.style={/tikz/out=0-90,/tikz/in=180-90,looseness=0.6},
      % -s'.- shapes are like s, but with a soften (customizable like o) angle.
      % The '. say if you are going up or down, and - forces a sharp angle (- is flat) on the side of the -
      ss./.style={/tikz/out=####1-90,/tikz/in=180-90+####1},
      ss./.default=\zxDefaultSoftAngleS,
      .ss/.style={/tikz/out=-####1-90,/tikz/in=180-90-####1},
      .ss/.default=\zxDefaultSoftAngleS,
      sIs./.style={/tikz/out=0-90,/tikz/in=180-90+####1},
      sIs./.default=\zxDefaultSoftAngleS,
      .sIs/.style={/tikz/out=0-90,/tikz/in=180-90-####1},
      .sIs/.default=\zxDefaultSoftAngleS,
      ss.I/.style={/tikz/out=####1-90,/tikz/in=180-90},
      ss.I/.default=\zxDefaultSoftAngleS,
      I.ss/.style={/tikz/out=-####1-90,/tikz/in=180-90},
      I.ss/.default=\zxDefaultSoftAngleS,
      %%%% Links with a N-shape, i.e. like s shape, but symetric against the diagonal. Equivalently, it's a soft 's' shape with a much wider angle (>45).
      N'/.style={/tikz/out=####1,/tikz/in=180+####1},
      N'/.default=\zxDefaultSoftAngleN,
      N./.style={/tikz/out=-####1,/tikz/in=180-####1},
      N./.default=\zxDefaultSoftAngleN,
      -N'/.style={/tikz/out=0,/tikz/in=180+####1},
      -N'/.default=\zxDefaultSoftAngleN,
      -N./.style={/tikz/out=0,/tikz/in=180-####1},
      -N./.default=\zxDefaultSoftAngleN,
      N'-/.style={/tikz/out=####1,/tikz/in=180},
      N'-/.default=\zxDefaultSoftAngleN,
      N.-/.style={/tikz/out=-####1,/tikz/in=180},
      N.-/.default=\zxDefaultSoftAngleN,
    },
    % No line but vdots/dots in between.
    3 vdots/.style={draw=none, "\makebox[0pt][r]{\footnotesize\smash{##1\,}}\scalebox{\zxScaleDots}{$\cvdotsCenterMathline$}" {anchor=center,zxNormalFont}},
    3 vdots/.default={},
    3 vdotsr/.style={draw=none, "\scalebox{\zxScaleDots}{$\cvdotsCenterMathline$}\makebox[0pt][l]{\footnotesize\smash{\,##1}}" {anchor=center,zxNormalFont}},
    3 vdotsr/.default={},
    3 dots/.style={draw=none, "\makebox[0pt][r]{\footnotesize\smash{##1}}\scalebox{\zxScaleDots}{$\chdots$}" {anchor=center,zxNormalFont}},
    3 dots/.default={},
    % Add a Hadmard/Z/X (no phase) in the middle of the line. Practical to add small nodes without creating
    % a new column/row. However, make sure the corresponding row/column is larger, using &[\zxHCol]
    % for columns and \\[\zxHRow] for rows (for Z/X style, use zxSCol and sxSRow), if you have both spiders
    % and Hadamard, use \zxHSCol and \zxHSRow.
    H/.style={"" {zxHSmall,anchor=center,##1}},
    H/.default={},
    Z/.style={"" {zxNoPhaseSmallZ,anchor=center,##1}},
    Z/.default={},
    X/.style={"" {zxNoPhaseSmallX,anchor=center,##1}},
    X/.default={},
    % Sometimes, it might be handy to specify an anchor only if no anchor was set before, notably when using "post arrow style if end node" (see
    % for instance the Divider).
    end anchor if not set/.code={%
      \ifx\tikzcd@endanchor\pgfutil@empty% If there is no anchor, we can set ours:
        \pgfkeysalso{/tikz/commutative diagrams/end anchor={##1}}%
      \fi%
    },
    start anchor if not set/.code={%
      \ifx\tikzcd@startanchor\pgfutil@empty% If there is no anchor, we can set ours:
        \pgfkeysalso{/tikz/commutative diagrams/start anchor=##1}%
      \fi%
    },
    % Arrow will go out from the center of the shape instead of from the border. Useful e.g.
    % when connecting nodes with different shapes, it will give back a symetric connection.
    wire centered/.style={
      on layer=edgelayer,
      /tikz/commutative diagrams/start anchor=center,
      /tikz/commutative diagrams/end anchor=center,
    },
    wire centered start/.style={
      on layer=edgelayer,
      /tikz/commutative diagrams/start anchor=center,
    },
    wire centered end/.style={
      on layer=edgelayer,
      /tikz/commutative diagrams/end anchor=center,
    },
    wc/.style={wire centered},
    wcs/.style={wire centered start},
    wce/.style={wire centered end},
    wire not centered/.style={
      /tikz/commutative diagrams/start anchor=,
      /tikz/commutative diagrams/end anchor=,
    },
  },
  %% Will be loaded in tikzcd options directly.
  /zx/styles/rounded style preload/.style={
    content fixed baseline/.style={
      /zx/user overlay nodes/.style={
        zxShort/.append style={content fixed baseline},
        % fixed baseline gives poor result in fractions, so we disable it here
        zxSpecificFrac/.append style={content vertically centered},
      },
    },
    content fixed baseline also frac/.style={
      /zx/user overlay nodes/.style={
        zxShort/.append style={content fixed baseline},
      },
    },
    content vertically centered/.style={
      /zx/user overlay nodes/.style={
        zxShort/.append style={content vertically centered}
      },
    },
    /zx/styles/rounded style common nodes and ZX,
  },
  % These styles make sense both in "ZX" options and in nodes options.
  /zx/styles/rounded style common nodes and ZX/.style={
    small minus/.code={
      \def\zxMinus{\zxShortMinus}%
      \def\zxMinusInShort{\zxShortMinus}%
    },
    big minus/.code={
      \def\zxMinus{-}%
      \def\zxMinusInShort{-}%
    },
  },
  /zx/styles/rounded style/.style={
    /zx/styles/rounded style common nodes and ZX,
    %% Can be redefined by user
    % Style for empty nodes
    zxSpacingInsideNodes/.style={
      execute at begin node={\thinmuskip=0mu\medmuskip=0mu\thickmuskip=0mu}, % Reduce space around +/-...
    },
    zxNormalSpacing/.style={
      execute at begin node={\thinmuskip=3.0mu\medmuskip=4.0mu plus 2.0mu minus 4.0mu\thickmuskip=5.0mu plus 5.0mu}, %
    },
    %% Normalize fonts to avoid different rendering when document has a different font shape.
    zxNormalFont/.style={
      font={\fontsize{10}{12}\selectfont},
    },
    zxTinyFont/.style={
      font={\fontsize{6}{6}\selectfont},
    },
    zxTinyFontAndSpacing/.style={
      zxSpacingInsideNodes,
      zxTinyFont,
    },
    % Practical styles to either vertically center the content, or fix the position of the baseline.
    content fixed baseline/.style={
      text height=1.5mm,%5.4pt,%1.5mm,%.45em,
      text depth=0.1mm,%1.8pt,%0.1mm,%.15em,
    },
    % Style for empty nodes
    content vertically centered/.style={
      text height=,
      text depth=,
    },
    bold/.style={very thick},
    B/.style={bold},
    non bold/.style={%
      line width=\zxDefaultLineWidth,
    },
    NB/.style={non bold},
    %% To set bold all wires connected to the current wire
    bold wires/.style={
      post arrow style if end node={B},
      post arrow style if start node={B},
    },
    wires bold/.style={bold wires},% backward compatible
    Bw/.style={wires bold},
    BBw/.style={bold, wires bold,},
    zxAllNodes/.style={
      shape=rectangle, % Otherwise nodes are asymetrical rectangle, which is not practical in our case. Gives notably anchor "center" which is really centered compared to asymatrical rectangles
      anchor=center, % Center cells
      line width=\zxDefaultLineWidth,
      % Normalize the font size (avoids different rendering of dots...)
      zxSpacingInsideNodes,
      zxNormalFont,
    },
    % Use this to denote an empty diagram
    zxEmptyDiagram/.style={
      zxAllNodes,
      draw,
      dashed,
      minimum size=4mm,
    },
    % Style to use when no node is drawn
    zxNone/.style={
      zxAllNodes,
      shape=coordinate, % A coordinate has just a center. Nothing more.
    },
    % Style to use when no node is drawn, but a bit of space is required not to make the diagram too small
    zxNone+/.style={
      zxAllNodes,
      inner sep=1mm,
      outer sep=0mm
    },
    % Like zxNone+, but without width (wold prefer |, but special car in |[]|...
    zxNoneI/.style={
      zxNone+,
      inner xsep=0mm,
    },
    % Like zxNone+, but without height
    zxNone-/.style={
      zxNone+,
      inner ysep=0mm,
    },
    % Style to use when no node is drawn, but a large space must be reserved (typically used to fake two
    % nodes on a single line) (for +I- versions)
    zxNoneDouble/.style={
      shape=coordinate
    },
    % Style to use when no node is drawn, but a bit of space is required not to make the diagram too small
    zxNoneDouble+/.style={
      zxAllNodes,
      inner sep=.6em,
      outer sep=0mm
    },
    % Like zxNoneDouble+, but without width (wold prefer |, but special car in |[]|...
    zxNoneDoubleI/.style={
      zxNoneDouble+,
      inner xsep=0mm,
    },
    % Like zxNoneDouble+, but without height
    zxNoneDouble-/.style={
      zxNoneDouble+,
      inner ysep=0mm,
    },
    % Used to compute the intersection with the node's boundary
    allow boundary intersection/.code={%
      \ifdefined\zxEnableIntersectionsNodes%
        \edef\zx@name@path{zx@name@path@\tikz@fig@name}%
        \zx@message{[ZX] The name of the path will be \zx@name@path.}%
        \pgfkeysalso{
          name path=\zx@name@path, % Used by intersection library
        }%
      \fi%
    },
    % Will be specific to all spiders
    zxSpiders/.style={
      draw=black,
      allow boundary intersection,% Used later to do math on it.
    },
    % Will use this style when drawing a X/Z node without phase (not for end user directly)
    zxNoPhase/.style={
      zxAllNodes,
      zxSpiders,
      inner sep=0mm,
      minimum size=2mm,
      shape=circle,
    },
    % Used only in decoration of wires, to add small empty X/Z nodes.
    zxNoPhaseSmall/.style={
      zxNoPhase
    },
    % Style for nodes that are small enough to fit in a circle, like $\zxMinus \frac{\pi}{4}$ or $- \alpha$
    zxShort/.style={
      zxAllNodes,
      zxSpiders,
      minimum size=5mm,
      font={\fontsize{8}{10}\selectfont\boldmath},
      rounded rectangle,
      inner sep=0.0mm,
      scale=0.8,
    },% negative outer sep would draw lines from inside...
    % Style for nodes that are bigger, like $\alpha+\beta$ or $(a\oplus b)\pi$
    zxLong/.style={zxShort, inner xsep=1.2mm},
    %%% Styles of the label when |phase in label| is used
    stylePhaseInLabel/.style={
      font={\fontsize{8}{10}\selectfont\boldmath},
      inner sep=2pt,
      outer sep=0pt,
      rounded rectangle,
      % node on layer=labellayer, %% Fails in tikzcd: https://tex.stackexchange.com/questions/618823/node-on-layer-style-in-tikz-matrix-tikzcd
    },
    stylePhaseInLabelZ/.style={
      stylePhaseInLabel,
      fill=green!20!white
    },
    stylePhaseInLabelX/.style={
      stylePhaseInLabel,
      fill=red!20!white
    },
    % Some styles should be applied only on frac
    zxSpecificFrac/.style={
    },
    %%%%%%%%%%% Style defined depending on above ones. Feel free to redefine.
    zxNoPhaseZ/.style={zxNoPhase,fill=colorZxZ},
    zxNoPhaseX/.style={zxNoPhase,fill=colorZxX},
    zxNoPhaseSmallZ/.style={zxNoPhaseSmall,fill=colorZxZ},
    zxNoPhaseSmallX/.style={zxNoPhaseSmall,fill=colorZxX},
    zxShortZ/.style={zxShort,fill=colorZxZ},
    zxShortX/.style={zxShort,fill=colorZxX},
    zxLongZ/.style={zxLong,fill=colorZxZ},
    zxLongX/.style={zxLong,fill=colorZxX},
    %%%%%%%%%%%
    %%% Instead of adding directly the style as the node's content (which would make
    %%% impossible styles that adds the phase in a label outside of the node)
    %%% add@Phase@Spider{,Z,X}={phase of the node} will be in charge of adding it.
    % add@Phase@Spider{emptyStyle}{ShortStyle}{LongStyle}{label style}{node content}
    add@Phase@Spider/.style n args={5}{%
      zx@emptyStyle/.style={##1},
      zx@shortStyle/.style={##2},
      zx@notEmptyStyle/.style={##3},
      zx@labelStyle/.style={##4},
      /zx/zx@content/.initial={##5},
      phase in content,
    },
    add@Phase@Spider@Frac/.style n args={8}{% add@Phase@Spider{emptyStyle}{NotEmptyStyle}{labelstyle}{sign}{above fraction (no parens)}{below fraction (no parens)}{above fraction (parens)}{below fraction (parens)}
      zx@emptyStyle/.style={##1},
      zx@notEmptyStyle/.style={##2},
      zx@labelStyle/.style={##3},
      % Useful to help "phase in content" to know if we are in Frac or not.
      /zx/zx@isInFrac/.initial={true},
      phase in content,
    },
    % #1 is the node content. Seems that storing it in /zx/zx@content is not enough because keys
    % seems to be local to nodes and are not transfered to label.
    zx@Execute@Very@End/.style n args={5}{
      zx@commandToExecuteVeryEnd/.try={##1}{##2}{##3}{##4}{##5},
    },
    zx@Execute@Very@End/.default={}{}{}{},
    %% zx@Execute@Very@End@Frac={emptystyle}{contentstyle}{labelstyle}{sign}{above frac (no parens)}{below frac (no parens)}{above frac (parens)}{below frac (parens)}
    zx@Execute@Very@End@Frac/.style n args={8}{
      zx@commandToExecuteVeryEndFrac/.try={##1}{##2}{##3}{##4}{##5}{##6}{##7}{##8},
    },
    zx@Execute@Very@End@Frac/.default={}{}{}{}{}{}{}{},
    %% /!\ WARNING: the following styles "phase..." must be loaded by or *after* add@Phase@Spider...
    %% To load it on the whole picture, prefer to do:
    %% \zx[/zx/user post preparation labels/.style={phase in label}]{
    %%   \zxZ{\alpha}
    %% }
    phase in content/.code={%
      % Check if we are in a Frac or not
      \ifthenelse{\equal{\pgfkeysvalueof{/zx/zx@isInFrac}}{true}}{%
        %%% ### We are in a fraction node!
        %% Modifies zx@commandToExecuteVeryEnd (which is executed at the very end by zx@Execute@Very@End)
        %% in order to add the good style
        \pgfkeysalso{
          % zx@commandToExecuteVeryEndFrac{emptystyle}{contentstyle}{labelstyle}{sign}{above frac (no parens)}{below frac (no parens)}{above frac (parens)}{below frac (parens)}
          zx@commandToExecuteVeryEndFrac/.style n args={8}{%
            execute at begin node={\zxConvertToFracInContent{####4}{####5}{####6}{####7}{####8}},%
          },%
          % Adds the style:
          zx@notEmptyStyle,
        }%
      }{ %%% ### We are NOT in a fraction node.
        %% Modifies zx@commandToExecuteVeryEnd (which is executed at the very end by zx@Execute@Very@End)
        %% in order to add the good style
        \pgfkeysalso{
          zx@commandToExecuteVeryEnd/.style n args={5}{%
            execute at begin node={####5},% ####4 = content
          }%
        }%
        % Checks if the content (stored by add@Phase@Spider in /zx/zx@content) is empty or not
        \ifthenelse{\equal{\pgfkeysvalueof{/zx/zx@content}}{}}{%
          \pgfkeysalso{%
            zx@emptyStyle,%
          }%
        }{% We check if we need to force "short mode" (zxShort instead of zxLong)
          \ifthenelse{\equal{\pgfkeysvalueof{/zx/zx@shortModeForced}}{true}}{%
            \pgfkeysalso{%
              zx@shortStyle,%
            }%
          }{% Otherwise, just use the default style.
            \pgfkeysalso{%
              zx@notEmptyStyle,%
            }%
          }%
        }%
      }%
    },
    phase in label/.code={
      % Check if we are in a Frac or not
      \ifthenelse{\equal{\pgfkeysvalueof{/zx/zx@isInFrac}}{true}}{%
        %%% ### We are in a fraction node!
        \pgfkeysalso{%
          % zx@commandToExecuteVeryEndFrac{emptystyle}{contentstyle}{labelstyle}{sign}{above frac (no parens)}{below frac (no parens)}{above frac (parens)}{below frac (parens)}
          zx@commandToExecuteVeryEndFrac/.code n args={8}{%
            \pgfkeysalso{
              label={[####3,##1] \zxConvertToFracInLabel{####4}{####5}{####6}{####7}{####8}},%
            }%
          },%
          zx@emptyStyle,
        }%
      }{%
        \pgfkeysalso{%
          % ##1 is the argument of "phase in label", i.e. the style of the label
          zx@commandToExecuteVeryEnd/.code n args={5}{% ####4: label style, ####5: content
            % Checks if the content (stored by add@Phase@Spider in /zx/zx@content) is empty or not
            \ifthenelse{\equal{####5}{}}{% Content is empty
            }{% Content is not empty
              \pgfkeysalso{
                label={[####4,##1] ####5},%
              }%
            }%
          },%
          zx@emptyStyle,
        }%
      }%
    },
    pil/.style={phase in label={##1}},
    phase in label below/.style={
      phase in label={label position=below,##1}
    },
    pilb/.style={phase in label below={##1}},
    phase in label above/.style={
      phase in label={label position=above,##1}
    },
    pila/.style={phase in label above={##1}},
    phase in label right/.style={
      phase in label={label position=right,##1}
    },
    pilr/.style={phase in label right={##1}},
    phase in label left/.style={
      phase in label={label position=left,##1}
    },
    pill/.style={phase in label left={##1}},
    %%% Was supposed to automatically find the good style depending on content... Can't find how to do.
    % Styles zxLong{X/Z} zxNoPhase{X/Z} are automatically selected by \zxZ4{...} and \zxX4{...} commands
    % and zxShort is selected for fractions only like in \zxFracZ-{\pi}{4}
    % zxZ/.style={zxChoose={##1},fill=colorZxZ},
    % zxX/.style={zxChoose={##1},fill=colorZxX},
    %%% First argument is additional style.
    %%% Second argument is the minus mode: "-" for minus sign, which forces short mode.
    %%% Third argument is "*" for forced short mode.
    %%% 4th argument is content of node.
    zx@spider/.code n args={8}{
      %% ##1: zxnophase style
      %% ##2: zxshort style
      %% ##3: zxlong style
      %% ##4: label style
      %% user provided:
      %% ##5: additional tikz options
      %% ##6: minus "-" or empty, like "-alpha"
      %% ##7: star "*" or empty, to force short mode
      %% ##8: content
      %%% The argument is a minus. Like \zxZ-{\alpha}, goal is to typeset "-\alpha" in short mode.
      \ifthenelse{\equal{##6}{-}}{% It's a minus!
        \pgfkeysalso{% We update the content to "-content"
          /zx/zx@shortModeForced/.initial={true},%
        }%
      }{}%
      \ifthenelse{\equal{##7}{*}}{% We force short mode!
        \pgfkeysalso{%
          /zx/zx@shortModeForced/.initial={true},%
        }%
      }{}%
      %  It's a minus, and LaTeX is... grrrr. How to get a cleaner code? Tried def/let, not working.
      \ifthenelse{\equal{##6}{-}}{%
        \pgfkeysalso{%
          add@Phase@Spider={##1}{##2}{##3}{##4}{\zxMinusInShort##8},%
        }%
      }{%
        \pgfkeysalso{%
          add@Phase@Spider={##1}{##2}{##3}{##4}{##8},%
        }%
      }%
      \pgfkeysalso{%
        /zx/post preparation labels,
        /zx/user post preparation labels,
        % /zx/user overlay nodes,
        ##5,
      }%
      \ifthenelse{\equal{##6}{-}}{% It's a minus, and LaTeX is... grrrr
        \pgfkeysalso{%
          zx@Execute@Very@End={##1}{##2}{##3}{##4}{\zxMinusInShort##8},%
        }%
      }{%
        \pgfkeysalso{%
          zx@Execute@Very@End={##1}{##2}{##3}{##4}{##8},%
        }%
      }%
    },
    %% ##1: additional tikz options
    %% ##2: minus "-" or empty, like "-alpha"
    %% ##3: star "*" or empty, to force short mode
    %% ##4: content
    zxZ4/.style n args={4}{
      zx@spider={zxNoPhaseZ}{zxShortZ}{zxLongZ}{stylePhaseInLabelZ}{##1}{##2}{##3}{##4}
    },
    zxX4/.style n args={4}{
      zx@spider={zxNoPhaseX}{zxShortX}{zxLongX}{stylePhaseInLabelX}{##1}{##2}{##3}{##4}
    },
    %% These take 6 arguments: additional style, sign (string "-" for minus, nothing for "+",
    %% otherwise inserted directly), above fraction (no parens), below fraction (no parens), above fraction (parens), below fraction (parens).
    zxFracZ6/.style n args={6}{
      add@Phase@Spider@Frac={zxNoPhaseZ}{zxShortZ,zxSpecificFrac}{stylePhaseInLabelZ}{##2}{##3}{##4}{##5}{##6},
      /zx/post preparation labels,
      /zx/user post preparation labels,
      ##1,
      zx@Execute@Very@End@Frac={zxNoPhaseZ}{zxShortZ,zxSpecificFrac}{stylePhaseInLabelZ}{##2}{##3}{##4}{##5}{##6},
    },
    zxFracX6/.style n args={6}{
      add@Phase@Spider@Frac={zxNoPhaseX}{zxShortX}{stylePhaseInLabelX}{##2}{##3}{##4}{##5}{##6},
      /zx/post preparation labels,
      /zx/user post preparation labels,
      ##1,
      zx@Execute@Very@End@Frac={zxNoPhaseX}{zxShortX}{stylePhaseInLabelX}{##2}{##3}{##4}{##5}{##6},
    },
    % For Hadamard
    zxH/.style={
      zxAllNodes,
      outer sep=0pt,
      fill=colorZxH,
      draw,
      inner sep=0.6mm,
      minimum height=1.5mm,
      minimum width=1.5mm,
      shape=rectangle},
    zxHSmall/.style={zxH},
  },
  % Default style. Can be changed by user
  /zx/default style nodes/.style={
    /zx/styles/rounded style
  },
  % Will be executed in tikzcd options. Useful to define a "global" option specific to a given style.
  /zx/default style nodes preload/.style={
    /zx/styles/rounded style preload,
  },
  % User can put here any additional property
  /zx/user overlay nodes/.style={
  },
  % User can put here any global ZX options
  /zx/user overlay/.style={
  },
  % Any additional property that needs to be loaded after add@Phase@Spider (by script, not by user).
  /zx/post preparation labels/.style={
  },
  % User can put here any additional property that needs to be loaded after add@Phase@Spider
  /zx/user post preparation labels/.style={
  },
  % Default wire style. Can be changed by user.
  /zx/default style wires/.style={
  },
  % User can add stuff in this style to improve wire styles
  /zx/user overlay wires/.style={
  },
  /zx/every arrow pre/.style={%
    /utils/exec={% We provide a way to disable the default behavior
      \ifdefined\zxDisablePreRunStart\else%
        %%% If it is an alias, we replace it with the original node:
        \ifcsname zx@alias@\zxCurrentDiagram @\tikzcd@ar@start\endcsname% This is an alias: let's rename it to the original node
          \edef\tikzcd@ar@start{\csname zx@alias@\zxCurrentDiagram @\tikzcd@ar@start\endcsname}%
        \fi%
        %%% Load the arrow style set by the start node:
        \edef\zx@tmp{% Make sure that the style exists and load it:
          /zx/internals/preRunIfIamStartNode-\tikzcd@ar@start/.append style={},%
          /zx/internals/preRunIfIamStartNode-\tikzcd@ar@start,%
        }%
        \pgfkeysalsofrom{\zx@tmp}%
      \fi%
      %%% Load the arrow style set by the end node:
      \ifdefined\zxDisablePreRunEnd\else%
        %%% If it is an alias, we replace it with the original node:
        \ifcsname zx@alias@\zxCurrentDiagram @\tikzcd@ar@target\endcsname% This is an alias: let's rename it to the original node
          \edef\tikzcd@ar@target{\csname zx@alias@\zxCurrentDiagram @\tikzcd@ar@target\endcsname}%
        \fi%
        \edef\zx@tmp{% Make sure that the style exists and load it:
          /zx/internals/preRunIfIamEndNode-\tikzcd@ar@target/.append style={},%
          /zx/internals/preRunIfIamEndNode-\tikzcd@ar@target,%
        }%
        \pgfkeysalsofrom{\zx@tmp}%
      \fi%
    }%
  },%
  /zx/defaultEnv/.style={
    %tiny depends on the size of the font... which we want to keep consistent look
    column sep=\zxDefaultColumnSep,
    row sep=\zxDefaultRowSep,
    zx column sep/.code={%
      \def\zxDefaultColumnSep{##1}%
      \pgfkeysalso{
        column sep=##1,
      }%
    },
    zx row sep/.code={%
      \def\zxDefaultRowSep{##1}%
      \pgfkeysalso{%
        row sep=##1,
      }%
    },
    %column sep=tiny,
    %row sep=tiny,
    % center on the math axis
    baseline={([yshift=-axis_height]current bounding box.center)},
    % Fix 1-row diagram baseline
    % By default, 1-row diagrams have a different baseline... This package does not want a special case for 1-row diagrams.
    1-row diagram/.style={},
    %% usage: math baseline=wantedBaseline, where you have somewhere \zxX[a=wantedBaseline]{\beta}
    math baseline/.style={baseline={([yshift=-axis_height]##1)}},
    % math baseline row/.style={math baseline=\zxCurrentDiagram-##1-1},
    math baseline row/.code={%
      \IfInteger{##1}{%
        \pgfkeysalso{math baseline=\zxCurrentDiagram-##1-1}%
      }{%
        % We have a non-integer value like 1.5 that means between both values.
        % We first extract these values:
        \pgfmathtruncatemacro{\zx@first@index}{floor(##1)}%
        \pgfmathtruncatemacro{\zx@second@index}{floor(##1)+1}%
        \pgfmathsetmacro{\zx@percent}{##1-floor(##1)}%
        \pgfkeysalso{math baseline/.expanded={$(\zxCurrentDiagram-\zx@first@index-1)!\zx@percent!(\zxCurrentDiagram-\zx@second@index-1)$}}% 
      }%
    },
    math baseline row/.default={1},
    mbr/.style={math baseline row={##1}},
    mbr/.default={1},
    thick lines/.style={
      /tikz/zx thickness wires style/.style={thick, line cap=round},
    },
    % In circuits (mostly gates), the default separation feels too small, so we adjust it here
    circuit/.style={
      column sep=\zxDefaultColumnSepCircuit,
      row sep=\zxDefaultRowSepCircuit,
    },
    bold/.style={
      /tikz/commutative diagrams/arrows={B},
      /zx/user overlay nodes/.style={
        zxSpiders/.append style={BBw},
        zxH/.append style={BBw},
      },
      /utils/exec={\def\zxEnableAllElementsBold{}},
    },
    B/.style={
      bold
    },
    debug mode/.code={\def\zxDebugMode{}},
    no debug mode/.code={\let\zxDebugMode\undefined},
    amp/.style={
      ampersand replacement=\&,
    },
    %%% Include a diagram in another diagram.
    use diagram/.style 2 args={
      execute at end picture={
        \node[fit=##2,anchor=center,inner sep=0pt,yshift=-axis_height]{\zxUseDiagram{##1}};
      }
    },
    %%% Include a diagram in another diagram.
    use diagram advanced/.style n args={3}{
      execute at end picture={
        \node[fit=##2,anchor=center,inner sep=0pt,yshift=-axis_height,##3]{\zxUseDiagram{##1}};
      }
    },
    % Load (thanks ".search also") our own style
    /tikz/every node/.style={%
      % Otherwise, the empty nodes take some space… and create weird bugs. But I don't know how to add empties
      % https://tex.stackexchange.com/questions/675958/tikz-empty-nodes-with-different-style-coordinate
      % ==> solved, see "empty"
      nodes={
        zxAllNodes,
        inner sep=0pt,
        outer sep=0mm,
      },
      % For quickly adding alias, and displaying this alias in debug mode.
      a/.code={%
        \pgfkeysalso{%
          % alias=####1,%
          % alias seems to fail with nodes in pic… let's try another method. See also:
          % https://tex.stackexchange.com/questions/679647/tikz-pic-alias-to-internal-node
          %% TODO: according to the above discussion, this append after command should be done after the
          %% add anchor to node is finished, as an alias is basically copying all the macros of the initial node,
          %% so you want to do it once its finished (in particular to preserve the anchors). But maybe
          %% append after command is already avoiding that issue. In any case it would be good to make it more
          %% robust (\tikzlastnode might change, so defining my own \tikzlastnode might help, use link instead
          %% of copy alias to be sure the anchors are preserved etc...)
          append after command={\pgfextra{%
              \pgfnodealias{####1}{\tikzlastnode}%
              % There is no way anymore to remember the link between the alias and the non-alias. Let's rembember it
              % manually
              \expandafter\xdef\csname zx@alias@\zxCurrentDiagram @####1\endcsname{\tikzlastnode}%
            }%
          },
        }%
        %% name path won't work with alias... So let's create two name path ^^
        %% https://tex.stackexchange.com/questions/619622
        \ifdefined\zxEnableIntersectionsNodes%
          \edef\zx@name@path{zx@name@path@####1}%
          \zx@message{[ZX] Name given using an alias: \zx@name@path.}%
          \pgfkeysalso{
            name path=\zx@name@path, % Used by intersection library
          }%
        \fi%
        \ifdefined\zxDebugMode%
          \pgfkeysalso{%
            label={[inner sep=0pt,overlay,red,font={\fontsize{5}{6}}]-45:\scalebox{.5}{####1}}
          }%
        \fi%
      },
      % runs a=foo only if we are at the origin (0,0) of a zxExecuteAtRegionRelative
      a if origin/.code={%
        \ifnum \the\pgfmatrixcurrentrow=\zxOriginalRow\relax%
          \ifnum \the\pgfmatrixcurrentcolumn=\zxOriginalCol\relax%
            \pgfkeysalso{a={####1}}%
          \fi%          
        \fi%
      },
      %% To allow blocks
      block/.style={
        zx set execute at begin node={\begin{varwidth}{15em}},
        zx set execute at end node={\end{varwidth}},
      },
      %% This allows nodes to specify styles for the arrows that start/end from them.
      %% This style will be loaded last as otherwise we don't know yet the starting and ending point.
      %% See how we apply it on the arrow side using "every arrow post"
      post arrow style if start node/.style={%
        append after command={% This expects a path, but we want to write a code without any path:
          \pgfextra% this syntax with \endpgfextra completely turns off the tikz syntax, that might be safer than \pgfextra{...}.
            \def\zx@arguments{####1}%
            \edef\zx@tmp{%
              /zx/internals/postRunIfIamStartNode-\tikzlastnode/.append style={\unexpanded\expandafter{\zx@arguments}},%
            }% https://tex.stackexchange.com/questions/47905/how-to-globally-tikzset-styles
            {%
              \globaldefs=1\relax%
              \pgfkeysalsofrom{\zx@tmp}%
            }%
          \endpgfextra%
        },%
      },
      post arrow style if end node/.style={%
        append after command={% This expects a path, but we want to write a code without any path:
          \pgfextra% this syntax with \endpgfextra completely turns off the tikz syntax, that might be safer than \pgfextra{...}.
            \def\zx@arguments{####1}%
            \edef\zx@tmp{%
              /zx/internals/postRunIfIamEndNode-\tikzlastnode/.append style={\unexpanded\expandafter{\zx@arguments}},%
            }% https://tex.stackexchange.com/questions/47905/how-to-globally-tikzset-styles
            {%
              \globaldefs=1\relax%
              \pgfkeysalsofrom{\zx@tmp}%
            }%
          \endpgfextra%
        },%
      },%
      pre arrow style if start node/.style={%
        append after command={% This expects a path, but we want to write a code without any path:
          \pgfextra% this syntax with \endpgfextra completely turns off the tikz syntax, that might be safer than \pgfextra{...}.
            \def\zx@arguments{####1}%
            \edef\zx@tmp{%
              /zx/internals/preRunIfIamStartNode-\tikzlastnode/.append style={\unexpanded\expandafter{\zx@arguments}},%
            }% https://tex.stackexchange.com/questions/47905/how-to-globally-tikzset-styles
            {%
              \globaldefs=1\relax%
              \pgfkeysalsofrom{\zx@tmp}%
            }%
          \endpgfextra%
        },%
      },
      pre arrow style if end node/.style={%
        append after command={% This expects a path, but we want to write a code without any path:
          \pgfextra% this syntax with \endpgfextra completely turns off the tikz syntax, that might be safer than \pgfextra{...}.
            \def\zx@arguments{####1}%
            \edef\zx@tmp{%
              /zx/internals/preRunIfIamEndNode-\tikzlastnode/.append style={\unexpanded\expandafter{\zx@arguments}},%
            }% https://tex.stackexchange.com/questions/47905/how-to-globally-tikzset-styles
            {%
              \globaldefs=1\relax%
              \pgfkeysalsofrom{\zx@tmp}%
            }%
          \endpgfextra%
        },%
      },
      /zx/default style nodes,
      /zx/user overlay nodes,
    },
    every arrow/.style={%
      /zx/wires definition,
      /zx/default style wires,
      /zx/every arrow pre,
      /zx/user overlay wires,
      %%% Load the post arrow style at the end:
      %% See also every arrow pre if you want to run it before the style of the arrow
      /tikz/commutative diagrams/every arrow post/.append style={%
        /utils/exec={% We provide a way to disable the default behavior
          \ifdefined\zxDisablePostRunStart\else%
            %%% If it is an alias, we replace it with the original node:
            \ifcsname zx@alias@\zxCurrentDiagram @\tikzcd@ar@start\endcsname% This is an alias: let's rename it to the original node
              \edef\tikzcd@ar@start{\csname zx@alias@\zxCurrentDiagram @\tikzcd@ar@start\endcsname}%
            \fi%
            %%% Load the arrow style set by the start node:
            \edef\zx@tmp{% Make sure that the style exists and load it:
              /zx/internals/postRunIfIamStartNode-\tikzcd@ar@start/.append style={},%
              /zx/internals/postRunIfIamStartNode-\tikzcd@ar@start,%
            }%
            \pgfkeysalsofrom{\zx@tmp}%
          \fi%
          %%% Load the arrow style set by the end node:
          \ifdefined\zxDisablePostRunEnd\else%
            %%% If it is an alias, we replace it with the original node:
            \ifcsname zx@alias@\zxCurrentDiagram @\tikzcd@ar@target\endcsname% This is an alias: let's rename it to the original node
              \edef\tikzcd@ar@target{\csname zx@alias@\zxCurrentDiagram @\tikzcd@ar@target\endcsname}%
            \fi%
            \edef\zx@tmp{% Make sure that the style exists and load it:
              /zx/internals/postRunIfIamEndNode-\tikzcd@ar@target/.append style={},%
              /zx/internals/postRunIfIamEndNode-\tikzcd@ar@target,%
            }%
            \pgfkeysalsofrom{\zx@tmp}%
          \fi%
        }%
      },%
      disable post run start/.code={\def\zxDisablePostRunStart{}},
      disable post run end/.code={\def\zxDisablePostRunEnd{}},
      disable post run/.style={
        disable post run start,
        disable post run end,
      },
      enable post run start/.code={\let\zxDisablePostRunStart\undefined},
      enable post run end/.code={\let\zxDisablePostRunEnd\undefined},
      enable post run/.style={
        enable post run start,
        enable post run end,
      },
      enable post run,
    },
    %% We add an empty coordinate in every node. This has multiple advantages: first this avoids errors about
    %% single ampersand when the matrix does not start with None, and it also avoids using \zxN{} everywhere.
    %% However, we can't just use 'nodes in empty cells=true' or it will add nodes that might take some space:
    % https://tex.stackexchange.com/questions/696622/tikz-allow-the-style-syntax-in-execute-at-empty-cells?noredirect=1#comment1729912_696622
    % /tikz/replace empty cell with/.code={%
    %   \def\tikz@at@emptycell{%
    %     \tikz@lib@matrix@start@cell
    %     #1%
    %     \tikz@lib@matrix@end@cell
    %   }%
    % },
    %   /tikz/replace empty cell with={%
    % /tikz/execute at empty cell/.code={%
    %   \expandafter\def\expandafter\tikz@at@emptycell\expandafter
    %   {\tikz@at@emptycell\expandafter\expandafter\expandafter
    %   \tikz@lib@matrix@start@cell##1\tikz@lib@matrix@end@cell}},
    % This is started at the beginning of non-empty cells
    every matrix/.append style={
      /tikz/execute at end cell={%
        \ifcsname zxExecuteAtCell-\zxCurrentDiagram-\the\pgfmatrixcurrentrow-\the\pgfmatrixcurrentcolumn\endcsname%
          \csname zxExecuteAtCell-\zxCurrentDiagram-\the\pgfmatrixcurrentrow-\the\pgfmatrixcurrentcolumn\endcsname%
        \fi%
      },%
    },%
    /tikz/execute at empty cell={%
      % We want to check if there is a
      \ifcsname zxExecuteAtCell-\zxCurrentDiagram-\the\pgfmatrixcurrentrow-\the\pgfmatrixcurrentcolumn\endcsname%
        % I wanted to allow a syntax with |[]| but first it seems to be quite hard (as it depends on the number of extensions):
        % %% https://tex.stackexchange.com/questions/696622/tikz-allow-the-style-syntax-in-execute-at-empty-cells
        % but anyway I guess it is better to turn everything into nodes so that \zxX{} can be inserted outside of a matrix.
        \csname zxExecuteAtCell-\zxCurrentDiagram-\the\pgfmatrixcurrentrow-\the\pgfmatrixcurrentcolumn\endcsname%
      \else%
        \ifdefined\zxCustomExecuteAtEmptyCell\zxCustomExecuteAtEmptyCell%
        \else%
          \coordinate[
          % yshift=axis_height, % We already do that elsewhere
          name=\zxCurrentDiagram-\the\pgfmatrixcurrentrow-\the\pgfmatrixcurrentcolumn];%
        \fi%
      \fi%
    },
    % When an arrow is present, it will not detect it as an empty cell… The below code tries to avoid this issue, but actually has another issue as it removes all nodes without text in it, like |[circle, draw, inner sep=3pt]|, so
    % we disable it for now.
    %% https://tex.stackexchange.com/questions/675958/tikz-empty-nodes-with-different-style-coordinate/675971#675971
    % /tikz/every node/.append code={%
    %   \tikz@addoption{%
    %     \ifdim\wd\pgfnodeparttextbox=0pt
    %       \pgfutil@ifx\tikz@shape\tikz@coordinate@text{}{%
    %         \def\tikz@shape{coordinate}%
    %         \pgftransformyshift{axis_height}}%
    %     \fi}%
    % },
    %%% To be used only in \zx{...} environment.
    %%% Exemple:
    phase in content/.style={
      /zx/post preparation labels/.append style={
        phase in content,
      }
    },
    phase in label/.style={
      /zx/post preparation labels/.append style={
        phase in label={##1},
      }
    },
    phase in label above/.style={
      /zx/post preparation labels/.append style={
        phase in label above={##1},
      }
    },
    phase in label below/.style={
      /zx/post preparation labels/.append style={
        phase in label below={##1},
      }
    },
    phase in label right/.style={
      /zx/post preparation labels/.append style={
        phase in label right={##1},
      }
    },
    phase in label left/.style={
      /zx/post preparation labels/.append style={
        phase in label left={##1},
      }
    },
    % for "Phase In Label"
    pil/.style={phase in label={##1}},
    pilb/.style={phase in label below={##1}},
    pila/.style={phase in label above={##1}},
    pilr/.style={phase in label right={##1}},
    pill/.style={phase in label left={##1}},
    % Load the global options style-specific
    /zx/default style nodes preload,
    /zx/user overlay,
  },
}

% % Call \zxSetStyleWhenEndingHere{your, zx, style} when you want to configure a particular style when arriving on a particular cell
% \NewExpandableDocumentCommand{\zxSetStyleWhenEndingHere}{m}{%
%   \pgfkeys{
%     /zx/internals/postRunIfIamEndNode-\zxCurrentDiagram-\the\pgfmatrixcurrentrow-\the\pgfmatrixcurrentcolumn/.append code={#1},
%     /zx/internals/postRunIfIamEndNode-\zxCurrentDiagram-\the\pgfmatrixcurrentrow-\the\pgfmatrixcurrentcolumn
%   }%
% }

%%%%%%%%%%%%%%%%%%%%%%%%%%%%%%
%%%% Helper functions
%%%%%%%%%%%%%%%%%%%%%%%%%%%%%%


% Defines a "on layer=nameoflayer" style. TODO: check if better to move it in /zx/
% https://tex.stackexchange.com/questions/20425/z-level-in-tikz/20426#20426
% For path: on layer=namelayer, for nodes "node on layer=..."
% /!\ node on layer fails in tikzcd: https://tex.stackexchange.com/questions/618823/node-on-layer-style-in-tikz-matrix-tikzcd
\pgfkeys{%
  /tikz/on layer/.code={
    \pgfonlayer{#1}\begingroup
    \aftergroup\endpgfonlayer
    \aftergroup\endgroup
  },
  /tikz/node on layer/.code={
    \gdef\node@@on@layer{%
      \setbox\tikz@tempbox=\hbox\bgroup\pgfonlayer{#1}\unhbox\tikz@tempbox\endpgfonlayer\egroup}
    \aftergroup\node@on@layer
  },
  /tikz/end node on layer/.code={
    \endpgfonlayer\endgroup\endgroup
  }
}
\def\node@on@layer{\aftergroup\node@@on@layer}

%%% Declare a symbol for a short minus (useful in fractions)
\DeclareMathSymbol{\zxShortMinus}{\mathbin}{AMSa}{"39}% Requires amssymb. This version should not be redefined by the user to obtain a normal minus (or "small minus" won't work anymore)
\def\zxMinus{\zxShortMinus}

%%% Checks if a function is a point or a node.
%%% Not sure if best solution (needed to dig into source of TeX), but can't find anything better in manual
%%% https://tex.stackexchange.com/questions/6189553
\def\ifPgfpointOrNode#1#2#3{%
  \pgfutil@ifundefined{pgf@sh@ns@#1}{%
    #2%
  }{%
    #3%
  }%
}

% shape anchor name if exists if not. Works, while doing \pgfutil@ifundefined{pgf@anchor@\shapenode{}@#2} fails.
% I guess it has to do with the way macro are expanded...
\def\ifAnchorExistsFromShape#1#2#3#4{%
  \pgfutil@ifundefined{pgf@anchor@#1@#2}{%
    #4%
  }{%
    #3%
  }%
}

% node name, anchor name, if exists, if not.
\def\ifAnchorExists#1#2#3#4{%
  %%% First we extract the shape of the node:
  \edef\pgf@node@name{#1}%
  \edef\shapenode{\csname pgf@sh@ns@\pgf@node@name\endcsname}% e.g. rectangle
  \ifAnchorExistsFromShape{\shapenode}{#2}{#3}{#4}%
}

%% If we manually add an anchor to a note, \ifAnchorExists will not work. Hence this improved version:
%% Note that #3 and #4 will run in a group
\def\ifAnchorExistsNodeSpecific#1#2#3#4{%
  \let\zxcalculus@tmp\undefined% This is used to run #3 and #4 outside the group
  {% group not to pollute stuff with pgf@sh@ma
    \ifcsname pgf@sh@ma@#1\endcsname% per node config
      \csname pgf@sh@ma@#1\endcsname% we run the config
    \fi%
    %%% First we extract the shape of the node:
    \edef\pgf@node@name{#1}%
    \edef\shapenode{\csname pgf@sh@ns@\pgf@node@name\endcsname}% e.g. rectangle
    \ifAnchorExistsFromShape{\shapenode}{#2}{\gdef\zxcalculus@tmp{}}{}%    
  }%
  \ifdefined\zxcalculus@tmp%
    #3%
  \else%
    #4%
  \fi%
}

% \zxGetSubnodeOfNode{\tikzcd@ar@start}{mysubnode}{True branch}{False branch} will set \zx@result to the corresponding node if it exits and run True branch
% otherwise it will run False branch, where \zx@error will contain more details on why it failed. This function also try to use \zxOriginalRow to find
% a possible parent.
\def\zxGetSubnodeOfNode#1#2#3#4{%
  \edef\zx@result{#1}%
  % First we try to see if there is an alias:
  \edef\zx@tmp{zx@alias@\zxCurrentDiagram @#1}%
  % First, rewrite #1 so that it equals the name of the current node without alias
  \ifcsname \zx@tmp\endcsname% There exists an alias
    \edef\zx@tmp@alias{(alias \zx@tmp) }% Mostly for the error message
    \edef\zx@result{\csname zx@alias@\zxCurrentDiagram @#1\endcsname}%
  \else%
    \edef\zx@tmp@alias{(no alias) }% Mostly for the error message
  \fi
  % First, we check if there is a node with this name on this node. If not, we will try with the parent node
  \@ifundefined{pgf@sh@ns@\zx@result #2}{%
    % The current node has no such subnode name: let's start its parent.
    \@ifundefined{\zx@result -zxOriginalRow}{%
      % There is no original coordinate: error
      \def\zx@error{the subnode #2 of node \zx@result  \zx@tmp@alias does not exist.}%
      #4%
    }{%
      % There is an original coordinate: we try to see if the original node admits a subnode.
      \edef\zx@tmp@originalRow{\csname \zx@result -zxOriginalRow\endcsname}%
      \edef\zx@tmp@originalCol{\csname \zx@result -zxOriginalCol\endcsname}%
      \@ifundefined{pgf@sh@ns@\zxCurrentDiagram-\zx@tmp@originalRow-\zx@tmp@originalCol #2}{%
        \def\zx@error{the subnode #2 of aliased node \zx@result {} \zx@tmp@alias does not exist (even in the parent node at coordinate \zx@tmp@originalRow - \zx@tmp@originalCol).}%
        #4%
      }{%
        \edef\zx@result{\zxCurrentDiagram-\zx@tmp@originalRow-\zx@tmp@originalCol #2}%%
        #3%
      }%
    }%
  }{% There exists a subnode in the current name
    \edef\zx@result{\zx@result #2}%%
    #3%
  }%
}

% \zxTryFirstSubnodeThenAnchorThenNode{\tikztostart}{true north}{north} will set \zx@result to the corresponding subnode if it exists, otherwise the true north anchor,
% otherwise the north anchor
\def\zxTryFirstSubnodeThenAnchorThenNode#1#2#3{%
  \zxGetSubnodeOfNode{#1}{#2}{%
    % The subnode exists: we forward \zx@result
  }{%
    % No submode exists:
    % Test if it is a point or a node, and define StartPoint accordingly.
    \ifPgfpointOrNode{#1}{%
      \edef\zx@result{#1}%
    }{% This is a node
      % "zx add anchor to node" might define per-node anchors, and they are stored in
      % a macro called like \pgf@sh@ma@tikz@f@3-2-2, so we need to read this first:
      % TODO: maybe use pp@name everywhere? (see comment beginning of this file)
      \begingroup% This group avoids pollution done by executing pgf@sh@ma@NODENAME
      \pgfutil@ifundefined{pgf@sh@ma@\tikz@pp@name{##1}}{}{%
        % We the per-node macro exists, we run it.
        \csname pgf@sh@ma@\tikz@pp@name{##1}\endcsname%
      }%
      % Test if "true north" etc exists:
      \ifAnchorExistsNodeSpecific{#1}{#2}{%
        \xdef\zx@result{#1.#2}%
      }{%
        \xdef\zx@result{#1.#3}%
      }%
      \endgroup%
    }%
  }
}

% Like zxTryFirstSubnodeThenAnchorThenNodeToCoordinate but \zx@result will contain coordinates of the final point.
\def\zxTryFirstSubnodeThenAnchorThenNodeToCoordinate#1#2#3{%
  \zxGetSubnodeOfNode{#1}{#2}{%
    % The subnode exists: we forward \zx@result after turning it in a coordinate
    \ifPgfpointOrNode{\zx@result}{}{%
      % this is a node, so we take the center
      \pgfpointanchor{\zx@result}{#3}%
      \def\zx@result{\the\pgf@x,\the\pgf@y}%
    }%
  }{%
    % No submode exists:
    % Test if it is a point or a node, and define StartPoint accordingly.
    \ifPgfpointOrNode{#1}{%
      \edef\zx@result{#1}%
    }{% This is a node
      % "zx add anchor to node" might define per-node anchors, and they are stored in
      % a macro called like \pgf@sh@ma@tikz@f@3-2-2, so we need to read this first:
      % TODO: maybe use pp@name everywhere? (see comment beginning of this file)
      \begingroup% This group avoids pollution done by executing pgf@sh@ma@NODENAME
      \pgfutil@ifundefined{pgf@sh@ma@\tikz@pp@name{#1}}{}{%
        % We the per-node macro exists, we run it.
        \csname pgf@sh@ma@\tikz@pp@name{#1}\endcsname%
      }%
      % Test if "true north" etc exists:
      \ifAnchorExistsNodeSpecific{#1}{#2}{%
        \pgfpointanchor{#1}{#2}%
      }{%
        \pgfpointanchor{#1}{#3}%
      }%
      \xdef\zx@result{\the\pgf@x,\the\pgf@y}%
      \endgroup%
    }%
  }
}


%%% Create different kinds of dots...
%% https://tex.stackexchange.com/questions/617959
%% https://tex.stackexchange.com/questions/528774/excess-vertical-space-in-vdots/528775#528775
\DeclareRobustCommand\cvdotsAboveBaseline{%
  \vbox{\baselineskip4\p@ \lineskiplimit\z@%
    \hbox{.}\hbox{.}\hbox{.}}
}

\DeclareRobustCommand{\cvdotsCenterMathline}{%
  % vcenter is used to center the argument on the 'math axis', which is at half the height of an 'x', or about the position of a minus sign.
  \vcenter{\cvdotsAboveBaseline}%
}

\DeclareRobustCommand{\cvdotsCenterBaseline}{%
  \raisebox{-.5\height}{%
    $\cvdotsAboveBaseline$%
  }%
}

\DeclareRobustCommand{\chdots}{%
  \raisebox{-.5\height}{%
    \rotatebox{90}{% Maybe better options than rotatebox...
      $\cvdotsAboveBaseline$%
    }%
  }%
}

\DeclareRobustCommand{\cvdots}{\cvdotsCenterMathline}

%%%%%%%%%%%%%%%%%%%%%%%%%%%%%%%%%%%%%%%%%%%%%%%%%%%%%%%%%%
%%% Practical functions to define multirow/multicols gates
%%%%%%%%%%%%%%%%%%%%%%%%%%%%%%%%%%%%%%%%%%%%%%%%%%%%%%%%%%
% We provide a generic set of functions to execute code later
% (in a later cell, or at the end of the picture), together with
% a way to pass any kind of variables for later

\NewExpandableDocumentCommand{\zxCoordinateMain}{O{}m}{
  \coordinate[#1,#2](\zxCurrentDiagram-\the\pgfmatrixcurrentrow-\the\pgfmatrixcurrentcolumn);
}

\NewExpandableDocumentCommand{\zxCoordinateSubnode}{O{}m}{
  \coordinate[#1](\zxCurrentDiagram-\the\pgfmatrixcurrentrow-\the\pgfmatrixcurrentcolumn#2);
}

\NewDocumentCommand{\zxMainNodeAlreadySet}{m}{\def\zxMainNodeAlreadySetMacro{}}

\tikzset{
  % When using the \node{}; instead of |[foo]| bar structure, no name is given to the node. This is fixed with
  % this:
  zx main node/.code={%
    % If we are in a matrix, we can give it the name of the current column
    \ifdefined\zxCurrentDiagram%
      %  Useful to temporarily disable this, for instance useful in matrix to change the size of a single cell.
      \ifdefined\zxMainNodeAlreadySetMacro\else%
        \tikzset{name=\zxCurrentDiagram-\the\pgfmatrixcurrentrow-\the\pgfmatrixcurrentcolumn}%
      \fi%
    \fi%
  },
  zx subnode/.code={
    \ifdefined\zxCurrentDiagram%
      \ifdefined\zxOriginalRow%
        \tikzset{alias=\zxCurrentDiagram-\zxOriginalRow-\zxOriginalCol#1}%
        % \tikzset{name=\zxCurrentDiagram-\zxOriginalRow-\zxOriginalCol#1}%
      \else%
        \tikzset{alias=\zxCurrentDiagram-\the\pgfmatrixcurrentrow-\the\pgfmatrixcurrentcolumn#1}%
        % \tikzset{name=\zxCurrentDiagram-\the\pgfmatrixcurrentrow-\the\pgfmatrixcurrentcolumn#1}%
      \fi%
    \else%
      \message{Warning: you or outside of a tikzcd environment, so you will not be able to implement subnodes for now. But you might want to try to define zxCurrentDiagram, pgfmatrixcurrentrow and pgfmatrixcurrentcolumn yourself, you might be lucky.}%
    \fi%
  },
  zx independent subnode/.style={
    name=\zxCurrentDiagram-\the\pgfmatrixcurrentrow-\the\pgfmatrixcurrentcolumn#1,%
  },
  zx apply style/.code={\tikzset{/tikz/.cd,#1}},
  zx style from variable/.style={%
    zx apply style/.expand twice={\csname zxGateMultiVariable-\zxCurrentDiagram-\zxOriginalRow-\zxOriginalCol-#1\endcsname},
    % \expandafter\scantokens\expandafter{\zx@tmp}%
    % \zx@tmp%
  },
}

\NewExpandableDocumentCommand{\zxGetNameAbsoluteNode}{mm}{%
  \tikzcdmatrixname-#1-#2%
}%

%% here you can use previously defined variables if you created them
%% will only work inside environments where \zxOriginalRow/Col are defined
\NewExpandableDocumentCommand{\zxGetNameRelativeNode}{mm}{%
  \ifdefined\zxOriginalRow%
    \zxCurrentDiagram-\the\numexpr\zxOriginalRow+#1\relax-\the\numexpr\zxOriginalCol+#2\relax%
  \else%
    \zxCurrentDiagram-\the\numexpr\the\pgfmatrixcurrentrow+#1\relax-\the\numexpr\the\pgfmatrixcurrentcolumn+#2\relax%
  \fi%
}%

\NewExpandableDocumentCommand{\zxGetNameRelativeNodeStartOne}{mm}{%
  \zxGetNameRelativeNode{\numexpr #1-1\relax}{\numexpr #2-1\relax}%
}%


% \zxExecuteAtEndPicture{\node{A};} will execute \node{A}; at the end of the picture (or rather after the draw of the matrix, but before the draw of the wires)
\NewExpandableDocumentCommand{\zxExecuteAtEndPicture}{m}{%
  \def\zx@tmp@args{#1}%
  \edef\zx@tmp{\noexpand\pgfutil@g@addto@macro\noexpand\tikzcd@before@paths@hook{%
      % We give random value just to avoid having errors when it is not yet defined, but you should always specify the
      % to and from.
      \noexpand\def\noexpand\tikzcd@currentrow{In execute at end picture}%
      \noexpand\def\noexpand\tikzcd@currentcolumn{In execute at end picture}%
      \noexpand\def\noexpand\tikzcd@lineno{In zxArrowNow}%
        \noexpand\def\noexpand\zxOriginalRow{\the\pgfmatrixcurrentrow}%
        \noexpand\def\noexpand\zxOriginalCol{\the\pgfmatrixcurrentcolumn}%
      \expandonce\zx@tmp@args%
    }%
  }%
  \zx@tmp%
}

\NewExpandableDocumentCommand{\zxExecuteWhenDrawingArrows}{m}{%
  \def\zx@tmp@args{#1}%
  \edef\zx@tmp{\noexpand\pgfutil@g@addto@macro\noexpand\tikzcd@savedpaths{%
        \noexpand\def\noexpand\zxOriginalRow{\the\pgfmatrixcurrentrow}%
        \noexpand\def\noexpand\zxOriginalCol{\the\pgfmatrixcurrentcolumn}%
      \expandonce\zx@tmp@args%
    }%
  }%
  \zx@tmp%
}

% like \ar but run it directly. Usually run it in \zxExecuteWhenDrawingArrows.
\NewExpandableDocumentCommand{\zxArrowNow}{O{}m}{%
  \path[/tikz/commutative diagrams/.cd,every arrow,#1] (\tikzcd@ar@start\tikzcd@startanchor) to (\tikzcd@ar@target\tikzcd@endanchor);%
}

% \zxExecuteAtCellAbsolute{1}{2}{\node{A};} will execute at cell row 1, column 2 the code \node{A};, just make sure that nothing is already present in that cell
\NewExpandableDocumentCommand{\zxExecuteAtCellAbsolute}{mmm}{%
  \def\zx@tmp@args{#3}%
  \edef\zx@tmp{zxExecuteAtCell-\zxCurrentDiagram-#1-#2}%
  % Make sure to define the macro as we can't add to it if it does not exist
  \ifcsname \zx@tmp\endcsname\else\expandafter\gdef\csname \zx@tmp\endcsname{}\fi%
  \edef\zx@tmp@code{
    \noexpand\expandafter\noexpand\pgfutil@g@addto@macro\noexpand\csname zxExecuteAtCell-\zxCurrentDiagram-#1-#2\noexpand\endcsname{%
      \noexpand\def\noexpand\zxOriginalRow{\the\pgfmatrixcurrentrow}%
      \noexpand\def\noexpand\zxOriginalCol{\the\pgfmatrixcurrentcolumn}%
      % To allow to recover \zxOriginalRow from the name of the node later, we create a new macro "namenode-zxOriginalRow" with the value of the original row, same for Col
      \noexpand\expandafter\noexpand\xdef\noexpand\csname\zxCurrentDiagram-\noexpand\the\noexpand\pgfmatrixcurrentrow-\noexpand\the\noexpand\pgfmatrixcurrentcolumn-zxOriginalRow\endcsname{\noexpand\zxOriginalRow}%
      \noexpand\expandafter\noexpand\xdef\noexpand\csname\zxCurrentDiagram-\noexpand\the\noexpand\pgfmatrixcurrentrow-\noexpand\the\noexpand\pgfmatrixcurrentcolumn-zxOriginalCol\endcsname{\noexpand\zxOriginalCol}%
      \expandonce{\zx@tmp@args}%
    }%
  }%
  \zx@tmp@code%
}

% TODO: maybe use instead /tikz/every cell={n}{m} that is always run even if the cell is not empty? Or execute at begin cell that applies to non-empty cells?
% \zxExecuteAtCellRelative{1}{2}{\node{A};} will execute at cell current row + 1, current column + 2 the code \node{A};, just make sure that nothing is already present in that cell
\NewExpandableDocumentCommand{\zxExecuteAtCellRelative}{mmm}{%
  \def\zx@tmp@args{#3}%
  % putting this in the csname directly does not work
  \edef\zx@tmp{zxExecuteAtCell-\zxCurrentDiagram-\the\numexpr \the\pgfmatrixcurrentrow+#1\relax-\the\numexpr\the\pgfmatrixcurrentcolumn+#2\relax}%
  % \noexpand\expandafter\noexpand\gdef\noexpand\csname \zx@tmp\noexpand\endcsname{%
  % Make sure to define the macro as we can't add to it if it does not exist
  \ifcsname \zx@tmp\endcsname\else\expandafter\gdef\csname \zx@tmp\endcsname{}\fi%
  \edef\zx@tmp@code{%
    \noexpand\expandafter\noexpand\pgfutil@g@addto@macro\noexpand\csname \zx@tmp\noexpand\endcsname{%
      \noexpand\def\noexpand\zxOriginalRow{\the\pgfmatrixcurrentrow}%
      \noexpand\def\noexpand\zxOriginalCol{\the\pgfmatrixcurrentcolumn}%
      % To allow to recover \zxOriginalRow from the name of the node later, we create a new macro "namenode-zxOriginalRow" with the value of the original row, same for Col
      \noexpand\expandafter\noexpand\xdef\noexpand\csname\zxCurrentDiagram-\noexpand\the\noexpand\pgfmatrixcurrentrow-\noexpand\the\noexpand\pgfmatrixcurrentcolumn-zxOriginalRow\endcsname{\noexpand\zxOriginalRow}%
      \noexpand\expandafter\noexpand\xdef\noexpand\csname\zxCurrentDiagram-\noexpand\the\noexpand\pgfmatrixcurrentrow-\noexpand\the\noexpand\pgfmatrixcurrentcolumn-zxOriginalCol\endcsname{\noexpand\zxOriginalCol}%
      \expandonce{\zx@tmp@args}%
    }%
  }%
  \zx@tmp@code%
  % \edef\zx@tmp{}%
  % \expandafter\gdef\csname \zx@tmp\endcsname{#3}%
}

% \zxExecuteAtRegionRelative{n}{m}{code} will execute code \zxExecuteAtCellRelative in all cells starting from the current cell (note that
% it will have no effect on the (0,0) cell since this cell is not empty and already processed
\NewExpandableDocumentCommand{\zxExecuteAtRegionRelative}{mmm}{%
  \pgfmathparse{#1-1}%
  \foreach \i in {0,...,\pgfmathresult}{%
    \pgfmathparse{#2-1}%
    \foreach \j in {0,...,\pgfmathresult}{%
      \zxExecuteAtCellRelative{\i}{\j}{#3}%
    }%
  }%
}

\NewExpandableDocumentCommand{\zxExecuteAtRegionRelativeAndOrigin}{mmm}{%
  % Just a trick or the tikz matrix parser will not understand that we are in a node if we start with def:
  \coordinate(zx@Useless);
  \def\zxOriginalRow{\the\pgfmatrixcurrentrow}%
  \def\zxOriginalCol{\the\pgfmatrixcurrentcolumn}%
  %#3%
  \zxExecuteAtRegionRelative{#1}{#2}{#3}%
}

%%%
%%% Variables that you can create in one cell and reuse later
%%

% \zxGateMultiNewVariable{my variable}{value}
\NewExpandableDocumentCommand{\zxSetVariable}{mm}{%
  % For some reasons it fails inside the csname directly
  \edef\zx@tmp{zxGateMultiVariable-\zxCurrentDiagram-\the\pgfmatrixcurrentrow-\the\pgfmatrixcurrentcolumn-#1}%
  \expandafter\gdef\csname \zx@tmp\endcsname{%
    #2%
  }%
}

\NewExpandableDocumentCommand{\zxSetVariableExpandOnce}{mm}{%
  % For some reasons it fails inside the csname directly
  \edef\zx@tmp{zxGateMultiVariable-\zxCurrentDiagram-\the\pgfmatrixcurrentrow-\the\pgfmatrixcurrentcolumn-#1}%
  \expandafter\xdef\csname \zx@tmp\endcsname{%
    \expandonce{#2}%
  }%
}

\NewExpandableDocumentCommand{\zxSetVariableExpand}{mm}{%
  % For some reasons it fails inside the csname directly
  \edef\zx@tmp{zxGateMultiVariable-\zxCurrentDiagram-\the\pgfmatrixcurrentrow-\the\pgfmatrixcurrentcolumn-#1}%
  \expandafter\xdef\csname \zx@tmp\endcsname{%
    #2%
  }%
}

\def\zxGetVariable#1{%
  \csname zxGateMultiVariable-\zxCurrentDiagram-\zxOriginalRow-\zxOriginalCol-#1\endcsname%
}

\def\zxShowVariable#1{%
  \expandafter\show\csname zxGateMultiVariable-\zxCurrentDiagram-\zxOriginalRow-\zxOriginalCol-#1\endcsname%
}

%%%%%%%%%%
%%% Slices
%%%%%%%%%%

\tikzset{
  /zx/slice label style/.style={
    align=center,pos=0, black, anchor=south,zxTinyFont,
  },
  /zx/slice line style/.style={
    dashed, red, thick,
  },
}

\NewDocumentCommand{\zxSlice}{O{}O{}m}{%
  \zxExecuteAtEndPicture{
    \draw[/zx/slice line style,#2]
    (\tikzcdmatrixname.north-|{$(\zxCurrentDiagram-1-\zxOriginalCol)!.5!(\zxCurrentDiagram-1-\the\numexpr\zxOriginalCol+1\relax)$})
    -- (\tikzcdmatrixname.south-|{$(\zxCurrentDiagram-1-\zxOriginalCol)!.5!(\zxCurrentDiagram-1-\the\numexpr\zxOriginalCol+1\relax)$})
    node[rotated/.style={anchor=south west,yshift=-.5ex,rotate=####1},rotated/.default=30, /zx/slice label style,#1]{#3};
  }
}

\tikzset{
  /zx/vertical slice label style/.style={
    align=center,pos=1, black, anchor=west,zxTinyFont,
  },
  /zx/vertical slice line style/.style={
    dashed, red, thick,
  },
}

\NewDocumentCommand{\zxVSlice}{O{}O{}m}{%
  \zxExecuteAtEndPicture{%
    % \pgfextra{
      \pgfpointanchor{\tikzcdmatrixname}{west}%
      \pgfgetlastxy{\zx@tmp@west@x}{\zx@tmp@west@y}%
      \pgfpointanchor{\tikzcdmatrixname}{east}%
      \pgfgetlastxy{\zx@tmp@east@x}{\zx@tmp@east@y}%
      \pgfmathparse{\zx@tmp@east@x-\zx@tmp@west@x}%
      \edef\zxMatrixWidth{\pgfmathresult pt}%
    % }%
    \draw[/zx/vertical slice line style,#2]
    (\tikzcdmatrixname.west|-{$(\zxCurrentDiagram-\zxOriginalRow-1)!.5!(\zxCurrentDiagram-\the\numexpr\zxOriginalRow+1\relax-1)$})
    -- (\tikzcdmatrixname.east|-{$(\zxCurrentDiagram-\zxOriginalRow-1)!.5!(\zxCurrentDiagram-\the\numexpr\zxOriginalRow+1\relax-1)$})
    node[rotated/.style={anchor=south west,yshift=-.5ex,rotate=####1},rotated/.default=30, /zx/vertical slice label style,#1]{#3};
  }
}


%%%%%%%%%%%%%%%%%%%%%%%%%%%%%%%%%%%%%%%%
%%% Practical functions to nest diagrams
%%%%%%%%%%%%%%%%%%%%%%%%%%%%%%%%%%%%%%%%

\NewDocumentCommand{\zxSaveDiagram}{mO{}m}{
  \newsavebox{#1}%
  \savebox{#1}{%
    \ifdefined\tikz@library@external@loaded% Library external is loaded...
      \tikzset{external/optimize=false}% Otherwise, tikz will try to optimize and replace the box with a text
    \fi%
    \begin{ZX}[#2]#3\end{ZX}
  }%
}

\NewDocumentCommand{\zxUseDiagram}{m}{
  \usebox{#1}%
}

%%%%%%%%%%%%%%%%%%%%%
%%% Main environments
%%%%%%%%%%%%%%%%%%%%%

% Quantikz has a bug which adds space automatically.
% https://tex.stackexchange.com/questions/618330
% Fixing that by copying the original (unpatched) functions, and reusing them later.
% Warning: you must load this package **before** quantikz otherwise the fix will not work.
\let\tikzcd@@originalCopyZx\tikzcd@
\let\endtikzcd@originalCopyZx\endtikzcd

%%%%% Main environment \begin{ZX}...\end{ZX}. However, we call it ZXNoExt because when using
%%%%% externalization (to save compilation time), we need to wrap it around a figure.
\NewDocumentEnvironment{ZXNoExt}{O{}}{%
  \bgroup%
  % Do not change the font size here or it will change axis_height... Use \fontsize{10}{12}\selectfont
  % when drawing some texts instead.
  % Add a switch in case someone really wants the current tikzcd version:
  \ifdefined\doNotPatchQuantikz% Do not patch tikzcd.
  \else% Restore locally original tikzcd.
    \let\tikzcd@\tikzcd@@originalCopyZx%
    \let\endtikzcd\endtikzcd@originalCopyZx%
  \fi%
  % we need to reset this or it will get data from previous pictures
  \global\let\tikzcd@before@paths@hook\pgfutil@empty%
  \global\let\tikzcd@savedpaths\pgfutil@empty%
  \ifdefined\zxDoNotPatchArrows% Do not patch arrows (we need sometimes "every arrow post")
  \else% Copy/pasted from https://github.com/astoff/tikz-cd/blob/master/tikzlibrarycd.code.tex, just added "every arrow post".
    %% I'm not sure which method is more resilient: patchcmd is likely to fail on minor library update,
    %% while redefining \tikzcd@ar@new might not apply some major library upgrades (note that we redefine it
    %% only locally). Since I don't know what is better, I do both depending on this macro
    \ifdefined\zxUsePatchCmdToPatchArrows%
      \patchcmd\tikzcd@ar@new%
      {\path[/tikz/commutative diagrams/.cd,every arrow,#1]}%
      {\path[/tikz/commutative diagrams/.cd,every arrow,#1,every arrow post]}%
      {}%
      {}%
    \else%
      \def\tikzcd@ar@new[##1]{% Make sure to turn #1 into ##1 as the command is nested
        \pgfutil@g@addto@macro\tikzcd@savedpaths{%
          \path[/tikz/commutative diagrams/.cd,every arrow,##1,every arrow post]%<--- we added every arrow post
          (\tikzcd@ar@start\tikzcd@startanchor) to (\tikzcd@ar@target\tikzcd@endanchor); }}%
    \fi%
  \fi%
  % We make sure "every arrow post" exists:
  \tikzset{%
    /tikz/commutative diagrams/every arrow post/.append style={}%
  }%
  \pgfsetlayers{background,belownodelayer,edgelayer,nodelayer,main,abovenodelayer,box,labellayer,foreground}% Layers are defined locally to avoid to disturb other drawings
  %% Provide a way to reference the current diagram. This way it is for instance possible to do
  %% \ar[from=Zbotright, to=\zxCurrentDiagram-1-1]
  %% For this to work, you should not modify the name of the matrix.
  %% Old way
  % \def\zxCurrentDiagram{tikz@f@\the\tikz@fig@count}%Does not exists yet... Hopefully it's updated soon enough
  % Should be equivalent but might even work if we change the name of the matrix, hopefully we only use it at the right time.
  % Also, when drawing the lines, we are not anymore inside the tikzmatrix, but tikzcdmatrixname contains the name of the matrix instead:
  \def\zxCurrentDiagram{\ifdefined\tikzmatrixname\tikzmatrixname\else\tikzcdmatrixname\fi}%
  % \pgfutil@g@addto@macro\tikzcd@before@paths@hook{%
  %   \def\zxCurrentDiagram{\tikzcdmatrixname}%
  % }%
  % the problem with \tikzcdmatrixname is that it is only defined at the end, once the matrix is built
  % We provide a way to get the current matrix name:
  \begin{tikzcd}[%
    /zx/defaultEnv,%
    #1]%
  }{\end{tikzcd}%
  \global\let\tikzcd@before@paths@hook\pgfutil@empty%
  \global\let\tikzcd@savedpaths\pgfutil@empty%
  \egroup}

% https://tex.stackexchange.com/a/619983/116348
\ExplSyntaxOn
%%%%% Shortcut macro \zxNoExt{...} equivalent to \begin{ZX}...\end{ZX}
%%%%% We will create an alias \zx, but when we use externalization we
%%%%% wrap it around a figure.
\NewDocumentCommand{\zxNoExt}{O{}+m}{% Warning, expl syntax removes space.
  \tl_rescan:nn { \char_set_catcode_active:N \& } { \begin{ZXNoExt}[#1] #2 \end{ZXNoExt} }
}
\ExplSyntaxOff

%% Version to use when using externalize (it wraps it around a 
\NewDocumentCommand{\zxExt}{O{}O{}O{}+m}{% Warning, expl syntax removes space.
  \begin{tikzpicture}[baseline=(zxlibrarymainnode.base),#2]%
    \node(zxlibrarymainnode)[inner sep=0pt,outer sep=0pt,#3]{\zxNoExt[#1]{#4}};%
  \end{tikzpicture}%
}

\NewDocumentEnvironment{ZXExt}{O{}O{}O{}+b}{%
  \zxExt[#1][#2][#2]{#4}%
}{}%

\NewDocumentEnvironment{ZX}{O{}O{}O{}+b}{%
  \zx[#1][#2][#3]{#4}%
}{}

% The external library is not compatible with tikzcd directly.
% So if:
% \zx@external@mode is enabled, we wrap the figure into another figure to be compatible with tikz externalize
% otherwise, we check if the external library is loaded, and disable it temporarily.
\NewDocumentCommand\zx{O{}O{}O{}+m}{%
  {%
    % Suffix for pictures made only with zx when using the external library
    \ifdefined\tikz@library@external@loaded% Library external is loaded...
      \ifdefined\zx@external@suffix%
        \tikzappendtofigurename{\zx@external@suffix}%
      \fi%
    \fi%
    \ifdefined\zx@external@mode%
      % We wrap everything around figures and enable tikz external (useful when many figures
      % are not compatible with external):
      \ifthenelse{\equal{\zx@external@mode}{wrapForceExt}}{%
        {\tikzexternalenable\zxExt[#1][#2][#3]{#4}}%
      }{%
        \ifthenelse{\equal{\zx@external@mode}{wrap}}{% We wrap everything around figures to be compatible with external
          \zxExt[#1][#2][#3]{#4}%
        }{% We don't wrap anything inside figures, so we lose external compatibility.
          \ifthenelse{\equal{\zx@external@mode}{noWrapNoExt}}{%
            % We disable the tikz external library temporarily (tikzcd is not compatible with it)
            \ifdefined\tikz@library@external@loaded% Library external is loaded...
              \tikzexternaldisable%
              \message{WARNING: you chose to disable temporarily the tikz external library.}%
            \fi%
          }{}%
          \zxNoExt[#1]{#4}%
        }%
      }%
    \else% Mode auto enabled
      {% If the tikz library is loaded:
        \ifdefined\tikz@library@external@loaded% Library external is loaded...
          \zxExt[#1][#2][#3]{#4}%
        \else%
          \zxNoExt[#1]{#4}%      
        \fi%
      }%
    \fi%
  }%
}

%% By default, the error displayed when compiling using "external"
%% is not meaningful and does not stop if a picture already exists.
%% https://tex.stackexchange.com/questions/633100/stop-at-error-meaningfull-errors-with-shell-escape-and-tikz-externalize/633121#633121
%% See also this bug, that shows that if you have an error and compile twice, the error disappears, but the old file is used instead.
%% https://github.com/pgf-tikz/pgf/issues/1137
\NewDocumentCommand\zxConfigureExternalSystemCall{O{}}{
  %% Code inspired by https://github.com/pgf-tikz/pgf/blob/a7b45b35e99af11bf7156aa3697b897b98870e5b/tex/generic/pgf/frontendlayer/tikz/libraries/tikzexternalshared.code.tex#L277
  \expandafter\def\csname zx@driver@pgfsys-luatex.def\endcsname{%
    \pgfkeyssetvalue{/tikz/external/system call}{%
      lualatex \tikzexternalcheckshellescape #1 -jobname "\image" "\texsource"%
    }%
  }%
  \expandafter\def\csname zx@driver@pgfsys-pdftex.def\endcsname{%
    \pgfutil@IfUndefined{directlua}{%
      \pgfkeyssetvalue{/tikz/external/system call}{%
        pdflatex \tikzexternalcheckshellescape #1 -jobname "\image" "\texsource"%
      }%
    }{%
      \pgfkeyssetvalue{/tikz/external/system call}{%
        lualatex \tikzexternalcheckshellescape #1 -jobname "\image" "\texsource"%
      }%
    }%
  }%
  \expandafter\def\csname zx@driver@pgfsys-xetex.def\endcsname{%
    \pgfkeyssetvalue{/tikz/external/system call}{%
      xelatex \tikzexternalcheckshellescape #1 -jobname "\image" "\texsource"%
    }%
  }%
  \expandafter\def\csname zx@driver@pgfsys-dvips.def\endcsname{%
    \pgfkeyssetvalue{/tikz/external/system call}{%
      latex \tikzexternalcheckshellescape #1 -jobname "\image" "\texsource" %
      && dvips -o "\image".ps "\image".dvi %
    }%
  }%
  % Auto-select a suitable default value fo 'system call':
  \pgfutil@ifundefined{tikzexternal@driver@\pgfsysdriver}{%
    % fallback. We do not know the driver here.
    \csname zx@driver@pgfsys-pdftex.def\endcsname
  }{%
    \csname zx@driver@\pgfsysdriver\endcsname
  }%
}

% Call zxConfigureExternalSystemCall with the same system call as the calling program
% Useful otherwise the error may be missed (for instance errorstopmode works in pdflatex but not in
% emacs, and nonstopmode works in emacs, but not in pdflatex).
% The current mode is stored in \interactionmode, 0, 1, 2, 3 corresponding respectively to batch, nonstop, scroll or errorstop. 
% https://tex.stackexchange.com/questions/91592/where-to-find-official-and-extended-documentation-for-tex-latexs-commandlin
\NewDocumentCommand\zxConfigureExternalSystemCallAuto{}{%
  \ifnum\interactionmode=0 %
    \zxConfigureExternalSystemCall[-interaction=batchmode]%
  \fi%
  \ifnum\interactionmode=1 %
    \zxConfigureExternalSystemCall[-interaction=nonstopmode]% emacs
  \fi%
  \ifnum\interactionmode=2 %
    \zxConfigureExternalSystemCall[-interaction=scrollmode]%
  \fi%
  \ifnum\interactionmode=3 %
    \zxConfigureExternalSystemCall[-interaction=errorstopmode]% pdflatex by default
  \fi%
}


\ifdefined\zxDoNotPatchSystemCall\else%
  \zxConfigureExternalSystemCallAuto
\fi

% \zx@external@mode can be undefined (=auto), wrap or nowrap.

%% Automatically check if the external library is loaded, and use it.
\NewDocumentCommand\zxExternalAuto{}{%
  \let\zx@external@mode\undefined%
}

%% Force to wrap the ZX figures around a figure to ensure compatibility with external.
\NewDocumentCommand\zxExternalWrap{}{%
  \def\zx@external@mode{wrap}%
}

%% Do not wrap the ZX figures around another figure (not compatible with external)
\NewDocumentCommand\zxExternalNoWrap{}{%
  \def\zx@external@mode{nowrap}%
}

%% Do not wrap the ZX figures, but disable external for these figures (semi-compatible with external)
\NewDocumentCommand\zxExternalNoWrapNoExt{}{%
  \def\zx@external@mode{noWrapNoExt}%
}

%% Wrap the ZX figures, and enable external for these figures (useful when most other environments
%% are not compatible with tikz external)
\NewDocumentCommand\zxExternalWrapForceExt{}{%
  \def\zx@external@mode{wrapForceExt}%
}

% Mode auto is enabled by default
\zxExternalAuto

\NewDocumentCommand\zxExternalSuffix{m}{
  \ifthenelse{\equal{#1}{}}{% Nothing submitted: we erase the suffix
    \let\zx@external@suffix\undefined%
  }{%
    \def\zx@external@suffix{#1}
  }%
}

\zxExternalSuffix{zx}



%%%%%%%%%%%%%%%%%%%%%%%%%%%%%%
%%% Custom nodes from Pic
%%%%%%%%%%%%%%%%%%%%%%%%%%%%%% 
%% It is quite complicated to define custom node shapes, especially when using anchors. Here, we define a
%% Way to easily create a custom node from a picture.

%% This code allows to add a new anchor to a given node:
%% https://tex.stackexchange.com/a/676090/116348
\tikzset{
  zx add anchor to node/.code n args={3}{%
    \edef\tikz@temp##1{% \tikz@pp@name/\tikzlastnode needs to be expanded
      %% This codes does the following thing:
      %% First, it modifies the "\pgf@anchor@rectangle@fake center west" macro (rectangle might be another shape,
      %% same for anchor). But since this macro might be different between nodes sharing the same shape,
      %% it adds it to a per-node macro called like \pgf@sh@ma@tikz@f@3-2-2 (tikz@f@3-2-2 is the name of the
      %% node).
      \noexpand\pgfutil@g@addto@macro\expandafter\noexpand\csname pgf@sh@ma@\tikz@pp@name{#1}\endcsname{%
        \def\expandafter\noexpand\csname pgf@anchor@\csname pgf@sh@ns@\tikz@pp@name{#1}\endcsname @#2\endcsname{##1}%
      }%
    }%
    \tikz@temp{#3}%
  },
  zx add anchor to pic default/.style={/tikz/zx add anchor to pic={#1}{#1}},
  zx add anchor to pic/.code 2 args={%
    % If base coordinate isn't created yet, do so at (0,0)
    % in the current coordinate system!
    \pgfutil@ifundefined{pgf@sh@ns@\tikz@pp@name{}}{%
      \pgfcoordinate{\tikz@pp@name{}}{\pgfpointorigin}%
    }{}%
    \begingroup
      % What is the distance between coordinate and base?
      % We have to do this in the coordinate system of the base coordinate.
      \pgfsettransform{\csname pgf@sh@nt@\tikz@pp@name{}\endcsname}%
      \pgf@process{\pgfpointanchor{\tikz@pp@name{#2}}{center}}%
      % This distance is the new anchors coordinate
      % in the base's coordinate system.
      % Adding an anchor to a node must be global
      % which is why we can do this inside the group.
      \pgfkeysalso{
        /tikz/zx add anchor to node/.expanded=%
          {}{#1}{\noexpand\pgfqpoint{\the\pgf@x}{\the\pgf@y}}
      }%
    \endgroup
  }%
}

%%% Create automatically a new pic-based node
%%% \zxNewNodeFromPic{nameOfNode}[default style before user][default style after user]{content of the pic}
%%% - In 'default style before user', we can put any style that is given to the pic (e.g. colors, rotate, scale…)
%%% or even custom properties that will be read by the pic (e.g. to allow the define the number of spikes of a
%%% star). This style can notably have "zx create anchors={a, list, of, coordinates, to, turn, into, anchors}"
%%% that will automatically turn any coordinate in the pic in the list of coordinates into an anchor of the
%%% parent node.
%%% - In 'default style after user', the style is run after the code provided by the user. In particular, you can
%%% use "invert top bottom"  to invert the "top" version with the "bottom" version (rotation is minus 90 instead
%%% of 90 for instance).
\NewDocumentCommand{\zxNewNodeFromPic}{mO{}O{}O{}O{}m}{%
  \tikzset{%
    #1Pic/.pic={%
      #6%
      % The user can even add more stuff to draw after if needed by redefining this macro:
      \ifdefined\zxCustomPicAdditionalPic\zxCustomPicAdditionalPic\fi%
      \tikzset{%
        % We store the list of anchors to create into \zxListOfNewAnchors. If it does not exist we create it:
        /utils/exec={\ifdefined\zxListOfNewAnchors\else\def\zxListOfNewAnchors{}\fi},
        % We add the coordinates in \zxListOfNewAnchors as anchors for the node
        zx add anchor to pic default/.list/.expanded={\zxListOfNewAnchors}}%
    },%
    #1/.style={%
      %shape=coordinate, %yshift=axis_height, %% already applied
      #4,
      append after command={%
        pic[
        % We configure the name and position of the pic
        at=(\tikzlastnode), name=\tikzlastnode,
        % We create a command to add anchors via zx create anchors={my, list, of, anchors}. This way, it also allows dynamic anchors using
        % the default style after user.
        zx create anchors/.store in={\zxListOfNewAnchors},
        zx create anchors={},
        % We create a new style to apply to the main node
        zx main node/.style={
          name=,
          zx forward to node/.append style={},
          zx forward to node,
        },
        % We load the other default arguments:
        #2,
        % We try to load a potentially global style configured by the user (mostly used to override this library)
        /zx/picCustomStyleBeforeUser#1/.append style={}, % Make sure it exists to avoid errors
        /zx/picCustomStyleBeforeUser#1,
        % We load the per-instance arguments:
        ##1,
        % We try to load a potentially global style configured by the user (mostly used to override this library)
        /zx/picCustomStyleAfterUser#1/.append style={}, % Make sure it exists to avoid errors
        /zx/picCustomStyleAfterUser#1,
        #3,
        % We try to load a potentially global style configured by the user (mostly used to override this library)
        /zx/picCustomStyleLastPic#1/.append style={}, % Make sure it exists to avoid errors
        /zx/picCustomStyleLastPic#1,
        ]{#1Pic}%
      },
    }%
  }%
  %% \ExpandArgs{c} is an equivalent of expandafter + csname, to expand the first argument before evaluating the function:
  %% https://tex.stackexchange.com/a/676103/116348
  %% ##1: pic additional style
  %% ##2: node additional style
  %% ##3: text to consume with \tikzpictext
  % \ExpandArgs{c}\NewExpandableDocumentCommand{zx#1}{t.t-t'O{}O{}m}{%
  %   |[#1={pic text={##6}, ##4}, ##5]|%
  % }%
  \ExpandArgs{c}\NewExpandableDocumentCommand{zx#1}{O{}O{}t.t-t't/t*e_m}{%
    \node[zx main node, #1={
      pic text={##9},
      %% The alias=… should rather be called on the second element. But it's quite handy to write it here, so here we go:
      a/.style={%
        zx forward to node/.append style={
          a={########1},
        },
      },
      %% The mode stored in \zxRotationMode contains the current rotation mode (i.e. the angle between 0 and 359, typically 0,90,180,270),
      %% to allow further tweaks (for instance, a pic might be mirrored instead of rotated by 180° when using the parameter -, and this
      %% can be configured by reading this variable).
      /utils/exec={%
        \def\zxCurrentRotationMode{0}%
        %%% Fake center is mostly for curves like <', and true north/ is mostly for curves like C, C'...
        %%% These anchor name are changed depending on the rotation: the north/… is before the rotation.
        \def\zxVirtualCenterNorth{fake center north}%
        \def\zxVirtualCenterSouth{fake center south}%
        \def\zxVirtualCenterEast{fake center east}%
        \def\zxVirtualCenterWest{fake center west}%
        \def\zxTrueNorth{true north}%
        \def\zxTrueSouth{true south}%
        \def\zxTrueEast{true east}%
        \def\zxTrueWest{true west}%
        \IfBooleanT{##3}{%
          \def\zxCurrentRotationMode{90}%
          \def\zxVirtualCenterNorth{fake center west}%
          \def\zxVirtualCenterSouth{fake center east}%
          \def\zxVirtualCenterEast{fake center north}%
          \def\zxVirtualCenterWest{fake center south}%
          \def\zxTrueNorth{true west}%
          \def\zxTrueSouth{true east}%
          \def\zxTrueEast{true north}%
          \def\zxTrueWest{true south}%
        }%
        \IfBooleanT{##4}{%
          \def\zxCurrentRotationMode{180}%
          \def\zxVirtualCenterNorth{fake center south}%
          \def\zxVirtualCenterSouth{fake center north}%
          \def\zxVirtualCenterEast{fake center west}%
          \def\zxVirtualCenterWest{fake center east}%
          \def\zxTrueNorth{true south}%
          \def\zxTrueSouth{true north}%
          \def\zxTrueEast{true west}%
          \def\zxTrueWest{true east}%
        }%
        \IfBooleanT{##5}{%
          \def\zxCurrentRotationMode{270}%
          \def\zxVirtualCenterNorth{fake center east}%
          \def\zxVirtualCenterSouth{fake center west}%
          \def\zxVirtualCenterEast{fake center south}%
          \def\zxVirtualCenterWest{fake center north}%
          \def\zxTrueNorth{true east}%
          \def\zxTrueSouth{true west}%
          \def\zxTrueEast{true south}%
          \def\zxTrueWest{true north}%
        }%
        \IfBooleanT{##6}{%
          \def\zxModeSlash{}%
        }%
        \IfBooleanT{##7}{%
          \def\zxModeStar{}%
        }%
        % Embellishments can contain something like \zxMynode_{Text A}{Text B}
        \IfNoValueTF{##8}{}{%
          \def\zxModeUnderscore{##8}%
        }%
      },
      rotate=\zxCurrentRotationMode,%
      %%% Sometimes it can be quite handy to invert the top and the bottom version if it does not match with the
      %%% intuitive position it should have.
      %%% Make sure to call this in the second optional parameter (after user style)
      invert top bottom/.style={
        rotate=-\zxCurrentRotationMode,
        /utils/exec={%
          \ifnum\zxCurrentRotationMode=90\relax
            \def\zxCurrentRotationMode{270}%
          \else
            \ifnum\zxCurrentRotationMode=270\relax
              \def\zxCurrentRotationMode{90}%
            \fi
          \fi
        },
        rotate=\zxCurrentRotationMode,%        
      },%
      invert right left/.style={
        rotate=-\zxCurrentRotationMode,
        /utils/exec={%
          \ifnum\zxCurrentRotationMode=0\relax
            \def\zxCurrentRotationMode{180}%
          \else
            \ifnum\zxCurrentRotationMode=180\relax
              \def\zxCurrentRotationMode{0}%
            \fi
          \fi
        },
        rotate=\zxCurrentRotationMode,%        
      },%
      ##1}, #5, ##2]{};%
  }%
}%

%% Useful to increase or decrease the bounding box of a custom node.
%% Parameters are like extend=5mm (that extends on all directions), and similarly for top/bottom/left/right/horizontal/vertical
\NewExpandableDocumentCommand{\zxExtendBoundingBox}{m}{%
  \pgfkeys{
    /zx/.cd,
    left/.store in=\zx@left,
    left=0pt,
    right/.store in=\zx@right,
    right=0pt,
    top/.store in=\zx@top,
    top=0pt,
    bottom/.store in=\zx@bottom,
    bottom=0pt,
    horizontal/.style={
      left=##1,
      right=##1,
    },
    vertical/.style={
      /zx/top=##1,
      /zx/bottom=##1,
    },
    extend/.style={
      vertical=##1,
      horizontal=##1,
    },
    #1,
  }%
  \path[use as bounding box] ([xshift=-\zx@left,yshift=\zx@top]current bounding box.north west)
    rectangle ([xshift=\zx@right,yshift=-\zx@bottom]current bounding box.south east);%
}


%%%%%%%%%%%%%%%%%%%%%%%%%%%%%%
%%% Some special nodes based on \zxNewNodeFromPic
%%%%%%%%%%%%%%%%%%%%%%%%%%%%%%

%% Ground symbol
\zxNewNodeFromPic{Ground}[scale=\zxGroundScale][invert top bottom]{
  \draw[line width=\zxDefaultLineWidth]
  (0,0) -- (1mm,0)
  -- +(0,1mm) -- +(0,-1mm)
  ++(.4mm,0)
  -- +(0,.7mm) -- +(0,-.7mm)
  ++(.4mm,0)
  -- +(0,.35mm) -- +(0,-.35mm);
  \coordinate() at (0,0); % Empty coordinate make the center start at the right position
}

%% From scalable ZX calculus
\zxNewNodeFromPic{Divider}[][
  % Because of the rounded corners, the lines will stop before touching the shape when arriving on the angle. So we force such nodes to be drawn below
  % the shape, targeting a point inside the node.
  %post arrow style if end node/.expanded={% we need to expand everything before setting the style, as this style will be executed much later in the arrow,
  post arrow style if end node={% if the text contains macro like \zxVirtualCenterWest use post arrow style if end node/.expanded instead as these macro
    % will not exist anymore once the line is drawn
    on layer=edgelayer,%
    end anchor if not set=fake center north,
  },%
  post arrow style if start node={% if the text contains macro like \zxVirtualCenterWest use post arrow style if end node/.expanded instead as these macro
    % will not exist anymore once the line is drawn
    on layer=edgelayer,%
    start anchor if not set=fake center south,
  },%
  zx create anchors={\zxVirtualCenterWest, \zxVirtualCenterEast, \zxVirtualCenterNorth, \zxVirtualCenterSouth},
  every node/.append style={transform shape}
]{
  \node[regular polygon, regular polygon sides=3,shape border rotate=90, draw=black,fill=gray!50, inner sep=1.6pt, rounded corners=0.8mm,zx main node] {};
  \coordinate(\zxVirtualCenterEast) at (.2mm,0); % Used to start lines on the side of the shape
  \coordinate(\zxVirtualCenterWest) at (-1mm,0);
  \coordinate(\zxVirtualCenterNorth) at (\zxVirtualCenterWest); % Synonyme, useful to determine where to start
  \coordinate(\zxVirtualCenterSouth) at (\zxVirtualCenterEast); % Synonyme, useful to determine where to start
}

%% Matrix symbol
\zxNewNodeFromPic{Matrix}
  [
    % As we want - to only swap the position of the label but not the rotation (we use * already for transpose)
    % we disable the above rotation:
    rotate=-\zxCurrentRotationMode,
    % Otherwise the space between colums is too big
    /utils/exec={\setlength\arraycolsep{1pt}}, 
  ][zx create anchors={true north,true south,true east,true west}]{
  \node[zx main node, draw, signal, fill=colorZxMatrix, inner sep=1pt, minimum height=6pt, transform shape,
    % The direction of the shape is turned when using the * symbol (transpose). 
    signal to/.expanded={\ifdefined\zxModeStar west\else east\fi},
    signal from/.expanded={\ifdefined\zxModeStar east\else west\fi},
    % We disabled the default rotation (* is for transpose), but we still want to rotate it to read to/down:
    rotate/.expanded={mod(-\zxCurrentRotationMode,180)},
    % If we want the syntax \zxMatrix{45:f} to put f at 45 degrees, we want to expand \tikzpictext
    % *before* to read the key, hence the need for expanded.
    label/.expanded={[
      inner sep=2pt,
      % The position of the label is independent of the rotation, but depends on the rotation mode ('.-):
      absolute,
      label position=\ifdefined\zxLabelAngle \zxLabelAngle\else 90+\zxCurrentRotationMode\fi,
      font=\noexpand\footnotesize, % We don't want to expand \footnotesize before defining the label
      % we use the / decoration
      overlay/.expanded={\ifdefined\zxModeSlash true\else false\fi},
      % Math mode by default:
      execute at begin node=$, execute at end node=$,
      /zx/picCustomStyleMatrixLabel/.append style={}, % Make sure it exists to avoid errors
      /zx/picCustomStyleMatrixLabel,
      ]
      % We put as the label the text directly. Note that \unexpanded\expandafter expands the next token only once
      % as if it gets expanded further, we can get errors (e.g.\ with pmatrix)
      \unexpanded\expandafter{\tikzpictext}
      \ifdefined\zxModeUnderscore\noexpand\begin{bmatrix}\noexpand\zxModeUnderscore\noexpand\end{bmatrix}\fi
    },
    /zx/picCustomStyleMatrixMainNode/.append style={}, % Make sure it exists to avoid errors
    /zx/picCustomStyleMatrixMainNode,
    ]{};
    \coordinate(true east) at (.east); % Used to start lines on the side of the shape
    \coordinate(true west) at (.west);
    \coordinate(true north) at (true west); % Synonyme, useful to determine where to start
    \coordinate(true south) at (true east); % Synonyme, useful to determine where to start
}

%% A bit a refactoring might be useful to avoid repeating ourself
% %% Box, that just draws... a box
\zxNewNodeFromPic{Box}[
  main/.style={
    /zx/picCustomStyleBoxMainNode/.append style={####1},
  },
  add label advanced/.style 2 args={
    main={
      label={[inner sep=1.8pt,zxTinyFont,####1]####2},
    },
  },
  block/.style={main={/tikz/block}},
  % label already exists
  add label/.style={
    add label advanced={}{####1},
  },
][zx create anchors={fake center north,fake center south,fake center east,fake center west}]{%
  \node[draw, inner sep=1.3mm, rectangle, zx filling style, zx thickness wires style, zx main node, execute at begin node=$, execute at end node=$,%alias=hello,
  minimum width=\zxBoxMinimumWidth, minimum height=\zxBoxMinimumHeight,
  /zx/picCustomStyleBoxMainNode/.append style={}, % Make sure it exists to avoid errors
    /zx/picCustomStyleBoxMainNode,
  ]{\tikzpictext};%
  \coordinate(fake center north) at (.center);%
  \coordinate(fake center west) at (.center);%
  \coordinate(fake center east) at (.center);%  
  \coordinate(fake center south) at (.center);%
}


\NewCommandCopy\zxGate\zxBox

% Like box, but don't draw the border. Useful for instance in \zxElt{\ket{\psi}}
\zxNewNodeFromPic{Elt}[
  main/.style={
    /zx/picCustomStyleEltMainNode/.append style={####1},
  },
][zx create anchors={fake center north,fake center south,fake center east,fake center west}]{%
  \node[inner sep=1.3mm, rectangle, zx main node, execute at begin node=$, execute at end node=$,%alias=hello,
  minimum width=\zxBoxMinimumWidth, minimum height=\zxBoxMinimumHeight,
  /zx/picCustomStyleEltMainNode/.append style={}, % Make sure it exists to avoid errors
    /zx/picCustomStyleEltMainNode,
  ]{\tikzpictext};%
  \coordinate(fake center north) at (.north);%
  \coordinate(fake center west) at (.west);%
  \coordinate(fake center east) at (.east);%  
  \coordinate(fake center south) at (.south);%
}

% \NewExpandableDocumentCommand{\zxEltMulti}{O{}mmm}{
%   \zxGateMulti[#1]{#2}{#3}{#4}
% }

%%%%%%%%%%%%%%%%%%%%%%%%%%%
%%% Gate Multi (user part)
%%%%%%%%%%%%%%%%%%%%%%%%%%%

% TODO: maybe try to clean: https://tex.stackexchange.com/questions/696958/simple-way-to-specify-a-pass-a-style-with-syntax-my-style-foo
\NewExpandableDocumentCommand{\zxGateMulti}{O{}mmm}{
  % The coordinate here is useful to help tikzcd to see that it is a node as it tries to search for path
  \path;\pgfkeys{
    /zx/gateMulti/.cd,
    content inner nodes/.store in=\myContent,
    content inner nodes={#4},
    a/.store in=\zxGateMultiAlias,
    a=,
    style inner nodes/.store in=\zxGateMultiStyle,
    style inner nodes={},
    % style for the main style
    main/.store in=\zxGateMultiMainStyle,
    main={},
    main text/.store in=\zxGateMultiMainTextStyle,
    main text={},
    % Run additional code after the last \node
    additional code/.store in=\zxGateMultiAdditionalCode,
    additional code=,   
    % This will compute the size of the content, and adapt the width of the inner nodes
    % leaving around a margin specified in the 2 first arguments (row/column), the last argument is the minimum width of the inner nodes
    % note that we take as a reference the content of \pgfmatrixcolumnsep which might not be constant: if you change the width of the columns,
    % you are on your own.
    fit content/.code n args={4}{
      % First, we compute the width of the content.
      % \newdimen\zx@totalheight% + total height of box + margin
      % \newdimen\zx@width% width of box + margin
      % we save the content in a box:
      % I asked here if there is a better approach:
      % https://tex.stackexchange.com/questions/696964/cannot-set-sbox-inside-tikz
      %% Old version (dirty)
      % \begin{lrbox}{\zx@tmp@box}%
      %   \let\selectfont\pgf@selectfontorig\begin{tikzpicture}\zxBox{#4}\end{tikzpicture}%
      %   % I don't know why, but adding a second, empty tikzpicture help, otherwise it sets a huge bounding box for the first element.
      %   % maybe tikzpicture's variable are not all set locally?
      %   \begin{tikzpicture}\end{tikzpicture}%
      % \end{lrbox}%
      % New version:
      \begin{lrbox}{\zx@tmp@box}%
        \begin{pgfinterruptpicture}%
          \begin{pgfpicture}%
            \zxBox{#4}%
          \end{pgfpicture}%
        \end{pgfinterruptpicture}%
      \end{lrbox}%
      % % we measure the content of the box (##x appears 2 times since we have left and right margins):
      \edef\zx@totalheight{\the\dimexpr\ht\zx@tmp@box + \dp\zx@tmp@box + ##1 + ##1\relax}%
      \edef\zx@width{\the\dimexpr\wd\zx@tmp@box + ##2 + ##2\relax}%
      \edef\zx@width{\the\dimexpr\wd\zx@tmp@box\relax}%
      % \show\zx@totalheight
      % \show\zx@width
      %% We compute now the width/total height of each node. Basically, we have #3 blocks, and #3-1 column separations, so we want to find x such that:
      %% x*#2 + (#2-1)*\pgfmatrixcolumnsep = \zx@width, i.e.
      %% x = (\zx@width - (#2-1)*\pgfmatrixcolumnsep)/#2,
      \pgfmathsetmacro{\zx@final@width@block}{max(##4,(\zx@width - (#3-1)*\pgfmatrixcolumnsep)/(#3))}%
      \pgfmathsetmacro{\zx@final@height@block}{max(##3,(\zx@totalheight - (#2-1)*\pgfmatrixrowsep)/(#2))}%
      \pgfkeysalso{%
        content inner nodes={},
        style inner nodes/.expanded={
          inner sep=0pt,
          outer sep=0pt,
          opacity=0,
          % fill=red,
          minimum width=\zx@final@width@block pt,
          minimum height=\zx@final@height@block pt,
        },
      }%
    },
    fit content/.default={0mm}{0mm}{\zxBoxMinimumHeight}{\zxBoxMinimumWidth},
    /utils/exec={
      \ifdefined\zxDisableFitContent\else%
        \pgfkeysalso{fit content}%
      \fi%
    },
    safe fit/.style={
      style inner nodes={},
      content inner nodes={#4},
    },
    add label advanced/.style 2 args={
      main={
        label={[align=center,inner sep=1.8pt,zxTinyFont,##1]##2},
      },
    },
    % label already exists
    add label/.style={
      add label advanced={}{##1},
    },
    #1,
    % % /tikz/column sep/.show code,
    % /utils/exec={
    %   \message{\pgfmatrixcolumnsep}
    %   \show\pgfmatrixcolumnsep
    % },
  }%
  % Store the content for letter use
  \zxSetVariableExpandOnce{content inner nodes}{\myContent}%
  \zxSetVariableExpandOnce{style inner nodes}{\zxGateMultiStyle}%
  \zxSetVariableExpandOnce{style main node}{\zxGateMultiMainStyle}%
  \zxSetVariableExpandOnce{style main text node}{\zxGateMultiMainTextStyle}%
  \zxSetVariableExpandOnce{gate multi alias}{\zxGateMultiAlias}%
  \zxSetVariableExpandOnce{additional code}{\zxGateMultiAdditionalCode}%
  \zxExecuteAtRegionRelativeAndOrigin{#2}{#3}{%
    \zxBox[%
    main={zx main node,
      draw,
      opacity=0,
      zx style from variable={style inner nodes},
    }]{\zxGetVariable{content inner nodes}}
  }%
  \zxExecuteAtEndPicture{
    % We cannot use fit directly as it misplace the text. According to the doc, we should place the text using another intermediate node:
    \node[
      draw,
      zx filling style,
      zx thickness wires style,
      a=\zxGetVariable{gate multi alias},
      inner sep=0pt,
      fit=(\zxGetNameRelativeNode{0}{0})(\zxGetNameRelativeNode{\the\numexpr#2-1\relax}{\the\numexpr#3-1\relax}),
      alias=zxMainNode,
      /zx/picCustomStyleBoxMainNode/.append style={}, % Make sure it exists to avoid errors
      /zx/picCustomStyleBoxMainNode,
      zx style from variable={style main node}]{};
      \node[rectangle, execute at begin node=$, execute at end node=$, at=(zxMainNode.center),alias=zxMainNodeText,zx style from variable={style main text node}]{#4};
      % \message{\zxGetVariable{additional code}}
      % \expandafter\show\let
      \scantokens{%
        \zxGetVariable{additional code}%
      }%
  }
}

%%%%%%%%%%%%%%%%%%%%%%%%%%%
%%% Circuit-related gates
%%%%%%%%%%%%%%%%%%%%%%%%%%%

\tikzset{
  zx filling style/.style={
    fill=white
  },
  % To change the thickness of wires and boxes at the same time
  zx thickness wires style/.style={
  },
}

\NewExpandableDocumentCommand{\zxCtrl}{O{}m}{%
  \node[zx main node, circle, inner sep=2pt, fill=black, #1]{};
}

% white ctrl
\NewExpandableDocumentCommand{\zxOCtrl}{O{}m}{%
  \node[zx main node, circle, inner sep=2pt, zx filling style, zx thickness wires style, draw, #1]{};
}

\NewExpandableDocumentCommand{\zxNot}{O{}m}{%
  \node[zx main node, circle, inner sep=2.5pt, draw, zx filling style, zx thickness wires style, append after command={(\tikzlastnode.north) edge[zx thickness wires style] (\tikzlastnode.south) (\tikzlastnode.east) edge[zx thickness wires style] (\tikzlastnode.west)}, #1]{};
}

\NewExpandableDocumentCommand{\zxCross}{O{}m}{%
  \zxCoordinateMain{}
  \node[at={(0,0)}, rectangle, inner sep=3pt, append after command={(\tikzlastnode.north east) edge[line cap=round,zx thickness wires style] (\tikzlastnode.south west) (\tikzlastnode.south east) edge[line cap=round,zx thickness wires style] (\tikzlastnode.north west)}, #1]{};
}

% Loosely inspired by
% https://tex.stackexchange.com/questions/416055/measurement-meter-quantum-circuit-with-tikz
% Avoid nesting tikz pictures, more efficient, more configurable
\newsavebox\zxMeterBox
\begin{lrbox}{\zxMeterBox}%
  \begin{tikzpicture}[line width=.5,scale=.5]%
    \draw[line cap=round] (0,0) to[bend left=50] (1,0);%
    % \draw[line cap=round,-latex] (.5,-.2) to (.7,.8);
    \draw[line cap=round,-{Stealth[length=6, round]}] (.5,-.15) to (.9,.75);%
    %\useasboundingbox (-0.1,-.16) (1.1,.9);
  \end{tikzpicture}%
\end{lrbox}

\NewExpandableDocumentCommand{\zxMeter}{O{1}m}{%
  \pgfmathparse{.8*\zxMeterScale*#1}%
  \scalebox{\pgfmathresult}{\usebox{\zxMeterBox}}%
}

% ---------------------------------
% --- (Circuit/Gate) inputs/outputs
% ---------------------------------
% I found no great way to pass the style to the label... but it seems like somehow the font color
% of the parent node is automatically transferred to the label, so the interface is not too bad

\pgfkeys{
  /zx/input style/.style={
    align=center,
    inner sep=2pt,
    zxNormalFont,
    anchor=east,
  },
}

\NewExpandableDocumentCommand{\zxInput}{O{}m}{%
  \zxExecuteAtEndPicture{\node[/zx/input style,#1] at (\zxGetNameRelativeNode{0}{0}){$#2$};}%
}

\pgfkeys{
  /zx/output style/.style={
    align=center,
    inner sep=2pt,
    zxNormalFont,
    anchor=west,
  },
}

\NewExpandableDocumentCommand{\zxOutput}{O{}m}{%
  \zxExecuteAtEndPicture{\node[/zx/output style,#1] at (\zxGetNameRelativeNode{0}{0}){$#2$};}%
}

\pgfkeys{
  /zx/input multi style/.style={
    decorate,
    decoration={brace, mirror, raise=2pt},
  },
  /zx/input multi label style/.style={swap,outer sep=5pt,zxNormalFont},
}

% https://tex.stackexchange.com/questions/696958/simple-way-to-specify-a-pass-a-style-with-syntax-my-style-foo/697043
%
\NewExpandableDocumentCommand{\zxInputMulti}{O{}O{}mm}{%
  \zxExecuteWhenDrawingArrows{%
    \zxArrowNow[
      from=\zxGetNameRelativeNode{0}{0},
      to=\zxGetNameRelativeNodeStartOne{#3}{1},
      /zx/input multi style/.try,
      #1,
      "#4" {/zx/input multi label style, #2},
      ]{}
  }%
}

\pgfkeys{
  /zx/output multi style/.style={
    decorate,
    decoration={brace, raise=2pt},
  },
  /zx/output multi label style/.style={outer sep=5pt,zxNormalFont},
}

% https://tex.stackexchange.com/questions/696958/simple-way-to-specify-a-pass-a-style-with-syntax-my-style-foo/697043
%
\NewExpandableDocumentCommand{\zxOutputMulti}{O{}O{}mm}{%
  \zxExecuteWhenDrawingArrows{%
    \zxArrowNow[
      from=\zxGetNameRelativeNode{0}{0},
      to=\zxGetNameRelativeNodeStartOne{#3}{1},
      /zx/output multi style/.try,
      #1,
      "#4" {/zx/output multi label style, #2},
      ]{}
  }%
}


\pgfkeys{
  /zx/gate input multi style/.style={
    decorate,
    decoration={brace, raise=2pt},
  },
  /zx/gate input multi label style/.style={
    outer sep=4pt,
    zxTinyFontAndSpacing
  },
}


\NewExpandableDocumentCommand{\zxGateInputMulti}{O{}O{}mm}{%
  \zxExecuteWhenDrawingArrows{%
    \zxArrowNow[
      from=\zxGetNameRelativeNode{0}{0}.west,
      to=\zxGetNameRelativeNodeStartOne{#3}{1}.west,
      /zx/gate input multi style/.try,
      #1,
      "#4" {/zx/gate input multi label style, #2},
      ]{}
  }%
}

\pgfkeys{
  /zx/gate output multi style/.style={
    decorate,
    decoration={brace, mirror, raise=2pt},
  },
  /zx/gate output multi label style/.style={
    swap,
    outer sep=4pt,
    zxTinyFontAndSpacing
  },
}

% https://tex.stackexchange.com/questions/696958/simple-way-to-specify-a-pass-a-style-with-syntax-my-style-foo/697043
%
\NewExpandableDocumentCommand{\zxGateOutputMulti}{O{}O{}mm}{%
  \zxExecuteWhenDrawingArrows{%
    \zxArrowNow[
      from=\zxGetNameRelativeNode{0}{0}.east,
      to=\zxGetNameRelativeNodeStartOne{#3}{1}.east,
      /zx/gate output multi style/.try,
      #1,
      "#4" {/zx/gate output multi label style, #2},
      ]{}
  }%
}


\pgfkeys{
  /zx/gate output style/.style={
    align=center,
    inner sep=2pt,
    zxTinyFontAndSpacing,
    anchor=east,
  },
}

\NewExpandableDocumentCommand{\zxGateOutput}{O{}m}{%
  \zxExecuteAtEndPicture{\node[/zx/gate output style,#1] at (\zxGetNameRelativeNode{0}{0}.east){$#2$};}%
}

\pgfkeys{
  /zx/gate input style/.style={
    align=center,
    inner sep=2pt,
    zxTinyFontAndSpacing,
    anchor=west,
  },
}

\NewExpandableDocumentCommand{\zxGateInput}{O{}m}{%
  \zxExecuteAtEndPicture{\node[/zx/gate input style,#1] at (\zxGetNameRelativeNode{0}{0}.west){$#2$};}%
}

% --- Same, but reading bottom to top

\pgfkeys{
  /zx/vertical input style/.style={
    align=center,
    inner sep=2pt,
    zxNormalFont,
    anchor=north,
  },
}

\NewExpandableDocumentCommand{\zxVInput}{O{}m}{%
  \zxExecuteAtEndPicture{\node[/zx/vertical input style,#1] at (\zxGetNameRelativeNode{0}{0}){$#2$};}%
}

\pgfkeys{
  /zx/vertical output style/.style={
    align=center,
    inner sep=2pt,
    zxNormalFont,
    anchor=south,
  },
}

\NewExpandableDocumentCommand{\zxVOutput}{O{}m}{%
  \zxExecuteAtEndPicture{\node[/zx/vertical output style,#1] at (\zxGetNameRelativeNode{0}{0}){$#2$};}%
}

\pgfkeys{
  /zx/vertical input multi style/.style={
    decorate,
    decoration={brace, mirror, raise=2pt},
  },
  /zx/vertical input multi label style/.style={swap,outer sep=5pt,zxNormalFont},
}

% https://tex.stackexchange.com/questions/696958/simple-way-to-specify-a-pass-a-style-with-syntax-my-style-foo/697043
%
\NewExpandableDocumentCommand{\zxVInputMulti}{O{}O{}mm}{%
  \zxExecuteWhenDrawingArrows{%
    \zxArrowNow[
      from=\zxGetNameRelativeNode{0}{0},
      to=\zxGetNameRelativeNodeStartOne{1}{2},
      % to=\zxGetNameRelativeNodeStartOne{1}{#3},
      /zx/vertical input multi style/.try,
      #1,
      "#4" {/zx/vertical input multi label style, #2},
      ]{}
  }%
}

\pgfkeys{
  /zx/vertical output multi style/.style={
    decorate,
    decoration={brace, raise=2pt},
  },
  /zx/vertical output multi label style/.style={outer sep=5pt,zxNormalFont},
}

% https://tex.stackexchange.com/questions/696958/simple-way-to-specify-a-pass-a-style-with-syntax-my-style-foo/697043
%
\NewExpandableDocumentCommand{\zxVOutputMulti}{O{}O{}mm}{%
  \zxExecuteWhenDrawingArrows{%
    \zxArrowNow[
      from=\zxGetNameRelativeNode{0}{0},
      to=\zxGetNameRelativeNodeStartOne{1}{#3},
      /zx/vertical output multi style/.try,
      #1,
      "#4" {/zx/vertical output multi label style, #2},
      ]{}
  }%
}


\pgfkeys{
  /zx/vertical gate input multi style/.style={
    decorate,
    decoration={brace, raise=2pt},
  },
  /zx/vertical gate input multi label style/.style={
    outer sep=4pt,
    zxTinyFontAndSpacing
  },
}


\NewExpandableDocumentCommand{\zxVGateInputMulti}{O{}O{}mm}{%
  \zxExecuteWhenDrawingArrows{%
    \zxArrowNow[
      from=\zxGetNameRelativeNode{0}{0}.south,
      to=\zxGetNameRelativeNodeStartOne{1}{#3}.south,
      /zx/vertical gate input multi style/.try,
      #1,
      "#4" {/zx/vertical gate input multi label style, #2},
      ]{}
  }%
}

\pgfkeys{
  /zx/vertical gate output multi style/.style={
    decorate,
    decoration={brace, mirror, raise=2pt},
  },
  /zx/vertical gate output multi label style/.style={
    swap,
    outer sep=4pt,
    zxTinyFontAndSpacing
  },
}

% https://tex.stackexchange.com/questions/696958/simple-way-to-specify-a-pass-a-style-with-syntax-my-style-foo/697043
%
\NewExpandableDocumentCommand{\zxVGateOutputMulti}{O{}O{}mm}{%
  \zxExecuteWhenDrawingArrows{%
    \zxArrowNow[
      from=\zxGetNameRelativeNode{0}{0}.north,
      to=\zxGetNameRelativeNodeStartOne{1}{#3}.north,
      /zx/vertical gate output multi style/.try,
      #1,
      "#4" {/zx/vertical gate output multi label style, #2},
      ]{}
  }%
}


\pgfkeys{
  /zx/vertical gate output style/.style={
    align=center,
    inner sep=2pt,
    zxTinyFontAndSpacing,
    anchor=north,
  },
}

\NewExpandableDocumentCommand{\zxVGateOutput}{O{}m}{%
  \zxExecuteAtEndPicture{\node[/zx/vertical gate output style,#1] at (\zxGetNameRelativeNode{0}{0}.north){$#2$};}%
}

\pgfkeys{
  /zx/vertical gate input style/.style={
    align=center,
    inner sep=2pt,
    zxTinyFontAndSpacing,
    anchor=south,
  },
}

\NewExpandableDocumentCommand{\zxVGateInput}{O{}m}{%
  \zxExecuteAtEndPicture{\node[/zx/vertical gate input style,#1] at (\zxGetNameRelativeNode{0}{0}.south){$#2$};}%
}

%% If people prefer to read from top:

\let\zxTOutput\zxVInput
\let\zxTInput\zxVOutput
\let\zxTOutputMulti\zxVInputMulti
\let\zxTInputMulti\zxVOutputMulti
\let\zxTGateOutputMulti\zxVGateInputMulti
\let\zxTGateInputMulti\zxVGateOutputMulti
\let\zxTGateInput\zxVGateOutput
\let\zxTGateOutput\zxVGateInput


%%%%%%%%%%%%%%%%%%%%%%%%%%%%%%
%%% 
%%% Now using a special command for fractions (easier to code, and more customizable)
%%%%%%%%%%%%%%%%%%%%%%%%%%%%%%


%%%%% Like \zx but uses \& instead of &. This WAS useful for instance in "align" environments
%%%%% since \zx{} was given an error (even without using &, changing the catcode was enough to
%%%%% break the function). However, in recent versions, \zx should work as it in align, and
%%%%% \begin{ZX}...\end{ZX} seems to always work in align without any issues.
%%%%% Anyway, if at some points you have troubles, either use
%%%%% \begin{ZX}[amp] ... \end{ZX} (amp is shortcut for "ampersand replacement=\&") or this function:
\NewDocumentCommand{\zxAmp}{O{}+m}{%
  \begin{ZX}[ampersand replacement=\&, #1]%
    #2%
  \end{ZX}%
}

%%%%%%%%%%%%%%%%%%%%%%%%%%%%%%%%%%%%%%%%%%%%%%%%%%%%%%%%%%%%%%%%%%%%%%%%%
%%% Practical macros to automatically choose appropriate style and arrows
%%%%%%%%%%%%%%%%%%%%%%%%%%%%%%%%%%%%%%%%%%%%%%%%%%%%%%%%%%%%%%%%%%%%%%%%%
% /!\ Warning: you should add {} at the end of all macros (except arrows)!
% Not using that may work for now, but it may break later...
% TODO: define them only in \zx environment.

% A swap on one line... Practical mostly to gain space. Must be used with large nodes tough...
% \NewExpandableDocumentCommand{\OneLineSwap}{}{%
%   \ar[r,s,start anchor=south,end anchor=north] \ar[r,s,start anchor=north,end anchor=south]
% }


\NewExpandableDocumentCommand{\zxLoop}{O{90}O{20}O{}m}{%
  \ar[loop,in=#1-#2,out=#1+#2,looseness=8,min distance=3mm,#3]
}

\NewExpandableDocumentCommand{\zxLoopAboveDots}{O{20}O{}m}{%
  \ar[loop,in=90-#1,out=90+#1,looseness=8,min distance=3mm,"\cvdotsCenterMathline" {zxNormalFont,scale=.6,anchor=north,yshift=-0.25mm},#2]
}


% Usage: node without any style, but may have space. Default is no space, \zxNone+{} is both horizontal
% and vertical, \zxNone-{} is only horizontal space, \zxNone|{} is only vertical space.
\NewExpandableDocumentCommand{\zxNone}{t+t-t|O{}m}{
  \IfBooleanTF{#1}{% \zxNone+
    \node[zx main node, zxNone+,#4] {#5};%
  }{
    \IfBooleanTF{#2}{% \zxNone-
      \node[zx main node, zxNone-,#4] {#5};%
    }{
      \IfBooleanTF{#3}{% \zxNone
        \node[zx main node, zxNoneI,#4] {#5};%
      }{% \zxNone
        \node[zx main node, zxNone,#4] {#5};%
      }%
    }%
  }%
}

% Usage: alias of \zxNone... To bad token can't be easily forwarded to another function.
\NewExpandableDocumentCommand{\zxN}{t+t-t|O{}m}{
  \IfBooleanTF{#1}{% \zxNone+
    \node[zx main node, zxNone+,#4] {#5};%
  }{
    \IfBooleanTF{#2}{% \zxNone-
      \node[zx main node, zxNone-,#4]{#5};%
    }{
      \IfBooleanTF{#3}{% \zxNone|
        \node[zx main node, zxNoneI,#4]{#5};%
      }{% \zxNone
        \node[zx main node, zxNone,#4]{#5};%
      }%
    }%
  }%
}


% Usage: can be used without {}. Wait, actually it MUST be used without {} which is disturbing… to fix
\def\zxNL{
  \zxN{} \rar \pgfmatrixnextcell[\zxwCol]%
}

% Usage: can be used without {}
\def\zxNR{
  \pgfmatrixnextcell[\zxwCol] \zxN{}  \ar[l]
}



% Cf \zxNone, but with larger space.
\NewExpandableDocumentCommand{\zxNoneDouble}{t+t-t|O{}m}{
  \IfBooleanTF{#1}{% \zxNoneDouble+
    \node[zx main node, zxNoneDouble+,#4]{#5};%
  }{
    \IfBooleanTF{#2}{% \zxNoneDouble-
      \node[zx main node, zxNoneDouble-,#4]{#5};%
    }{
      \IfBooleanTF{#3}{% \zxNoneDouble
        \node[zx main node, zxNoneDoubleI,#4]{#5};%
      }{% \zxNoneDouble
        \node[zx main node, zxNoneDouble,#4]{#5};%
      }
    }
  }
}

%% For maximum styling liberty, the content is given directly to the style.
% It allows the style to put the phase in a label.
\NewExpandableDocumentCommand{\zxZ}{O{}t*t-m}{
  \node[zx main node, zxZ4={#1}{\IfBooleanTF{#3}{-}{}}{\IfBooleanTF{#2}{*}{}}{#4}] {}; %
}

%% For maximum styling liberty, the content is given directly to the style.
%% It allows the style to put the phase in a label.
\NewExpandableDocumentCommand{\zxX}{O{}t*t-m}{
  \node[zx main node, zxX4={#1}{\IfBooleanTF{#3}{-}{}}{\IfBooleanTF{#2}{*}{}}{#4}]{};%
}

\NewExpandableDocumentCommand{\zxH}{O{}m}{
  \node[zx main node, zxH,#1]{};%
}

% Use like: \zxFracX{\pi}{4} for positive values or for negative \zxFracX-{\pi}{4}
\NewExpandableDocumentCommand{\zxFracZ}{O{}t-moom}{%
  \IfNoValueTF{#5}{% 2 arguments like: \zxFracZ{\pi}{2}
    \node[zx main node, zxFracZ6={#1}{\IfBooleanTF{#2}{\zxMinus}{}}{#3}{#6}{#3}{#6}]{};%
  }{% 4 arguments like \zxFracZ{a+b}[(a+b)][(c+d)]{c+d}
    \node[zx main node, zxFracZ6={#1}{\IfBooleanTF{#2}{\zxMinus}{}}{#3}{#6}{#4}{#5}]{};%
  }%
}

% Use like: \zxFracX{\pi}{4} for positive values or for negative \zxFracX-{\pi}{4}
\NewExpandableDocumentCommand{\zxFracX}{O{}t-moom}{%
  \IfNoValueTF{#5}{% 2 arguments like: \zxFracZ{\pi}{2}
    \node[zx main node, zxFracX6={#1}{\IfBooleanTF{#2}{\zxMinus}{}}{#3}{#6}{#3}{#6}]{};%
  }{% 4 arguments like \zxFracZ{a+b}[(a+b)][(c+d)]{c+d}
    \node[zx main node, zxFracX6={#1}{\IfBooleanTF{#2}{\zxMinus}{}}{#3}{#6}{#4}{#5}]{};%
  }%
}

\NewExpandableDocumentCommand{\zxEmptyDiagram}{}{
  \node[zx main node, zxEmptyDiagram]{};%
}


% % Example: \leftManyDots{n}
% Useful to put on the left of a node like "n \vdots", linked to the next node.  Example: \leftManyDots{n}.
% First optional argument is scale of text, second is scale of =.
\NewExpandableDocumentCommand{\leftManyDots}{O{1}O{\zxScaleDots}m}{%
  \node[zx main node, zxNone+,inner xsep=0pt]{\scalebox{#1}{$#3$\,}\makebox[0pt][l]{\scalebox{#2}{$\cvdotsCenterMathline$}}}; \ar[r,-N.,start anchor=north east] \ar[r,-N',start anchor=south east] \pgfmatrixnextcell[\zxwCol]%
}

% Useful to link two nodes and put a vdots in between.
\NewExpandableDocumentCommand{\middleManyDots}{}{%
  \ar[r,3 vdots] \ar[o',r] \ar[o.,r]%
}

% Like \leftManyDots but on the right. Do *not* create a new node, like in |[zxShortZ]| \alpha \rightManyDots{m}
\NewExpandableDocumentCommand{\rightManyDots}{O{1}O{\zxScaleDots}m}{%
  \ar[r,N'-,end anchor=north west] \ar[r,N.-,end anchor=south west] \pgfmatrixnextcell[\zxwCol] \node[zx main node, zxNone+,inner xsep=0pt]{\makebox[0pt][r]{\scalebox{#2}{$\cvdotsCenterMathline$}}\scalebox{#1}{\,$#3$}};
}

% Shortcut for frequent wires
\NewExpandableDocumentCommand{\zxDoubleO}{O{}m}{%
  \ar[r,o.,#1] \ar[r,o',#1]
}

% Version going down
\NewExpandableDocumentCommand{\zxDoubleOD}{O{}m}{%
  \ar[r,o-,#1] \ar[r,-o,#1]
}

\NewExpandableDocumentCommand{\zxTtripleO}{O{}m}{%
  \rar \ar[r,o.,#1] \ar[r,o',#1]
}

\NewExpandableDocumentCommand{\zxTripleOD}{O{}m}{%
  \rar \ar[r,-.,#1] \ar[r,o-,#1]
}


\NewExpandableDocumentCommand{\zxOneOverSqrtTwo}{O{}m}{%
  \zxZ{} \rar[#1] \ar[r,o.,#1] \ar[r,o',#1] \pgfmatrixnextcell \zxX{}
}

\NewExpandableDocumentCommand{\zxSqrtTwo}{O{}m}{%
  \zxZ{} \rar[#1] \pgfmatrixnextcell \zxX{}
}


%% Need AfterEndPreamble otherwise there is an error when using tikzexternalize.
\AfterEndPreamble{
  \zxSaveDiagram{\zxCteOneOverSqrtTwo}{\zxZ{} \rar \ar[r,o.] \ar[r,o'] \pgfmatrixnextcell \zxX{}}
  \zxSaveDiagram{\zxCteSqrtTwo}{\zxZ{} \rar \pgfmatrixnextcell \zxX{}}
}

%% Useful to give name to box
%% \namedBox{(node1)(node2)}{text}
%% \namedBox[additional style][color]{(node1)(node2)}[additional label options]{optional position:text}
%% \begin{ZX}[
%%   execute at end picture={
%%     \zxNamedBox{(measZ)(measZ)}{right:Bell measure}
%%   }
%%   ]
%%   \zxN{} \dar[C] \ar[rrr] & & & \zxN{} \\
%%   \zxN{} \ar[rr] &&\zxX[a=measX]{x\pi} \dar[C-]\\
%%   & \zxX{r\pi} \rar& \zxZ[a=measZ]{z\pi}
%% \end{ZX}

%% Helper
%% https://tex.stackexchange.com/questions/523579/tikz-fit-inner-sep-seperate-values-for-all-4-directions
% \tikzset{fit margins/.style={
%     /tikz/afit/.cd,
%     #1,
%     /tikz/.cd,
%     inner xsep=\pgfkeysvalueof{/tikz/afit/left}+\pgfkeysvalueof{/tikz/afit/right},
%     inner ysep=\pgfkeysvalueof{/tikz/afit/top}+\pgfkeysvalueof{/tikz/afit/bottom},
%     xshift=-\pgfkeysvalueof{/tikz/afit/left}+\pgfkeysvalueof{/tikz/afit/right},
%     yshift=-\pgfkeysvalueof{/tikz/afit/bottom}+\pgfkeysvalueof{/tikz/afit/top},
%     afit/.cd,
%     left/.initial=1pt,right/.initial=1pt,bottom/.initial=1pt,top/.initial=1pt
%   }}

\tikzset{
  fit margins/.style={
    /tikz/afit/.cd,#1,
    /tikz/.cd,
    inner xsep=\pgfkeysvalueof{/tikz/afit/left}+\pgfkeysvalueof{/tikz/afit/right},
    inner ysep=\pgfkeysvalueof{/tikz/afit/top}+\pgfkeysvalueof{/tikz/afit/bottom},
    xshift=-\pgfkeysvalueof{/tikz/afit/left}+\pgfkeysvalueof{/tikz/afit/right},
    yshift=-\pgfkeysvalueof{/tikz/afit/bottom}+\pgfkeysvalueof{/tikz/afit/top}
  },
  afit/.cd,
  left/.initial=1pt,
  right/.initial=1pt,
  bottom/.initial=1pt,
  top/.initial=1pt,
  horizontal/.style={
    left=#1,
    right=#1,
  },
  vertical/.style={
    top=#1,
    bottom=#1,
  },
  all/.style={
    horizontal=#1,
    vertical=#1,
  },
  /tikz/.cd
}

\NewExpandableDocumentCommand{\zxNamedBox}{O{}O{blue}mO{}m}{
  \node[l/.style={fit margins={all=1mm}},inner sep=2pt, node on layer=background, rounded corners, draw=#2,dashed, fill={#2!50!white}, opacity=.5, fit=#3,label={[#4]#5},#1]{};
}

% Cont is for container, equivalent of zxNamedBox but inside the matrix directly
\NewExpandableDocumentCommand{\zxCont}{O{}O{blue}mmO{}O{}m}{%
  \zxExecuteAtEndPicture{%
    \zxNamedBox[zx subnode=wrapper,#1][#2]{(\zxGetNameRelativeNode{0}{0})(\zxGetNameRelativeNode{\numexpr #3-1\relax}{\numexpr #4-1\relax})#6}[#5]{#7}%
  }%
}

% Cont is for container, equivalent of zxNamedBox but inside the matrix directly
\NewExpandableDocumentCommand{\zxGroupGates}{O{}mmO{}O{}m}{%
  \zxCont[fill=none,rounded corners=0pt,solid,opacity=1,zx thickness wires style,draw=black,rectangle,fit margins={all=.5mm},#1]{#2}{#3}[#4][#5]{#6}
}

% Cont is for container, equivalent of zxNamedBox but inside the matrix directly
% this is the version where you specify the name of the node and not the relative pos
\NewExpandableDocumentCommand{\zxContName}{O{}O{blue}mO{}m}{%
  \zxExecuteAtEndPicture{%
    \zxNamedBox[#1][#2]{(\zxGetNameRelativeNode{0}{0})#3}[#4]{#5}%
  }%
}



%%%%%%%%%%%%%%%%%%%%%%%%%%%%%%
%%% Old code that tried to automatically find if zxShort or zxLong should be used...
%%% Now using a special command for fractions (easier to code, and more customizable)
%%%%%%%%%%%%%%%%%%%%%%%%%%%%%%
\newsavebox\zx@box % Temporary box to compute height/width/depth

\newlength{\zxMaxDepthPlusHeight}\setlength{\zxMaxDepthPlusHeight}{2em}
\def\zxMaxRatio{1.3} % Ratio width/(height+depth)

\NewExpandableDocumentCommand{\zxChooseStyle}{mmmm}{%
  % #1=text,#2=empty style,#3=short style,#4=long style
  \savebox\zx@box{#1}%
  % Check if width is 0pt:
  \ifdimcomp{\wd\zx@box}{=}{0pt}{% Return empty style if box is empty
    #2%
  }{% Else compute size of thext
    % Check if height+depth < zxMaxDepthPlusHeight to see if short style applies
    \ifdimcomp{\dimexpr\dp\zx@box+\ht\zx@box\relax}{<}{\zxMaxDepthPlusHeight}{%
      % Check if width < ratio*(height+depth) to see if short style applies
      \ifdimcomp{\wd\zx@box}{<}{\dimexpr \zxMaxRatio\ht\zx@box + \zxMaxRatio\dp\zx@box\relax}{%
        #3% Short style is used
      }{% Else
        #4% Long style is used
      }%
    }{%
      #4% Long style is used
    }%
  }%
}

\makeatother
% Loads the great package that produces tikz-like manual (see also tikzcd for examples)
\input{pgfmanual-en-macros.tex} % Is supposed to be included in recent TeX distributions, but I get errors...
\usepackage{makeidx} % Produces an index of commands.
\makeindex % Useful or not index will be created
\usepackage{alertmessage} % For warning, info...
\usepackage[hidelinks]{hyperref}
\newcommand{\mylink}[2]{\href{#1}{#2}\footnote{\url{#1}}}
\usepackage{verbatim}
\usepackage{cleveref}
\usepackage{mathtools}
\usepackage{listings}

%%%% Bibliography
% I want to have separate references for appendix and main body
% https://tex.stackexchange.com/questions/98660/
\usepackage[style=trad-alpha,
  sortcites=true,
  doi=false,
  url=false,
  giveninits=true, % Bob Foo --> B. Foo
  isbn=false,
  url=false,
  eprint=false,
  sortcites=false, % \cite{B,A,C}: [A,B,C] --> [B,A,C]
]{biblatex}
\renewcommand{\multicitedelim}{, } % [ABC96; DEF12] -> [ABC96, DEF12]
% [unknown_ref] => [??]
% https://tex.stackexchange.com/a/352573
\makeatletter
\protected\def\abx@missing#1{\textbf{??}}
\makeatother
\renewcommand*{\bibfont}{\normalfont\small} % BibLatex font seems bigger than Bibtex?
\addbibresource{biblio.bib}

%%%%%%%%%%%%%%%%%%%%%%%%%%%%%%
%%% Documentation
%%%%%%%%%%%%%%%%%%%%%%%%%%%%%%

\begin{document}
%%% Title: thanks tikzcd for the styling
\begin{center}
  \vspace*{1em} % Thanks tikzcd
  \tikz\node[scale=1.2]{%
    \color{gray}\Huge\ttfamily \char`\{\raisebox{.09em}{\textcolor{red!75!black}{zx\raisebox{-0.1em}{-}calculus}}\char`\}};

  \vspace{0.5em}
  {\Large\bfseries ZX-calculus with \tikzname}

  \vspace{1em}
  {Léo Colisson \quad Version 2023/09/27+unstable}\\[3mm]
  {\href{https://github.com/leo-colisson/zx-calculus}{\texttt{github.com/leo-colisson/zx-calculus}}}
\end{center}

\tableofcontents

\section{Introduction}

This library (based on the great \tikzname{} and \tikzname-cd packages) allows you to typeset ZX-calculus and diagrams for diagrammatic reasoning~\cite{CK17_PicturingQuantumProcesses,van20_ZXcalculusWorkingQuantum} directly in \LaTeX{}. It comes with a default---but highly customizable---style:
\begin{codeexample}[]
  \begin{ZX}
    \zxZ{\alpha} \arrow[r] & \zxFracX-{\pi}{4}
  \end{ZX}
\end{codeexample}
Even if this has not yet been tested a lot, you can also use a ``phase in label'' style, without really changing the code:
\begin{codeexample}[]
  \begin{ZX}[phase in label right]
    \zxZ{\alpha} \arrow[d] \\
    \zxFracX-{\pi}{4}
  \end{ZX}
\end{codeexample}

Since 24/02/2023, we also provide ways to easily create new, highly customizable shapes, with text, anchors, sub-nodes, rotations, and more:
{\catcode`\|=12 % Ensures | is not anymore \verb|...|
\begin{codeexample}[width=0pt]
% Define a reusable node
\zxNewNodeFromPic{MyDivider}[][zx create anchors={\zxVirtualCenterWest, \zxVirtualCenterEast},
  every node/.append style={transform shape}
  ]{
  \node[regular polygon, regular polygon sides=3,shape border rotate=90,%shape border rotate=90,
        draw=black,fill=gray!50, inner sep=1.6pt, rounded corners=0.8mm,zx main node] {};
  \coordinate(\zxVirtualCenterEast) at (.2mm,0); % Used to start lines on the side of the shape
  \coordinate(\zxVirtualCenterWest) at (-1mm,0);
}
% Use the node
\begin{ZX}
                            & [2mm]                           & [3mm] \zxN{} \\[\zxZeroRow]
 \zxZ[B]{} \rar[Bn'=n+m, wc] & \zxMyDivider{}
                              \rar[<',ru,Bn'Args={n}{pos=.7}]
                              \rar[Bn.Args={m}{pos=.7},<.,rd] &\\[\zxZeroRow]
                            &                                 & \zxN{}
\end{ZX}
\end{codeexample}
}

Since 2023-09-27, you can also do advanced operations on multi-gate elements or use our default style:
\begin{codeexample}[]
\begin{ZX}[circuit]
  \rar & \zxGate{H} \rar & \zxGateMulti{2}{3}{H} &  & \rar & \\
  \rar & \zxGate{H} \rar &                       &  & \rar & 
\end{ZX}
\end{codeexample}

and create more traditional circuits:

{\catcode`\|=12 % Ensures | is not anymore \verb|...|
\begin{codeexample}[width=0pt]
\begin{ZX}[circuit]
  \zxElt{\ket{\psi}} \rar & \zxBox{H} \rar         & \zxCtrl{} \dar \rar & \zxCross{} \dar \rar
  & \zxBox[add label={Measure}]{\zxMeter{}} \ar[dr,classical,connect -|] \\
  \ar[r]                  & \zxOCtrl{} \rar \ar[u] & \zxNot{} \rar       & \zxCross{} \ar[rr]
  &  & \zxBox{H} \rar & 
\end{ZX}    
\end{codeexample}
}

The goal is to provide an alternative to the great |tikzit| package: we wanted a solution that does not require the creation of an additional file, the use of an external software, and which automatically adapts the width of columns and rows depending on the content of the nodes (in |tikzit| one needs to manually tune the position of each node, especially when dealing with large nodes). Our library also provides a default style and tries to separate the content from the style: that way it should be easy to globally change the styling of a given project without redesigning all diagrams. However, it should be fairly easy to combine tikzit and this library: when some diagrams are easier to design in tikzit, then it should be possible to directly load the style of this library inside tikzit.

This library is quite young, so feel free to propose improvements or report issues on \href{https://github.com/leo-colisson/zx-calculus/issues}{\texttt{github.com/leo-colisson/zx-calculus/issues}}. We will of course try to maintain backward compatibility as much as possible, but we can't guarantee at 100\% that small changes (spacing, wire looks\dots{}) won't be made later. In case you want a completely unalterable style, just copy the two files of this library in your project forever (see installation)!

\section{Installation}

If your CTAN distribution is recent enough, you can directly insert in your file:
% verse indents stuff, index adds to the index of command at the end of the file, || is a shortcut of \verb||
\begin{verse}
  \index{zx@\protect\texttt{zx-calculus} package}%
  \index{Packages and files!zx-calculus@\protect\texttt{zx-calculus}}%
  |\usepackage{zx-calculus}|%
\end{verse}
or load \tikzname{} and then use:
\begin{verse}%
   \index{cd@\protect\texttt{zx-calculus} library}%
   \index{Libraries!cd@\protect\texttt{zx-calculus}}%
   |\usetikzlibrary{zx-calculus}|%
\end{verse}
If this library is not yet packaged into CTAN (which is very likely in 2021), you must first download \mylink{https://github.com/leo-colisson/zx-calculus/blob/main/tikzlibraryzx-calculus.code.tex}{\texttt{tikzlibraryzx-calculus.code.tex}} and \mylink{https://github.com/leo-colisson/zx-calculus/blob/main/zx-calculus.sty}{\texttt{zx-calculus.sty}} (right-click on ``Raw'' and ``Save link as'') and save them at the root of your project.

\section{Quickstart}

You can create a diagram either with |\zx[options]{matrix}|, |\zxAmp[options]{matrix}| or with:
\begin{verse}
  |\begin{ZX}[options]|\\
    |  matrix|\\
  |\end{ZX}|
\end{verse}
The matrix is composed of rows separated by |\\| and columns separated by |&| (except in |\zxAmp| where columns are separated with |\&|).

% \alertwarning{ % Can't use |...| or \verb|...| in this environment :'(
%   Due to \LaTeX{} restrictions, \texttt{\&} can sometimes cause some troubles. \texttt{\textbackslash{}zxAmp} is always guaranteed to work (just make sure to use \texttt{\textbackslash{}\&} instead of \texttt{\&}). See \cref{subsec:addDiagram} for more details.
% }

This matrix is basically a \tikzname{} matrix of nodes (even better, a |tikz-cd| matrix, so you can use all the machinary of |tikz-cd|), so cells can be created using \verb#|[tikz style]| content#. However, the users does not usually need to use this syntax since many nodes like |\zxZ{spider phase}| have been created for them (including |\zxN{}| which is an empty node):

\begin{codeexample}[width=0pt]
\begin{ZX}
  \zxZ{} & \zxZ{\alpha} & \zxZ-{\alpha} & \zxZ{\alpha+\beta} & \zxFracZ{\pi}{2} & \zxFracZ-{\pi}{2}\\
  \zxX{} & \zxX{\alpha} & \zxX-{\alpha} & \zxX{\alpha+\beta} & \zxFracX{\pi}{2} & \zxFracX-{\pi}{2}\\
  \zxN{} & \zxH{}
\end{ZX}
\end{codeexample}

Note that if a node has no argument like |\zxN|, you should still end it like |\zxN{}| to make sure you code will be backward compatible and will behave correctly.

To link the nodes, you should use |\arrow[options]| (|\ar[options]| for short) at the end of a cell (you can put many arrows). The options can contain a direction, made of a string of |r| (for ``right''), |l| (for ``left''), |d| (for `down''), |u| (for ``up'') letters. That way, |\ar[rrd]| would be an arrow going right, right, and down:
\begin{codeexample}[]
\begin{ZX}
  \zxZ{} \ar[rrd] & \zxX{}\\
                  &        & \zxX-{\alpha}
\end{ZX}
\end{codeexample}
See how the alignment of your matrix helps reading it: in emacs |M-x align| is your friend (or even better, if you are tired of selecting the lines to align, bind \mylink{https://tex.stackexchange.com/a/64566/116348}{this \texttt{align-environment} function} to some shortcuts, like |C-<tab>| and you will just have to do a single key-press to align your matrix).

You may also encounter some shortcuts, like |\rar| instead of |\ar[r]|. Since straight lines are boring, we created many styles that you can just add in the options. For instance, a measured Bell-pair can be created using the |C| style (note also how the |*| argument forces the node to be tighter), as the name of the style tries to mimic the shape of the wire:
\begin{codeexample}[]
\begin{ZX}
  \zxZ*{a \pi} \ar[d,C]\\
  \zxZ*{b \pi}
\end{ZX}
\end{codeexample}

\alertinfo{
  \textbf{Tips}: on small diagrams, describing the arrow position using relative position like \texttt{rrb} works really nicely, but on big diagrams that you need to rewrite, it can quickly be hard to manage, as moving a node will break all links to it. While it is possible to specify both |from| and |to| in an absolute way, the experience shows that it is easier to work with a ``semi-relative'' addressing, where only the |to| is specified in an absolute setting, while the |from| is automatically derived from the current position of the arrow. You can use \texttt{a=somename} to give a name for a node (\texttt{a} being the short name of alias).}

This way, on non-trivial diagrams, we recommend to format graphs like:
\begin{codeexample}[]
\begin{ZX}
  \zxN{}                                      & \zxZ[a=beta]{\beta} \\
  \zxZ{\alpha} \ar[to=beta,N] \ar[to=gamma,N] &                     \\
                                              & \zxZ[a=gamma]{\gamma}
\end{ZX}
\end{codeexample}
Note that you can also set |debug mode| to display the name of the nodes for an easier addressing:
\begin{codeexample}[]
\begin{ZX}[debug mode]
  % See how the code stayed unchanged despite changing the position of the nodes
  \zxN{}                                      & \zxZ[a=beta]{\beta} \\
  \zxZ{\alpha} \ar[to=beta,N] \ar[to=gamma,N] &                     \\
                                              &                     \\
                                              &  & \zxZ[a=gamma]{\gamma}
\end{ZX}
\end{codeexample}

We also introduce many other styles, like |N| for wires that arrive and leave at wide angle (yeah, the |N| is the best letter I could find to fit that shape):
\begin{codeexample}[]
\begin{ZX}
  \zxN{}                           & \zxZ{\beta}\\
  \zxZ{\alpha} \ar[ru,N] \ar[rd,N] &\\
                                   & \zxZ{\gamma}
\end{ZX}
\end{codeexample}
Or |s| for wires that arrive and leave at sharp angles\footnote{Note: on older versions, a |\textbackslash zxN\{\}| might be needed on the first cell as it seems that the first cell of a matrix can't be empty… this should be fixed in latest versions.}:
\begin{codeexample}[]
\begin{ZX}
                                   & \zxZ{\beta} \\
  \zxZ{\alpha} \ar[ru,s] \ar[rd,s] &             \\
                                   & \zxZ{\gamma}
\end{ZX}
\end{codeexample}
You have then different variations of a style depending on the shape and/or direction of it. For instance, if we want the arrival of the |N| wire to be flat, use |N-|:
\begin{codeexample}[]
\begin{ZX}
  \zxZ{\alpha} \ar[rd,N-] \\
                         & \zxZ{\beta}
\end{ZX}
\end{codeexample}
Similarly |o'| is a style for wires that have the shape of the top part of the circle, and comes with variations depending on the part of the circle that must be kept:
\begin{codeexample}[width=0pt]
\begin{ZX}
  \zxZ{\alpha} \ar[r,o',green] \ar[r,o.,red] \ar[d,o-,blue] \ar[d,-o,purple] & \zxZ{\beta}\\
  \zxZ{\beta}
\end{ZX}
\end{codeexample}
Note that the position of the embellishments (|'|, |-|, |.|\dots{}) tries to graphically represent the shape of the node. That way |-o| means ``take the left part (position of |-|) of the circle |o|''. Applied to |C|, this gives:
\begin{codeexample}[]
\begin{ZX}
  \zxX{} \ar[d,C] \ar[r,C']  & \zxZ{} \ar[d,C-]\\
  \zxZ{} \ar[r,C.]           & \zxX{}
\end{ZX}
\end{codeexample}

You also have styles which automatically add another node in between, for instance |H| adds a Hadamard node in the middle of the node:
\begin{codeexample}[]
\begin{ZX}
  \zxZ{\alpha} \ar[r,o',H] \ar[r,o.,H] &[\zxHCol] \zxZ{\beta}
\end{ZX}
\end{codeexample}
Note that we used also |&[\zxHCol]| instead of |&| (on the first line). This is useful to add an extra space between the columns to have a nicer look. The same applies for rows (see the |*Row| instead of |*Col|):
\begin{codeexample}[]
\begin{ZX}
  \zxZ{\alpha} \ar[d,-o,Z] \ar[d,o-,X] \\[\zxSRow]
  \zxX{\beta}
\end{ZX}
\end{codeexample}
The reason for this is that it is hard to always get exactly the good spacing by default (for instance here \tikzname{} has no idea that a |H| node will be inserted when it starts to build the diagram), and sometimes the spacing needs some adjustments. However, while you could manually tweak this space using something like |&[1mm]| (it adds |1mm| to the column space), it is better to use some pre-configured spaced that can be (re)-configured document-wise to keep a uniform spacing. You could define your own spacing, but we already provide a list for the most important spacings. They all start with |zx|, then you find the type of space: |H| for Hadamard, |S| for spiders, |W| when you connect only |\zxNone| nodes (otherwise the diagram will be too shrinked), |w| when one side of the row contains only |\zxNone|\dots{} and then you find |Col| (for columns spacing) or |Row| (for rows spacing). For instance we can use the |\zxNone| style (|\zxN| for short) style and the above spacing to obtain this:
\begin{codeexample}[]
\begin{ZX}
  \zxN{} \rar &[\zxwCol] \zxH{} \rar &[\zxwCol] \zxN{}
\end{ZX}
\end{codeexample}
\noindent or that:
\begin{codeexample}[]
\begin{ZX}
  \zxN{} \ar[d,C] \ar[dr,s] &[\zxWCol] \zxN{} \\[\zxWRow]
  \zxN{} \ar[ru,s]          &          \zxN{} \\
\end{ZX}
\end{codeexample}


When writing equations, you may also want to change the baseline to align properly your diagrams on a given line like that (since march 2023):
\begin{codeexample}[]
  $\zx[mbr=2]{ % mbr is a shortcut for "math baseline row"
    \zxX{}\\
    \zxZ[a=myZ]{}
  }
  = \zx{\zxX{} & \zxZ{}}$
\end{codeexample}
You can also specify (this works on older versions) a specific node like that (|a=blabla| gives the alias name |blabla| to the node, and configure tools useful for debugging):
\begin{codeexample}[]
  $\zx[math baseline=myZ]{
    \zxX{}\\
    \zxZ[a=myZ]{}
  }
  = \zx{\zxX{} & \zxZ{}}$
\end{codeexample}

We also provide easy methods like |phase in label right| to change the labelling of a note (per-node, per-picture or document wise) to move the phase in a label automatically:
\begin{codeexample}[]
  \begin{ZX}[phase in label right]
    \zxZ{\alpha} \arrow[d] \\
    \zxFracX-{\pi}{4}
  \end{ZX}
\end{codeexample}

Now you should know enough to start your first diagrams. The rest of the documentation will go through all the styles, customizations and features, including the one needed to obtain:
\begin{codeexample}[width=3cm]
\begin{ZX}
  \leftManyDots{n} \zxX{\alpha} \zxLoopAboveDots{} \middleManyDots{} \ar[r,o'={a=75}]
      & \zxX{\beta} \zxLoopAboveDots{} \rightManyDots{m}
\end{ZX}
\end{codeexample}
\noindent You will also see some tricks (notably based on alias) to create clear bigger diagrams, like this debug mode which turns
{
\begin{ZX}[zx row sep=1pt,
  execute at begin picture={%
    %%% Definition of long items (the goal is to have a small and readable matrix
    % (warning: macro can't have numbers in TeX. Also, make sure not to use existing names)
    \def\Zpifour{\zxFracZ[a=Zpi4]-{\pi}{4}}%
    \def\mypitwo{\zxFracX[a=mypi2]{\pi}{2}}%
  }
  ]
  %%% Matrix: in emacs "M-x align" is practical to automatically format it. a is for 'alias'
  & \zxN[a=n]{}  & \zxZ[a=xmiddle]{}       &            & \zxN[a=out1]{} \\
  \zxN[a=in1]{} & \Zpifour{}   & \zxX[a=Xdown]{}         & \mypitwo{} &                \\
  &              &                         &            & \zxN[a=out2]{} \\
  \zxN[a=in2]{} & \zxX[a=X1]{} & \zxZ[a=toprightpi]{\pi} &            & \zxN[a=out3]{}
  %%% Arrows
  % Column 1
  \ar[from=in1,to=X1,s]
  \ar[from=in2,to=Zpi4,.>]
  % Column 2
  \ar[from=X1,to=xmiddle,N']
  \ar[from=X1,to=toprightpi,H]
  \ar[from=Zpi4,to=n,C] \ar[from=n,to=xmiddle,wc]
  \ar[from=Zpi4,to=Xdown]
  % Column 3
  \ar[from=xmiddle,to=Xdown,C-]
  \ar[from=xmiddle,to=mypi2,)]
  % Column 4
  \ar[from=mypi2,to=toprightpi,(']
  \ar[from=mypi2,to=out1,<']
  \ar[from=mypi2,to=out2,<.]
  \ar[from=Xdown,to=out3,<.]
\end{ZX} into %
{%
  \def\zxDebugMode{}%%%%
  \begin{ZX}[zx row sep=1pt,
    execute at begin picture={%
      %%% Definition of long items (the goal is to have a small and readable matrix
      % (warning: macro can't have numbers in TeX. Also, make sure not to use existing names)
      \def\Zpifour{\zxFracZ[a=Zpi4]-{\pi}{4}}%
      \def\mypitwo{\zxFracX[a=mypi2]{\pi}{2}}%
    }
    ]
    %%% Matrix: in emacs "M-x align" is practical to automatically format it. a is for 'alias'
    & \zxN[a=n]{}  & \zxZ[a=xmiddle]{}       &            & \zxN[a=out1]{} \\
    \zxN[a=in1]{} & \Zpifour{}   & \zxX[a=Xdown]{}         & \mypitwo{} &                \\
    &              &                         &            & \zxN[a=out2]{} \\
    \zxN[a=in2]{} & \zxX[a=X1]{} & \zxZ[a=toprightpi]{\pi} &            & \zxN[a=out3]{}
    %%% Arrows
    % Column 1
    \ar[from=in1,to=X1,s]
    \ar[from=in2,to=Zpi4,.>]
    % Column 2
    \ar[from=X1,to=xmiddle,N']
    \ar[from=X1,to=toprightpi,H]
    \ar[from=Zpi4,to=n,C] \ar[from=n,to=xmiddle,wc]
    \ar[from=Zpi4,to=Xdown]
    % Column 3
    \ar[from=xmiddle,to=Xdown,C-]
    \ar[from=xmiddle,to=mypi2,)]
    % Column 4
    \ar[from=mypi2,to=toprightpi,(']
    \ar[from=mypi2,to=out1,<']
    \ar[from=mypi2,to=out2,<.]
    \ar[from=Xdown,to=out3,<.]
  \end{ZX}
} \ (of course it only helps during the construction).\\

You will also see how you can customize the styles, and how you can easily extend this library to get any custom diagram:
{\catcode`\|=12 % Ensures | is not anymore \verb|...|
\begin{codeexample}[width=0pt]
{ % \usetikzlibrary{shadows}
  \tikzset{
    my bloc/.style={
      anchor=center,
      inner sep=2pt,
      inner xsep=.7em,
      minimum height=3em,
      draw,
      thick,
      fill=blue!10!white,
      double copy shadow={opacity=.5},tape,
    }
  }
  \zx{|[my bloc]| f \rar &[1mm] |[my bloc]| g \rar &[1mm] \zxZ{\alpha} \rar & \zxNone{}}
}
\end{codeexample}
}

If you have some questions, suggestions, or bugs, please report them on \texttt{\url{https://github.com/leo-colisson/zx-calculus/issues}}.

\textbf{Tips}: if you are unsure of the definition of a style in an example, just click on it, a link will point to its definition. Also, if your pdf viewer does not copy/paste these examples correctly, you can copy them from the source code of this documentation available \mylink{https://github.com/leo-colisson/zx-calculus/blob/main/doc/zx-calculus.tex}{here} (to find the example, just use the ``search'' function of your web browser).

\section{Usage}

\subsection{Add a diagram}\label{subsec:addDiagram}
\begin{pgfmanualentry}
  \extractcommand\zx\opt{\oarg{options}}\marg{your diagram}\@@
  \extractenvironement{ZX}\opt{\oarg{options}}\@@
  \extractcommand\zxAmp\opt{\oarg{options}}\marg{your diagram}\@@
  \pgfmanualbody
  You can create a new ZX-diagram either with a macro (quicker for inline diagrams) or with an environment. All these commands are mostly equivalent, except that in |\zxAmp| columns are separated with |\&| instead of |&| (this was useful before as |&| was not usable in |align| or inside macros. Now it should be fixed.). The \meta{options} can be used to locally change the style of the diagram, using the same options as the |{tikz-cd}| environment (from the \mylink{https://www.ctan.org/pkg/tikz-cd}{\texttt{tikz-cd} package}). The \meta{your diagram} argument, or the content of |{ZX}| environment is a \tikzname{} matrix of nodes, exactly like in the |tikz-cd| package: rows are separated using |\\|, columns using |&| (except for |\zxAmp| where columns are separated using |\&|), and nodes are created using \verb#|[tikz style]| node content# or with shortcut commands presented later in this document (recommended). Wires can be added like in |tikz-cd| (see more below) using |\arrow| or |\ar|: we provide later recommended styles to quickly create different kinds of wires which can change with the configured style. Content is typeset in math mode by default, and diagrams can be included in any equation.
{\catcode`\|=12 % Ensures | is not anymore \verb|...|
% Do not indent not to put space in final code
\begin{codeexample}[]
Spider \zx{\zxZ{\alpha}}, equation $\zx{\zxZ{}} = \zx{\zxX{}}$ %
and custom diagram: %
\begin{ZX}[red]
  \zxZ{\beta} \arrow[r]                           & \zxZ{\alpha} \\
  |[fill=pink,draw]| \gamma \arrow[ru,bend right]
\end{ZX}
\end{codeexample}
}
\end{pgfmanualentry}

\begin{stylekey}{/zx/defaultEnv/amp}
  In a previous version (before 2022/02/09), it was not possible to use |&| inside macros and |align| due to \LaTeX{} limitations. However, we found a solution by re-scanning the tokens, so now no special care should be taken in align or macros. But in case you need to deal with an environment having troubles with |&|, either use the |ampersand replacement=\&| option (whose shortcut is |amp|) or |\zxAmp| (in any case, replace |&| with |\&|).
\begin{codeexample}[vbox]
An aligned equation:
\begin{align}
  \zxAmp{\zxZ{} \arrow[r] \& \zxX{}} &= \begin{ZX}[amp] \zxX{} \arrow[r] \& \zxZ{} \end{ZX}
\end{align}
This limitation does not apply anymore:
\begin{align}
  \zx{\zxZ{} \arrow[r] & \zxX{}} &= \begin{ZX} \zxX{} \arrow[r] & \zxZ{} \end{ZX}
\end{align}
even in macros: {\setlength{\fboxsep}{0pt} \fbox{\zx{\zxZ{} \rar & \zxX{}}}}
\end{codeexample}
\end{stylekey}

\subsection{Nodes}
\subsubsection{Spiders}
The following commands are useful to create different kinds of nodes. Always add empty arguments like |\example{}| if none are already present, otherwise if you type |\example| we don't guarantee backward compatibility.

\begin{command}{\zxEmptyDiagram{}}
  Create an empty diagram.
\begin{codeexample}[width=3cm]
\begin{ZX}
  \zxEmptyDiagram{}
\end{ZX}
\end{codeexample}
\end{command}


\begin{pgfmanualentry}
  \extractcommand\zxNone\opt{-\textbar+}\marg{text}\@@
  \extractcommand\zxN\opt{-\textbar+}\marg{text}\@@
  \extractcommand\zxNL\@@
  \extractcommand\zxNR\@@
  \pgfmanualbody
  Adds an empty node with |\zxNone{}| (alias |\zxN{}|). The \verb#-|+# decorations are used to add a bit of horizontal (\verb#\zxNone-{}#), vertical (\verb#\zxNone|{}#) and both (\verb#\zxNone+{}#) spacing.

  |\zxNone| is just a coordinate (and therefore can't have any text inside, but when possible this node should be preferred over the other versions since it has really zero width), but |\zxNone-{}| and \verb#\zxNone|{}# are actually nodes with |inner sep=0| along one direction. For that reason, they still have a tiny height or width (impossible to remove as far as I know). If you don't want to get holes when connecting multiple wires to them, it is therefore necessary to use |\zxNone{}| or the |wire centered| style (alias |wc|) (if you are using the |IO| mode, see also the |between none| style). But anyway you should mostly use |\zxNone|.

  Moreover, you should also add column and row spacing |&[\zxWCol]| and |\\[\zxWRow]| to avoid too shrinked diagrams when only wires are involved.
\begin{codeexample}[width=3cm]
\begin{ZX}
  \zxNone{} \ar[C,d] \ar[rd,s] &[\zxWCol] \zxNone{}\\[\zxWRow]
  \zxNone{}          \ar[ru,s] &          \zxNone{}
\end{ZX}
\end{codeexample}
Use |&[\zxwCol]| (on the first line) and/or |\\[\zxwRow]| when a single None node is connected to the wire to add appropriate spacing (this spacing can of course be redefined to your preferences):
\begin{codeexample}[]
Compare \begin{ZX}
  \zxN{} \rar & \zxZ{} \rar & \zxN{}
\end{ZX} with \begin{ZX}
  \zxN{} \rar &[\zxwCol] \zxZ{} \rar &[\zxwCol] \zxN{}
\end{ZX}
\end{codeexample}
This kind of code is so common that there is an alias for it: |\zxNL| and |\zxNR| automatically add a |\zxN{}| node, configure the column space (for this reason don't add an additional |&|, and be aware that emacs won't align them properly. Note also that the space will only be taken into account if it is on the first line) and add a straight arrow. The |L/R| part of the name is just to specify if the node is on the right or left of the diagram to put the column and arrow on the right side:
\begin{codeexample}[]
\zx{\zxNL \zxX{} \zxNR}
\end{codeexample}
Note that these two alias can be used without |{}|. But they are the only ones.

The \verb!\zxN|{text}! and \verb!\zxN-{text}! may be useful to display some texts:
{\catcode`\|=12 % Ensures | is not anymore \verb|...|
\begin{codeexample}[]
  \begin{ZX}[content fixed baseline]
    & \zxN|{\dots} \dar                                                                 \\
    \zxZ{\theta_i} \rar & \zxZ{} \dar \rar & \zxZ{-(\delta_i+\theta_i+r\pi)}\rar & \zxH{} \rar & \zxX{a\pi} \\
    & \zxN|{\dots}                                                                      \\
  \end{ZX}
\end{codeexample}
}

When the top left cell is empty, you may get an error at the compilation |Single ampersand used with wrong catcode| (this error should be fixed in latest releases) or |<symbol> allowed only in math mode| (not sure why). To solve this issue, you can add an empty node on the very first cell:
\begin{codeexample}[]
\begin{ZX}
  \zxN{}         &[\zxwCol] \zxN{} \ar[d]\\[\zxwRow]
  \zxNone{} \rar & \zxZ{}
\end{ZX}
\end{codeexample}
\end{pgfmanualentry}

You may also get the error |Single ampersand used with wrong catcode| when |&| has already a different meaning, for instance in |align|, in that case you may change the |&| character into |\&| using |[ampersand replacement=\&]|. Note however that in recent versions ($\geq$ 2022/02/09) this should not happen anymore.
\begin{codeexample}[vbox]
\begin{align}
  \begin{ZX}[ampersand replacement=\&]
    \zxN{} \rar \&[\zxWCol] \zxN{}
  \end{ZX}
  &= \begin{ZX}[ampersand replacement=\&]
    \zxN{} \rar \&[\zxwCol] \zxZ{} \rar \&[\zxwCol] \zxN{}
  \end{ZX}\\
  &= \begin{ZX}[ampersand replacement=\&]
    \zxN{} \rar \&[\zxwCol] \zxX{} \rar \&[\zxwCol] \zxN{}
  \end{ZX}
\end{align}
\end{codeexample}

\begin{command}{\zxNoneDouble\opt{-\textbar+}\marg{text}}
  Like |\zxNone|, but the spacing for \verb#-|+# is large enough to fake two lines in only one. Not extremely useful (or one needs to play with |start anchor=south,end anchor=north|).
{\catcode`\|=12 % Ensures | is not anymore \verb|...|
\begin{codeexample}[width=3cm]
\begin{ZX}
  \zxNoneDouble|{} \ar[r,s,start anchor=north,end anchor=south] \ar[r,s,start anchor=south,end anchor=north] &[\zxWCol] \zxNoneDouble|{}
\end{ZX}
\end{codeexample}
}
\end{command}

\begin{command}{\zxFracZ\opt{-}\marg{numerator}\opt{\oarg{numerator with parens}\oarg{denominator with parens}}\marg{denominator}}
  Adds a Z node with a fraction, use the minus decorator to add a small minus in front (a normal minus would be too big, but you can configure the symbol).
\begin{codeexample}[width=3cm]
\begin{ZX}
  \zxFracZ{\pi}{2} & \zxFracZ-{\pi}{2}
\end{ZX}
\end{codeexample}
The optional arguments are useful when the numerator or the denominator need parens when they are written inline (in that case optional arguments must be specified): it will prove useful when using a style that writes the fraction inline, for instance the default style for labels:
\begin{codeexample}[]
Compare %
\begin{ZX}
  \zxFracZ{a+b}[(a+b)][(c+d)]{c+d}
\end{ZX} with %
\begin{ZX}[phase in label right]
  \zxFracZ{a+b}[(a+b)][(c+d)]{c+d}
\end{ZX}
\end{codeexample}
\end{command}

\begin{command}{\zxFracX\opt{-}\marg{numerator}\marg{denominator}}
  Adds an X node with a fraction.
\begin{codeexample}[width=3cm]
\begin{ZX}
  \zxFracX{\pi}{2} & \zxFracX-{\pi}{2}
\end{ZX}
\end{codeexample}
\end{command}

\begin{command}{\zxZ\opt{\oarg{other styles}*-}\marg{text}}
  Adds a Z node. \meta{other styles} are optional \tikzname{} arguments (the same as the one provided to |tikz-cd|) They should be use with care, and if possible moved to the style directly to keep a consistent look across the paper.
\begin{codeexample}[width=3cm]
\begin{ZX}
  \zxZ{} & \zxZ{\alpha} & \zxZ{\alpha + \beta} & \zxZ[fill=blue!50!white,text=red]{(a \oplus b)\pi}
\end{ZX}
\end{codeexample}
The optional |-| optional argument is to add a minus sign (customizable, see |\zxMinusInShort|) in front of a very short expression and try to keep a circular shape. This is recommended notably for single letter expressions.
\begin{codeexample}[width=3cm]
  Compare \zx{\zxZ{-\alpha}} with \zx{\zxZ-{\alpha}}. Labels:
  \zx[pila]{\zxZ{-\alpha}} vs \zx[pila]{\zxZ-{\alpha}}.
\end{codeexample}
The |*| optional argument is to force a condensed style, no matter what is the text inside. This can be practical \emph{sometimes}:
\begin{codeexample}[width=3cm]
  Compare \zx{\zxN{} \rar &[\zxwCol] \zxZ{a\pi}} with \zx{\zxN{} \rar &[\zxwCol] \zxZ*{a\pi}}.
\end{codeexample}
\noindent but you should use it as rarely as possible (otherwise, change the style directly). See that it does not always give nice results:
\begin{codeexample}[width=3cm]
  Compare \zx{\zxZ{-\alpha} \rar & \zxZ{\alpha+\beta}}
  with \zx{\zxZ*{-\alpha} \rar & \zxZ*{\alpha+\beta}}.
  Labels:
  \zx[pila]{\zxZ{-\alpha} \rar & \zxZ{\alpha+\beta}}
  vs \zx[pila]{\zxZ*{-\alpha} \rar & \zxZ*{\alpha+\beta}}.
\end{codeexample}
\end{command}


\begin{command}{\zxX\opt{\oarg{other styles}*-}\marg{text}}
  Adds an X node, like for the Z node.
\begin{codeexample}[width=3cm]
\begin{ZX}
  \zxX{} & \zxX{\alpha} & \zxX-{\alpha} & \zxX{\alpha + \beta}
  & \zxX[text=green]{(a \oplus b)\pi}
\end{ZX}
\end{codeexample}
\end{command}

\begin{command}{\zxH\opt{\oarg{other styles}}}
  Adds an Hadamard node. See also |H| wire style.
\begin{codeexample}[width=3cm]
\begin{ZX}
  \zxNone{} \rar & \zxH{} \rar & \zxNone{}
\end{ZX}
\end{codeexample}
\end{command}



\begin{command}{\leftManyDots\opt{\oarg{text scale}\oarg{dots scale}}\marg{text}}
  Shortcut to add a dots and a text next to it. It automatically adds the new column, see more examples below. Internally, it uses |3 dots| to place the dots, and can be reproduced using the other nodes around. Note that this node automatically adds a new cell, so you should \emph{not} use |&|.
\begin{codeexample}[]
\begin{ZX}
  \leftManyDots{n} \zxX{\alpha}
\end{ZX}
\end{codeexample}
\end{command}

\begin{command}{\leftManyDots\opt{\oarg{text scale}\oarg{dots scale}}\marg{text}}
  Shortcut to add a dots and a text next to it. It automatically adds the new column, see more examples below.
\begin{codeexample}[width=3cm]
\begin{ZX}
  \zxX{\alpha} \rightManyDots{m}
\end{ZX}
\end{codeexample}
\end{command}

\begin{command}{\middleManyDots{}}
  Shortcut to add a dots and a text next to it, see more examples below.
\begin{codeexample}[width=3cm]
\begin{ZX}
  \zxX{\alpha} \middleManyDots{} & \zxX{\beta}
\end{ZX}
\end{codeexample}
\end{command}

\begin{command}{\zxLoop\opt{\oarg{direction angle}\oarg{opening angle}\oarg{other styles}}}
  Adds a loop in \meta{direction angle} (defaults to $90$), with opening angle \meta{opening angle} (defaults to $20$).
\begin{codeexample}[width=3cm]
\begin{ZX}
  \zxX{\alpha} \zxLoop{} & \zxX{} \zxLoop[45]{} & \zxX{} \zxLoop[0][30][red]{}
\end{ZX}
\end{codeexample}
\end{command}

\begin{command}{\zxLoopAboveDots\opt{\oarg{opening angle}\oarg{other styles}}}
  Adds a loop above the node with some dots.
\begin{codeexample}[width=3cm]
\begin{ZX}
  \zxX{\alpha} \zxLoopAboveDots{}
\end{ZX}
\end{codeexample}
\end{command}

\noindent The previous commands can be useful to create this figure:
\begin{codeexample}[width=0pt]% Forces code/example on two lines.
\begin{ZX}
  \leftManyDots{n} \zxX{\alpha} \zxLoopAboveDots{} \middleManyDots{} \ar[r,o'={a=75}]
      & \zxX{\beta} \zxLoopAboveDots{} \rightManyDots{m}
\end{ZX}
\end{codeexample}


\begin{pgfmanualentry}
  \makeatletter
  \def\extrakeytext{style, }
  \extractkey/zx/styles/rounded style/content vertically centered\@nil%
  \extractkey/zx/styles/rounded style/content fixed baseline\@nil%
  \extractkey/zx/styles/rounded style preload/content vertically centered\@nil%
  \extractkey/zx/styles/rounded style preload/content fixed baseline\@nil%
  \extractkey/zx/styles/rounded style preload/content fixed also frac\@nil%
  \makeatother
  \pgfmanualbody
  By default the content of the nodes are vertically centered. This can be nice to have as much space as possible around the text, but when using several nodes with letters having different height or depth, the baseline of each node won't be aligned (this is particularly visible on nodes with very high text, like |b'|):
\begin{codeexample}[width=0pt]
\begin{ZX}
  \zxX[a=start]{\alpha} & \zxX{\beta} & \zxX{a} & \zxX{b} & \zxX*{a\pi} & \zxX*{b\pi}
  & \zxX*{b'\pi} & \zxZ*{'b\pi} & \zxZ{(a \oplus b )\pi} & \zxFracX-{\pi}{4}
  & \zxFracZ{\pi}{4} & \zxZ-{\delta} & \zxZ[a=end]-{\gamma}
   \ar[from=start,to=end,on layer=background]
\end{ZX}
\end{codeexample}
Using |content fixed baseline|, it is however possible to fix the height and depth of the text to make sure the baselines are aligned (|content vertically centered| is use to come back to the default behavior). When used as a ZX option, it avoids setting this on fractions since it renders poorly. Use |content fixed baseline also frac| if you also want to fix the baseline of all fractions as well (this last style is useful only as a ZX option, since |content fixed baseline| works on all nodes).
\begin{codeexample}[width=0pt]
\begin{ZX}[content fixed baseline]
  \zxX[a=start]{\alpha} & \zxX{\beta} & \zxX{a} & \zxX{b} & \zxX*{a\pi} & \zxX*{b\pi}
  & \zxX*{b'\pi} & \zxZ*{'b\pi} & \zxZ{(a \oplus b )\pi} & \zxFracX-{\pi}{4}
  & \zxFracZ{\pi}{4} & \zxZ-{\delta} & \zxZ[a=end]-{\gamma}
   \ar[from=start,to=end,on layer=background]
\end{ZX}
\end{codeexample}
Note however that the height is really hardcoded (not sure how to avoid that) and is quite small (otherwise nodes quickly become too large), so too large content may overlay on top of the node (this is visible on the |'b\pi| node). You can use this style either on a per-picture basis (it's what we just did), on a per-node basis (just use it in the options of the node), or globally:
\begin{codeexample}[width=0pt]
\tikzset{
  /zx/user overlay/.style={
    content fixed baseline,
  },
}
\begin{ZX}
  \zxX[a=start]{\alpha} & \zxX{\beta} & \zxX{a} & \zxX{b} & \zxX*{a\pi} & \zxX*{b\pi}
  & \zxX*{b'\pi} & \zxZ*{'b\pi} & \zxZ{(a \oplus b )\pi} & \zxFracX-{\pi}{4}
  & \zxFracZ{\pi}{4} & \zxZ-{\delta} & \zxZ[a=end]-{\gamma}
   \ar[from=start,to=end,on layer=background]
\end{ZX}
\end{codeexample}
It can also be practical to combine it with |small minus|:
\begin{codeexample}[]
\begin{ZX}
  \zxZ-{\delta_j} & \zxZ[content fixed baseline]-{\delta_j} &
  \zxZ[small minus]-{\delta_j} & \zxZ[content fixed baseline,small minus]-{\delta_j}
\end{ZX}
\end{codeexample}
\end{pgfmanualentry}

\subsubsection{Phase in label style}

We also provide styles to place the phase on a label next to an empty node (not yet very well tested):

\begin{pgfmanualentry}
  \makeatletter
  \def\extrakeytext{style, }
  \extractkey/zx/styles/rounded style/phase in content\@nil%
  \extractkey/zx/styles/rounded style/phase in label=style (default {})\@nil%
  \extractkey/zx/styles/rounded style/pil=style (default {})\@nil%
  \extractkey/zx/styles/rounded style/phase in label above=style (default {})\@nil%
  \extractkey/zx/styles/rounded style/pila=style (default {})\@nil%
  \extractkey/zx/styles/rounded style/phase in label below=style (default {})\@nil%
  \extractkey/zx/styles/rounded style/pilb=style (default {})\@nil%
  \extractkey/zx/styles/rounded style/phase in label right=style (default {})\@nil%
  \extractkey/zx/styles/rounded style/pilr=style (default {})\@nil%
  \extractkey/zx/styles/rounded style/phase in label left=style (default {})\@nil%
  \extractkey/zx/styles/rounded style/pill=style (default {})\@nil%
  \makeatother
  \pgfmanualbody
  The above styles are useful to place a spider phase in a label outside the node. They can either be put on the style of a node to modify a single node at a time:
\begin{codeexample}[]
  \zx{\zxX[phase in label]{\alpha} \rar & \zxX{\alpha}}
\end{codeexample}
\noindent It can also be configured on a per-figure basis:
\begin{codeexample}[]
\zx[phase in label right]{
  \zxZ{\alpha} \dar \\
  \zxX{\alpha} \dar \\
  \zxZ{}}
\end{codeexample}
\noindent or globally:
\begin{codeexample}[]
\tikzset{
  /zx/user overlay/.style={
    phase in label={label position=-45, text=purple,fill=none}
  }
}
\zx{
  \zxFracX-{\pi}{2}
}
\end{codeexample}
Note that we must use |user post preparation labels| and not |/zx/user overlay nodes| because this will be run after all the machinery for labels has been setup.

  While |phase in content| forces the content of the node to be inside the node instead of inside a label (which is the default behavior), all other styles are special cases of |phase in label|. The \meta{style} parameter can be any style made for a tikz label:
\begin{codeexample}[width=3cm]
  \zx{
    \zxX[phase in label={label position=45, text=purple}]{\alpha}
  }
\end{codeexample}

For ease of use, the special cases of label position |above|, |below|, |right| and |left| have their respective shortcut style. The |pil*| versions are shortcuts of the longer style written above. For instance, |pilb| stands for |phase in label below|. Note also that by default labels will take some space, but it's possible to make them overlay without taking space using the |overlay| label style\dots{} however do it at your own risks as it can overlay the content around (also the text before and after):
\begin{codeexample}[width=0pt]
  \zx{
    \zxZ[pilb]{\alpha+\beta} \rar & \zxX[pilb]{\gamma} \rar & \zxZ[pilb=overlay]{\gamma+\eta}
  }
\end{codeexample}
The above also works for fractions:
\begin{codeexample}[]
\zx{\zxFracX[pilr]-{\pi}{2}}
\end{codeexample}
For fractions, you can configure how you want the label text to be displayed, either in a single line (default) or on two lines, like in nodes. The function |\zxConvertToFracInLabel| is in charge of that conversion, and can be changed to your needs to change this option document-wise. To use the same notation in both content and labels, you can do:
\begin{codeexample}[width=0pt]
  Compare
  \begin{ZX}[phase in label right]
    \zxFracZ{\pi}{2} \dar \\
    \zxFracZ{a+b}[(a+b)][(c+d)]{c+d}
  \end{ZX} with
{\RenewExpandableDocumentCommand{\zxConvertToFracInLabel}{mmmmm}{
    \zxConvertToFracInContent{#1}{#2}{#3}{#4}{#5}%
  }
  \begin{ZX}[phase in label right]
    \zxFracZ{\pi}{2} \dar \\
    \zxFracZ{a+b}[(a+b)][(c+d)]{c+d}
  \end{ZX} (exact same code!)
}
\end{codeexample}
Note that in |\zxFracZ{a+b}[(a+b)][(c+d)]{c+d}| the optional arguments are useful to put parens appropriately when the fraction is written inline.
\end{pgfmanualentry}

\begin{stylekey}{/zx/defaultEnvdebug mode}
  If this macro is defined, debug mode is active. See below how it can be useful (here is a quick example).
\begin{codeexample}[width=3cm]
\begin{ZX}[debug mode]
  \rar[B] & [\zxwCol] \zxDivider[a=divTop]{} & \\
  &                                  & \zxZ[a=Ztopleft]{}
  \ar[from=divTop, to=Ztopleft]
\end{ZX}
\end{codeexample}
\end{stylekey}

\begin{command}{\zxDebugMode{}}
  If this macro is defined, debug mode is active. See below how it can be useful.
\end{command}
\begin{stylekey}{/tikz/every node/a=alias}
  Shortcut to add an |alias| to a wire, and in debug mode it also displays the alias of the nodes next to it (very practical to quickly add wires as we will see later). To enable debug mode, just type |\def\zxDebugMode{}| before your drawing, potentially in a group like |{\def\zxDebugMode{} your diagram...}| if you want to apply it to a single diagram.

  This will be very practical later when using names instead of directions to connect wires (this can improve readability and maintainability). This is added automatically in |/tikz/every node| style. Note that debug mode is effective only for |a| and not |alias|.
\begin{codeexample}[width=3cm]
  \begin{ZX}
    \zxX[a=A]{} & \zxZ[a=B]{\beta}
    \ar[from=A,to=B]
  \end{ZX}
  {\def\zxDebugMode{} %% Enable debug mode for next diagram%
    \begin{ZX}
      \zxX[a=A]{} & \zxZ[a=B]{\beta}
      \ar[from=A,to=B]
    \end{ZX}
  }
\end{codeexample}
\end{stylekey}

\begin{stylekey}{/zx/defaultEnv/math baseline=node alias}
  You can easily change the default baseline which defaults to:
  \begin{verse}
    |baseline={([yshift=-axis_height]current bounding box.center)}|
  \end{verse}
  (|axis_height| is the distance to use to center equations on the ``mathematical axis'') by using this in the \meta{options} field of |\zx[options]{...}|. However, this can be a bit long to write, so |math baseline=yourAlias| is a shorcut to |baseline={([yshift=-axis_height]yourAlias)}|:
\begin{codeexample}[width=0pt]
  Compare $\begin{ZX}
    \leftManyDots{n} \zxX{\alpha} \zxLoopAboveDots{} \middleManyDots{} \ar[r,o'={a=75}]
    & \zxX{\beta} \zxLoopAboveDots{} \rightManyDots{m}
  \end{ZX}
  = {\def\zxDefaultSoftAngleS{20} % useful to make the angle in \leftManyDots{} nicer.
    \begin{ZX}
      \leftManyDots{n} \zxX{\alpha+\beta} \rightManyDots{m}
    \end{ZX}}$ with $\begin{ZX}[math baseline=wantedBaseline]
    \leftManyDots{n} \zxX{\alpha} \zxLoopAboveDots{} \middleManyDots{} \ar[r,o'={a=75}]
    %% See here --v the node chosen as the baseline
    & \zxX[a=wantedBaseline]{\beta} \zxLoopAboveDots{} \rightManyDots{m}
  \end{ZX}
  = {\def\zxDefaultSoftAngleS{20} % useful to make the angle in \leftManyDots{} nicer.
    \begin{ZX}
      \leftManyDots{n} \zxX{\alpha+\beta} \rightManyDots{m}
    \end{ZX}}$
\end{codeexample}
Also, if you find your diagram a bit ``too high'', check that you did not forget to remove a trailing |\\| at the end of the last line:
\begin{codeexample}[width=3cm]
  Compare $\begin{ZX}
    \zxZ{} \rar[o'] \rar[o.]      & \zxX{}\\
    \zxZ{} \rar[o'] \rar[o.] \rar & \zxX{}\\ %% <--- remove last \\
  \end{ZX} = \zx{\zxEmptyDiagram}$ with $\begin{ZX}
    \zxZ{} \rar[o'] \rar[o.]      & \zxX{}\\
    \zxZ{} \rar[o'] \rar[o.] \rar & \zxX{}
  \end{ZX}  = \zx{\zxEmptyDiagram}$
\end{codeexample}
\end{stylekey}

\begin{stylekey}{/zx/defaultEnv/math baseline row=row to center}
  You can also choose directly a line to center on: for instance to center on the first line, use: |math baseline row=1|, or, equivalently |mbr=1| or directly |mbr|:
\begin{codeexample}[width=0pt]
  $\begin{ZX}[mbr]
    \zxN{} \rar[B] & [\zxwCol] \zxMatrix{A} \rar[B] &[\zxwCol] \zxN{}
  \end{ZX} = A$
\end{codeexample}
\end{stylekey}


\subsubsection{Ground}

\textbf{NB}: this functionality, based on custom nodes (\cref{subsec:customNodes}) was added on 13/03/2023.

\begin{pgfmanualentry}
  \def\extrakeytext{style, }
  \extractcommand\zxGround\oarg{picture style}\opt{.}\opt{-}\opt{'}\marg{}\@@
  \extractcommand\zxGroundScale\@@
  \pgfmanualbody
The ground symbol can be used to denote a discarding operation… but is also useful to compute the norm of a state or denote a measurement by first copying the state and discarding one copy. This way, a measurement in the computational basis can be represented as:
\begin{codeexample}[width=0pt]
  \begin{ZX}
              & [\zxwCol]                    & [\zxwCol] \zxN{} \rar & \zxN{} \\
    \zxN{} \rar & \zxZ{} \ar[ur,<'] \ar[dr,<'] &                            \\[\zxZeroRow]
              &                              & \zxGround{}              
  \end{ZX}
\end{codeexample}
You can also change the direction using the alternative names:
\begin{codeexample}[]
  \begin{ZX}
    \zxGround-{}\rar & \zxGround{} \\
    \zxGround'{}\dar\\
    \zxGround.{}
  \end{ZX}
\end{codeexample}
Internally, the ground symbol is drawn using a |pic|, so you can customize it as any other pic:
\begin{codeexample}[]
  \begin{ZX}
    \zxN{} \rar &[\zxwCol] \zxGround[scale=1.5,red,rotate=45]{}
  \end{ZX}
\end{codeexample}
Note that you can also redefine |\def\zxGroundScale{1.8}| to change the default scale on a whole document.
Moreover, by default the |pic| takes some space (this way it will not overlap with the next symbol, or the text below/after), but you want sometimes to make it |overlay|, for instance to preserve the symmetry with an empty wire (then, you might need to add some column space |&[yourspace]| or row space |\\[yourspace]| to avoid overlap with text around it):
\begin{codeexample}[width=0pt]
  Compare
  \begin{ZX}
                & [\zxwCol]                    & [\zxwCol] \zxN{} \rar & \zxN{} \\
    \zxN{} \rar & \zxZ{} \ar[ur,<'] \ar[dr,<'] &                            \\
                &                              & \zxGround[overlay]{}           
  \end{ZX}
  with
  \begin{ZX}
                & [\zxwCol]                    & [\zxwCol] \zxN{} \rar & \zxN{} \\
    \zxN{} \rar & \zxZ{} \ar[ur,<'] \ar[dr,<'] &                            \\
                &                              & \zxGround{}              
  \end{ZX}
\end{codeexample}
\end{pgfmanualentry}

\subsubsection{Scalable ZX}

We provide some notations coming from the scalable ZX calculus~\cite{CHP19_SZXcalculusScalableGraphical}. 

\begin{pgfmanualentry}
  \def\extrakeytext{style, }
  \extractcommand\zxDivider\oarg{picture style}\oarg{picture style}\opt{.}\opt{-}\opt{'}\marg{}\@@
  \pgfmanualbody
  Dividers can be used to split (or gather) groups of wires. The |.|,|-|,|'| modifiers are used, respectively, to denote the bottom/right/top versions:
{\catcode`\|=12 % Ensures | is not anymore \verb|...|
\begin{codeexample}[width=0pt]
We provide some pre-defined symbols in both horizontal: %
\begin{ZX}
  \zxZ[B]{} \rar[Bn'=n+m, wc] &[\zxwCol] \zxDivider{} 
                                  \rar[o',Bn'Args={n}{}]
                                  \rar[o.,Bn.Args={m}{}] &[\zxWCol] \zxDivider-{} \rar[B,wc] & \zxZ[B]{}
\end{ZX}
 and vertical mode:
\begin{ZX}
  \zxZ[B]{} \dar[Bn=n+m, wc] \\[\zxwRow]
  \zxDivider'{} \dar[-o,BnArgs={n}{}] \dar[o-,Bn-Args={m}{}] \\[\zxWRow]
  \zxDivider.{} \dar[B,wc]\\
  \zxZ[B]{}
\end{ZX}
\end{codeexample}
}
\end{pgfmanualentry}


\begin{pgfmanualentry}
  \def\extrakeytext{style, }
  \extractcommand\zxMatrix\oarg{pic style}\oarg{node style}\opt{.}\opt{-}\opt{'}\opt{/}\opt{*}\opt{\_\{text in pmatrix\}}\marg{matrix name}\@@
  \pgfmanualbody
  Matrices are represented using arrows: |matrix name| is the content of the label of the node:
{\catcode`\|=12 % Ensures | is not anymore \verb|...|
\begin{codeexample}[]
\begin{ZX}
  \zxN{} \rar & \zxMatrix{A} \rar & \zxN{}
\end{ZX}
\end{codeexample}
}
The |*| option is used to reverse the direction of the arrow, typically for the transpose (note how we can give an alias (with |a| in |node style|) to combine with |math baseline|, or its quicker variants |mbr=nb line to center on| to properly vertically align the node):
{\catcode`\|=12 % Ensures | is not anymore \verb|...|
\begin{codeexample}[width=0pt]
We define %
$\begin{ZX}[mbr]
  \zxN{} \rar[B] & \zxMatrix*{A} \rar[B] & \zxN{}
\end{ZX} \coloneqq
\begin{ZX}[mbr=2]
            & [\zxWCol] \zxN{} \ar[B,dr,s]        & [\zxWCol] \zxN{} \\[\zxZeroCol+.3mm]
\ar[B,dr,s] & \zxMatrix{A} \ar[B,u,C] \ar[B,d,C-] &                  \\
            &                                     & \zxN{}           \\
\end{ZX}$.
\end{codeexample}
}
The position of the label can be changed with |-| is in horizontal wires, and in vertical wires we use |'| and |.| (putting the label respectively on right and left).
{\catcode`\|=12 % Ensures | is not anymore \verb|...|
\begin{codeexample}[width=0pt]
\begin{ZX}
  \zxN{} \rar & \zxMatrix{} \rar & \zxMatrix{A} \rar & \zxMatrix*{A} \rar
  & \zxMatrix-{A} \rar & \zxMatrix-*{A} \rar & \zxN{}
\end{ZX}
\end{codeexample}
}
Similarly in horizontal wires:
{\catcode`\|=12 % Ensures | is not anymore \verb|...|
\begin{codeexample}[]
\begin{ZX}
  \zxN{} \dar \\
  \zxMatrix'{} \dar \\
  \zxMatrix.{A} \dar \\
  \zxMatrix'*{A} \dar \\
  \zxMatrix'{A} \dar \\
  \zxMatrix.*{A} \dar \\
  \zxN{}
\end{ZX}
\end{codeexample}
}
If you want to change the position of the label to a more advanced position (e.g.\ with an angle), the simpler solution is to add |yourAngle:| in front of the label (see tikz labels for more details):
{\catcode`\|=12 % Ensures | is not anymore \verb|...|
\begin{codeexample}[]
\begin{ZX}
  \zxZ{} \rar & \zxMatrix{45:A} \rar & \zxN{}
\end{ZX}
\end{codeexample}
}
Note that it might be useful to put the label as an overlay using |/| (i.e. it is not counted in the bounding box of the cell, and might overlap with content around) in order to reduce space if we know there is nothing on the nearby cell: (we can also manually change the row/colum sep with negative values… but it might be better to avoid this kind of manual tweaks):
{\catcode`\|=12 % Ensures | is not anymore \verb|...|
\begin{codeexample}[]
Compare %
\begin{ZX}
  \zxZ{} \rar \dar & \zxMatrix-{A} \rar & \zxN{} \\
  \zxX{}
\end{ZX} with %
\begin{ZX}
  \zxZ{} \rar \dar & \zxMatrix-/{A} \rar & \zxN{} \\
  \zxX{}
\end{ZX}
\end{codeexample}
}
It is also useful to put a |pmatrix| inside. While it is possible to write the full |pmatrix|, we can use the |_{}| embelishment to automatically wrap the text with |\begin{bmatrix} \end{bmatrix}|:
{\catcode`\|=12 % Ensures | is not anymore \verb|...|
\begin{codeexample}[width=0pt]
This %
\begin{ZX}[mbr]
  \zxN{} \rar & \zxMatrix_{A & B \\ C & D}{} \rar & \zxN{}
\end{ZX} %
is a shortcut for %
\begin{ZX}[math baseline row=1]
  \zxN{} \rar & \zxMatrix{\begin{bmatrix} A & B \\ C & D \end{bmatrix}} \rar & \zxN{}
\end{ZX}
\end{codeexample}
}
Note that it seems that some environments do not play well with the way we handle |&| (our changes were needed to make them compatible with |align|, and to provide an easy interface with the |external| library… but it seems to not fit well with all environments, e.g.\ arrays). In that case you should use |\begin{ZXNoExt}| together with |[ampersand replacement=\&]| (of course, use |\&| instead of |&| in the rest of the matrix):
{\catcode`\|=12 % Ensures | is not anymore \verb|...|
\begin{codeexample}[width=0pt]
$\begin{ZXNoExt}[ampersand replacement=\&]
  \zxN{} \rar \& \zxMatrix{
    \begin{bmatrix}
      \begin{array}{c|c}
        A & B \\
        \hline
        C & D
      \end{array}
    \end{bmatrix}} \rar \& \zxN{}
\end{ZXNoExt}$
\end{codeexample}
}
Here is a demo to check if the true north etc anchors are placed correctly:
{\catcode`\|=12 % Ensures | is not anymore \verb|...|
\begin{codeexample}[width=0pt]
\begin{ZX}
  \zxH{} \ar[to=Gll,C'] \ar[to=Gll,C.] & \zxMatrix[a=Gll]'{G}
\end{ZX} %
\begin{ZX}
  \zxH{} \ar[to=Gll,C'] \ar[to=Gll,C.] & \zxMatrix[a=Gll]'*{G}
\end{ZX} %
\begin{ZX}
  \zxH{} \ar[to=Gll,C'] \ar[to=Gll,C.] & \zxMatrix[a=Gll].{G}
\end{ZX} %
\begin{ZX}
  \zxH{} \ar[to=Gll,C'] \ar[to=Gll,C.] & \zxMatrix[a=Gll].*{G}
\end{ZX} %
\begin{ZX}
  \zxH{} \ar[to=Gll,C] \ar[to=Gll,C-] \\
  \zxMatrix[a=Gll]{G}
\end{ZX} %
\begin{ZX} %
  \zxH{} \ar[to=Gll,C] \ar[to=Gll,C-] \\
  \zxMatrix[a=Gll]*{G}
\end{ZX} %
\begin{ZX}
  \zxH{} \ar[to=Gll,C-] \ar[to=Gll,C] \\
  \zxMatrix[a=Gll]-{G}
\end{ZX} %
\begin{ZX}
  \zxH{} \ar[to=Gll,C] \ar[to=Gll,C-] \\
  \zxMatrix[a=Gll]-*{G}
\end{ZX}
\end{codeexample}
}


\end{pgfmanualentry}

\begin{pgfmanualentry}
  \def\extrakeytext{style, }
  \makeatletter% should not be letter for \@@... strange
  \extractkey/zx/picCustomStyleMatrixMainNode\@nil%
  \extractkey/zx/picCustomStyleMatrixLabel\@nil%
  \extractkey/zx/picCustomStyleBeforeUserMatrix\@nil%
  \extractkey/zx/picCustomStyleAterUserMatrix\@nil%
  \extractkey/zx/picCustomStyleLastPicMatrix\@nil%
  \makeatother
  \pgfmanualbody
If you would like to override some settings, note that you can use all customization options provided by our custom node system (\cref{subsec:customNodes}). In particular, you can use |pic style| to change the options of the pic used to draw the node, including scale, rotation…, |node style| to style the parent node of the pic (less used, mostly to give alias names to the shape). We additionally provide special styles to configure the nodes more precisely: |/zx/picCustomStyleMatrixMainNode| and |/zx/picCustomStyleMatrixLabel| to configure more specifically the :
{\catcode`\|=12 % Ensures | is not anymore \verb|...|
\begin{codeexample}[width=0pt]
\begin{ZX}
  \zxN{} \rar & \zxMatrix{A} \rar &
  \zxMatrix[
    scale=2,
    /zx/picCustomStyleMatrixMainNode/.style={fill=blue!50},
    /zx/picCustomStyleMatrixLabel/.style={red}]{A}
  \rar & \zxN{}
\end{ZX}
\end{codeexample}
}
You can also set globally the styles like |/zx/picCustomStyleBeforeUserMatrix| (automatically provided, see details in \cref{subsec:customNodes}) to automatically add a style to your picture :
{\catcode`\|=12 % Ensures | is not anymore \verb|...|
\begin{codeexample}[width=0pt]
\tikzset{
  /zx/picCustomStyleBeforeUserMatrix/.style={
    scale=2,
    /zx/picCustomStyleMatrixMainNode/.style={fill=blue!50},
    /zx/picCustomStyleMatrixLabel/.style={red,circle,draw,inner sep=1pt},
    % \zxCustomPicAdditionalPic can be any tikz code to run after the creation of the pic:
    /utils/exec={\def\zxCustomPicAdditionalPic{%
        % the main node has empty name, so .center is the center of the main node
        \node[draw,circle,inner sep=2pt,fill=pink] at (.center) {};%
      }%
    },
  },
}
\begin{ZX}
  \zxZ{} \rar & \zxMatrix{A} \rar & \zxMatrix*{B} \rar & \zxN{}
\end{ZX}
\end{codeexample}
}
\end{pgfmanualentry}


\subsubsection{Circuit-related}

Since 2023-09-17, we provide a number of options to typeset circuits.

\begin{pgfmanualentry}
  \def\extrakeytext{style, }
  \extractcommand\zxBox\opt{\oarg{style}}\marg{box text}\@@
  \extractcommand\zxGate\opt{\oarg{style}}\marg{box text}\@@
  \pgfmanualbody
  (both commands are alias) You can add simple boxes using |\zxBox{X}| (in math mode), possibly adding additional styling to the main box using |main={your style}|:
{\catcode`\|=12 % Ensures | is not anymore \verb|...|
\begin{codeexample}[]
\begin{ZX}
  \zxBox{G} \rar[B]                    & [\zxwCol] \zxN{}\\
  \zxBox[main={fill=green}]{G_A} \rar[B] & [\zxwCol] \zxN{}
\end{ZX}
\end{codeexample}
}
\end{pgfmanualentry}

\begin{pgfmanualentry}
  \extractcommand\zxDefaultColumnSepCircuit\@@
  \extractcommand\zxDefaultRowSepCircuit\@@
  \makeatletter
  \def\extrakeytext{style, }
  \extractkey/zx/defaultEnv/circuit\@nil%
  \makeatother
  \pgfmanualbody
  This style enables the circuit mode, which is simply changing the default spacing between lines and columns. Compare:
{\catcode`\|=12 % Ensures | is not anymore \verb|...|
\begin{codeexample}[]
\begin{ZX}
  \rar & \zxGate{H} \rar & \zxGateMulti{2}{3}{H} &  & \rar & \\
  \rar & \zxGate{H} \rar &                       &  & \rar & 
\end{ZX}
\end{codeexample}
}
with
{\catcode`\|=12 % Ensures | is not anymore \verb|...|
\begin{codeexample}[]
\begin{ZX}[circuit]
  \rar & \zxGate{H} \rar & \zxGateMulti{2}{3}{H} &  & \rar & \\
  \rar & \zxGate{H} \rar &                       &  & \rar & 
\end{ZX}
\end{codeexample}
}
You can customize the default spacing with |\def\zxDefaultColumnSepCircuit{4mm}| and |\def\zxDefaultRowSepCircuit{4mm}|.
\end{pgfmanualentry}

\begin{pgfmanualentry}
  \def\extrakeytext{style, }
  \extractcommand\zxGateMulti\opt{\oarg{style}}\marg{number rows}\marg{number columns}\marg{box text}\@@
  \makeatletter
  \def\extrakeytext{style, }
  \extractkey/zx/gateMulti/a=\marg{alias main node}\@nil%
  \extractkey/zx/gateMulti/add label=\marg{label}\@nil%
  \extractkey/zx/gateMulti/add label advanced=\marg{style}\marg{label}\@nil%
  \extractkey/zx/gateMulti/content inner nodes=\marg{content inner nodes}\@nil%
  \extractkey/zx/gateMulti/main=\marg{style}\@nil%
  \extractkey/zx/gateMulti/main text=\marg{style}\@nil%
  \extractkey/zx/gateMulti/additional code=\marg{code to add new nodes}\@nil%
  \extractkey/zx/gateMulti/fit content=\opt{\marg{additional row margin}\marg{additional column margin}\marg{minimum height inner nodes}\marg{minimum width inner nodes}}\@nil%
  \extractkey/zx/gateMulti/safe fit\@nil%
  \makeatother
  \pgfmanualbody
  Create a multi-line/row gate:
{\catcode`\|=12 % Ensures | is not anymore \verb|...|
\begin{codeexample}[]
\begin{ZX}[circuit]
  \rar & \zxGate{H} \rar & \zxGateMulti{2}{3}{H} &  & \rar & \\
  \rar & \zxGate{H} \rar &                       &  & \rar & 
\end{ZX}
\end{codeexample}
}
{\catcode`\|=12 % Ensures | is not anymore \verb|...|
\begin{codeexample}[]
  \begin{ZX}
    \rar & \zxBox{H} \rar & \zxGateMulti{1}{3}{H} \dar &           & \rar      & \zxBox{H} &  & \\
         &                & \zxBox{H} \rar             & \zxBox{H} & \zxBox{H} &           &  & 
  \end{ZX}
\end{codeexample}
}
Note that the column will adapt to fit the content if it is too long:
\begin{codeexample}[]
\begin{ZX}[circuit]
  \rar       & \zxGateMulti{1}{2}{HHHHHHHHHH} & \rar       & \\
  \zxGate{H} & \zxGate{H}                     & \zxGate{H} & \zxGate{H} & 
\end{ZX}
\end{codeexample}
You can also customize its behavior in a number of ways using the above listed styles. |a| will just give the main node an alias, which can be used to refer to that node later using |to=nameAlias|, |main| and |main text| allows you to change the style of the main node and the main text:
\begin{codeexample}[]
\begin{ZX}[circuit]
A                                 & B                                    & C & D    & E              &  & \\
\zxBox{H} \rar                    & \zxGateMulti[a=secondGate]{2}{3}{H}  &   & \rar & \zxBox{X}      &  & \\
\zxBox{H} \rar                    &                                      &   &      & \zxBox{X} \lar &  & \\
\zxBox{H} \rar \ar[dr,bend right] & \zxBox{H} \rar[to=secondGate]        &   &      &                &  & \\
                                  & \zxGateMulti[main={fill=orange!50!white}, main text={red}]{1}{2}{H} & & 
\end{ZX}
\end{codeexample}
while |additional code| allows you to write arbitrary code after the creation of the main node (|zxMainNode| and |zxMainNodeText| are the alias of the main node and text node):
\begin{codeexample}[]
\begin{ZX}[circuit]
  \zxBox{H} \rar       & \zxGateMulti[
    main={fill=orange!50!white},
    main text={red},
    additional code={
      \node[
        circle,
        inner sep=2pt,
        fill=red,
        at={($(zxMainNode.east)+(0mm,1.5mm)$)},
        zx subnode={redCircle}
      ]
      {};
    }%
    ]{1}{2}{H}         & \rar[start subnode={redCircle}, to=lastH, C-] & \\
    \zxBox[a=lastH]{H} &                                              
\end{ZX}
\end{codeexample}
The |add label| and |add label advanced| allows you to add labels:
\begin{codeexample}[width=0pt]
\begin{ZX}[circuit]
  \rar & \zxGateMulti[add label={Bell \\measurement}]{2}{1}{\zxMeter{}} \rar[classical] & \\
  \rar & \rar[cl]                                                                       & 
\end{ZX}
\end{codeexample}
Internally, in order to place the matrix, we place a number of inner nodes in the region of the matrix. You can customize them using |content inner nodes| and |style inner nodes|. This can be practical if you want to force the node to take less space for instance:
\begin{codeexample}[width=0pt]
  \begin{ZX}[circuit]
    &  &                       &  &  &  &  & \\
    & \zxGate{H} \rar & \zxGateMulti[content inner nodes={}, style inner nodes={red,
      minimum width=2pt, minimum height=2pt,
      test/.style={fill=#1,opacity=.3}, test=red}]{2}{3}{HHHH} &  &  &  &  & \\
    &  &                       &  &  &  &  & 
  \end{ZX}
\end{codeexample}
By default, we run |fit content/.default={0mm}{0mm}{\zxBoxMinimumHeight}{\zxBoxMinimumWidth}| which works by evaluating the width of the content, and by automatically setting the size of the inner nodes so that the content fits nicely at the end (it takes the default column sep of the matrix, so it will not see custom adjustment of the column size with |&[1cm]|), with a minimum size. You can also call instead |safe fit| that will just set all inner nodes to be equal to the size of the content (which might not be optimal in term of space). While it works well with nodes of small size, it gives weird results on large nodes:
Compare
\begin{codeexample}[width=0pt]
\begin{ZX}[circuit]
 & A              & B                        & C         & D         & E         &  & \\
 & \zxBox{H} \rar & \zxGateMulti{1}{3}{HHHH} &           & \rar      & \zxBox{H} &  & \\
 &                & \zxBox{H}                & \zxBox{H} & \zxBox{H} &           &  & \\
 &                &                          &           &           &           &  & \\
 &                &                          &           &           &           &  & \\
\end{ZX}
\end{codeexample}
with
\begin{codeexample}[width=0pt]
\begin{ZX}[circuit]
  & A              & B                                        & C         & D         & E         &  & \\
  & \zxBox{H} \rar & \zxGateMulti[safe fit]{1}{3}{HHHH} &           & \rar      & \zxBox{H} &  & \\
  &                & \zxBox{H}                                & \zxBox{H} & \zxBox{H} &           &  & \\
  &                &                                          &           &           &           &  & \\
  &                &                                          &           &           &           &  & \\
\end{ZX}
\end{codeexample}
Note that even if you enable |safe fit|, we first run |fit content| even before reading the user input, but you can disable it for efficiency reasons by setting |\def\zxDisableFitContent{}| before |\zxGateMulti| (|fit content| is significantly slower due to internal computations, but if time is an issue for you see our section on externalization \cref{sec:externalization}).
\end{pgfmanualentry}

%{\catcode`\|=12 % Ensures | is not anymore \verb|...|
\begin{pgfmanualentry}
  \def\extrakeytext{style, }
  \extractcommand\zxElt\opt{\oarg{style}}\marg{box text}\@@
  \extractcommand\zxCtrl\opt{\oarg{style}}\marg{}\@@
  \extractcommand\zxOCtrl\opt{\oarg{style}}\marg{}\@@
  \extractcommand\zxNot\opt{\oarg{style}}\marg{}\@@
  \extractcommand\zxCross\opt{\oarg{style}}\marg{}\@@
  \extractcommand\zxMeter\opt{\oarg{scale}}\marg{}\@@
  \makeatletter
  \def\extrakeytext{style, }
  \extractkey/zx/wires definition/classical\@nil%
  \extractkey/zx/wires definition/cl\@nil%
  \extractkey/zx/wires definition/connect =-\textbar\@nil%
  \extractkey/zx/wires definition/connect =\textbar-\@nil%
  \makeatother
  \pgfmanualbody
  |\zxElt| is like |\zxGate| but without any border (useful to add spacing around elements like |\ket{\psi}|), and the other commands just create the corresponding symbols that you can link together as you want as usual using |\ar| (note that |\zxMeter{}| is node a node but just an icon, so place it yourself inside |\zxGate| or |\zxGateMulti|). The style are wire styles, and are useful to print classical wires (|cl| is an alias for |classical|) and wires bent with a 90 degrees angle. Note that I could not write the right name in the doc, you should remove the equal sign like in \texttt{connect -\textbar}.
{\catcode`\|=12 % Ensures | is not anymore \verb|...|
\begin{codeexample}[width=0pt]
\begin{ZX}[circuit]
  \zxElt{\ket{\psi}} \rar & \zxBox{H} \rar         & \zxCtrl{} \dar \rar & \zxCross{} \dar \rar
  & \zxBox[add label={Measure}]{\zxMeter{}} \ar[dr,classical,connect -|] \\
  \ar[r]                  & \zxOCtrl{} \rar \ar[u] & \zxNot{} \rar       & \zxCross{} \ar[rr]
  &  & \zxBox{H} \rar & 
\end{ZX}    
\end{codeexample}
}
\end{pgfmanualentry}

\subsection{Wires}

\subsubsection{Creating wires and debug mode}

\begin{pgfmanualentry}
  \extractcommand\arrow\opt{\oarg{options}}\@@
  \extractcommand\ar\opt{\oarg{options}}\@@
  \pgfmanualbody
  These synonym commands (actually coming from |tikz-cd|) are used to draw wires between nodes. We refer to |tikz-cd| for an in-depth documentation, but what is important for our purpose is that the direction of the wires can be specified in the \meta{options} using a string of letters |r| (right), |l| (left), |u| (up), |d| (down). It's also possible to specify a node alias as a source or destination as shown below.
\begin{codeexample}[]
\zx{\zxZ{} \ar[r] & \zxX{}} = \zx{\zxX{} \arrow[rd] \\ & \zxZ{}}
\end{codeexample}
  \meta{options} can also be used to add any additional style, either custom ones, or the ones defined in this library (this is recommended since it can be easily changed document-wise by simply changing the style). Multiple wires can be added in the same cell. Other shortcuts provided in |tikz-cd| like |\rar|\dots{} can be used.
{\catcode`\|=12 % Ensures | is not anymore \verb|...|
\begin{codeexample}[width=0pt]
\begin{ZX}
  \zxZ{\alpha} \arrow[d, C] % C = Bell-like wire
               \ar[r,H,o']  % o' = top part of circle
               % H adds Hadamard, combine with \zxHCol
               \ar[r,H,o.] &[\zxHCol] \zxZ{\gamma}\\
  \zxZ{\beta}  \rar        & \zxX{} \ar[ld,red,"\circ" {marking,blue}] \ar[rd,s] \\
  \zxFracX-{\pi}{4}        & &\zxZ{}
\end{ZX}
\end{codeexample}
}
\end{pgfmanualentry}

As explained in |tikz-cd|, there are further shortened forms:
\begin{pgfmanualentry}
  \extractcommand\rar\opt{\oarg{options}}\@@
  \extractcommand\lar\opt{\oarg{options}}\@@
  \extractcommand\dar\opt{\oarg{options}}\@@
  \extractcommand\uar\opt{\oarg{options}}\@@
  \extractcommand\drar\opt{\oarg{options}}\@@
  \extractcommand\urar\opt{\oarg{options}}\@@
  \extractcommand\dlar\opt{\oarg{options}}\@@
  \extractcommand\ular\opt{\oarg{options}}\@@
  \pgfmanualbody
\end{pgfmanualentry}
The first one is equivalent to
\begin{verse}
  |\arrow|{\oarg{options}}|{r}|
\end{verse}
and the other ones work analogously.

Note that sometimes, it may be practical to properly organize big diagrams to increase readability. To that end, one surely wants to have a small and well indented matrix (emacs |M-x align-current| or |M-x align| (for selected lines) commands are very practical to indent matrices automatically). Unfortunately, adding wires inside the matrix can make the line really long and hard to read. Similarly, some nodes involving fractions or long expressions can also be quite long. It is however easy to increase readability (and maintainability) by moving the wires at the very end of the diagram, using |a| (like |alias|, but with a debug mode) to connect nodes and |\def| to create shortcuts. Putting inside a macro with |\def| long node definitions can also be useful to keep small items in the matrix:
\begin{codeexample}[width=0pt]
\begin{ZX}[zx row sep=1pt,
  execute at begin picture={%
    %%% Definition of long items (the goal is to have a small and readable matrix
    % (warning: macro can't have numbers in TeX. Also, make sure not to use existing names)
    \def\Zpifour{\zxFracZ[a=Zpi4]-{\pi}{4}}%
    \def\mypitwo{\zxFracX[a=mypi2]{\pi}{2}}%
  }
  ]
  %%% Matrix: in emacs "M-x align-current" is practical to automatically format it.
  %%% a is for 'alias'... but also provides a debug mode, see below.
                &              &                   &                 & \zxZ[a=toprightpi]{\pi} \\
  \zxN[a=in1]{} & \zxX[a=X1]{} &                   &                 &  & \zxN[a=out1]{}       \\
                &              & \zxZ[a=xmiddle]{} & \mypitwo{}      &  & \zxN[a=out2]{}       \\
  \zxN[a=in2]{} & \Zpifour{}   &                   & \zxX[a=Xdown]{} &  & \zxN[a=out3]{}
  %%% Arrows
  % Column 1
  \ar[from=in1,to=X1]
  \ar[from=in2,to=Zpi4]
  % Column 2
  \ar[from=X1,to=xmiddle,(.]
  \ar[from=X1,to=toprightpi,<',H]
  \ar[from=Zpi4,to=xmiddle,(']
  \ar[from=Zpi4,to=Xdown,o.]
  % Column 3
  \ar[from=xmiddle,to=Xdown,s.]
  \ar[from=xmiddle,to=mypi2]
  % Column 4
  \ar[from=mypi2,to=toprightpi,(']
  \ar[from=mypi2,to=out1,<']
  \ar[from=mypi2,to=out2]
  \ar[from=Xdown,to=out3]
\end{ZX}
\end{codeexample}
In that setting, it is often useful to enable the debug mode via |\def\zxDebugMode{}| as explained above to quickly visualize the alias given to each node (note that debug mode works with |a=| but not with |alias=|). For instance, it was easy to rewrite the above diagram by moving nodes in the matrix and arrows after checking their name on the produced pdf (NB: you can increase |column sep| and |row sep| temporarily to make the debug information more visible):
\begin{codeexample}[width=0pt]
{
  \def\zxDebugMode{}%%%%
  \begin{ZX}[zx row sep=1pt,
    execute at begin picture={%
      %%% Definition of long items (the goal is to have a small and readable matrix
      % (warning: macro can't have numbers in TeX. Also, make sure not to use existing names)
      \def\Zpifour{\zxFracZ[a=Zpi4]-{\pi}{4}}%
      \def\mypitwo{\zxFracX[a=mypi2]{\pi}{2}}%
    }
    ]
    %%% Matrix: in emacs "M-x align" is practical to automatically format it. a is for 'alias'
    & \zxN[a=n]{}  & \zxZ[a=xmiddle]{}       &            & \zxN[a=out1]{} \\
    \zxN[a=in1]{} & \Zpifour{}   & \zxX[a=Xdown]{}         & \mypitwo{} &                \\
    &              &                         &            & \zxN[a=out2]{} \\
    \zxN[a=in2]{} & \zxX[a=X1]{} & \zxZ[a=toprightpi]{\pi} &            & \zxN[a=out3]{}
    %%% Arrows
    % Column 1
    \ar[from=in1,to=X1,s]
    \ar[from=in2,to=Zpi4,.>]
    % Column 2
    \ar[from=X1,to=xmiddle,N']
    \ar[from=X1,to=toprightpi,H]
    \ar[from=Zpi4,to=n,C] \ar[from=n,to=xmiddle,wc]
    \ar[from=Zpi4,to=Xdown]
    % Column 3
    \ar[from=xmiddle,to=Xdown,C-]
    \ar[from=xmiddle,to=mypi2,)]
    % Column 4
    \ar[from=mypi2,to=toprightpi,(']
    \ar[from=mypi2,to=out1,<']
    \ar[from=mypi2,to=out2,<.]
    \ar[from=Xdown,to=out3,<.]
  \end{ZX}
}
\end{codeexample}

\subsubsection{Wire styles (new generation)}

We give now a list of wire styles provided in this library (|/zx/wires definition/| is an automatically loaded style). We recommend using them instead of manual styling to ensure they are the same document-wise, but they can of course be customized to your need. Note that the name of the styles are supposed (ahah, I do my best with what ASCII provides) to graphically represent the action of the style, and some characters are added to precise the shape: typically |'| means top, |.| bottom, |X-| is right to X (or should arrive with angle 0), |-X| is left to X (or should leave with angle zero). These shapes are usually designed to work when the starting node is left most (or above of both nodes have the same column). But they may work both way for some of them.

Note that the first version of that library (which appeared one week before this new version\dots{} hopefully backward compatibility won't be too much of a problem) was using |in=| and |out=| to create these styles. However, it turns out to be not very reliable since the shape of the wire was changing (sometimes importantly) depending on the position of the nodes. This new version should be more reliable, but the older styles are still available by using |IO, nameOfWirestyle| (read more in \cref{subsub:IOwires}).

\begin{pgfmanualentry}
  \makeatletter
  \def\extrakeytext{style, }
  \extractkey/zx/wires definition/C=radius ratio (default 1)\@nil%
  \extractkey/zx/wires definition/C.=radius ratio (default 1)\@nil%
  \extractkey/zx/wires definition/C'=radius ratio (default 1)\@nil%
  \extractkey/zx/wires definition/C-=radius ratio (default 1)\@nil%
  \makeatother
  \pgfmanualbody
  Bell-like wires with an arrival at ``right angle'', |C| represents the shape of the wire, while |.| (bottom), |'| (top) and |-| (side) represent (visually) its position. Combine with |wire centered| (|wc|) to avoid holes when connecting multiple wires (not required with |\zxNone{}|, alias |\zxN{}|).
\begin{codeexample}[]
  A Bell pair \zx{\zxNone{} \ar[d,C] \\[\zxWRow]
                \zxNone{}}
  , a swapped Bell pair
  \begin{ZX}
    \zxN{} \ar[d,C] \ar[rd,s] &[\zxWCol] \zxN{} \\[\zxWRow]
    \zxN{}          \ar[ru,s] &          \zxN{}
  \end{ZX}
  and a funny graph
  \begin{ZX}
    \zxX{} \ar[d,C] \ar[r,C']  & \zxZ{} \ar[d,C-]\\
    \zxZ{} \ar[r,C.]           & \zxX{}
  \end{ZX}.
\end{codeexample}
Note that this style is actually connecting the nodes using a perfect circle (it is \emph{not} based on |curve to|), and therefore should \emph{not} be used together with |in|, |out|, |looseness|\dots{} (this is the case also for most other styles except the ones in |IO|). It has the advantage of connecting nicely nodes which are not aligned or with different shapes:
\begin{codeexample}[]
  \begin{ZX}
    \zxX{\alpha} \ar[dr,C]\\
    & \zxNone{}
  \end{ZX}
\end{codeexample}
The \meta{radius ratio} parameter can be used to turn the circle into an ellipse using this ratio between both axis:
\begin{codeexample}[]
  \begin{ZX}
    \zxX{\alpha}
      \ar[dr,C=0.5,red]
      \ar[dr,C,green]
      \ar[dr,C=2,blue]
      \ar[dr,C=3,purple]\\
                        & \zxNone{}
  \end{ZX}
  \begin{ZX}
    \zxX{} \ar[d,C=2] \ar[r,C'=2]  & \zxZ{} \ar[d,C-=2,H]\\
    \zxZ{} \ar[r,C.=2]           & \zxX{}
  \end{ZX}.
\end{codeexample}
\end{pgfmanualentry}


\begin{pgfmanualentry}
  \makeatletter
  \def\extrakeytext{style, }
  \extractkey/zx/wires definition/o'=angle (default 40)\@nil%
  \extractkey/zx/wires definition/o.=angle (default 40)\@nil%
  \extractkey/zx/wires definition/o-=angle (default 40)\@nil%
  \extractkey/zx/wires definition/-o=angle (default 40)\@nil%
  \makeatother
  \pgfmanualbody
  Curved wire, similar to |C| but with a soften angle (optionally specified via \meta{angle}, and globally editable with |\zxDefaultLineWidth|). Again, the symbols specify which part of the circle (represented with |o|) must be kept.
\begin{codeexample}[width=3cm]
  \begin{ZX}
    \zxX{} \ar[d,-o] \ar[d,o-]\\
    \zxZ{} \ar[r,o'] \ar[r,o.] & \zxX{}
  \end{ZX}.
\end{codeexample}
 Note that these wires can be combined with |H|, |X| or |Z|, in that case one should use appropriate column and row spacing as explained in their documentation:
\begin{codeexample}[width=3cm]
  \begin{ZX}
    \zxX{\alpha} \ar[d,-o,H] \ar[d,o-,H]\\[\zxHRow]
    \zxZ{\beta} \rar & \zxZ{} \ar[r,o',X] \ar[r,o.,Z] &[\zxSCol] \zxX{}
  \end{ZX}.
\end{codeexample}
\end{pgfmanualentry}

\begin{pgfmanualentry}
  \makeatletter
  \def\extrakeytext{style, }
  \extractkey/zx/wires definition/(=angle (default 30)\@nil%
  \extractkey/zx/wires definition/)=angle (default 30)\@nil%
  \extractkey/zx/wires definition/('=angle (default 30)\@nil%
  \extractkey/zx/wires definition/('=angle (default 30)\@nil%
  \makeatother
  \pgfmanualbody
  Curved wire, similar to |o| but can be used for diagonal items. The angle is, like in |bend right|, the opening angle from the line which links the two nodes. For the first two commands, the |(| and |)| symbols must be imagined as if the starting point was on top of the parens, and the ending point at the bottom.
\begin{codeexample}[width=3cm]
  \begin{ZX}
    \zxX{} \ar[rd,(] \ar[rd,),red]\\
    & \zxZ{}
  \end{ZX}.
\end{codeexample}
Then, |('|=|(| and |(.|=|)|; this notation is, I think, more intuitive when linking nodes from left to right. |('| is used when going to top right and |(.| when going to bottom right.
\begin{codeexample}[width=3cm]
\begin{ZX}
  \zxN{}                       & \zxX{}\\
  \zxZ{} \ar[ru,('] \ar[rd,(.] & \\
                               & \zxX{}
\end{ZX}
\end{codeexample}
When the nodes are too far appart, the default angle of |30| may produce strange results as it will go above (for |('|) the vertical line. Either choose a shorter angle, or see |<'| instead. Note that for now this node is based on |in| and |out|, but it may change later. So if you want to change looseness, or really rely on the precise specified angle, prefer to use |IO,(| instead (which takes the |IO| version, guaranteed to stay untouched).
\end{pgfmanualentry}

\begin{pgfmanualentry}
  \makeatletter
  \def\extrakeytext{style, }
  \extractkey/zx/wires definition/start fake center north\@nil%
  \extractkey/zx/wires definition/start fake center south\@nil%
  \extractkey/zx/wires definition/start fake center east\@nil%
  \extractkey/zx/wires definition/start fake center west\@nil%
  \extractkey/zx/wires definition/start real center\@nil
  \extractkey/zx/wires definition/end fake center north\@nil%
  \extractkey/zx/wires definition/end fake center south\@nil%
  \extractkey/zx/wires definition/end fake center east\@nil%
  \extractkey/zx/wires definition/end fake center west\@nil%
  \extractkey/zx/wires definition/end real center\@nil
  \extractkey/zx/wires definition/left to right\@nil%
  \extractkey/zx/wires definition/right to left\@nil%
  \extractkey/zx/wires definition/up to down\@nil%
  \extractkey/zx/wires definition/down to up\@nil%
  \extractkey/zx/wires definition/force left to right\@nil%
  \extractkey/zx/wires definition/force right to left\@nil%
  \extractkey/zx/wires definition/force up to down\@nil%
  \extractkey/zx/wires definition/force down to up\@nil%
  \extractkey/zx/wires definition/no fake center\@nil%
  \makeatother
  \pgfmanualbody
  Usually each wire should properly use these functions, so the end user should not need that too often (during a first reading, you can skip this paragraph). We added 4 anchors to nodes: |fake center north|, |fake center south|, |fake center east| and |fake center west|. These anchors are used to determine the starting point of the wires depending on the direction of the wire. Because some nodes may not have these anchors, we can't directly set |start anchor=fake center north, on layer=edgelayer| (but the user can do that if they are using only nodes with these anchors) or the code may fail on some nodes. For that reason, we check that these anchors exist while drawing our wires (which, at the best of my knowledge, can only be done while drawing the path). The |start/end fake center *| code is responsible to configure that properly (|start real center| will use the real center), and |left to right| (and similar) just configure both the |start| and |end| point to ensure the node starts at the appropriate anchor. However this won't work for style not defined in this library: in case you are sure that these anchors exists and want to use your own wire styles, you can then set the anchors manually and use |on layer=edgelayer|, or use |force left to right| (and similar) which will automatically do that for the |start| and |end| points.
\begin{codeexample}[]
\begin{ZX}
  \zxX{\alpha+\beta} \ar[r,o',no fake center] & \zxZ{\alpha+\beta}\\
  \zxX{\alpha+\beta} \ar[r,o'] & \zxZ{\alpha+\beta}
\end{ZX}
\end{codeexample}
\end{pgfmanualentry}

\begin{pgfmanualentry}
  \makeatletter
  \def\extrakeytext{style, }
  \extractkey/zx/args/-andL/-=x\@nil%
  \extractkey/zx/args/-andL/1-=x\@nil%
  \extractkey/zx/args/-andL/2-=x\@nil%
  \extractkey/zx/args/-andL/L=y\@nil%
  \extractkey/zx/args/-andL/1L=y\@nil%
  \extractkey/zx/args/-andL/2L=y\@nil%
  \extractkey/zx/args/-andL/1 angle and length=\marg{angle}\marg{length}\@nil%
  \extractkey/zx/args/-andL/1al=\marg{angle}\marg{length}\@nil%
  \extractkey/zx/args/-andL/2 angle and length=\marg{angle}\marg{length}\@nil%
  \extractkey/zx/args/-andL/2al=\marg{angle}\marg{length}\@nil%
  \extractkey/zx/args/-andL/angle and length=\marg{angle}\marg{length}\@nil%
  \extractkey/zx/args/-andL/al=\marg{angle}\marg{length}\@nil%
  \extractkey/zx/args/-andL/1 angle=\marg{angle}\@nil%
  \extractkey/zx/args/-andL/1a=\marg{angle}\@nil%
  \extractkey/zx/args/-andL/2 angle=\marg{angle}\@nil%
  \extractkey/zx/args/-andL/1a=\marg{angle}\marg{length}\@nil%
  \extractkey/zx/args/-andL/angle=\marg{angle}\@nil%
  \extractkey/zx/args/-andL/a=\marg{angle}\@nil%
  \extractkey/zx/args/-andL/symmetry-L\@nil%
  \extractkey/zx/args/-andL/symmetry\@nil%
  \extractkey/zx/args/-andL/negate1L\@nil%
  \extractkey/zx/args/-andL/negate2L\@nil%
  \extractkey/zx/args/-andL/negateL\@nil%
  \extractkey/zx/args/-andL/negate1-\@nil%
  \extractkey/zx/args/-andL/negate2-\@nil%
  \extractkey/zx/args/-andL/negate-\@nil%
  \extractkey/zx/args/-andL/oneMinus1-\@nil%
  \extractkey/zx/args/-andL/oneMinus2-\@nil%
  \extractkey/zx/args/-andL/oneMinus1L\@nil%
  \extractkey/zx/args/-andL/oneMinus2L\@nil%
  \makeatother
  \pgfmanualbody
  The next wires can take multiple options. They are all based on the same set of options for now, namely |/zx/args/-andL/|. The |1*| options are used to configure the starting point, the |2*| to configure the ending point, if no number is given both points are updated. |-| and |L| are used to place two anchors of a Bezier curve. They are expressed in relative distance (so they are typically between $0$ and $1$, but can be pushed above $1$ or below $0$ for stronger effects), |-| is typically on the |x| axis and |L| on the |y| axis (the name represents ``graphically'' the direction). They are however not named |x| and |y| because some wires use them slighlty differently, notably |o| which uses |-| for the direction of the arrow and |L| for the direction perpendicular to the arrow (again the shape of |L| represents a perpendicular line). Each wire interprets |-| and |L| to ensure that $0$ should lead to a straight line, and that a correct shape is obtained when |1-| equals |2-|, |1L| equals |2L| (except for non-symmetric shapes of course), and both |-| and |L| are positive.

  The other expressions involving |angle| (or the shortcut |a|) allow you to define |1-|,|1L|\dots{} using a maybe more intuitive ``polar'' notation, i.e.\ an ``angle'' and a relative length (if not specified, like in |1 angle|, the length defaults to $0.6$). Note that the angle is not really an angle (it is an angle only when the nodes are placed at $45$ degrees, or for the |bezier x/y| variations), but a ``squeezed angle'' (when nodes are not at $45$ degrees, the shape is squeezed horizontally or vertically not to change the wire) and similarly for the length. In the above list, the meaning of each expression should be clear from the name: for instance |1angle and length={45}{.8}| will setup a squeezed angle of $45$ and a relative length of $.8$ for the first point, i.e.\ this is equivalent to $1-=.8\cos(45)$ and $1L=.8\sin(45)$, and |angle=45| will change the angle of both points to $45$, with a relative length of $.6$. In the above list, each long expression has below it a shorter version, for intance |a=45| is equivalent to |angle=45|.

  The last expressions (|symmetry-L|, |symmetry|\dots) are used internally to do some math. Of course if you need to do symmetries at some point you can use these keys (|symmetry-L| exchange |-| and |L|, and |symmetry| exchanges |1| and |2|), |negateX| just negates |X| and |oneMinusX| replaces |X| with |1-X|. Each of the following nodes have default values which can be configured as explained in \cref{subsec:wirecustom}.
\end{pgfmanualentry}

\begin{pgfmanualentry}
  \makeatletter
  \def\extrakeytext{style, }
  \extractkey/zx/wires definition/s=-andL config (default defaultS)\@nil%
  \extractkey/zx/wires definition/s'=-andL config (default defaultS')\@nil%
  \extractkey/zx/wires definition/s.=-andL config (default defaultS')\@nil%
  \extractkey/zx/wires definition/-s=-andL config (default default-S)\@nil%
  \extractkey/zx/wires definition/-s'=-andL config (default \{defaultS',default-S\})\@nil%
  \extractkey/zx/wires definition/-s.=-andL config (default \{defaultS',default-S\})\@nil%
  \extractkey/zx/wires definition/s-=-andL config (default \{defaultS',default-S,symmetry\})\@nil%
  \extractkey/zx/wires definition/s'-=-andL config (default \{defaultS',default-S,symmetry\})\@nil%
  \extractkey/zx/wires definition/s.-=-andL config (default \{defaultS',default-S,symmetry\})\@nil%
  \extractkey/zx/wires definition/-S=-andL config (default \{defaultS',default-S\})\@nil%
  \extractkey/zx/wires definition/-S'=-andL config (default \{defaultS',default-S\})\@nil%
  \extractkey/zx/wires definition/-S.=-andL config (default \{defaultS',default-S\})\@nil%
  \extractkey/zx/wires definition/S-=-andL config (default \{defaultS',default-S,symmetry\})\@nil%
  \extractkey/zx/wires definition/S'-=-andL config (default \{defaultS',default-S,symmetry\})\@nil%
  \extractkey/zx/wires definition/S.-=-andL config (default \{defaultS',default-S,symmetry\})\@nil%
  \makeatother
  \pgfmanualbody
  |s| and |S| are used to create a s-like wire (|s| is smoother than |S| that arrives and leave horizontally), to have nicer diagonal lines between nodes. Other versions are soften versions (the input and output angles are not as sharp. Adding |'| or |.| specifies if the wire is going up-right or down-right, however as of today if it mostly used for backward compatibility since, for instance, |-s'| is the same as |-s| (but some styles may want to do a difference later). The only exception is for |s|/|s'|/|s.|: |s| has a sharper output angle than |s'| and |s.| (which are both equals).
\begin{codeexample}[width=3cm]
  \begin{ZX}
    \zxX{\alpha} \ar[s,rd] \\
                           & \zxZ{\beta}\\
    \zxX{\alpha} \ar[s.,rd] \\
                           & \zxZ{\beta}\\
                           & \zxZ{\alpha}\\
    \zxX{\beta} \ar[S,ru] \\
                           & \zxZ{\alpha}\\
    \zxX{\beta} \ar[s',ru] \\
  \end{ZX}
\end{codeexample}
|-| forces the angle on the side of |-| to be horizontal. Because for now the wires start inside the node, this is not very visible. For that reason, versions with a capital |S| have an anchor on the side of |-| lying on the surface of the node (|S| has two such anchors since both inputs and outputs arrives horizontally) instead of on the |fake center *| anchor (see explanation on |fake center| anchors above).
\begin{codeexample}[width=3cm]
  \begin{ZX}
    \zxX{\alpha} \ar[s.,rd] \\
                           & \zxZ{\beta}\\
    \zxX{\alpha} \ar[-s.,rd] \\
                           & \zxZ{\beta}\\
    \zxX{\alpha} \ar[s.-,rd] \\
                           & \zxZ{\beta}\\
    \zxX{\alpha} \ar[S,rd] \\
                           & \zxZ{\beta}\\
  \end{ZX}
\end{codeexample}
It is possible to configure it using the options in |-andL config| as explained above (default values are given in \cref{subsec:wirecustom}), where |-| is the (relative) position of the horizontal Bezier anchor and |L| its relative vertical position (to keep a |s|-shape, you should have |-|$>$|L|).
\begin{codeexample}[width=3cm]
\begin{ZX}
  \zxX{\alpha} \ar[rd,s.] \\
  & \zxZ{\beta}\\
  % same as s., configure globally using defaultS'\\
  \zxX{\alpha} \ar[rd,s.={-=.8,L=.2}]\\
                 & \zxZ{\beta}\\
  \zxX{\alpha} \ar[rd,s.={L=.4}] \\
                 & \zxZ{\beta}\\
  \zxX{\alpha} \ar[rd,s.={L=0.1,-=1}] \\
                 & \zxZ{\beta}\\
  \zxX{\alpha} \ar[rd,edge above, control points visible,s.={-=2}] \\
                 & \zxZ{\beta}
\end{ZX}
\end{codeexample}
For the non-symmetric versions (involving a vertical arrival), you can configure each point separately using |1-| and |1L| (first point) and |2-| and |2L| (second points). You can also specify the ``squeezed angle'' and ``length'' of each point, for instance using the |1 angle and length={10}{.8}| option (short version is |1al={10}{.8}|) or both at the same time using |al={10}{.6}| (this last command being itself equivalent to |a=10|). As explained later |edge above| and |control points visible| can help you to visualize the control points of the underlying Bezier curve.
\begin{codeexample}[width=3cm]
  \begin{ZX}
    \zxZ{} \ar[dr,s.={al={10}{.8}}]\\ &\zxZ{}\\
    \zxZ{} \ar[edge above,control points visible,dr,s.={a=10}]\\ &\zxZ{}
  \end{ZX}
\end{codeexample}
\end{pgfmanualentry}

\begin{pgfmanualentry}
  \makeatletter
  \def\extrakeytext{style, }
  \extractkey/zx/wires definition/ss=-andL config (default \{defaultS,symmetry-L\})\@nil%
  \extractkey/zx/wires definition/SS=-andL config (default \{defaultS,symmetry-L\})\@nil%
  \extractkey/zx/wires definition/ss.=-andL config (default \{defaultS',symmetry-L\})\@nil%
  \extractkey/zx/wires definition/.ss=-andL config (default \{defaultS',symmetry-L\}30)\@nil%
  \extractkey/zx/wires definition/sIs.=-andL config (default defaultSIS)\@nil%
  \extractkey/zx/wires definition/.sIs=-andL config (default \{defaultS',defaultSIS\})\@nil%
  \extractkey/zx/wires definition/ss.I-=-andL config (default \{defaultS',defaultSIS,symmetry\})\@nil%
  \extractkey/zx/wires definition/I.ss-=-andL config (default \{defaultS',defaultSIS,symmetry\})\@nil%
  \extractkey/zx/wires definition/SIS=-andL config (default \{defaultS',defaultSIS\})\@nil%
  \extractkey/zx/wires definition/.SIS=-andL config (default \{defaultS',defaultSIS\})\@nil%
  \extractkey/zx/wires definition/ISS=-andL config (default \{defaultS',defaultSIS,symmetry\})\@nil%
  \extractkey/zx/wires definition/SS.I=-andL config (default \{defaultS',defaultSIS,symmetry\})\@nil%
  \extractkey/zx/wires definition/I.SS=-andL config (default \{defaultS',defaultSIS,symmetry\})\@nil%
  \extractkey/zx/wires definition/SSI=-andL config (default \{defaultS',defaultSIS,symmetry\})\@nil%
  \makeatother
  \pgfmanualbody
  |ss| is similar to |s| except that we go from top to bottom instead of from left to right. The position of |.| says if the node is wire is going bottom right (|ss.|) or bottom left (|.ss|).
\begin{codeexample}[width=3cm]
  \begin{ZX}
    \zxX{\alpha} \ar[ss,rd] \\
                           & \zxZ{\beta}\\
    \zxX{\alpha} \ar[ss.,rd] \\
                           & \zxZ{\beta}\\
                           & \zxX{\beta} \ar[.ss,dl] \\
    \zxZ{\alpha}\\
                           & \zxX{\beta} \ar[.ss={},dl] \\
    \zxZ{\alpha}\\
  \end{ZX}
\end{codeexample}
|I| forces the angle above (if in between the two |s|) or below (if on the same side as |.|) to be vertical.
\begin{codeexample}[width=3cm]
  \begin{ZX}
    \zxX{\alpha} \ar[ss,rd] \\
                           & \zxZ{\beta}\\
    \zxX{\alpha} \ar[sIs.,rd] \\
                           & \zxZ{\beta}\\
    \zxX{\alpha} \ar[ss.I,rd] \\
                           & \zxZ{\beta}\\
                           & \zxX{\beta} \ar[.sIs,dl] \\
    \zxZ{\alpha}\\
                           & \zxX{\beta} \ar[I.ss,dl] \\
    \zxZ{\alpha}\\
  \end{ZX}
\end{codeexample}
The |S| version forces the anchor on the vertical line to be on the boundary.
\begin{codeexample}[width=3cm]
  \begin{ZX}
    \zxX{\alpha} \ar[SS,rd] \\
                           & \zxZ{\beta}\\
    \zxX{\alpha} \ar[SIS,rd] \\
                           & \zxZ{\beta}\\
    \zxX{\alpha} \ar[SSI,rd] \\
                           & \zxZ{\beta}\\
                           & \zxX{\beta} \ar[.sIs,dl] \\
    \zxZ{\alpha}\\
                           & \zxX{\beta} \ar[I.ss,dl] \\
    \zxZ{\alpha}\\
  \end{ZX}
\end{codeexample}
As for |s| it can be configured:
\begin{codeexample}[width=3cm]
\begin{ZX}
  \zxX{\alpha} \ar[rd,SIS] \\
                             & \zxZ{\beta}\\
  \zxX{\alpha} \ar[rd,SIS={1L=.4}] \\
                             & \zxZ{\beta}\\
  \zxX{\alpha} \ar[rd,SIS={1L=.8}] \\
                             & \zxZ{\beta}\\
  \zxX{\alpha} \ar[rd,SIS={1L=1,2L=1}] \\
                             & \zxZ{\beta}\\
\end{ZX}
\end{codeexample}
\end{pgfmanualentry}

\begin{pgfmanualentry}
  \makeatletter
  \def\extrakeytext{style, }
  \extractkey/zx/wires definition/N=-andL config (default defaultN)\@nil%
  \extractkey/zx/wires definition/N'=-andL config (default defaultN)\@nil%
  \extractkey/zx/wires definition/N.=-andL config (default defaultN)\@nil%
  \extractkey/zx/wires definition/-N=-andL config (default \{defaultN,defaultN-\})\@nil%
  \extractkey/zx/wires definition/-N'=-andL config (default \{defaultN,defaultN-\})\@nil%
  \extractkey/zx/wires definition/-N.=-andL config (default \{defaultN,defaultN-\})\@nil%
  \extractkey/zx/wires definition/N-=-andL config (default \{defaultN,defaultN-,symmetry\})\@nil%
  \extractkey/zx/wires definition/N'-=-andL config (default \{defaultN,defaultN-,symmetry\})\@nil%
  \extractkey/zx/wires definition/N.-=-andL config (default \{defaultN,defaultN-,symmetry\})\@nil%
  \extractkey/zx/wires definition/Nbase=-andL config (default defaultN)\@nil%
  \makeatother
  \pgfmanualbody
  |N| is used to create a left-to-right wire leaving at wide angle and arriving at wide angle (it's named |N| because it roughly have the shape of a capital |N|). In older versions, |'| and |.| was required to specify if the wire should go up-right or down-right, but it is not useful anymore (we keep it for compatibilty with |IO| styles and in case some styles want to do a distinction later).
\begin{codeexample}[width=3cm]
  \begin{ZX}
    \zxX{\alpha} \ar[N,rd] \\
                               & \zxZ{\beta}\\
                               & \zxZ{\alpha}\\
    \zxX{\beta} \ar[N,ru]
  \end{ZX}
\end{codeexample}
|-| forces the angle on the side of |-| to be horizontal.
\begin{codeexample}[width=3cm]
  \begin{ZX}
    \zxX{\alpha} \ar[-N,rd] \\
                               & \zxZ{\beta}\\
                               & \zxZ{\alpha}\\
    \zxX{\beta} \ar[N-,ru]
  \end{ZX}
\end{codeexample}
Like other wires, it can be configured using |-| (horizontal relative position of anchor points) and |L| (vertical relative position of anchor points, make sure to have |-|$<$|L| to have a |N|-looking shape), |al={angle}{relative length}|\dots{}
\begin{codeexample}[width=3cm]
  \begin{ZX}
    \zxX{\alpha} \ar[N,rd] \\
                               & \zxZ{\beta}\\
    \zxX{\alpha} \ar[N={L=1.2},rd] \\
                               & \zxZ{\beta}
  \end{ZX}
\end{codeexample}
All these styles are based on Nbase (which should not be used directly), including the styles like |<|. If you wish to overwrite later |N|-like commands, but not |<|-like, then change |N|. If you wish to also update |<| commands, use |Nbase|.
\end{pgfmanualentry}



\begin{pgfmanualentry}
  \makeatletter
  \def\extrakeytext{style, }
  \extractkey/zx/wires definition/NN=-andL config (default \{defaultN,symmetry-L,defaultNN\})\@nil%
  \extractkey/zx/wires definition/NN.=-andL config (default \{defaultN,symmetry-L,defaultNN\})\@nil%
  \extractkey/zx/wires definition/.NN=-andL config (default \{defaultN,symmetry-L,defaultNN\})\@nil%
  \extractkey/zx/wires definition/NIN=-andL config (default \{defaultN,symmetry-L,defaultNN,defaultNIN\})\@nil%
  \extractkey/zx/wires definition/INN=-andL config (default \{defaultN,symmetry-L,defaultNN,defaultNIN,symmetry\})\@nil%
  \extractkey/zx/wires definition/NNI=-andL config (default \{defaultN,symmetry-L,defaultNN,defaultNIN,symmetry\})\@nil%
  \makeatother
  \pgfmanualbody
  Like |N| but for diagrams read up-to-down or down-to-up. The |.| are maintly used for backward compatibility with |IO| style.
% \begin{codeexample}[width=3cm]
%   \begin{ZX}
%     \zxX{\alpha} \ar[NN,rd] \\
%                                & \zxZ{\beta}\\
%                                & \zxZ{\alpha}\\
%     \zxX{\beta} \ar[NN,ru]
%   \end{ZX}
% \end{codeexample}
% |I| forces the angle on the side of |I| to be vertical.
% \begin{codeexample}[width=3cm]
%   \begin{ZX}
%     \zxX{\alpha} \ar[NIN,rd] \\
%                                & \zxZ{\beta}\\
%                                & \zxZ{\alpha}\\
%     \zxX{\beta} \ar[NNI,ru]
%   \end{ZX}
% \end{codeexample}
% It can be configured like |N| using |-|, |L|\dots{}
\end{pgfmanualentry}

\begin{pgfmanualentry}
  \makeatletter
  \def\extrakeytext{style, }
  \extractkey/zx/wires definition/<'=-andL config (default like N-)\@nil%
  \extractkey/zx/wires definition/<.=-andL config (default like N-)\@nil%
  \extractkey/zx/wires definition/'>=-andL config (default like -N)\@nil%
  \extractkey/zx/wires definition/.>=-andL config (default like -N)\@nil%
  %\extractkey/zx/wires definition/^.=-andL config (default 60)\@nil%
  %\extractkey/zx/wires definition/.^=-andL config (default 60)\@nil%
  \extractkey/zx/wires definition/'v=-andL config (default like INN)\@nil%
  \extractkey/zx/wires definition/v'=-andL config (default like NNI)\@nil%
  \makeatother
  \pgfmanualbody
  |<'| and |<.| are similar to |N-|, except that the anchor of the vertical line is put on the boundary (similarly for |*>| and |-N|, |*v*| and |INN|, and |*^*| and |NIN|: |.^| and |^.| were not possible to put in this documentation since the documentation package does not like the |^| character). The position of |'| and |.| does not really matters anymore in new versions, but for backward compatibility with |IO| styles, and maybe forward compatibility (another style may need this information), it's cleaner to put |.| or |'| on the direction of the wire. It also helps the reader of your diagrams to see the shape of the wire.
\begin{codeexample}[width=0cm]
\begin{ZX}
  \zxN{}                         & \zxZ{}\\
  \zxX{} \ar[ru,<'] \ar[rd,<.] \\
  \zxN{}                         & \zxZ{}\\
\end{ZX}
\end{codeexample}
\begin{codeexample}[width=0cm]
\begin{ZX}
  \zxN{}                         & \zxZ{}\\
  \zxX{} \ar[ru,.>] \ar[rd,'>] \\
  \zxN{}                         & \zxZ{}\\
\end{ZX}
\end{codeexample}
\begin{codeexample}[width=0cm]
\begin{ZX}
  \zxN{} & \zxFracX{\pi}{2} \ar[dl,.^] \ar[dr,^.] & \\
  \zxZ{} &                                & \zxX{}
\end{ZX}
\end{codeexample}
\begin{codeexample}[width=0cm]
\begin{ZX}
  \zxZ{} &                                & \zxX{}\\
  \zxN{} & \zxX{} \ar[ul,'v] \ar[ur,v'] &
\end{ZX}
\end{codeexample}
\end{pgfmanualentry}


\begin{pgfmanualentry}
  \makeatletter
  \def\extrakeytext{style, }
  \extractkey/zx/wires definition/3 dots=text (default =)\@nil%
  \extractkey/zx/wires definition/3 vdots=text (default =)\@nil%
  \makeatother
  \pgfmanualbody
  The styles put in the middle of the wire (without drawing the wire) $\dots$ (for |3 dots|) or $\vdots$ (for |3 vdots|). The dots are scaled according to |\zxScaleDots| and the text \meta{text} is written on the left. Use |&[\zxDotsRow]| and |\\[\zxDotsRow]| to properly adapt the spacing of columns and rows.
\begin{codeexample}[width=3cm]
\begin{ZX}
  \zxZ{\alpha} \ar[r,o'] \ar[r,o.]
               \ar[r,3 dots]
               \ar[d,3 vdots={$n$\,}] &[\zxDotsCol] \zxFracX{\pi}{2}\\[\zxDotsRow]
  \zxZ{\alpha} \rar             & \zxFracX{\pi}{2}
\end{ZX}
\end{codeexample}
\end{pgfmanualentry}

\begin{pgfmanualentry}
  \makeatletter
  \def\extrakeytext{style, }
  \extractkey/zx/wires definition/H=style (default {})\@nil%
  \extractkey/zx/wires definition/Z=style (default {})\@nil%
  \extractkey/zx/wires definition/X=style (default {})\@nil%
  \makeatother
  \pgfmanualbody
  Adds a |H| (Hadamard), |Z| or |X| node (without phase) in the middle of the wire. Width of column or rows should be adapted accordingly using |\zxNameRowcolFlatnot| where |Name| is replaced by |H|, |S| (for ``spiders'', i.e.\ |X| or |Z|), |HS| (for both |H| and |S|) or |W|, |Rowcol| is either |Row| (for changing row sep) or |Col| (for changing column sep) and |Flatnot| is empty or |Flat| (if the wire is supposed to be a straight line as it requires more space). For instance:
\begin{codeexample}[width=3cm]
\begin{ZX}
  \zxZ{\alpha} \ar[d] \ar[r,o',H] \ar[r,o.,H] &[\zxHCol] \zxX{\beta}\\
  \zxZ{\alpha}  \ar[d,-o,X] \ar[d,o-,Z]                        \\[\zxHSRow]
  \zxX{\gamma}
\end{ZX}
\end{codeexample}
The \meta{style} parameter can be used to add additional \tikzname{} style to the nodes, notably a position using |\ar[rd,-N.,H={pos=.35}]|. The reason for using that is that the wires start inside the nodes, therefore the ``middle'' of the wire is closer to the node when the other side is on an empty node.
\begin{codeexample}[width=0pt]
\begin{ZX}[zx row sep=0pt]
 \zxN{} \ar[rd,-N.,H={pos=.35}] &[\zxwCol,\zxHCol]  &[\zxwCol,\zxHCol] \zxN{} \\[\zxNRow]%%
                                & \zxX{\alpha}
                                     \ar[ru,N'-,H={pos=1-.35}]
                                     \ar[rd,N.-,H={pos=1-.35}] &  \\[\zxNRow]
 \zxN{} \ar[ru,-N',H={pos=.35}]                                &  & \zxN{}
\end{ZX}
\end{codeexample}
Note that it's possible to automatically start wires on the border of the node, but it is slower and create other issues, see \cref{subsec:wiresInsideOutside} for more details. The second option (also presented in this section) is to manually define the |start anchor| and |end anchor|, but it can change the shape of the wire).
\end{pgfmanualentry}

\begin{pgfmanualentry}
  \makeatletter
  \def\extrakeytext{style, }
  \extractkey/zx/wires definition/wire centered\@nil%
  \extractkey/zx/wires definition/wc\@nil%
  \extractkey/zx/wires definition/wire centered start\@nil%
  \extractkey/zx/wires definition/wcs\@nil%
  \extractkey/zx/wires definition/wire centered end\@nil%
  \extractkey/zx/wires definition/wce\@nil%
  \makeatother
  \pgfmanualbody
  Forces the wires to start at the center of the node (|wire centered start|, alias |wcs|), to end at the center of the node (|wire centered end|, alias |wce|) or both (|wire centered|, alias |wc|). This may be useful, for instance in the old |IO| mode (see below) when nodes have different sizes (the result looks strange otherwise), or with some wires (like |C|) connected to |\ZxNone+| (if possible, use |\zxNone| (without any embellishment) since it does not suffer from this issue as it is a coordinate).

  See also |between none| to also increase looseness when connecting only wires (use |between none| \emph{only} in |IO| mode).
\begin{codeexample}[width=3cm]
\begin{ZX}
  \zxZ{} \ar[IO,o',r] \ar[IO,o.,r]       & \zxX{\alpha}\\
  \zxZ{} \ar[IO,o',r,wc] \ar[IO,o.,r,wc] & \zxX{\alpha}
\end{ZX}
\end{codeexample}
% Without |wc| (note that because there is no node, we need to use |&[\zxWCol]| (for columns) and |\\[\zxWRow]| (for rows) to get nicer spacing):
% \begin{codeexample}[width=3cm]
% \zx{\zxNone{} \rar &[\zxWCol] \zxNone{} \rar &[\zxWCol] \zxNone{} }
% \end{codeexample}
% With |wc|:
% \begin{codeexample}[width=3cm]
% \zx{\zxNone{} \rar[wc] &[\zxWCol] \zxNone{} \rar[wc] &[\zxWCol] \zxNone{}}
% \end{codeexample}
\end{pgfmanualentry}


\begin{pgfmanualentry}
  \makeatletter
  \def\extrakeytext{style, }
  \extractkey/zx/wires definition/bezier=-andL config \@nil%
  \extractkey/zx/wires definition/bezier x=-andL config \@nil%
  \extractkey/zx/wires definition/bezier y=-andL config \@nil%
  \extractkey/zx/wires definition/bezier 4=\{x1\}\{y1\}\{x2\}\{y2\} \@nil%
  \extractkey/zx/wires definition/bezier 4 x=\{x1\}\{y1\}\{x2\}\{y2\} \@nil%
  \extractkey/zx/wires definition/bezier 4 y=\{x1\}\{y1\}\{x2\}\{y2\} \@nil%
  \makeatother
  \pgfmanualbody
  To create a bezier wire. These styles are not really meant to be used for the final user because they are long to type and would not be changed document-wise when the style is changed, but most styles are based on these styles. For the |bezier 4 *| versions, the two first arguments are the relative position of the first anchor (|x| and |y| position), the next two of the second anchor. In the |bezier *| versions, the value of |1-| will be the relative |x| position of the first point, |1L| the relative position of the second, and |2-| and |2L| will be for the second point (the advantage of this is that it is also possible to specify angles using |1al={angle}{length}|\dots{} as explained in the |-andL| syle). They are said to be relative in the sense that |{0}{0}| is the coordinate of the first point, and |{1}{1}| the second point. The |bezier x| and |bezier 4 y| are useful when the node are supposed to be horizontally or vertically aligned: the distance are now expressed as a fraction of the horizontal (respectively vertical) distance between the two nodes. Using relative coordinates has the advantage that if the nodes positions are moved, the aspect of the wire does not change (it is just squeezed), while this is not true with |in|/|out| wires which preserves angles but not shapes.
\end{pgfmanualentry}


\subsubsection{IO wires, the old styles}\label{subsub:IOwires}

\begin{stylekey}{/zx/wires definition/IO}
As explained above, wires were first defined using |in|, |out| and |looseness|, but this turned out to be sometimes hard to use since the shape of the wire was changing depending on the position. For example consider the differences between the older version:
\begin{codeexample}[]
\begin{ZX}
  \zxN{}                       & \zxZ{} \\
  \zxZ{} \ar[ru,IO,N'] \ar[rd,IO,N.] &\\
                               & \zxZ{} \\
\end{ZX}
\begin{ZX}
  \zxN{}                       & \zxZ{} \\[-3pt]
  \zxZ{} \ar[ru,IO,N'] \ar[rd,IO,N.] &\\[-3pt]
                               & \zxZ{} \\
\end{ZX}
\begin{ZX}
  \zxN{}                       & \zxZ{} \\[-5pt]
  \zxZ{} \ar[ru,IO,N'] \ar[rd,IO,N.] &\\[-5pt]
                               & \zxZ{} \\
\end{ZX}
\end{codeexample}
\begin{codeexample}[]
\begin{ZX}
  \zxN{}                       & \zxZ{} \\
  \zxZ{} \ar[ru,N'] \ar[rd,N.] &\\
                               & \zxZ{} \\
\end{ZX}
and the newer:
\begin{ZX}
  \zxN{}                       & \zxZ{} \\[-3pt]
  \zxZ{} \ar[ru,N'] \ar[rd,N.] &\\[-3pt]
                               & \zxZ{} \\
\end{ZX}
\begin{ZX}
  \zxN{}                       & \zxZ{} \\[-5pt]
  \zxZ{} \ar[ru,N'] \ar[rd,N.] &\\[-5pt]
                               & \zxZ{} \\
\end{ZX}
\end{codeexample}
Here is another example:
\begin{codeexample}[]
Before \begin{ZX}
  \zxNone{} \ar[IO,C,d,wc] \ar[rd,IO,s] &[\zxWCol] \zxNone{} \\[\zxWRow]
  \zxNone{}                \ar[ru,IO,s] &          \zxNone{}
\end{ZX} after \begin{ZX}
  \zxNone{} \ar[C,d] \ar[rd,s] &[\zxWCol] \zxNone{} \\[\zxWRow]
  \zxNone{}          \ar[ru,s] &          \zxNone{}
\end{ZX}
\end{codeexample}
This example led to the creation of the |bn| style, in order to try to find appropriate looseness values depending on the case\dots{} but it is harder to use and results are less predictable.

The new method also allowed us to use |N| for both |N.| and |N'| styles (however we kept both versions for backward compatibility and in case later we want to make a distinction between nodes going doing or up).

However, if you prefer the old style, you can just use them by adding |IO,| in front of the style name (styles are nested inside |IO|). Note however that the customization options are of course different.
\end{stylekey}

We list now the older |IO| styles:

\begin{pgfmanualentry}
  \makeatletter
  \def\extrakeytext{style, }
  \extractkey/zx/wires definition/IO/C\@nil%
  \extractkey/zx/wires definition/IO/C\@nil%
  \extractkey/zx/wires definition/IO/C'\@nil%
  \extractkey/zx/wires definition/IO/C-\@nil%
  \makeatother
  \pgfmanualbody
  |IO| mode for the |C| wire (used for Bell-like shapes).
\begin{codeexample}[]
  Bell pair \zx{\zxNone{} \ar[d,IO,C] \\[\zxWRow]
                \zxNone{}}
  and funny graph
  \begin{ZX}
    \zxX{} \ar[d,IO,C] \ar[r,C']  & \zxZ{} \ar[d,IO,C-]\\
    \zxZ{} \ar[r,IO,C.]           & \zxX{}
  \end{ZX}.
\end{codeexample}
Note that the |IO| version cannot really be used when nodes are not aligned (using |wc| can sometimes help with the alignment):
\begin{codeexample}[]
  Normal \begin{ZX}
    \zxX{\alpha} \ar[dr,C]\\
    & \zxNone{}
  \end{ZX}, and IO \begin{ZX}
    \zxX{\alpha} \ar[dr,IO,C]\\
    & \zxNone{}
  \end{ZX}
\end{codeexample}
\end{pgfmanualentry}

\begin{pgfmanualentry}
  \makeatletter
  \def\extrakeytext{style, }
  \extractkey/zx/wires definition/IO/o'=angle (default 40)\@nil%
  \extractkey/zx/wires definition/IO/o.=angle (default 40)\@nil%
  \extractkey/zx/wires definition/IO/o-=angle (default 40)\@nil%
  \extractkey/zx/wires definition/IO/-o=angle (default 40)\@nil%
  \makeatother
  \pgfmanualbody
  |IO| version of |o|. Curved wire, similar to |C| but with a soften angle (optionally specified via \meta{angle}, and globally editable with |\zxDefaultLineWidth|). Again, the symbols specify which part of the circle (represented with |o|) must be kept.
\begin{codeexample}[width=3cm]
  \begin{ZX}
    \zxX{} \ar[d,IO,-o] \ar[d,IO,o-]\\
    \zxZ{} \ar[r,IO,o'] \ar[r,IO,o.] & \zxX{}
  \end{ZX}.
\end{codeexample}
 Note that these wires can be combined with |H|, |X| or |Z|, in that case one should use appropriate column and row spacing as explained in their documentation:
\begin{codeexample}[width=3cm]
  \begin{ZX}
    \zxX{\alpha} \ar[d,IO,-o,H] \ar[d,IO,o-,H]\\[\zxHRow]
    \zxZ{\beta} \rar & \zxZ{} \ar[r,IO,o',X] \ar[r,IO,o.,Z] &[\zxSCol] \zxX{}
  \end{ZX}.
\end{codeexample}
\end{pgfmanualentry}

\begin{pgfmanualentry}
  \makeatletter
  \def\extrakeytext{style, }
  \extractkey/zx/wires definition/IO/(=angle (default 30)\@nil%
  \extractkey/zx/wires definition/IO/)=angle (default 30)\@nil%
  \extractkey/zx/wires definition/IO/('=angle (default 30)\@nil%
  \extractkey/zx/wires definition/IO/('=angle (default 30)\@nil%
  \makeatother
  \pgfmanualbody
  |IO| version of |(| (so far they are the same, but it may change later, use this version if you want to play with |looseness|). Curved wire, similar to |o| but can be used for diagonal items. The angle is, like in |bend right|, the opening angle from the line which links the two nodes. For the first two commands, the |(| and |)| symbols must be imagined as if the starting point was on top of the parens, and the ending point at the bottom.
\begin{codeexample}[width=3cm]
  \begin{ZX}
    \zxX{} \ar[rd,IO,(] \ar[rd,IO,),red]\\
    & \zxZ{}
  \end{ZX}.
\end{codeexample}
Then, |('|=|(| and |(.|=|)|; this notation is, I think, more intuitive when linking nodes from left to right. |('| is used when going to top right and |(.| when going to bottom right.
\begin{codeexample}[width=3cm]
\begin{ZX}
  \zxN{}                       & \zxX{}\\
  \zxZ{} \ar[ru,IO,('] \ar[IO,rd,(.] & \\
                               & \zxX{}
\end{ZX}
\end{codeexample}
When the nodes are too far appart, the default angle of |30| may produce strange results as it will go above (for |('|) the vertical line. Either choose a shorter angle, or see |<'| instead.
\end{pgfmanualentry}

\begin{pgfmanualentry}
  \makeatletter
  \def\extrakeytext{style, }
  \extractkey/zx/wires definition/IO/s\@nil%
  \extractkey/zx/wires definition/IO/s'=angle (default 30)\@nil%
  \extractkey/zx/wires definition/IO/s.=angle (default 30)\@nil%
  \extractkey/zx/wires definition/IO/-s'=angle (default 30)\@nil%
  \extractkey/zx/wires definition/IO/-s.=angle (default 30)\@nil%
  \extractkey/zx/wires definition/IO/s'-=angle (default 30)\@nil%
  \extractkey/zx/wires definition/IO/s.-=angle (default 30)\@nil%
  \makeatother
  \pgfmanualbody
  |IO| version of |s|. |s| is used to create a s-like wire, to have nicer soften diagonal lines between nodes. Other versions are soften versions (the input and output angles are not as sharp, and the difference angle can be configured as an argument or globally using |\zxDefaultSoftAngleS|). Adding |'| or |.| specifies if the wire is going up-right or down-right.
\begin{codeexample}[width=3cm]
  \begin{ZX}
    \zxX{\alpha} \ar[IO,s,rd] \\
                               & \zxZ{\beta}\\
    \zxX{\alpha} \ar[IO,s.,rd] \\
                               & \zxZ{\beta}\\
                               & \zxZ{\alpha}\\
    \zxX{\beta} \ar[IO,s,ru] \\
                               & \zxZ{\alpha}\\
    \zxX{\beta} \ar[IO,s',ru] \\
  \end{ZX}
\end{codeexample}
|-| forces the angle on the side of |-| to be horizontal.
\begin{codeexample}[width=3cm]
  \begin{ZX}
    \zxX{\alpha} \ar[IO,s.,rd] \\
                           & \zxZ{\beta}\\
    \zxX{\alpha} \ar[IO,-s.,rd] \\
                           & \zxZ{\beta}\\
    \zxX{\alpha} \ar[IO,s.-,rd] \\
                           & \zxZ{\beta}\\
  \end{ZX}
\end{codeexample}
\end{pgfmanualentry}

\begin{pgfmanualentry}
  \makeatletter
  \def\extrakeytext{style, }
  \extractkey/zx/wires definition/IO/ss\@nil%
  \extractkey/zx/wires definition/IO/ss.=angle (default 30)\@nil%
  \extractkey/zx/wires definition/IO/.ss=angle (default 30)\@nil%
  \extractkey/zx/wires definition/IO/sIs.=angle (default 30)\@nil%
  \extractkey/zx/wires definition/IO/.sIs=angle (default 30)\@nil%
  \extractkey/zx/wires definition/IO/ss.I-=angle (default 30)\@nil%
  \extractkey/zx/wires definition/IO/I.ss-=angle (default 30)\@nil%
  \makeatother
  \pgfmanualbody
  |IO| version of |ss|. |ss| is similar to |s| except that we go from top to bottom instead of from left to right. The position of |.| says if the node is wire is going bottom right (|ss.|) or bottom left (|.ss|).
\begin{codeexample}[width=3cm]
  \begin{ZX}
    \zxX{\alpha} \ar[IO,ss,rd] \\
                           & \zxZ{\beta}\\
    \zxX{\alpha} \ar[IO,ss.,rd] \\
                           & \zxZ{\beta}\\
                           & \zxX{\beta} \ar[IO,.ss,dl] \\
    \zxZ{\alpha}\\
                           & \zxX{\beta} \ar[IO,.ss,dl] \\
    \zxZ{\alpha}\\
  \end{ZX}
\end{codeexample}
|I| forces the angle above (if in between the two |s|) or below (if on the same side as |.|) to be vertical.
\begin{codeexample}[width=3cm]
  \begin{ZX}
    \zxX{\alpha} \ar[IO,ss,rd] \\
                           & \zxZ{\beta}\\
    \zxX{\alpha} \ar[IO,sIs.,rd] \\
                           & \zxZ{\beta}\\
    \zxX{\alpha} \ar[IO,ss.I,rd] \\
                           & \zxZ{\beta}\\
                           & \zxX{\beta} \ar[IO,.sIs,dl] \\
    \zxZ{\alpha}\\
                           & \zxX{\beta} \ar[IO,I.ss,dl] \\
    \zxZ{\alpha}\\
  \end{ZX}
\end{codeexample}
\end{pgfmanualentry}

\begin{pgfmanualentry}
  \makeatletter
  \def\extrakeytext{style, }
  \extractkey/zx/wires definition/IO/<'=angle (default 60)\@nil%
  \extractkey/zx/wires definition/IO/<.=angle (default 60)\@nil%
  \extractkey/zx/wires definition/IO/'>=angle (default 60)\@nil%
  \extractkey/zx/wires definition/IO/.>=angle (default 60)\@nil%
  %\extractkey/zx/wires definition/IO/^.=angle (default 60)\@nil%
  %\extractkey/zx/wires definition/IO/.^=angle (default 60)\@nil%
  \extractkey/zx/wires definition/IO/'v=angle (default 60)\@nil%
  \extractkey/zx/wires definition/IO/v'=angle (default 60)\@nil%
  \makeatother
  \pgfmanualbody
  These keys are a bit like |('| or |(.| but the arrival angle is vertical (or horizontal for the |^| (up-down) and |v| (down-up) versions). As before, the position of the decorator |.|,|'| denote the direction of the wire.
\begin{codeexample}[width=0cm]
\begin{ZX}
  \zxN{}                         & \zxZ{}\\
  \zxX{} \ar[IO,ru,<'] \ar[IO,rd,<.] \\
  \zxN{}                         & \zxZ{}\\
\end{ZX}
\end{codeexample}
\begin{codeexample}[width=0cm]
\begin{ZX}
  \zxN{} & \zxFracX{\pi}{2} \ar[IO,dl,.^] \ar[IO,dr,^.] & \\
  \zxZ{} &                                & \zxX{}
\end{ZX}
\end{codeexample}
\begin{codeexample}[width=0cm]
\begin{ZX}
  \zxZ{} &                                & \zxX{}\\
  \zxN{} & \zxX{} \ar[IO,ul,'v] \ar[IO,ur,v'] &
\end{ZX}
\end{codeexample}
\end{pgfmanualentry}

\begin{pgfmanualentry}
  \makeatletter
  \def\extrakeytext{style, }
  \extractkey/zx/wires definition/IO/N'=angle (default 60)\@nil%
  \extractkey/zx/wires definition/IO/N.=angle (default 60)\@nil%
  \extractkey/zx/wires definition/IO/-N'=angle (default 60)\@nil%
  \extractkey/zx/wires definition/IO/-N.=angle (default 60)\@nil%
  \extractkey/zx/wires definition/IO/N'-=angle (default 60)\@nil%
  \extractkey/zx/wires definition/IO/N.-=angle (default 60)\@nil%
  \makeatother
  \pgfmanualbody
  |IO| version of |N|. |N| is used to create a wire leaving at wide angle and arriving at wide angle. Adding |'| or |.| specifies if the wire is going up-right or down-right.
\begin{codeexample}[width=3cm]
  \begin{ZX}
    \zxX{\alpha} \ar[IO,N.,rd] \\
                               & \zxZ{\beta}\\
                               & \zxZ{\alpha}\\
    \zxX{\beta} \ar[IO,N',ru] \\
  \end{ZX}
\end{codeexample}
|-| forces the angle on the side of |-| to be horizontal.
\begin{codeexample}[width=3cm]
  \begin{ZX}
    \zxX{\alpha} \ar[IO,-N.,rd] \\
                               & \zxZ{\beta}\\
                               & \zxZ{\alpha}\\
    \zxX{\beta} \ar[IO,N'-,ru] \\
  \end{ZX}
\end{codeexample}
\end{pgfmanualentry}

\begin{stylekey}{/zx/wires definition/ls=looseness}
  Shortcut for |looseness|, allows to quickly redefine looseness. Use with care (or redefine style directly), and \emph{do not use on styles that are not in |IO|}, since they don't use the |in|/|out| mechanism (only |(|-like styles use |in|/|out|\dots{} for now. In case you want to change looseness of |(|, prefer to use |IO,(| as it is guaranteed to be backward compatible). We may try later to give a key |looseness| for these styles, but it's not the case for now. For |IO| styles, you can also change yourself other values, like |in|, |out|\dots
\begin{codeexample}[]
\begin{ZX}
  \zxZ{} \ar[rd,s] \\
                   & \zxX{}\\
  \zxZ{} \ar[rd,IO,s] \\
                   & \zxX{}\\
  \zxZ{} \ar[rd,IO,s,ls=3] \\
                   & \zxX{}
\end{ZX}
\end{codeexample}
\end{stylekey}

\begin{pgfmanualentry}
  \makeatletter
  \def\extrakeytext{style, }
  \extractkey/zx/wires definition/between none\@nil%
  \extractkey/zx/wires definition/bn\@nil%
  \makeatother
  \pgfmanualbody
  When drawing only |IO| wires (normal wires would suffer from this parameter), the default looseness may not be good looking and holes may appear in the line. This style (whose alias is |bn|) should therefore be used when curved wires \emph{from the |IO| path} are connected together. In that case, also make sure to separate columns using |&[\zxWCol]| and rows using |\\[\zxWRow]|.
\begin{codeexample}[width=3cm]
A swapped Bell pair is %
\begin{ZX}
  \zxNone{} \ar[IO,C,d,wc] \ar[rd,IO,s,bn] &[\zxWCol] \zxNone{} \\[\zxWRow]
  \zxNone{}                \ar[ru,IO,s,bn] &          \zxNone{}
\end{ZX}
\end{codeexample}
\end{pgfmanualentry}

\begin{pgfmanualentry}
  \makeatletter
  \def\extrakeytext{style, }
  \extractkey/zx/wires definition/bold\@nil%
  \extractkey/zx/wires definition/B\@nil%
  \extractkey/zx/wires definition/non bold\@nil%
  \extractkey/zx/wires definition/NB\@nil%
  \extractkey/zx/wires definition/boldn\@nil%
  \extractkey/zx/wires definition/boldn-\@nil%
  \extractkey/zx/wires definition/boldn'\@nil%
  \extractkey/zx/wires definition/boldn.\@nil%
  \extractkey/zx/wires definition/Bn=message (default n)\@nil%
  \extractkey/zx/wires definition/Bn-=message (default n)\@nil%
  \extractkey/zx/wires definition/Bn'=message (default n)\@nil%
  \extractkey/zx/wires definition/Bn.=message (default n)\@nil%
  \extractkey/zx/wires definition/BnArgs=\{message\}\{style\} (default \{n\}\{\})\@nil%
  \extractkey/zx/wires definition/Bn-Args=\{message\}\{style\} (default \{n\}\{\})\@nil%
  \extractkey/zx/wires definition/Bn'Args=\{message\}\{style\} (default \{n\}\{\})\@nil%
  \extractkey/zx/wires definition/Bn.Args=\{message\}\{style\} (default \{n\}\{\})\@nil%
  \makeatother
  \pgfmanualbody
  Creates a bold (or non-bold) wire (|B| and |NB| being short aliases). The versions with a |n| at the end adds a $n$ on the right/left/above/below (for the respective symbols empty, |-|, |'|, |.|), that you can overwrite by providing an option:
{\catcode`\|=12 % Ensures | is not anymore \verb|...|
\begin{codeexample}[width=0pt]
\begin{ZX}
  \zxX[bold]{} \rar[Bn',o'] \rar[Bn.=m,o.] &[\zxwCol] \zxZ{} \rar &[\zxwCol] \zxN{}
\end{ZX}
\end{codeexample}
}
Finally, the last variations with args like |BnArgs| allows you to provide an additional label style, notably arguments like |pos=.7| to change the position of the line in the path (when the lines are drawn from the middle of the nodes, which is the case for many styles by default for efficiency reasons, the middle of the path may appear poorly centered: using |pos| is one method to center it back, see also \cref{subsec:wiresInsideOutside}).
{\catcode`\|=12 % Ensures | is not anymore \verb|...|
\begin{codeexample}[width=0pt]
\begin{ZX}
  \zxX[bold]{} \rar[<.,Bn.Args={n}{pos=.6},dr] &[\zxwCol] \\
                                               & \zxN{}
\end{ZX}
\end{codeexample}
}
Note that |bold| and its alias |B| can also be used as an argument to the |ZX| environment to turn all spiders and wires in bold:
\begin{codeexample}[width=0pt]
\begin{ZX}[bold]
  \zxX{} \rar[o'] \rar[o.] & \zxZ{} \rar &[\zxwCol] \zxN{}
\end{ZX}
\end{codeexample}
You might also want to combine it with the |non bold| option to temporarily set a wire non-bold:
\begin{codeexample}[width=0pt]
\begin{ZX}[bold]
  \zxX{} \rar[o'] \rar[o.] & \zxZ{} \rar[non bold] &[\zxwCol] \zxN{}
\end{ZX}
\end{codeexample}
\end{pgfmanualentry}

\begin{pgfmanualentry}
  \makeatletter
  \def\extrakeytext{style, }
  \extractkey/zx/styles/rounded style/wires bold\@nil%
  \extractkey/zx/styles/rounded style/Bw\@nil%
  \extractkey/zx/styles/rounded style/BBw\@nil%
  \makeatother
  \pgfmanualbody
  If you set |wires bold| (alias |Bw|) to a node, all wired connected to that node will be bold. |BBw| additionally turn the current node bold.
\begin{codeexample}[width=0pt]
\begin{ZX}
  \zxX{} \rar[o'] \rar[o.] & \zxZ[BBw]{} \rar &[\zxwCol] \zxN{}
\end{ZX}
\end{codeexample}
\end{pgfmanualentry}


\subsection{Custom nodes}\label{subsec:customNodes}

\textbf{NB}: this functionality was added on 24/02/2023.

Technically, there is nothing preventing a user to create any style to create custom nodes like:
{\catcode`\|=12 % Ensures | is not anymore \verb|...|
\begin{codeexample}[width=0pt]
  \begin{ZX}
    \zxN{} \rar & |[draw,fill=blue!50,draw=blue,inner sep=3mm]| \text{my custom node}
  \end{ZX}
\end{codeexample}
}
possibly using |\tikzset| and/or |\NewExpandableDocumentCommand| to avoid repetitions:
{\catcode`\|=12 % Ensures | is not anymore \verb|...|
\begin{codeexample}[width=0pt]
  \tikzset{
    my custom node/.style={
      draw,fill=blue!50,draw=blue,inner sep=3mm
    },
  }
  \NewExpandableDocumentCommand{\myCustomNode}{m}{|[my custom node]| \text{#1}}
  \begin{ZX}
    \zxN{} \rar & \myCustomNode{Hey}
  \end{ZX}
\end{codeexample}
}
However, this method has several limitations: first, it is non-trivial to define complex shapes this way (you have a single node, so you need to use more advanced tikz primitives to make it right), it is even harder to create variations of this shape without repeating yourself (like different rotations of the same shape), and it is quite hard to define anchors to start/stop arrows like |\rar[<']| at the right position\footnote{In that case, the center of the shape will be used even if it might not be desirable}… not even mentioning multiple anchors. We therefore provide below some helper functions.

\begin{pgfmanualentry}
  \def\extrakeytext{style, }
  \extractcommand\zxNewNodeFromPic\marg{name new node}\oarg{style before user}\oarg{style after user}\marg{code}\@@
  \makeatletter% should not be letter for \@@... strange
  \extractkey/tikz/zx create anchors=\marg{list, of, coordinates, to, turn, in, anchor}\@nil%
  \extractkey/tikz/invert top bottom\@nil%
  \extractkey/tikz/start subnode=\marg{name of subnode}\@nil%
  \extractkey/tikz/stop subnode=\marg{name of subnode}\@nil%
  \makeatother
  \pgfmanualbody
  |\zxNewNodeFromPic{Name}[style before][style after]{code}| is used to create a new command with name |\zxName|. |code| is the code to draw the node (that you would use in a |\pic| environment, and that will be centered on |(0,0)|):
{\catcode`\|=12 % Ensures | is not anymore \verb|...|
\begin{codeexample}[width=0pt]
  \zxNewNodeFromPic{MyNode}{
    \node[draw,fill=blue!50,draw=blue,inner sep=3mm, zx main node, alias=mynode] at (0,0) {};
    \node[draw,fill=red!50,draw=blue,inner sep=1mm] at (mynode.north east) {};
  }
  \begin{ZX}
    \zxN{} \rar & \zxMyNode{} \rar & \zxMyNode{}
  \end{ZX}
\end{codeexample}
}
Note how giving |zx main node| to a node makes it the target of arrows (note that you should make sure that your picture has at least one main node if you want features like |debug mode| to work). Without this, the lines would go to the coordinate |(0,0)|, and go through all the nodes:
{\catcode`\|=12 % Ensures | is not anymore \verb|...|
\begin{codeexample}[width=0pt]
  \zxNewNodeFromPic{MyNode}{
    \node[draw,fill=blue!50,draw=blue,inner sep=3mm, alias=mynode] at (0,0) {};
    \node[draw,fill=red!50,draw=blue,inner sep=1mm] at (mynode.north east) {};
  }
  \begin{ZX}
    \zxN{} \rar & \zxMyNode{}
  \end{ZX}
\end{codeexample}
}
Note that sometimes, one might want to draw lines to a fixed point, while leaving the line below the object (for instance if you use a |rounded corners| option and try to enter by the corner, tikz will actually stop before which is not beautiful). A simple solution is to use |wc| (short for |wire centered|) on the arrow, and it will automatically move the path on a layer behind (see also |wcs| and |wce| to configure only the starting or ending points, and some other nodes also use this, notably most styles based on |bezier|, like |s|):
{\catcode`\|=12 % Ensures | is not anymore \verb|...|
\begin{codeexample}[width=0pt]
\zxNewNodeFromPic{MyDivider}{
  \node[regular polygon, regular polygon sides=3, shape border rotate=90,
        draw=black,fill=gray!50, inner sep=1.6pt, rounded corners=0.8mm,zx main node] {};
}
Compare %
\begin{ZX}
  \zxZ{} \rar[very thick]  &\zxMyDivider{} \rar[very thick] & \zxN{}
\end{ZX} %
and %
\begin{ZX}
  \zxZ{} \rar[very thick,wce]  &\zxMyDivider{} \rar[very thick] & \zxN{} 
\end{ZX}
\end{codeexample}
}
Note that I'm working on a better solutions to move nodes on different layers, or automatically see which anchor to choose depending on the custom node… but it's not that \mylink{https://tex.stackexchange.com/questions/618823/node-on-layer-style-in-tikz-matrix-tikzcd}{easy}.

We can similarly put the |zx main node| on the other node to point to the other node by default:
{\catcode`\|=12 % Ensures | is not anymore \verb|...|
\begin{codeexample}[width=0pt]
  \zxNewNodeFromPic{MyNode}{
    \node[draw,fill=blue!50,draw=blue,inner sep=3mm, alias=mynode] at (0,0) {};
    \node[draw,fill=red!50,draw=blue,inner sep=1mm, zx main node] at (mynode.north east) {};
  }
  \begin{ZX}
    A \rar[end anchor=center,on layer=main] & \zxMyNode{} \rar & B & \zxMyNode{} & \zxMyNode{}
  \end{ZX}
\end{codeexample}
}

\paragraph{Add margin around nodes.} You might want to add empty paths to increase the ``margin'' around your picture, or |overlay| some nodes if you don't want them to be counted when computing the bounding box. For simplicity, we provide a command |\zxExtendBoundingBox| that extends the current bounding box. This command takes as argument a list of arguments like |left|, |right|, |top|, |bottom|, |horizontal|, |vertical|, |extend|, where each argument takes a distance and extends to the corresponding direction (|extend| extends in all directions). For instance, below we increase the box by 1cm horizontally and by 3mm on each side on the y axis:
{\catcode`\|=12 % Ensures | is not anymore \verb|...|
\begin{codeexample}[width=0pt]
  \zxNewNodeFromPic{MyNode}{
    \node[draw,fill=blue!50,draw=blue,inner sep=3mm, zx main node] at (0,0) {};
    \zxExtendBoundingBox{horizontal=1cm, vertical=3mm}
  }
  \begin{ZX}
    A \rar & \zxMyNode{} \rar & B\\
  \end{ZX}
\end{codeexample}
}

\paragraph{Give text as argument.} It is also possible to pass text as an argument by simply adding |\tikzpictext| where you want the text to appear, and calling then |\zxMyNode{your text}|:
{\catcode`\|=12 % Ensures | is not anymore \verb|...|
\begin{codeexample}[width=0pt]
  \zxNewNodeFromPic{MyNode}{
    \node[draw,fill=blue!50,draw=blue,inner sep=3mm, zx main node, alias=mynode] at (0,0) {\tikzpictext};
    \node[draw,fill=red!50,draw=blue,inner sep=1mm] at (mynode.north east) {};
  }
  \begin{ZX}
    \zxMyNode{Hello} & \zxMyNode{do you see} & \zxMyNode{the text} & ?
  \end{ZX}
\end{codeexample}
}

\paragraph{Pass arbitrary arguments.} You can actually pass virtually any argument(s) by using the optional argument of the created |\zxMyNode| (that is just an additional argument to the pic options, inserted before \emph{style after user}, but after \emph{style before user} that we can use to create additional arguments with default values using the tikz/pgf power):
{\catcode`\|=12 % Ensures | is not anymore \verb|...|
\begin{codeexample}[width=0pt]
% Requires \usetikzlibrary {shapes.geometric} for the star shape
\zxNewNodeFromPic{MyStar}[
  nb spikes/.store in=\myNbSides, % When users type nb spikes=42, this puts 42 into \myNbSides
  nb spikes=5, % <- Default value
]{
  \node[star, star points=\myNbSides, minimum size=6mm, draw, fill=red] at (0,0) {};
}
\begin{ZX}
  \text{Default:} & \zxMyStar{} & \text{More !} & \zxMyStar[nb spikes=8]{} & \zxMyStar[nb spikes=12]{}
\end{ZX}
\end{codeexample}
}

\paragraph{Overriding existing parameters.}

You might want to override a custom node, for instance because this custom node was provided by an external library (e.g.\ the ground and scalable ZX symbols are made using custom-nodes, and you might want to customize them). We provide for that a series of styles: if |XXXX| is the name on the node (e.g.\ |Ground|), the |/zx/picCustomStyleBeforeUserXXXX| style is sent to the picture style before loading the options of the user, |/zx/picCustomStyleAfterUserXXXX| is loaded after the input of the user, |/zx/picCustomStyleLastPicXXXX| is loaded as the last argument to the picture (after the default style). You can also set the command |\zxCustomPicAdditionalPic| to any code to draw at the end of the picture:
{\catcode`\|=12 % Ensures | is not anymore \verb|...|
\begin{codeexample}[width=0pt]
\tikzset{
  /zx/picCustomStyleBeforeUserMatrix/.style={
    scale=2,
    % \zxCustomPicAdditionalPic can be any tikz code to run after the creation of the pic:
    /utils/exec={\def\zxCustomPicAdditionalPic{%
        % the main node has empty name, so .center is the center of the main node
        \node[draw,circle,inner sep=2pt,fill=pink] at (.center) {};%
      }%
    },
  },
}
\begin{ZX}
  \zxZ{} \rar & \zxMatrix{A} \rar & \zxMatrix*{B} \rar & \zxN{}
\end{ZX}
\end{codeexample}
}

When designing your own custom nodes, you can also put:\\
|NameOfYourStyle/.append style={}, NameOfYourStyle,|\\
where you want to allow users to load |NameOfYourStyle| (|NameOfYourStyle/.append style={}| is used to create the style if it does not exist).

If you want to change a create a new node style without overriding a style, you can of course create your own command that sets the parameters for you:
{\catcode`\|=12 % Ensures | is not anymore \verb|...|
\begin{codeexample}[width=0pt]
\tikzset{
  /zx/picCustomStyleBeforeUserMatrix/.style={
    scale=2,
    % \zxCustomPicAdditionalPic can be any tikz code to run after the creation of the pic:
    /utils/exec={\def\zxCustomPicAdditionalPic{%
        % the main node has empty name, so .center is the center of the main node
        \node[draw,circle,inner sep=2pt,fill=pink] at (.center) {};%
      }%
    },
  },
}
\begin{ZX}
  \zxZ{} \rar & \zxMatrix{A} \rar & \zxMatrix*{B} \rar & \zxN{}
\end{ZX}
\end{codeexample}
}

\paragraph{Local style.}

Sometimes, you might want to define styles that are available only locally: for instance, you might prefer to type |main={fill=green}| instead of:\\
|/zx/picCustomStyleBoxMainNode/.append style={fill=green}|\\
Nothing prevents you from doing so directly in the default style. Just, make sure to use, e.g.\ |####1|, if you want to call an argument (internally we use nested styles and functions):

{\catcode`\|=12 % Ensures | is not anymore \verb|...|
\begin{codeexample}[width=0pt]
\zxNewNodeFromPic{MyBox}[
  main/.style={
    /zx/picCustomStyleMyBoxMainNode/.append style={####1},
  },
]{%
  \node[draw, inner sep=1.3mm, rectangle, zx main node, execute at begin node=$, execute at end node=$,
    /zx/picCustomStyleMyBoxMainNode/.append style={}, % Make sure it exists to avoid errors
    /zx/picCustomStyleMyBoxMainNode,
  ]{\tikzpictext};%
}

\begin{ZX}
  \zxMyBox{G} \rar[B]                    & [\zxwCol] \zxN{}\\
  \zxMyBox[main={fill=green}]{G} \rar[B] & [\zxwCol] \zxN{}
\end{ZX}
\end{codeexample}
}

\paragraph{Rotations.} Note that often, we want to rotate the nodes. The automatically generated macro to add new nodes comes with multiple flavors by adding the usual symbols |-|,|'|,|.|: by default, |\myNode-| will ``rotate'' by 180 degrees the pic, |\myNode'| will rotate it by 270 degrees to fake a node put ``above'', and |\myNode.| will rotate it by 90 degrees to fake a node put below. Importantly, by default tikz rotates/scales the coordinates, but \textbf{not the nodes or text}, so you might want to specify \textbf{transform shape} on each node you want to rotate and scale according to the parent transform, or using the |every node/.append style={transform shape}| (\textbf{WARNING}: make sure to use |append| or otherwise |zx main node| you not work and you will not be able to give a name to a node) option to apply it on all nodes:
{\catcode`\|=12 % Ensures | is not anymore \verb|...|
\begin{codeexample}[width=0pt]
  \zxNewNodeFromPic{MyTriangle}[
    % To also rotate/scale the node, otherwise only it's coordinate is rotated
    every node/.append style={transform shape},
    % Just to show how to quickly scale a node by default, to quickly update the size of a whole picture:
    scale=1.3
  ]{
      \node[regular polygon, regular polygon sides=3, shape border rotate=90, draw=black,fill=blue!50, inner sep=1mm,rounded corners, zx main node] {};
  }
  \begin{ZX}
    \zxMyTriangle{} \rar & \zxMyTriangle-{} & \zxMyTriangle'{} \dar \\
                         &                  & \zxMyTriangle.{}
  \end{ZX}
\end{codeexample}
}
Note however that you can customize how rotations is applied. By default, the |'-.| variants updates |\zxCurrentRotationMode| to the rotation angle (from 0 to 90,180,270), and do |rotate=\zxCurrentRotationMode|. You can however easily change that behavior, by first undoing this rotation (in the \emph{style before user}) with |rotate=-\zxCurrentRotationMode| and do any action you like depending on the value of |\zxCurrentRotationMode|. Here is a nice use-case: if you rotate a node with |transform shape|, it will also rotate the text inside:
{\catcode`\|=12 % Ensures | is not anymore \verb|...|
\begin{codeexample}[width=0pt]
  \zxNewNodeFromPic{MyTriangle}[
    every node/.append style={transform shape},
  ]{
      \node[regular polygon, regular polygon sides=3, shape border rotate=90, draw=black,fill=blue!50, font=\footnotesize,inner sep=1pt,rounded corners, zx main node] {\tikzpictext};
  }
  \begin{ZX}
    \zxMyTriangle{A} \rar   & \zxMyTriangle-{A}   & \zxMyTriangle'{A} \dar   \\
                            &                     & \zxMyTriangle.{A}        \\
  \end{ZX}
\end{codeexample}
}
To avoid that issue, we will undo the default rotation, and apply on the node of interest a rotation with |shape border rotate| (that rotates the border but not the text inside):
{\catcode`\|=12 % Ensures | is not anymore \verb|...|
\begin{codeexample}[width=0pt]
  \zxNewNodeFromPic{MyTriangle}[
    % We undo the default rotation:
    rotate=-\zxCurrentRotationMode,
  ]{
    \node[
      regular polygon, regular polygon sides=3, draw=black,fill=blue!50, font=\footnotesize,
      inner sep=1pt,rounded corners, zx main node,
      shape border rotate=90+\zxCurrentRotationMode, % Rotates the shape but not the text
    ] {\tikzpictext};
  }
  \begin{ZX}
    \zxMyTriangle{A} \rar   & \zxMyTriangle-{A}   & \zxMyTriangle'{A} \dar   \\
                            &                     & \zxMyTriangle.{A}        \\
  \end{ZX}
\end{codeexample}
}
Note that one might want to exchange |\zxMyNode|/|\zxMyNode-| or |\zxMyNode'|/|\zxMyNode.| (for instance, it might be more natural to denote |'| in place of |.| and |.| in place |'|, since these symbols visually represent the top/bottom place of the node). In that case, you just need to apply the style |invert top bottom| or |invert right left| in your \emph{style before user} and it will invert them automatically.

\paragraph{Add anchors to node.} We can automatically (and easily) add multiple anchors to our nodes (the |fake center east/…| anchors are used for instance by shapes like |\ar[<']| to see where the node should start from, while the |true north/…| anchors are used to notify where is the true north after the rotation (after rotating a node, the east might actually be on the north etc), which is for instance used by shapes like |C'|): just add coordinates with the name of the desired anchor in your graph (say |anchorA| and |anchorB|), and in \emph{style \textbf{after} user} add this list of anchor like: |zx create anchors={anchorA,anchorB}| (actually this list is expanded using the |\foreach| syntax so you can use more advanced syntax like |anchor1,anchor...,anchor10| to avoid repeating the 10 names of the anchor). For instance, we can do this way:
{\catcode`\|=12 % Ensures | is not anymore \verb|...|
\begin{codeexample}[width=0pt]
  \zxNewNodeFromPic{MyStar}[][
    % Note that this is the SECOND optional argument!
    zx create anchors={anchorA,anchorB},
  ]{
      \node[star, minimum size=6mm, draw, alias=mystar, fill=red, zx main node] at (0,0) {};
      \coordinate (anchorA) at (mystar.outer point 1);
      \coordinate (anchorB) at (mystar.outer point 2);
  }
  \begin{ZX}
    A \rar \rar[->,end anchor=anchorA,bend left] \rar[->,end anchor=anchorB,bend left]  &[5mm] \zxMyStar{}
  \end{ZX}
\end{codeexample}
}
The ZX library typically uses some special anchors like |fake center east/…| and |true east/…| to determine where some wires should start (not all kinds of wires follow this convention, for instance |<'| uses |fake center| since the wire is supposed to leave close to the center, while |C| follows |true west| since the wire is supposed to leave from the west). Because the north anchor is not anymore in the north after applying a rotation, we provide also |\zxVirtualCenterWest| (same for |East|, |North|, |South|) that will ``counter balance'' the rotation, to ensure the real north anchor is always in the north, by renaming the anchors appropriately. This way, by using these ``virtual anchors'', you only need to place your anchor when the shape is not rotated, and it should automatically rotate the anchors appropriately when needed (see remarks regarding |every node/.append style={transform shape}|):
{\catcode`\|=12 % Ensures | is not anymore \verb|...|
\begin{codeexample}[width=0pt]
% Define a reusable node
  \zxNewNodeFromPic{MyDivider}[][zx create anchors={\zxVirtualCenterWest, \zxVirtualCenterEast},
  every node/.append style={transform shape}
  ]{
  \node[regular polygon, regular polygon sides=3,shape border rotate=90,%shape border rotate=90,
        draw=black,fill=gray!50, inner sep=1.6pt, rounded corners=0.8mm,zx main node] {};
  \coordinate(\zxVirtualCenterEast) at (.2mm,0); % Used to start lines on the side of the shape
  \coordinate(\zxVirtualCenterWest) at (-1mm,0);
}
% Use the node horizontally
\begin{ZX}
  \zxZ[B]{} \rar[Bn'=n+m, wc] & \zxMyDivider{} 
                                  \rar[o',Bn'Args={n}{}]
                                  \rar[o.,Bn.Args={m}{}] &[\zxWCol] \zxMyDivider-{} \rar[B,wc] & \zxZ[B]{}
\end{ZX}
% Use the node vertically                         
\begin{ZX}
  \zxZ[B]{} \dar[Bn=n+m, wc] \\
  \zxMyDivider'{} \dar[-o,BnArgs={n}{}] \dar[o-,Bn-Args={m}{}] \\[\zxWRow]
  \zxMyDivider.{} \dar[B,wc]\\
  \zxZ[B]{}
\end{ZX}
\end{codeexample}
}

You can even dynamically add anchors depending on the input of the user. For instance, if the user can choose the number of spikes of the star, you might want to create as many anchors as there are spikes:
{\catcode`\|=12 % Ensures | is not anymore \verb|...|
\begin{codeexample}[width=0pt]
\zxNewNodeFromPic{MyStar}[
  nb spikes/.store in=\myNbSides,
  nb spikes=5,
  every node/.append style={transform shape},
  ][
    % zx create anchors is parsed ultimately by \foreach so we can use lists like
    % this to create anchors spike-1, spike-2...
    zx create anchors/.expanded={spike-1,spike-...,spike-\myNbSides},
  ]{%
    \node[star, fill=red, star points=\myNbSides, minimum size=1cm, draw, alias=mystar]
      at (0,0) {\tikzpictext};
    % We need to place coordinates where we want to final anchors to be.
    \foreach \i in {1,...,\myNbSides}{
      \coordinate (spike-\i) at (mystar.outer point \i);
    }%
}
\begin{ZX}
  B \foreach \i in {1,...,8}{\expanded{\noexpand\rar[->,end anchor=spike-\i,bend left]}}  &[5cm] \zxMyStar[nb spikes=8]{?!?}
\end{ZX}
\end{codeexample}
}

Note that this library uses some anchors called |fake center {east,west,north,south}|, to indicate for some curve style where the line should start and stop (|\ar| alone does not use any anchor, and not all curves use them: |S| for instance does not use it, while |s| does: in any case, you can force it for any curve with |force left to right| and alike, see the corresponding documentation Note also that in this mode the curve is drawn by default below the shapes, but this can be changed, see the corresponding section for details). 
{\catcode`\|=12 % Ensures | is not anymore \verb|...|
\begin{codeexample}[width=0pt]
\zxNewNodeFromPic{MyNode}[][zx create anchors/.expanded={fake center west}]{
  \node[draw,fill=blue!50,draw=blue,inner sep=3mm, zx main node, alias=mynode] at (0,0) {};
  \node[draw,fill=red!50,draw=blue,inner sep=1mm] at (mynode.north east) {};
  \coordinate(fake center west) at (mynode.north east);
}
\begin{ZX}
  \zxZ{} \ar[rd,s] & \\
  A \rar           & \zxMyNode{}  
\end{ZX}
\end{codeexample}
}

\paragraph{Link to a sub-node.} Sometimes, we might prefer to draw a link to a node instead of an anchor (an anchor can only be a single coordinate). To that end, just name your nodes as usual, and use |start subnode| or |end subnode| to do the link
{\catcode`\|=12 % Ensures | is not anymore \verb|...|
\begin{codeexample}[width=0pt]
  \zxNewNodeFromPic{MyDoubleNode}{
      \node[fill=blue!20, zx main node, inner sep=3mm] at (0,0) {};
      \node[circle,fill=red!70, inner sep=.8mm,name=redNode] at (0,1.5mm) {};
      \node[circle,fill=orange!70, inner sep=.8mm, name=orangeNode] at (0,-1.5mm) {};
  }
  \begin{ZX}
    A \rar[end subnode=redNode] \rar[end subnode=orangeNode] &[5mm] \zxMyDoubleNode{}\\
    B \ar[ru, end subnode=orangeNode]
  \end{ZX}
\end{codeexample}
}

Note that you can also combine it with |to| and |alias|:
{\catcode`\|=12 % Ensures | is not anymore \verb|...|
\zxNewNodeFromPic{MyDoubleNode}{
  \node[fill=blue!20, zx main node, inner sep=3mm] at (0,0) {};
  \node[circle,fill=red!70, inner sep=.8mm,name=redNode] at (0,1.5mm) {};
  \node[circle,fill=orange!70, inner sep=.8mm, name=orangeNode] at (0,-1.5mm) {};
}
\begin{ZX}
  \zxZ{} \ar[to=buttons, end subnode=orangeNode]  \\
  \zxX{} \rar & \zxMyDoubleNode[a=buttons]{} \rar & 
\end{ZX}
}

\end{pgfmanualentry}




\subsection{Create your own multi-column/row gate}

We provide since 2023/09/23 a number of functions in order to deal with gates spawning multiple rows or columns, including to create your own multi-columns/rows gates. We present first the individual functions, and show after how to combine them to create more advanced styles. Note that the following define commands, but you can easily insert them in a style using:
\begin{verbatim}
/utils/exec={\yourCommands}
\end{verbatim}
or by creating a new style with |my style/.code={\yourCommand}|.

\begin{pgfmanualentry}
  \extractcommand\zxGetNameAbsoluteNode\marg{row}\marg{column}\@@
  \pgfmanualbody%
  Function to get the name of a cell using its absolute coordinate in the matrix (start at 1):
\begin{codeexample}[]
\begin{ZX}[
  execute at end picture={
    \node[draw, rounded corners, fill=orange,
    node on layer=background,
    fit=(\zxGetNameAbsoluteNode{1}{1})(\zxGetNameAbsoluteNode{1}{2})
    ]{};
  }]
  A & B
\end{ZX}
\end{codeexample}
\end{pgfmanualentry}

\begin{pgfmanualentry}
  \extractcommand\zxExecuteAtEndPicture\marg{code}\@@
  \extractcommand\zxOriginalRow\@@
  \extractcommand\zxOriginalCol\@@
  \extractcommand\zxGetNameRelativeNode\marg{relative row}\marg{relative column}\@@
  \pgfmanualbody%
  |execute at end picture| is first annoying to use, but it also runs after the definition of the wires, meaning that it is not possible to point to nodes defined in that portion of the code. To solve these issues, we define |\zxExecuteAtEndPicture{your code}| that you can insert inside a node. Note that in |your code| you can use |\zxOriginalRow| and |\zxOriginalCol| that will contain the position of the column and row where you inserted that command, and |\zxGetNameRelativeNode| will allow you to get the name of the neighbour nodes by specifying the difference of columns and the difference of lines. Note that these variables and commands are usable in |\zxExecuteAtCellRelative| and |\zxExecuteAtCellAbsolute| as well.
\begin{codeexample}[]
\NewExpandableDocumentCommand{\myFitWithBelowNeighbour}{}{
  \zxExecuteAtEndPicture{%
    \node[draw, rounded corners, fill=orange,
      node on layer=background,
      fit=(\zxGetNameRelativeNode{0}{0})(\zxGetNameRelativeNode{1}{0})
    ]{};
  }%
}
\begin{ZX}
  A & B                          & C & D \\
  A & B \myFitWithBelowNeighbour & C & D \\
  A & B                          & C & D \\
\end{ZX}
\end{codeexample}
\end{pgfmanualentry}

\begin{pgfmanualentry}
  \def\extrakeytext{style, }
  \extractcommand\zxExecuteAtCellAbsolute\marg{row}\marg{column}\marg{code}\@@
  \extractcommand\zxExecuteAtCellRelative\marg{relative row}\marg{relative column}\marg{code}\@@
  \extractcommand\zxExecuteAtRegionRelative\marg{relative row}\marg{relative column}\marg{code}\@@
  \pgfmanualbody%
  Often, just adding some nodes at the end of the picture is not enough: you might want to add other nodes automatically in the matrix. For instance, if you want to implement yourself a gate spawning multiple rows, a simple way to achieve that would be to add invisible boxes inside the matrix and use them to fit the final component. You can place nodes where you want using something like this (note that you should use |zx main node| that is basically a shortcut for:
\begin{verbatim}
\tikzset{name=\zxCurrentDiagram-\the\pgfmatrixcurrentrow-\the\pgfmatrixcurrentcolumn}
\end{verbatim}
otherwise |\rar| will not be able to know which node to join: note that it is automatically added in nodes like |\zxX{}|):
\begin{codeexample}[]
\NewExpandableDocumentCommand{\mySimpleMultigate}{m}{
  \zxBox{#1}
  \zxExecuteAtCellRelative{1}{-1}{\node[zx main node, green]{\text{I am relative}};}
  \zxExecuteAtCellRelative{0}{1}{\node[zx main node, green]{\text{I am also relative}};}
  \zxExecuteAtCellAbsolute{2}{4}{\node[zx main node, red]{\text{I am absolute}};}
}
\begin{ZX}
A & B                     & C  \\
  & \mySimpleMultigate{X} &  & \\
  &                       &    \\
\end{ZX}
\end{codeexample}
(note that the relative version adds it to cells relative to the current cell while the absolute version executes it at the absolute cell coordinate)
\textbf{Important}: for all these manipulation to work, be sure to only specify coordinates after the main node, either later on the same line or on next rows since previous cells have already been parsed.

You can also do it for a whole region of size $n \times m$ using |\zxExecuteAtRegionRelative{n}{m}{code}| like this (just make sure that the cells exist):
\begin{codeexample}[]
\NewExpandableDocumentCommand{\myRegion}{}{
  \zxExecuteAtRegionRelative{2}{3}{%
    \zxBox{\text{\tiny Original row: \zxOriginalRow{}, column:\zxOriginalCol{}
        % We compute the relative position to print it:
        (rel: \the\numexpr\the\pgfmatrixcurrentrow  -\zxOriginalRow\relax
        x\the\numexpr\the\pgfmatrixcurrentcolumn-\zxOriginalCol\relax).}}}
}
\begin{ZX}
  A & B & C         & D & E & F & G \\
  &   & \myRegion &   &   &   &   \\
  &   &           &   &   &   &   \\
  &   &           &   &   &   & 
\end{ZX}
\end{codeexample}
(note that if you use |\zxExecuteAtRegionRelative| and want to define one alias for the top-left node, you should use |a if origin=alias for the node| instead of |a=alias name for the node| since it would give the alias to the last node instead of the first node). Note that if you execute it multiple times on the same cell, both codes will be executed.
\end{pgfmanualentry}

\begin{pgfmanualentry}
  \def\extrakeytext{style, }
  \extractcommand\zxSetVariable\marg{name variable}\marg{content variable}\@@
  \extractcommand\zxSetVariableExpandOnce\marg{name variable}\marg{content variable}\@@
  \extractcommand\zxSetVariableExpand\marg{name variable}\marg{content variable}\@@
  \extractcommand\zxGetVariable\marg{name variable}\@@
  \makeatletter% should not be letter for \@@... strange
  \extractkey/tikz/zx style from variable=\marg{name variable}\@nil%
  \makeatother
  \pgfmanualbody%
  You might want at some point to pass data to the node you placed elsewhere, for instance the style it should take etc. For simple setups, you can directly pass the argument like:
\begin{codeexample}[]
  \NewExpandableDocumentCommand{\mySimpleMultigate}{mmmm}{
    \zxBox{#1}
    \zxExecuteAtCellRelative{1}{-1}{\node[zx main node, #2]{\text{I am relative}};}
    \zxExecuteAtCellRelative{0}{1}{\zxBox[main={zx main node}]{\text{I am also relative}}}
    \zxExecuteAtCellAbsolute{2}{4}{\node[zx main node, #4]{\text{I am absolute}};}
  }
  \begin{ZX}
 &                    &    \\
 & \mySimpleMultigate{X}{red}{pink}{orange} &  & \\
 &                    &    \\
  \end{ZX}
\end{codeexample}
but for more complicated setups, you might want to use some variables. This can be done using the above commands (set it in the main node, and get it in other nodes) (or you can also directly give it the argument): this should just create a new macro and evaluate it.
\begin{codeexample}[]
\NewExpandableDocumentCommand{\mySimpleMultigate}{mmm}{
  \zxX{#1}
  \zxSetVariable{my variable}{Hey #2}
  \zxSetVariable{my style}{#3}
  \zxExecuteAtCellRelative{1}{-1}{\node[zx style from variable={my style}]{\text{\zxGetVariable{my variable}}};}
}
\begin{ZX}
A & B                                         & C  \\
  & \mySimpleMultigate{\alpha}{Alice}{purple,opacity={.3}} &  & \\
  &                                           &    \\
\end{ZX}
\end{codeexample}
The difference between |\zxSetVariable|, |\zxSetVariableExpandOnce| and |\zxSetVariableExpand| is that they respectively don't expand their input like |\gdef|, expand it once, or expand it completely like |\gxdef|. This can be useful in more complex setups, for instance if you store the style/variable in a macro first using pgfkeys:
\begin{codeexample}[]
\NewExpandableDocumentCommand{\MyGateMulti}{O{}mmm}{
  % The path; here is useful to help tikzcd to see that it is a node as it tries to search for path and \pgfkeys does not look like a path
  % If you forget it, you will get something like \node is not defined
  \path;\pgfkeys{
    /zx/gateMulti/.cd,
    content inner nodes/.store in=\myContent,
    content inner nodes=#4,
    a/.store in=\zxGateMultiAlias,
    a=,
    style inner nodes/.store in=\zxGateMultiStyle,
    style inner nodes={},
    #1,
  }%
  % Store the content for letter use
  \zxSetVariableExpandOnce{content inner nodes}{\myContent}
  \zxSetVariableExpandOnce{style inner nodes}{\zxGateMultiStyle}
  \zxExecuteAtRegionRelativeAndOrigin{#2}{#3}{%
  \zxBox[main={zx main node,
    % a if origin=\zxGateMultiAlias,
    draw,
    zx style from variable={style inner nodes},
    %/zx/gateMulti/style inner nodes aux
  }]{\zxGetVariable{content inner nodes}};
  }%
}

\begin{ZX}
 &  &                       &  &  &  &  & \\
 &  & \MyGateMulti[style inner nodes={red,test/.style={fill=#1,opacity=.3},test=red}]{2}{3}{H} &  &  &  &  & \\
 &  &                       &  &  &  &  & \\
 &  &                       &  &  &  &  & \\
 &  &                       &  &  &  &  & \\
\end{ZX}
\end{codeexample}
\end{pgfmanualentry}

With all of this, we can now create a simple multi-columns/rows gate mechanism (but you certainly want to use the one provided by this library if you actually care about multi-columns/rows gates:


\begin{codeexample}[width=0pt]
  % \mySimpleMultigate{rows}{cols}{content}
  \NewExpandableDocumentCommand{\mySimpleMultigate}{mmm}{
    \node[zx main node, inner sep=1mm]{#3};
    \zxExecuteAtCellRelative{#1}{#2}{%
      \node[zx main node, inner sep=1mm]{#3};
    }
    \zxExecuteAtEndPicture{
      \node[node on layer=foreground,
        fill=white, inner sep=0pt, draw,
        fit=(\zxGetNameRelativeNode{0}{0})(\zxGetNameRelativeNode{#1}{#2}),
        label={[node on layer=foreground]center:{#3}}]{};
    }
  }
  \begin{ZX}
A                             & B                           & C & D & E & F                          \\
\rar                          & \mySimpleMultigate{2}{3}{X} &   &   &   & \lar                       \\
\rar                          &                             &   &   &   & \lar                       \\
\zxN{} \dar[C] \rar           &                             &   &   &   & \zxN[a=bottomright]{} \lar \\
\zxN{} \ar[to=bottomright,C-] &                                                                      
\end{ZX}
\end{codeexample}

Note that you can also define subnodes in the final node and point to them. You can either use its alias, or point to any node added via region relative or cell relative, it will automatically find the parent node:

\begin{codeexample}[width=0pt]
  % \mySimpleMultigate{rows}{cols}{content}
  \NewExpandableDocumentCommand{\myFancyGate}{O{}mmm}{
    % a=#1 make sure to add the alias on the first element
    \node[zx main node, inner sep=1mm, a=#1]{#4};
    \zxExecuteAtRegionRelative{#2}{#3}{%
      \node[zx main node, inner sep=1mm]{#4};
    }%
    \zxExecuteAtEndPicture{
      \node[node on layer=foreground,
        alias=tmp,
        opacity=.5,
        fill=white, inner sep=0pt, draw,
        fit=(\zxGetNameRelativeNode{0}{0})(\zxGetNameRelativeNode{\the\numexpr#2-1\relax}{\the\numexpr#3-1\relax}),
        label={[node on layer=foreground]center:{#4}}]
        {};
      \node[node on layer=foreground, fill=red, inner sep=.9mm, circle, zx subnode=redCircle] at (tmp.south east) {};
    }
  }
  \begin{ZX}
A                                                     & B                                  & C & D & E    & F    \\
\rar[end subnode=redCircle, blue, bend right]         & \myFancyGate[aliasMyGate]{3}{4}{X} &   &   & \rar &      \\
\rar[purple]                                          &                                    &   &   &      & \lar \\
\rar[end subnode=redCircle, pink, bend right]         &                                    &   &   &      & \lar \\
\ar[to=aliasMyGate,red]                               &                                                          \\
\ar[to=aliasMyGate,end subnode=redCircle,green,bend right] & 
\end{ZX}
\end{codeexample}

You can also force all nodes pointing to a region-relative or cell-relative to point to a specific anchor:

\begin{pgfmanualentry}
  \def\extrakeytext{style, }
  \makeatletter% should not be letter for \@@... strange
  \extractkey/every node/pre arrow style if start node=\marg{style, to, apply}\@nil%
  \extractkey/every node/pre arrow style if end node=\marg{style, to, apply}\@nil%
  \extractkey/every node/post arrow style if start node=\marg{style, to, apply}\@nil%
  \extractkey/every node/post arrow style if end node=\marg{style, to, apply}\@nil%
  \makeatother
  \pgfmanualbody%
  This will apply the corresponding style to any wire leaving from the current cell or arriving to the current cell depending on the variation (node that you can use these functions in any node, there is no special role played by |zxExecuteAtRegionRelativeAndOrigin| here), |pre| variants being run before the user input (\textbf{WARNING}: at the step, the target of the node is not known!) while |post| variants are run after the user input:

\begin{codeexample}[width=0pt]
% \myFancyGate[alias]{rows}{cols}{content}
\NewExpandableDocumentCommand{\myFancyGate}{O{}mmm}{
  \zxExecuteAtRegionRelativeAndOrigin{#2}{#3}{%
    \node[zx main node, inner sep=1mm, a if origin=#1,
      %% These two lines says to connect any line coming/leaving from here to the redCircle subnode
      post arrow style if start node={start subnode=redCircle},
      post arrow style if end node={end subnode=redCircle},
    ]{#4};
  }%
  \zxExecuteAtEndPicture{
    \node[node on layer=foreground,
      alias=tmp,
      opacity=.5,
      fill=white, inner sep=0pt, draw,
      % The expression "\the\numexpr#3-1\relax" removes 1 to the current column or it will draw too many
      % columns (size n x m means that we stop at element (n-1) x (m-1) since we start from zero
      fit=(\zxGetNameRelativeNode{0}{0})(\zxGetNameRelativeNode{\the\numexpr#2-1\relax}{\the\numexpr#3-1\relax}),
      label={[node on layer=foreground]center:{#4}}]
      {};
    \node[node on layer=foreground, fill=red, inner sep=.9mm, circle, zx subnode=redCircle] at (tmp.south east) {};
  }
}
\begin{ZX}
A                                             & B                                  & C & D & E    & F    \\
\rar                                          & \myFancyGate[aliasMyGate]{3}{4}{X} &   &   & \rar &      \\
\rar[purple]                                  &                                    &   &   &      & \lar \\
\rar[end subnode=redCircle, pink, bend right] &                                    &   &   &      & \lar \\
\ar[to=aliasMyGate,red]                       &                                                          \\
\ar[to=aliasMyGate,end subnode=redCircle,green,bend right] 
\end{ZX}
\end{codeexample}
\end{pgfmanualentry}


\subsection{Caching pictures via an externalization library}\label{sec:externalization}

The pictures built with this package can easily take maybe 0.5 seconds per picture to be built, which can easily add up to a long compilation time. Caching externalization can save you a lot of time: I went from a tiny draft taking 30 seconds to be built to a built time of 2.5 seconds.

\subsubsection{robust-externalize: recommended}

While \tikzname{} provides an external library, it has many drawbacks that make it unusable in practice. To solve that issue, I created my own library called \mylink{https://github.com/leo-colisson/robust-externalize}{\texttt{robust-externalize}}. It's still very young, but has been working really reliably in my tests (note that a major update occurred in september 2023, so we updated the instruction accordingly).

To use it, first install |robust-externalize| by copy/pasting robust-externalize.sty in your project (unless your \LaTeX{} distribution already has it), make sure you have version 1.1, and load it using:
\begin{codeexample}[code only]
\usepackage{robust-externalize}
\robExtConfigure{
  % We create a new preset for zx pictures
  ZX/.style={
    latex, % we inherit from the latex preset
    add to preamble={  % we make sure the zx library is loaded when building cached images
      \usepackage{amsmath}
      \usepackage{zx-calculus}
    },
    dependencies={}, % Add here any file that must induce a recompilation of your file if it changes.
  },
}

% say to cache by default all ZX environments (can be changed later) and \zx{} commands
\cacheEnvironment{ZX}{ZX}
\cacheCommand{zx}{ZX}

% You can pass options to robust-externalize via <>
\begin{ZX}<add to preamble={\def\name{Alice}}>
  \zxX{\text{\name}}
\end{ZX}
\end{codeexample}

Now, just recompile your project: it should cache the pictures. See \mylink{https://github.com/leo-colisson/robust-externalize}{\texttt{robust-externalize}} for more details.

\subsubsection{Tikz external: not recommended}

\textbf{WARNING}: I added some options to use tikz |external| library to save compilation time... And then I realized that tikz |external| was quite close to be unusable in practice as it has many caveats. I ended up coding my own replacement to externalize any operation, and it works much better! See comment in the previous section.

Since 2022/02/08, it is possible to use the tikz |external| library to save compilation time. To load it, you need to add the following tikz libraries:
\begin{verse}
  |\usetikzlibrary{external}|\\
  |\usetikzlibrary{zx-calculus}|\\
  |\tikzexternalize|\\
  |\zxConfigureExternalSystemCallAuto|\\
\end{verse}
Then, compile with shell-escape, for instance using |pdflatex -shell-escape yourfile.tex|.

\textbf{WARNING}: if |external| is loaded before |zx-calculus|, you don't need to run |\zxConfigureExternalSystemCallAuto|. This command is only useful to ensure the system call used by |external| displays errors appropriately by configuring the interaction mode to match the one used by the parent compilation command. If you prefer to disable this feature to use |external|'s default, define `\def\zxDoNotPatchSystemCall{}` before loading the |zx-calculus| library.

\alertwarning{Note however that this has not yet being extensively tested, and the \texttt{external} library has a few caveats presented below}
\begin{itemize}
\item If you change the order of the diagrams, or add a diagram in the middle of the document, all subsequent diagrams will be recompiled. This issue has been reported \mylink{https://github.com/pgf-tikz/pgf/issues/758}{here} and is caused by the fact that the figures are called \texttt{figure0},\dots,\texttt{figureN}. To limit this issue, you can regularly insert \texttt{\textbackslash{}tikzsetfigurename\{nameprefix\}} in your document with different names to avoid a full recompilation of the file (or using groups to change it for a single newly added equation).
\item The \texttt{external} library uses the main file to recompile each picture, so if your file is large or loads a lot of libraries, it may \mylink{https://tex.stackexchange.com/questions/633175/tikz-externalize-is-much-slower-than-tikz-on-first-run}{take a while to compile a single diagram} (to give an example, this library takes 41 seconds to compile without the externalize library, with the externalize library it takes 14mn for the first run and 3 seconds for the next runs). For this reason, you may want to use \texttt{\textbackslash{}tikzset\{external/export=false\}} or |\tikzexternaldisable| (the latter won't fail if your elements are not parsable by tikz external) in a group to disable temporarily the external library while you are writing your diagram. You may like the |list and make| option of tikz external that produces a Makefile that one can compile separately to build in parallel all pictures.
\item If you compile once a diagram without any error, and recompile it after inserting an error, you will see an error while compiling. But if you recompile again, the error will disappear and the diagram that lastly succeeded to build will be inserted instead of the newly buggy diagram. This has been reported \mylink{https://github.com/pgf-tikz/pgf/issues/1137}{here}.
\item Sometimes, |external| cuts some parts of the picture, namely when parts are drawn outside of the bounding box. I've not experienced that with zx diagrams directly (we don't go beyond the bounding box), but the example at the end of this document with the |double copy shadow| has such issues because the shadow is drawn outside of the bounding box. One should therefore disable the externalization (or increase the bounding box) in these cases.
\item It seems that sometimes the inner sep of some labels in |external| mode defaults to zero (see the CNOT example below), I'm not sure why. Adding explicitly the value of the label fixes this.
\end{itemize}

\begin{pgfmanualentry}
  \extractcommand\zxExternalAuto\@@
  \extractcommand\zxExternalWrap\@@
  \extractcommand\zxExternalNoWrap\@@
  \extractcommand\zxExternalNoWrapNoExt\@@
  \extractcommand\zxExternalWrapForceExt\@@
  \pgfmanualbody
  Also, the library |external| forces us to wrap our diagrams into a basically empty tikz-pictures to make it work. The current library will automatically wrap the diagrams when the |external| library is enabled, but you can customize how diagrams are wrapped manually: |\zxExternalAuto| (default) will wrap it automatically if |external| is enabled, |\zxExternalWrap| will always wrap it, |\zxExternalNoWrap| will never wrap it (you will get errors if you use |external|), |\zxExternalNoWrapNoExt| will not wrap the figure, but will disable temporary the externalization for diagrams to avoid errors and |\zxExternalWrapForceExt| will wrap the figures and enable tikz |external| locally only for zx-diagrams (using |\tikzexternalenable|). This last option is particularly useful when using |external| while most other environments are not compatible with externalization (like tikzcd, quantikz, blochsphere, maybe cryptocode\dots{}): the idea is to disable tikzexternalize everywhere, except for zx diagrams:
\begin{verse}
|\tikzexternalize|\\
|\tikzexternaldisable|\\
|\zxExternalWrapForceExt|
\end{verse}
\end{pgfmanualentry}

\begin{pgfmanualentry}
  \extractcommand\zxExternalSuffix\marg{suffix}\@@
  \pgfmanualbody
By default, the library adds a suffix |zx| to figures corresponding to zx diagrams (it avoids to recompile diagrams when a normal figure is added in between two zx diagrams). You can change (or remove) this suffix using |\zxExternalSuffix{yoursuffix}|, where the suffix can also be empty.
\end{pgfmanualentry}

Note that if you get an error:\\
|Argument of \tikzexternal@laTeX@collect@until@end@tikzpicture has an extra }|\\
it is likely that you have an element that is not handled by tikz external, like a raw |tikzcd| environment, a |blochsphere| environment\dots{} Either disable temporary tikz external around it:\\
|{\tikzexternaldisable your code not compatible with external}|\\
or wrap your element around |\begin{tikzpicture}\node{your figure};\end{tikzpicture}| (people usually don't like nested tikz pictures\dots{} but it often works nicely). In case you also care about the baseline, you may prefer to wrap it using:\\
|\begin{tikzpicture}[baseline=(mynode.base)]\node(mynode){your figure};\end{tikzpicture}|

\subsection{How to visually group multiple nodes}\label{subsec:decoration}


\begin{pgfmanualentry}
  \def\extrakeytext{style, }
  \extractcommand\zxCont\oarg{style}\oarg{color}\marg{nb rows}\marg{nb columns}\oarg{label style}\marg{Text of label}\@@
  \extractcommand\zxContName\oarg{style}\oarg{color}\marg{(nodes)(to)(fit)}\oarg{label style}\marg{Text of label}\@@
  \extractcommand\zxNamedBox\oarg{style}\oarg{color}\marg{(nodes)(to)(fit)}\oarg{label style}\marg{Text of label}\@@
  \pgfmanualbody
  These commands are used to highlight a part of a diagram (|Cont| is for ``Container''), the main difference is that the |Cont| version can be used inside the matrix and will automatically contain the current node (you must then either by specifying the size of the box or the name of the other nodes to fit), while |\zxNamedBox| must be inserted later (so you might not use it much), like:
\begin{codeexample}[width=0cm]
  \begin{ZX}
    \zxN{} \rar & [\zxwCol] \zxZ{} \zxCont{2}{1}{\textsc{cnot}} \dar \rar & [\zxwCol] \zxN{} \\
    \zxN{} \rar & \zxX{} \rar                                             & [\zxwCol] \zxN{} \\
  \end{ZX}
\end{codeexample}
\begin{codeexample}[width=0cm]
  \begin{ZX}
    \zxN{} \rar & [\zxwCol] \zxZ{} \zxContName{(endCnot)}{\textsc{cnot}} \dar \rar & [\zxwCol] \zxN{} \\
    \zxN{} \rar & \zxX[a=endCnot]{} \rar                                             & [\zxwCol] \zxN{} \\
  \end{ZX}
\end{codeexample}
\begin{codeexample}[width=0cm]
  \begin{ZX}[
    % \zxCont offers a more practical interface
    execute at end picture={
      \zxNamedBox{(cnot1)(cnot2)}{
        below:\textsc{cnot}
      }
    }
    ]
    \zxN{} \rar &[\zxwCol] \zxZ[alias=cnot1]{} \dar \rar &[\zxwCol] \zxN{}\\
    \zxN{} \rar & \zxX[alias=cnot2]{} \rar      &[\zxwCol] \zxN{}\\
  \end{ZX}
\end{codeexample}

Note that the |fit| library cannot fit a wire. So if you want to include the wires around a node, the simpler thing might be to manually increase the size of the box in the style. We provide for that an helper command |fit margins={top=1mm,bottom=2mm,right=2.2mm,left=2.2mm}| (the |inner sep| and |shift| commands of tikz are not really practical to move a single border), you can also use |horizontal|, |vertical|, or |all|, and |l| is a shorcut to add larger margins instead of |fit margins={all=1mm}|.
\begin{codeexample}[width=0cm]
  \begin{ZX}[
    execute at end picture={
      \zxNamedBox[fit margins={right=2.2mm}]{(measX)(measZ)}{
        below:\footnotesize Bell measurement
      }
      \zxNamedBox[fit margins={bottom=2pt,top=2pt,left=2.0mm}][green!80!black]{(bellA)(bellB)}{
        left:\footnotesize Bell pair:
      }
    }
    ]
    \zxN[a=bellA]{} \rar \dar[C] & [\zxwCol] \zxX*{x\pi} \rar   & \zxZ*{z\pi} \rar & \zxN{} \\[\zxwCol]
    \zxN[a=bellB]{} \rar         & \zxX[a=measX]{x\pi} \dar[C-] &                           \\
    \zxN{} \rar                  & \zxZ[a=measZ]{z\pi}
  \end{ZX}
\end{codeexample}
\end{pgfmanualentry}

\section{Advanced styling}

\subsection{Overlaying or creating styles}

It is possible to arbitrarily customize the styling, create or update ZX or tikz styles\dots{} First, any option that can be given to a |tikz-cd| matrix can also be given to a |ZX| environment (we refer to the manual of |tikz-cd| for more details). We also provide overlays to quickly modify the ZX style.

\begin{stylekey}{/zx/default style nodes}
  This is where the default style must be loaded. By default, it simply loads the (nested) style packed with this library, |/zx/styles/rounded style|. You can change the style here if you would like to globally change a style.

  Note that a style must typically define at least |zxZ4|, |zxX4|, |zxFracZ6|, |zxFracX6|, |\zxH|, |zxHSmall|, |zxNoPhaseSmallZ|, |zxNoPhaseSmallX|, |zxNone{,+,-,I}|, |zxNoneDouble{,+,-,I}| and all the |phase in label*|, |pil*| styles (see code on how to define them). Because the above styles (notably |zxZ*| and |zxFrac*|) are slightly complex to define (this is needed in order to implement |phase in label|, |-| versions\dots{}), it may be quite long to implement them all properly by yourself.

  For that reason, it may be easier to load our default style and overlay only some of the styles we use (see example in |/zx/user overlay nodes| right after). You can check our code in |/zx/styles/rounded style| to see what you can redefine (intuitively, the styles like |my style name| should be callable by the end user, |myStyleName| may be redefined by users or used in tikzit, and |my@style@name| are styles that should not be touched by the user). The styles that have most interests are |zxNoPhase| (for Z and X nodes without any phase inside), |zxShort| (for Z and X nodes for fractions typically), |zxLong| (for other Z and X nodes) and |stylePhaseInLabel| (for labels when using |phase in label|). These basic styles are extended to add colors (just add |Z|/|X| after the name) like |zxNoPhaseZ|\dots{} You can change them, but if you just want to change the color, prefer to redefine |colorZxZ|/|colorZxZ| instead (note that this color does not change |stylePhaseInLabelZ/X|, so you are free to redefine these styles as well). All the above styles can however be called from inside a tikzit style, if you want to use tikzit internally (make sure to load this library then in |*.tikzdefs|).

  Note however that you should avoid to call these styles from inside |\zx{...}| since |\zx*| and |\zxFrac*| are supposed to choose automatically the good style for you depending on the mode (fractions, labels in phase\dots{}). For more details, we encourage the advanced users too look at the code of the library, and examples for simple changes will be presented now.
\end{stylekey}

\begin{stylekey}{/zx/user overlay nodes}
  If a user just wants to overlay some parts of the node styles, add your changes here.
\begin{codeexample}[]
  {\tikzset{
      /zx/user overlay nodes/.style={
        zxH/.append style={dashed,inner sep=2mm}
      }}
    \zx{\zxNone{} \rar & \zxH{} \rar & \zxNone{}}
  }
\end{codeexample}
You can also change it on a per-diagram basis:
\begin{codeexample}[]
  \zx[text=yellow,/zx/user overlay nodes/.style={
    zxSpiders/.append style={thick,draw=purple}}
  ]{\zxX{} \rar & \zxX{\alpha} \rar & \zxFracZ-{\pi}{2}}
\end{codeexample}
The list of keys that can be changed will be given below in |/zx/styles/rounded style/*|.
\end{stylekey}

\begin{stylekey}{/zx/user overlay}
  This key will be loaded like it if were inside the options of the picture |\zx[options]{...}|. To avoid always typing |\zx[content vertically centered]{...}|, you can therefore use:
\begin{codeexample}[width=0pt]
\tikzset{
  /zx/user overlay/.style={
    content vertically centered,
  },
}
\begin{ZX}
  \zxX{\alpha} & \zxX{\beta} & \zxX{a} & \zxX{b} & \zxX*{a\pi} & \zxX*{b\pi} & \zxX*{b'\pi} & \zxX*{'b\pi} & \zxZ{(a \oplus b )\pi}
\end{ZX}
\end{codeexample}
\end{stylekey}

\begin{stylekey}{/zx/default style wires}
  Default style for wires. Note that |/zx/wires definition/| is always loaded by default, and we don't add any other style for wires by default. But additional styles may use this functionality.
\end{stylekey}

\begin{stylekey}{/zx/user overlay wires}
  The user can add here additional styles for wires.
\begin{codeexample}[]
\begin{ZX}[/zx/user overlay wires/.style={thick,->,C/.append style={dashed}}]
  \zxNone{} \ar[d,C] \rar[] &[\zxWCol] \zxNone{}\\[\zxWRow]
  \zxNone{} \rar[] & \zxNone{}
\end{ZX}
\end{codeexample}
\end{stylekey}

\begin{stylekey}{/zx/styles/rounded style}
  This is the style loaded by default. It contains internally other (nested) styles that must be defined for any custom style.
\end{stylekey}

We present now all the properties that a new node style must have (and that can overlayed as explained above).
\begin{stylekey}{/zx/styles/rounded style/zxAllNodes}
  Style applied to all nodes.
\end{stylekey}

\begin{stylekey}{/zx/styles/rounded style/zxEmptyDiagram}
  Style to draw an empty diagram.
\end{stylekey}

\begin{pgfmanualentry}
  \makeatletter
  \def\extrakeytext{style, }
  \extractkey/zx/styles/rounded style/zxNone\@nil%
  \extractkey/zx/styles/rounded style/zxNone+\@nil%
  \extractkey/zx/styles/rounded style/zxNone-\@nil%
  \extractkey/zx/styles/rounded style/zxNoneI\@nil%
  \makeatother
  \pgfmanualbody
  Styles for None wires (no inner sep, useful to connect to wires). The |-|,|I|,|+| have additional horizontal, vertical, both spaces.
\end{pgfmanualentry}

\begin{pgfmanualentry}
  \makeatletter
  \def\extrakeytext{style, }
  \extractkey/zx/styles/rounded style/zxNoneDouble\@nil%
  \extractkey/zx/styles/rounded style/zxNoneDouble+\@nil%
  \extractkey/zx/styles/rounded style/zxNoneDouble-\@nil%
  \extractkey/zx/styles/rounded style/zxNoneDoubleI\@nil%
  \makeatother
  \pgfmanualbody
  Like |zxNone|, but with more space to fake two nodes on a single line (not very used).
\end{pgfmanualentry}

\begin{stylekey}{/zx/styles/rounded style/zxSpiders}
  Style that apply to all circle spiders.
\end{stylekey}

\begin{stylekey}{/zx/styles/rounded style/zxNoPhase}
  Style that apply to spiders without any angle inside. Used by |\zxX{}| when the argument is empty.
\end{stylekey}

\begin{stylekey}{/zx/styles/rounded style/zxNoPhaseSmall}
  Like |zxNoPhase| but for spiders drawn in between wires.
\end{stylekey}

\begin{stylekey}{/zx/styles/rounded style/zxShort}
  Spider with text but no inner space. Used notably to obtain nice fractions.
\end{stylekey}

\begin{stylekey}{/zx/styles/rounded style/zxLong}
  Spider with potentially large text. Used by |\zxX{\alpha}| when the argument is not empty.
\end{stylekey}

\begin{pgfmanualentry}
  \makeatletter
  \def\extrakeytext{style, }
  \extractkey/zx/styles/rounded style/zxNoPhaseZ\@nil%
  \extractkey/zx/styles/rounded style/zxNoPhaseX\@nil%
  \extractkey/zx/styles/rounded style/zxNoPhaseSmallZ\@nil%
  \extractkey/zx/styles/rounded style/zxNoPhaseSmallX\@nil%
  \extractkey/zx/styles/rounded style/zxShortZ\@nil%
  \extractkey/zx/styles/rounded style/zxShortX\@nil%
  \extractkey/zx/styles/rounded style/zxLongZ\@nil%
  \extractkey/zx/styles/rounded style/zxLongX\@nil%
  \makeatother
  \pgfmanualbody
  Like above styles, but with colors of |X| and |Z| spider added. The color can be changed globally by updating the |colorZxX| color. By default we use:
  \begin{verse}
    |\definecolor{colorZxZ}{RGB}{204,255,204}|\\
    |\definecolor{colorZxX}{RGB}{255,136,136}|\\
    |\definecolor{colorZxH}{RGB}{255,255,0}|
  \end{verse}
  as the second recommendation in \href{https://zxcalculus.com/accessibility.html}{\texttt{zxcalculus.com/accessibility.html}}.
\end{pgfmanualentry}

\begin{stylekey}{/zx/styles/rounded style/zxH}
  Style for Hadamard spiders, used by |\zxH{}| and uses the color |colorZxH|.
\end{stylekey}

\begin{stylekey}{/zx/styles/rounded style/zxHSmall}
  Like |zxH| but for Hadamard on wires, (see |H| style).
\end{stylekey}

\begin{pgfmanualentry}
  \extractcommand\zxConvertToFracInContent\marg{sign}\marg{num no parens}\marg{denom no parens}\marg{nom parens}\marg{denom parens}\@@
  \extractcommand\zxConvertToFracInLabel\@@
  \pgfmanualbody
  These functions are not meant to be used, but redefined using something like (we use |\zxMinus| as a shorter minus compared to $-$):
\begin{verse}
  |\RenewExpandableDocumentCommand{\zxConvertToFracInLabel}{mmmmm}{%|\\
  |  \ifthenelse{\equal{#1}{-}}{\zxMinus}{#1}\frac{#2}{#3}%|\\
  |}|
\end{verse}
This is used to change how the library does the conversion between |\zxFrac| and the actual written text (either in the node content or in the label depending on the function). The first argument is the sign (string |-| for minus, anything else must be written in place of the minus), the second and third argument are the numerator and denominator of the fraction when used in |\frac{}{}| while the last two arguments are the same except that they include the parens which should be added when using an inline version. For instance, one could get a call |\zxConvertToFracInLabel{-}{a+b}{c+d}{(a+b)}{(c+d)}|. See part on labels to see an example of use.
\end{pgfmanualentry}

\begin{pgfmanualentry}
  \def\extrakeytext{style, }
  \extractcommand\zxMinusUnchanged\@@
  \extractcommand\zxMinus\@@
  \extractcommand\zxMinusInShort\@@
  \makeatletter% should not be letter for \@@... strange
  \extractkey/zx/defaultEnv/zx column sep=length\@nil%
  \extractkey/zx/styles/rounded style preload/small minus\@nil%
  \extractkey/zx/styles/rounded style preload/big minus\@nil%
  \extractkey/zx/styles/rounded style/small minus\@nil%
  \extractkey/zx/styles/rounded style/big minus\@nil%
  \makeatother
  \pgfmanualbody
  |\zxMinus| is the minus sign used in fractions, |\zxMinusInShort| is used in |\zxZ-{\alpha}| and |\zxMinusUnchanged| is a minus sign shorter than $-$. You can redefine them, for instance:
\begin{codeexample}[]
Compare {\def\zxMinusInShort{-}
  \zx{\zxZ-{\alpha}}
} and
{\def\zxMinusInShort{\zxMinus}
  \zx{\zxZ-{\alpha}}
}
\end{codeexample}
You can also choose to always use a big or a small minus, either on a per-node, per-figure, or document-wise.
\begin{codeexample}[]
\begin{ZX}
  \zxFracZ-{\pi}{4} & \zxZ-{\alpha} & \zxZ-{\delta_i} & \zxZ[small minus]-{\delta_i}
\end{ZX} Picture-wise %
\begin{ZX}[small minus]
  \zxZ-{\alpha} & \zxZ-{\delta_i} & \zxZ-{\delta_i} & \zxFracZ-{\pi}{4}
\end{ZX} Document-wise %
\tikzset{/zx/user overlay/.style={small minus}}%
\begin{ZX}
  \zxZ-{\alpha} & \zxZ-{\delta_i} & \zxZ[big minus]-{\delta_i} & \zxFracZ[big minus]-{\pi}{4}
\end{ZX}
\end{codeexample}
\end{pgfmanualentry}

\noindent We also define several spacing commands that can be redefined to your needs:
\begin{pgfmanualentry}
  \extractcommand\zxHCol{}\@@
  \extractcommand\zxHRow{}\@@
  \extractcommand\zxHColFlat{}\@@
  \extractcommand\zxHRowFlat{}\@@
  \extractcommand\zxSCol{}\@@
  \extractcommand\zxSRow{}\@@
  \extractcommand\zxSColFlat{}\@@
  \extractcommand\zxSRowFlat{}\@@
  \extractcommand\zxHSCol{}\@@
  \extractcommand\zxHSRow{}\@@
  \extractcommand\zxHSColFlat{}\@@
  \extractcommand\zxHSRowFlat{}\@@
  \extractcommand\zxWCol{}\@@
  \extractcommand\zxWRow{}\@@
  \extractcommand\zxwCol{}\@@
  \extractcommand\zxwRow{}\@@
  \extractcommand\zxDotsCol{}\@@
  \extractcommand\zxDotsRow{}\@@
  \extractcommand\zxZeroCol{}\@@
  \extractcommand\zxZeroRow{}\@@
  \extractcommand\zxNCol{}\@@
  \extractcommand\zxNRow{}\@@
  \pgfmanualbody
  These are spaces, to use like |&[\zxHCol]| or |\\[\zxHRow]| in order to increase the default spacing of rows and columns depending on the style of the wire. |H| stands for Hadamard, |S| for Spiders, |W| for Wires only, |w| is you link a |zxNone| to a spider (goal is to increase the space), |N| is when you have a |\zxN| and want to reduce the space between columns, |HS| for both Spiders and Hadamard, |Dots| for the 3 dots styles, |Zero| completely resets the default column sep. And of course |Col| for columns, |Row| for rows.
\begin{codeexample}[width=3cm]
\begin{ZX}
 \zxN{} \ar[rd,-N.] &[\zxwCol]    &[\zxwCol] \zxN{} \\[\zxNRow]%%
                    & \zxX{\alpha}
                      \ar[ru,N'-]
                      \ar[rd,N.-] &  \\[\zxNRow]
 \zxN{} \ar[ru,-N']               &  & \zxN{}
\end{ZX}
\end{codeexample}
Note that you can add multiple of them by separating with commas (see |\pgfmatrixnextcell|'s documentation for more details). For instance to have a column separation of exactly |2mm|, do |&[\zxZeroCol,2mm]| (if you just do |&[2mm]| the column will be |2mm| larger). This can also be useful to avoid having a huge space when nodes have multiple empty outputs:
\begin{codeexample}[]
  Compare:
  \begin{ZX}
    \zxN{}      &                              & \zxN{} \\
    \zxN{} \rar & \zxZ{} \ar[ur,<'] \ar[dr,<'] &        \\
                &                              & \zxN{} \\
  \end{ZX}
  with
  \begin{ZX}
    \zxN{}      &                              & \zxN{} \\[\zxZeroRow]
    \zxN{} \rar & \zxZ{} \ar[ur,<'] \ar[dr,<'] &        \\[\zxZeroRow]
                &                              & \zxN{} 
  \end{ZX}
\end{codeexample}
\end{pgfmanualentry}

\begin{pgfmanualentry}
  \def\extrakeytext{style, }
  \extractcommand\zxDefaultColumnSep{}\@@
  \extractcommand\zxDefaultRowSep{}\@@
  \makeatletter% should not be letter for \@@... strange
  \extractkey/zx/defaultEnv/zx column sep=length\@nil%
  \extractkey/zx/defaultEnv/zx row sep=length\@nil%
  \makeatother
  \pgfmanualbody
  |\zxDefaultColumn/RowSep| are the column and row space, and the corresponding styles are to change a single matrix. Prefer to change these parameters compared to changing the |row sep| and |column sep| (without |zx|) of the matrix directly since other spacing styles like |\zxZeroCol| or |\zxNCol| depend on |\zxDefaultColumn|.
\end{pgfmanualentry}


\begin{pgfmanualentry}
  \extractcommand\zxDefaultSoftAngleS{}\@@
  \extractcommand\zxDefaultSoftAngleO{}\@@
  \extractcommand\zxDefaultSoftAngleChevron{}\@@
  \pgfmanualbody
  Default opening angles of |S|, |o| and |v|/|<| wires. Defaults to respectively $30$, $40$ and $45$.
\end{pgfmanualentry}

\subsection{Wire customization}\label{subsec:wirecustom}

\begin{pgfmanualentry}
  \makeatletter
  \def\extrakeytext{style, }
  \extractkey/zx/args/-andL/\@nil%
  \extractkey/zx/args/-andL/defaultO (default {-=.2,L=.4})\@nil%
  \extractkey/zx/args/-andL/defaultN (default {-=.2,L=.8})\@nil%
  \extractkey/zx/args/-andL/defaultN- (default {1-=.4,1L=0})\@nil%
  \extractkey/zx/args/-andL/defaultNN (default {})\@nil%
  \extractkey/zx/args/-andL/defaultNIN (default {1-=0,1L=.6})\@nil%
  \extractkey/zx/args/-andL/defaultS (default {-=.8,L=0})\@nil%
  \extractkey/zx/args/-andL/defaultS' (default {-=.8,L=.2})\@nil%
  \extractkey/zx/args/-andL/default-S (default {1-=.8,1L=0})\@nil%
  \extractkey/zx/args/-andL/defaultSIS (default {1-=0,1L=.8})\@nil%
  \makeatother
  \pgfmanualbody
  Default values used by wires (). You can customize them globally using something like:
  \begin{verse}
    |\tikzset{|\\
    |  /zx/args/-andL/.cd,|\\
    |  defaultO/.style={-=.2,L=.4}|\\
    |}|
   \end{verse}
   Basically |defaultO| will configure all the |o| familly, |defaultS'| will configure all the ``soft'' versions of |s|, |default-S| will configure the anchor on the side of the vertical arrival\dots{} For more details or which wire uses which configuration, check the default value given in each style definition.
\end{pgfmanualentry}

\subsection{Wires starting inside or on the boundary of the node}\label{subsec:wiresInsideOutside}

This library provides multiple methods to draw the wires between the nodes (for all curves depending on |bezier|, which is basically everything but |C| and straight lines).

\paragraph{Default drawing method.}
By default the lines will be drawn behind the node and the starting and ending points will be defined to be a |fake center *| anchor (if it exists, the exact chosen anchors (north, south\dots{}) depending on the direction). Because this anchor lies behind the node, we put them on the |edgelayer| layer. For debugging purpose, it can be practical to display them:
\begin{pgfmanualentry}
  \def\extrakeytext{style, }
  \extractcommand\zxEdgesAbove\@@
  \makeatletter% should not be letter for \@@... strange
  \extractkey/zx/wires definition/edge above\@nil%
  \extractkey/zx/wires definition/edge not above\@nil%
  \makeatother
  \pgfmanualbody
  If the macro |\zxEdgesAbove| is undefined (using |\let\zxEdgesAbove\undefined|) edges will be drawn above the nodes. To change it on a per-edge basis, use |edge above| (or its contrary |edge not above|) \emph{before the name of the wire}. This is mostly useful to understand how lines are drawn and for debugging purpose.
\begin{codeexample}[]
  What you see:
  \zx{\zxZ{\alpha+\beta} \ar[dr,s] \\
                         & \zxZ{\alpha+\beta}}
  What is drawn:
  \zx{\zxZ{\alpha+\beta} \ar[dr,edge above,s] \\
                         & \zxZ{\alpha+\beta}}
\end{codeexample}
(you can note the fact that the wire does not start at the center but at a |fake center *| anchor to provide a nicer look)
\end{pgfmanualentry}

\begin{pgfmanualentry}
  \def\extrakeytext{style, }
  \extractcommand\zxControlPointsVisible\@@
  \makeatletter% should not be letter for \@@... strange
  \extractkey/zx/wires definition/control points visible\@nil%
  \extractkey/zx/wires definition/control points not visible\@nil%
  \makeatother
  \pgfmanualbody
  Similarly, it can be useful for debugging to see the control points of the curves (note that |C|, straight lines and |(| wires are not based on our curve system, so it won't do anything for them). If the macro |\zxControlPointsVisible| is defined (using |\def\zxEdgesAbove{}|) control points will be drawn. To change it on a per-edge basis, use |control points visible| (or its contrary |control points not visible|). This is mostly useful to understand how lines are drawn and for debugging purpose.
\begin{codeexample}[]
  Controls not visible:
  \zx{\zxZ{\alpha+\beta} \ar[dr,s] \\
                         & \zxZ{\alpha+\beta}}
  Control visible:
  \zx{\zxZ{\alpha+\beta} \ar[dr,control points visible,s] \\
                         & \zxZ{\alpha+\beta}}
  Control visible + edge above:
  \zx{\zxZ{\alpha+\beta} \ar[dr,edge above,control points visible,s] \\
                         & \zxZ{\alpha+\beta}}
\end{codeexample}
\textbf{WARNING}: this command adds some points in the wire path, and in particular if you have a |H| wire (Hadamard in the middle of the wire), this option will not place it correctly. But it's not really a problem since it's just to do a quick debugging.
\end{pgfmanualentry}

Unfortunately, the default drawing method also has drawbacks. For instance, when using the |H| edge between a spider and an empty node, the ``middle'' of the edge will appear too close to the center by default (we draw the first edge above to illustrate the reason of this visual artifact):
\begin{codeexample}[width=0cm]
\begin{ZX}
 \zxN{} \ar[rd,edge above,-N.,H] &[\zxwCol,\zxHCol]  &[\zxwCol,\zxHCol] \zxN{} \\[\zxNRow]%%
                                 & \zxX{\alpha}
                                   \ar[ru,N'-,H]
                                   \ar[rd,N.-,H] &  \\[\zxNRow]
 \zxN{} \ar[ru,-N',H]                        &  & \zxN{}
\end{ZX}
\end{codeexample}
To solve that issue, you need to manually position the |H| node as shown before:
\begin{codeexample}[width=0cm]
\begin{ZX}
 \zxN{} \ar[rd,edge above,-N.,H={pos=.35}] &[\zxwCol,\zxHCol]  &[\zxwCol,\zxHCol] \zxN{} \\[\zxNRow]%%
                                           & \zxX{\alpha}
                                             \ar[ru,N'-,H={pos=1-.35}]
                                             \ar[rd,N.-,H={pos=1-.35}] &  \\[\zxNRow]
 \zxN{} \ar[ru,-N',H={pos=.35}]                                  &  & \zxN{}
\end{ZX}
\end{codeexample}
Or manually position the anchor outside the node (you can use angles, centered on the real center on the shape), but be aware that it can change the shape of the node (see below):
\begin{codeexample}[width=0cm]
\begin{ZX}
 \zxN{} \ar[rd,edge above,-N.,H,end anchor=180-45] &[\zxwCol,\zxHCol]  &[\zxwCol,\zxHCol] \zxN{} \\[\zxNRow]%%
                                           & \zxX{\alpha}
                                             \ar[ru,N'-,H,start anchor=45]
                                             \ar[rd,N.-,H,start anchor=-45] &  \\[\zxNRow]
 \zxN{} \ar[ru,-N',H,end anchor=180+45]                                  &  & \zxN{}
\end{ZX}
\end{codeexample}

A second drawback is that it is not possible to add arrows on the curved wires (except |C| which uses a different approach), since they will be hidden behind the node:
\begin{codeexample}[]
  What you see:
  \zx{\zxZ{\alpha+\beta} \ar[dr,s,<->] \\
                         & \zxZ{\alpha+\beta}}
  What is drawn:
  \zx{\zxZ{\alpha+\beta} \ar[dr,edge above,s,<->] \\
                         & \zxZ{\alpha+\beta}}
\end{codeexample}
Here, the only solution (without changing the drawing mode) is to manually position the anchor as before\dots{} but note that on nodes with a large content |45| degrees is actually nearly on the top since the angle is not taken from a fake center but from the real center of the node.
\begin{codeexample}[]
  \zx{\zxZ{\alpha+\beta} \ar[dr,s,<->,start anchor=-45,end anchor=180-45] \\
                         & \zxZ{\alpha+\beta}}
  \zx{\zxZ{\alpha+\beta} \ar[dr,s,<->,start anchor=-15,end anchor=180-15] \\
                         & \zxZ{\alpha+\beta}}
\end{codeexample}
 Note that the shape of the wire may be a bit different since the ending and leaving parts was hidden before, and the current styles are not designed to look nicely when starting on the border of a node. For that reason, you may need to tweak the style of the wire yourself using |-|, |L| options.

 \paragraph{The ``intersection'' drawing methods}

 We also define other modes to draw wires (they are very new and not yet tested a lot). In the first mode, appropriate |fake center *| is taken, then depending on the bezier control points, a point is taken on the border of the shape (starting from the fake center and using the direction of the bezier control point). Then the node is drawn. Here is how to enable this mode:

\begin{pgfmanualentry}
  \def\extrakeytext{style, }
  \extractcommand\zxEnableIntersections\marg{}\@@
  \extractcommand\zxDisableIntersections\marg{}\@@
  \extractcommand\zxEnableIntersectionsNodes\@@
  \extractcommand\zxEnableIntersectionsWires\@@
  \makeatletter% should not be letter for \@@... strange
  \extractkey/zx/wires definition/use intersections\@nil%
  \extractkey/zx/wires definition/dont use intersections\@nil%
  \makeatother
  \pgfmanualbody
  The simpler method to enables or disable intersections is to call |\zxEnableIntersections{}| or |\zxDisableIntersections{}| (potentially in a group to have a local action only). Note that \emph{this does not automatically adapt the styles}, see |ui| to adapt the styles automatically.
\begin{codeexample}[width=0cm]
{% Enable intersections (but does not load our custom "intersections" style, see ui).
  \zxEnableIntersections{}% Small space left = artifact of the documentation
  \begin{ZX}
    \zxN{} \ar[rd,edge above,-N.,H] &[\zxwCol,\zxHCol] &[\zxwCol,\zxHCol] \zxN{} \\[\zxNRow]%%
                                    & \zxX{\alpha}
                                      \ar[ru,N'-,H]
                                      \ar[rd,N.-,H]    &  \\[\zxNRow]
    \zxN{} \ar[ru,-N',H]            &                  & \zxN{}
  \end{ZX}
}
\end{codeexample}
(The |edge above| is just to show that the wire does not go inside.) However, this method enable intersections for the whole drawing. You can disable it for a single arrow using the |dont use intersections| style. But it is possible instead to enable it for a single wire. To do that, first define |\def\zxEnableIntersectionsNodes{}| (it will automatically add a |name path| on each node. If you don't care about optimizations, you can just define it once at the beginning of your project), and then use |use intersections| on the wires which should use intersections:
\begin{codeexample}[width=0cm]
{% Create the machinary needed to compute intersections, but does not enable it.
  \def\zxEnableIntersectionsNodes{}% Small space left = artifact of the documentation
  \begin{ZX}
    \zxN{} \ar[rd,edge above,-N.,H, %% "use intersections" does not load any style, cf ui.
            use intersections] &[\zxwCol,\zxHCol]  &[\zxwCol,\zxHCol] \zxN{} \\[\zxNRow]%%
                                                   & \zxX{\alpha}
                                                     \ar[ru,edge above,N'-,H,use intersections]
                                                        \ar[rd,edge above,N.-,H]  &  \\[\zxNRow]
    \zxN{} \ar[ru,edge above,-N',H]                &                              & \zxN{}
  \end{ZX}
}
\end{codeexample}
\end{pgfmanualentry}

\begin{pgfmanualentry}
  \def\extrakeytext{style, }
  \makeatletter% should not be letter for \@@... strange
  \extractkey/zx/wires definition/ui\@nil%
  \makeatother
  \pgfmanualbody
  This method has however a few drawbacks. One of the first reason that explains why we don't use it by default is that it is quite long to compute (it involves the |intersections| library to obtain the bezier point to start at and my code may also be not very well optimized as I'm a beginner with \LaTeX{} and \tikzname{} programming\dots{} and what a strange language!). Secondly, it has not yet been tested a lot. Note also that the default wire styles have not been optimized for this setup and the results may vary compared to the default drawing mode (sometimes they are ``better'', sometimes they are not). We have however tried to define a second style |/zx/args/-andL/ui/| that have nicer results. To load it, just type |ui| \emph{before the wire style name}, it will automatically load |use intersections| together with our custom styles (see below how to use |user overlay wires| to load it by default):
%%% I'm not sure why, but inside codeexample if I write zxEnableIntersections{}%<go to line>\begin{ZX}... then an additional space is added... Not sure why.
\begin{codeexample}[]
{%
  \def\zxEnableIntersectionsNodes{}
  Before \begin{ZX}
    \zxX{\beta} \ar[r,o'] & \zxX{}
  \end{ZX} after \begin{ZX}
    \zxX{\beta} \ar[r,o',use intersections] & \zxX{}
  \end{ZX} corrected manually \begin{ZX}
    \zxX{\beta} \ar[r,edge above, use intersections, o'={-=.2,L=.15}] & \zxX{}
  \end{ZX} or with our custom style \begin{ZX}
    \zxX{\beta} \ar[r,edge above, ui, o'] & \zxX{}
  \end{ZX}.
}
\end{codeexample}
Here are further comparisons:
\begin{codeexample}[]
{%
  \def\zxEnableIntersectionsNodes{}
  Before \begin{ZX}
    \zxX{} \ar[r,o'] & \zxX{}
  \end{ZX} ui \begin{ZX}
    \zxX{} \ar[r, ui, o'] & \zxX{}
  \end{ZX}. Before \begin{ZX}
    \zxX{\alpha} \ar[r,o'] & \zxZ{\beta}
  \end{ZX} ui \begin{ZX}
    \zxX{\alpha} \ar[r, ui, o'] & \zxZ{\beta}
  \end{ZX}. Before \begin{ZX}
    \zxX{\alpha+\beta} \ar[r,o'] & \zxZ{\alpha+\beta}
  \end{ZX} ui \begin{ZX}
    \zxX{\alpha+\beta} \ar[r, ui, o'] & \zxZ{\alpha+\beta}
  \end{ZX}.
}
\end{codeexample}
With |N|:
\begin{codeexample}[]
{%
  \def\zxEnableIntersectionsNodes{}
  \begin{ZX}
    \zxX{} \ar[rd,N]\\ & \zxX{}
  \end{ZX} $\Rightarrow $ ui \begin{ZX}
    \zxX{} \ar[rd,ui,N]\\ & \zxX{}
  \end{ZX}.
  \begin{ZX}
    \zxX{\alpha} \ar[rd,N]\\ & \zxX{}
  \end{ZX} $\Rightarrow $ ui \begin{ZX}
    \zxX{\alpha} \ar[rd,ui,N]\\ & \zxX{}
  \end{ZX}.
  \begin{ZX}
    \zxX{\alpha} \ar[rd,N]\\ & \zxX{\beta}
  \end{ZX} $\Rightarrow $ ui \begin{ZX}
    \zxX{\alpha} \ar[rd,ui,N]\\ & \zxX{\beta}
  \end{ZX}.
  \begin{ZX}
    \zxX{\alpha+\beta} \ar[rd,N]\\ & \zxX{\alpha+\beta}
  \end{ZX} $\Rightarrow $ ui \begin{ZX}
    \zxX{\alpha+\beta} \ar[rd,ui,N]\\ & \zxX{\alpha+\beta}
  \end{ZX}.
}
\end{codeexample}

With |N-|:
\begin{codeexample}[]
{%
  \def\zxEnableIntersectionsNodes{}
  \begin{ZX}
    \zxX{} \ar[rd,N-]\\ & \zxX{}
  \end{ZX} $\Rightarrow $ ui \begin{ZX}
    \zxX{} \ar[rd,ui,N-]\\ & \zxX{}
  \end{ZX}.
  \begin{ZX}
    \zxX{\alpha} \ar[rd,N-]\\ & \zxX{}
  \end{ZX} $\Rightarrow $ ui \begin{ZX}
    \zxX{\alpha} \ar[rd,ui,N-]\\ & \zxX{}
  \end{ZX}.
  \begin{ZX}
    \zxX{\alpha} \ar[rd,N-]\\ & \zxX{\beta}
  \end{ZX} $\Rightarrow $ ui \begin{ZX}
    \zxX{\alpha} \ar[rd,ui,N-]\\ & \zxX{\beta}
  \end{ZX}.
  \begin{ZX}
    \zxX{\alpha+\beta} \ar[rd,N-]\\ & \zxX{\alpha+\beta}
  \end{ZX} $\Rightarrow $ ui \begin{ZX}
    \zxX{\alpha+\beta} \ar[rd,ui,N-]\\ & \zxX{\alpha+\beta}
  \end{ZX}.
}
\end{codeexample}

With |<.|:
\begin{codeexample}[]
{%
  \def\zxEnableIntersectionsNodes{}
  \begin{ZX}
    \zxX{} \ar[rd,<.]\\ & \zxX{}
  \end{ZX} $\Rightarrow $ ui \begin{ZX}
    \zxX{} \ar[rd,ui,<.]\\ & \zxX{}
  \end{ZX}.
  \begin{ZX}
    \zxX{\alpha} \ar[rd,<.]\\ & \zxX{}
  \end{ZX} $\Rightarrow $ ui \begin{ZX}
    \zxX{\alpha} \ar[rd,ui,<.]\\ & \zxX{}
  \end{ZX}.
  \begin{ZX}
    \zxX{\alpha} \ar[rd,<.]\\ & \zxX{\beta}
  \end{ZX} $\Rightarrow $ ui \begin{ZX}
    \zxX{\alpha} \ar[rd,ui,<.]\\ & \zxX{\beta}
  \end{ZX}.
  \begin{ZX}
    \zxX{\alpha+\beta} \ar[rd,<.]\\ & \zxX{\alpha+\beta}
  \end{ZX} $\Rightarrow $ ui \begin{ZX}
    \zxX{\alpha+\beta} \ar[rd,ui,<.]\\ & \zxX{\alpha+\beta}
  \end{ZX}.
}
\end{codeexample}

With |NIN|:
\begin{codeexample}[]
{%
  \def\zxEnableIntersectionsNodes{}
  \begin{ZX}
    \zxX{} \ar[rd,NIN]\\ & \zxX{}
  \end{ZX} $\Rightarrow $ ui \begin{ZX}
    \zxX{} \ar[rd,ui,NIN]\\ & \zxX{}
  \end{ZX}.
  \begin{ZX}
    \zxX{\alpha} \ar[rd,NIN]\\ & \zxX{}
  \end{ZX} $\Rightarrow $ ui \begin{ZX}
    \zxX{\alpha} \ar[rd,ui,NIN]\\ & \zxX{}
  \end{ZX}.
  \begin{ZX}
    \zxX{\alpha} \ar[rd,NIN]\\ & \zxX{\beta}
  \end{ZX} $\Rightarrow $ ui \begin{ZX}
    \zxX{\alpha} \ar[rd,ui,NIN]\\ & \zxX{\beta}
  \end{ZX}.
  \begin{ZX}
    \zxX{\alpha+\beta} \ar[rd,NIN]\\ & \zxX{\alpha+\beta}
  \end{ZX} $\Rightarrow $ ui \begin{ZX}
    \zxX{\alpha+\beta} \ar[rd,ui,NIN]\\ & \zxX{\alpha+\beta}
  \end{ZX}.
}
\end{codeexample}

With this mode |s| behaves basically like |S| since the only difference is the anchor:
\begin{codeexample}[]
{%
  \def\zxEnableIntersectionsNodes{}
  \begin{ZX}
    \zxX{} \ar[rd,s]\\ & \zxX{}
  \end{ZX} $\Rightarrow $ ui \begin{ZX}
    \zxX{} \ar[rd,ui,s]\\ & \zxX{}
  \end{ZX}.
  \begin{ZX}
    \zxX{\alpha} \ar[rd,s]\\ & \zxX{}
  \end{ZX} $\Rightarrow $ ui \begin{ZX}
    \zxX{\alpha} \ar[rd,ui,s]\\ & \zxX{}
  \end{ZX}.
  \begin{ZX}
    \zxX{\alpha} \ar[rd,s]\\ & \zxX{\beta}
  \end{ZX} $\Rightarrow $ ui \begin{ZX}
    \zxX{\alpha} \ar[rd,ui,s]\\ & \zxX{\beta}
  \end{ZX}.
  \begin{ZX}
    \zxX{\alpha+\beta} \ar[rd,s]\\ & \zxX{\alpha+\beta}
  \end{ZX} $\Rightarrow $ ui \begin{ZX}
    \zxX{\alpha+\beta} \ar[rd,ui,s]\\ & \zxX{\alpha+\beta}
  \end{ZX}.
}
\end{codeexample}

With |s'|:
\begin{codeexample}[]
{%
  \def\zxEnableIntersectionsNodes{}
  \begin{ZX}
    \zxX{} \ar[rd,s']\\ & \zxX{}
  \end{ZX} $\Rightarrow $ ui \begin{ZX}
    \zxX{} \ar[rd,ui,s']\\ & \zxX{}
  \end{ZX}.
  \begin{ZX}
    \zxX{\alpha} \ar[rd,s']\\ & \zxX{}
  \end{ZX} $\Rightarrow $ ui \begin{ZX}
    \zxX{\alpha} \ar[rd,ui,s']\\ & \zxX{}
  \end{ZX}.
  \begin{ZX}
    \zxX{\alpha} \ar[rd,s']\\ & \zxX{\beta}
  \end{ZX} $\Rightarrow $ ui \begin{ZX}
    \zxX{\alpha} \ar[rd,ui,s']\\ & \zxX{\beta}
  \end{ZX}.
  \begin{ZX}
    \zxX{\alpha+\beta} \ar[rd,s']\\ & \zxX{\alpha+\beta}
  \end{ZX} $\Rightarrow $ ui \begin{ZX}
    \zxX{\alpha+\beta} \ar[rd,ui,s']\\ & \zxX{\alpha+\beta}
  \end{ZX}.
}
\end{codeexample}

With |-s|:
\begin{codeexample}[]
{%
  \def\zxEnableIntersectionsNodes{}
  \begin{ZX}
    \zxX{} \ar[rd,-s]\\ & \zxX{}
  \end{ZX} $\Rightarrow $ ui \begin{ZX}
    \zxX{} \ar[rd,ui,-s]\\ & \zxX{}
  \end{ZX}.
  \begin{ZX}
    \zxX{\alpha} \ar[rd,-s]\\ & \zxX{}
  \end{ZX} $\Rightarrow $ ui \begin{ZX}
    \zxX{\alpha} \ar[rd,ui,-s]\\ & \zxX{}
  \end{ZX}.
  \begin{ZX}
    \zxX{\alpha} \ar[rd,-s]\\ & \zxX{\beta}
  \end{ZX} $\Rightarrow $ ui \begin{ZX}
    \zxX{\alpha} \ar[rd,ui,-s]\\ & \zxX{\beta}
  \end{ZX}.
  \begin{ZX}
    \zxX{\alpha+\beta} \ar[rd,-s]\\ & \zxX{\alpha+\beta}
  \end{ZX} $\Rightarrow $ ui \begin{ZX}
    \zxX{\alpha+\beta} \ar[rd,ui,-s]\\ & \zxX{\alpha+\beta}
  \end{ZX}.
}
\end{codeexample}

With |SIS|:
\begin{codeexample}[]
{%
  \def\zxEnableIntersectionsNodes{}
  \begin{ZX}
    \zxX{} \ar[rd,SIS]\\ & \zxX{}
  \end{ZX} $\Rightarrow $ ui \begin{ZX}
    \zxX{} \ar[rd,ui,SIS]\\ & \zxX{}
  \end{ZX}.
  \begin{ZX}
    \zxX{\alpha} \ar[rd,SIS]\\ & \zxX{}
  \end{ZX} $\Rightarrow $ ui \begin{ZX}
    \zxX{\alpha} \ar[rd,ui,SIS]\\ & \zxX{}
  \end{ZX}.
  \begin{ZX}
    \zxX{\alpha} \ar[rd,SIS]\\ & \zxX{\beta}
  \end{ZX} $\Rightarrow $ ui \begin{ZX}
    \zxX{\alpha} \ar[rd,ui,SIS]\\ & \zxX{\beta}
  \end{ZX}.
  \begin{ZX}
    \zxX{\alpha+\beta} \ar[rd,SIS]\\ & \zxX{\alpha+\beta}
  \end{ZX} $\Rightarrow $ ui \begin{ZX}
    \zxX{\alpha+\beta} \ar[rd,ui,SIS]\\ & \zxX{\alpha+\beta}
  \end{ZX}.
}
\end{codeexample}

With |^.|:
\begin{codeexample}[]
{%
  \def\zxEnableIntersectionsNodes{}
  \begin{ZX}
    \zxX{} \ar[rd,^.]\\ & \zxX{}
  \end{ZX} $\Rightarrow $ ui \begin{ZX}
    \zxX{} \ar[rd,ui,^.]\\ & \zxX{}
  \end{ZX}.
  \begin{ZX}
    \zxX{\alpha} \ar[rd,^.]\\ & \zxX{}
  \end{ZX} $\Rightarrow $ ui \begin{ZX}
    \zxX{\alpha} \ar[rd,ui,^.]\\ & \zxX{}
  \end{ZX}.
  \begin{ZX}
    \zxX{\alpha} \ar[rd,^.]\\ & \zxX{\beta}
  \end{ZX} $\Rightarrow $ ui \begin{ZX}
    \zxX{\alpha} \ar[rd,ui,^.]\\ & \zxX{\beta}
  \end{ZX}.
  \begin{ZX}
    \zxX{\alpha+\beta} \ar[rd,^.]\\ & \zxX{\alpha+\beta}
  \end{ZX} $\Rightarrow $ ui \begin{ZX}
    \zxX{\alpha+\beta} \ar[rd,ui,^.]\\ & \zxX{\alpha+\beta}
  \end{ZX}.
}
\end{codeexample}

Now using our favorite drawing. Here we illustrate how we apply our custom style to all arrows.
\begin{codeexample}[width=0pt]
\def\zxEnableIntersectionsNodes{}%
\tikzset{
  /zx/user overlay wires/.style={
    ui, % Other method
  }
}
\begin{ZX}[
  execute at begin picture={%
    %%% Definition of long items (the goal is to have a small and readable matrix
    % (warning: macro can't have numbers in TeX. Also, make sure not to use existing names)
    \def\Zpifour{\zxFracZ[a=Zpi4]-{\pi}{4}}%
    \def\mypitwo{\zxFracX[a=mypi2]{\pi}{2}}%
  }
  ]
  %%% Matrix: in emacs "M-x align" is practical to automatically format it. a is for 'alias'
  & \zxN[a=n]{}  & \zxZ[a=xmiddle]{}       &            & \zxN[a=out1]{} \\
  \zxN[a=in1]{} & \Zpifour{}   & \zxX[a=Xdown]{}         & \mypitwo{} &                \\
  &              &                         &            & \zxN[a=out2]{} \\
  \zxN[a=in2]{} & \zxX[a=X1]{} & \zxZ[a=toprightpi]{\pi} &            & \zxN[a=out3]{}
  %%% Arrows
  % Column 1
  \ar[from=in1,to=X1,s]
  \ar[from=in2,to=Zpi4,.>]
  % Column 2
  \ar[from=X1,to=xmiddle,N']
  \ar[from=X1,to=toprightpi,H]
  \ar[from=Zpi4,to=n,C] \ar[from=n,to=xmiddle,wc]
  \ar[from=Zpi4,to=Xdown]
  % Column 3
  \ar[from=xmiddle,to=Xdown,C-]
  \ar[from=xmiddle,to=mypi2,'>]
  % Column 4
  \ar[from=toprightpi,to=mypi2,-N]
  \ar[from=mypi2,to=out1,<']
  \ar[from=mypi2,to=out2,<.]
  \ar[edge above,use intersections,from=Xdown,to=out3,<.]
\end{ZX}
\end{codeexample}

The same using |control points visible| to check if the good styles are applied:
\begin{codeexample}[width=0pt]
\def\zxEnableIntersectionsNodes{}%
\tikzset{
  /zx/user overlay wires/.style={
    ui, % Enable our style on all
    edge above, % For debugging
    control points visible % For debugging
  }
}
\def\zxDebugMode{}
\def\zxControlPointsVisible{}
\begin{ZX}[
  execute at begin picture={%
    %%% Definition of long items (the goal is to have a small and readable matrix
    % (warning: macro can't have numbers in TeX. Also, make sure not to use existing names)
    \def\Zpifour{\zxFracZ[a=Zpi4]-{\pi}{4}}%
    \def\mypitwo{\zxFracX[a=mypi2]{\pi}{2}}%
  }
  ]
  %%% Matrix: in emacs "M-x align" is practical to automatically format it. a is for 'alias'
  & \zxN[a=n]{}  & \zxZ[a=xmiddle]{}       &            & \zxN[a=out1]{} \\
  \zxN[a=in1]{} & \Zpifour{}   & \zxX[a=Xdown]{}         & \mypitwo{} &                \\
  &              &                         &            & \zxN[a=out2]{} \\
  \zxN[a=in2]{} & \zxX[a=X1]{} & \zxZ[a=toprightpi]{\pi} &            & \zxN[a=out3]{}
  %%% Arrows
  % Column 1
  \ar[from=in1,to=X1,s]
  \ar[from=in2,to=Zpi4,.>]
  % Column 2
  \ar[from=X1,to=xmiddle,N']
  \ar[from=X1,to=toprightpi,H]
  \ar[from=Zpi4,to=n,C] \ar[from=n,to=xmiddle,wc]
  \ar[from=Zpi4,to=Xdown]
  % Column 3
  \ar[from=xmiddle,to=Xdown,C-]
  \ar[from=xmiddle,to=mypi2,'>]
  % Column 4
  \ar[from=toprightpi,to=mypi2,-N]
  \ar[from=mypi2,to=out1,<']
  \ar[from=mypi2,to=out2,<.]
  \ar[edge above,use intersections,from=Xdown,to=out3,<.]
\end{ZX}
\end{codeexample}

Now, we can also globally enable the |ui| style and the intersection only for some kinds of arrows. For instance (here we enable it for all styles based on |N|, i.e.\ |*N*| and |>|-like wires). See that the |s| node is note using the intersections mode:
\begin{codeexample}[width=0pt]
\def\zxEnableIntersectionsNodes{}%
\tikzset{
  /zx/user overlay wires/.style={
    %% Nbase changes both N-like and >-like styles.
    %% Use N/.append to change only N-like.
    Nbase/.append style={%
      ui, % intersection only for arrows based on N (N and <)
    },
    edge above, % For debugging
    control points visible % For debugging
  }
}
\def\zxDebugMode{}
\def\zxControlPointsVisible{}
\begin{ZX}[
  execute at begin picture={%
    %%% Definition of long items (the goal is to have a small and readable matrix
    % (warning: macro can't have numbers in TeX. Also, make sure not to use existing names)
    \def\Zpifour{\zxFracZ[a=Zpi4]-{\pi}{4}}%
    \def\mypitwo{\zxFracX[a=mypi2]{\pi}{2}}%
  }
  ]
  %%% Matrix: in emacs "M-x align" is practical to automatically format it. a is for 'alias'
  & \zxN[a=n]{}  & \zxZ[a=xmiddle]{}       &            & \zxN[a=out1]{} \\
  \zxN[a=in1]{} & \Zpifour{}   & \zxX[a=Xdown]{}         & \mypitwo{} &                \\
  &              &                         &            & \zxN[a=out2]{} \\
  \zxN[a=in2]{} & \zxX[a=X1]{} & \zxZ[a=toprightpi]{\pi} &            & \zxN[a=out3]{}
  %%% Arrows
  % Column 1
  \ar[from=in1,to=X1,s]
  \ar[from=in2,to=Zpi4,.>]
  % Column 2
  \ar[from=X1,to=xmiddle,N']
  \ar[from=X1,to=toprightpi,H]
  \ar[from=Zpi4,to=n,C] \ar[from=n,to=xmiddle,wc]
  \ar[from=Zpi4,to=Xdown]
  % Column 3
  \ar[from=xmiddle,to=Xdown,C-]
  \ar[from=xmiddle,to=mypi2,'>]
  % Column 4
  \ar[from=toprightpi,to=mypi2,-N]
  \ar[from=mypi2,to=out1,<']
  \ar[from=mypi2,to=out2,<.]
  \ar[edge above,use intersections,from=Xdown,to=out3,<.]
\end{ZX}
\end{codeexample}

\end{pgfmanualentry}

\begin{pgfmanualentry}
  \def\extrakeytext{style, }
  \extractcommand\zxIntersectionLineBetweenStartEnd\@@
  \makeatletter% should not be letter for \@@... strange
  \extractkey/zx/wires definition/intersections mode between start end\@nil%
  \extractkey/zx/wires definition/intersections mode bezier controls\@nil%
  \makeatother
  \pgfmanualbody

Node that we also defined another intersection mechanism, in which the intersection with the node boundary is computed using the line that links the two fake centers of the starting and ending point. To use it, either define |\def\zxIntersectionLineBetweenStartEnd{}| or use the style |intersections between start end| (or to come back to the normal intersection mode |intersections bezier controls|). Note that this just changes the mode of computing intersections, but does not enable intersections, you still need to enable intersections as explained above (for instance using |use intersection|, or |ui| if you also want to load our style). Note however that we don't spent too much effort in this mode as the result is often not really appealing, in particular the |o| shapes, and therefore we designed no special style for it and made only a few tests.
\begin{codeexample}[]
{%
  \def\zxEnableIntersectionsNodes{}
  \begin{ZX}
    \zxX{\alpha} \ar[rd,N]\\ & \zxX{\beta}
  \end{ZX} $\Rightarrow $ between start end \begin{ZX}
    \zxX{\alpha} \ar[edge above,rd,ui,intersections mode between start end,N]\\ & \zxX{\beta}
  \end{ZX}.
  \begin{ZX}
    \zxX{\alpha} \ar[r,o'] \ar[r,o.] & \zxX{\beta}
  \end{ZX} $\Rightarrow $ between start end \begin{ZX}
    \zxX{\alpha} \ar[r,ui,intersections mode between start end,o']
                 \ar[r,ui,intersections mode between start end,o.] & \zxX{\beta}
  \end{ZX}.
}
\end{codeexample}
\end{pgfmanualentry}

\subsection{Nested diagrams}


If you consider this example:

\begin{codeexample}[]
Alone \begin{ZX}
  \zxX{} \rar \ar[r,o'] \ar[r,o.] & \zxZ{}
\end{ZX} %
in a diagram %
\begin{ZX}[math baseline=myb]
                     &[\zxwCol] \zxFracX-{\pi}{2} & \zxX{}
                                                    \rar
                                                    \ar[r,o']
                                                    \ar[r,o.] & \zxZ{} &[\zxwCol]\\
  \zxN[a=myb]{} \rar & \zxFracZ{\pi}{2} \rar & \zxFracX{\pi}{2} \rar & \zxFracZ{\pi}{2} \rar & \zxN{}
\end{ZX}
\end{codeexample}

You can see that the constant looks much wider in the second picture, due to the fact that the nodes below increase the column size. One solution I found for this problem is to use |savebox| to create your drawing \emph{before} the diagram, and then use the |fit| library to include it where you want in the matrix (see below for a command that does that automatically):

\begin{codeexample}[width=0pt]
%% Create a new box
\newsavebox{\myZXbox}
%% Save our small diagram.
%% Warning: on older versions, you needed to use \& instead of &
%% (the char '&' cause troubles in functions), but I fixed it 2022/02/09
\sbox{\myZXbox}{%
  % add \tikzset{external/optimize=false} if you use tikz "external" library %
  \zx{ % you may need 
    \zxX{} \rar \ar[r,o'] \ar[r,o.] & \zxZ{}
  }%
}

$x = \begin{ZX}[
  execute at end picture={
    %% Add our initial drawing at the end:
    \node[fit=(start)(end),yshift=-axis_height] {\usebox{\myZXbox}};
  },
  math baseline=myb]
                     &[\zxwCol] \zxFracX-{\pi}{2} & \zxN[a=start]{}       & \zxN[a=end]{}\\
  \zxN[a=myb]{} \rar & \zxFracZ{\pi}{2} \rar      & \zxFracX{\pi}{2} \rar & \zxFracZ{\pi}{2} \rar & \zxN{}
\end{ZX}$
\end{codeexample}

Note that this has the advantage of preserving the baseline of the big drawing. However, it is a bit cumbersome to use, so we provide here a wrapper that automatically does the following code (this has not been tested extensively, and may be subject to changes):

\begin{pgfmanualentry}
  \def\extrakeytext{style, }
  \extractcommand\zxSaveDiagram\marg{name with backslash}\opt{\oarg{zx options}}\marg{diagram with ampersand \textbackslash\&}\@@
  \makeatletter% should not be letter for \@@... strange
  \extractkey/zx/wires definition/use diagram=\marg{name with \textbackslash}\marg{(nodes)(to)(fit)}\@nil%
  \makeatother
  \pgfmanualbody
  Use |\zxSaveDiagram| to save and name a diagram (must be done before the diagram you want to insert this diagram into, also \textbf{do not forget the backslash before the name} (note that before 2022/09/02, you needed to use |\&| instead of |&|, but this has been fixed after 2022/02/09. In case you really care about backward compatibility, either download the sty file in your folder, or add |ampersand replacement=\&| in the options of the |\zxSaveDiagram| function and use |\&| as before). Then, |use diagram| to insert it inside a diagram (the second argument is a list of nodes given to the |fit| library):
\begin{codeexample}[width=0pt]
%%             v---- note the backslash
\zxSaveDiagram{\myZXconstant}{\zxX{} \rar \ar[r,o'] \ar[r,o.] & \zxZ{}}
$x = \begin{ZX}[use diagram={\myZXconstant}{(start)(end)}, math baseline=myb]
                     &[\zxwCol] \zxFracX-{\pi}{2} & \zxN[a=start]{}       & \zxN[a=end]{}\\
  \zxN[a=myb]{} \rar & \zxFracZ{\pi}{2} \rar      & \zxFracX{\pi}{2} \rar & \zxFracZ{\pi}{2} \rar & \zxN{}
\end{ZX}$
\end{codeexample}
Note that if you need more space to insert the drawing, you can use |\zxN+[a=start,minimum width=2cm]{}| instead of |\zxN|.
\end{pgfmanualentry}

Sometimes, you may also find useful to stack diagram (for instance because the matrix of the first diagram is not related to the matrix of the second diagram), or position multiple diagrams relative to each others. You can use arrays to do that, but it will not preserve the baseline. Another solution is to put the |ZX| environment inside nodes contained in |tikzpicture| (I know that tikz does not like nesting\dots{} but it works nice for what I tried. I will also try to provide an helper function to do that later). For example here we use it to add the constants below the second diagram:
\begin{codeexample}[vbox]
\begin{equation*}
  \begin{ZX}
    &[\zxwCol]       &                                      &[\zxwCol] \zxN{} \ar[dd,3 vdots] \\[\zxNCol]
    \zxN{} \rar & \zxZ{\pi} \rar & \zxX{\alpha} \ar[ru,N'-] \ar[rd,N.-] &                        \\[\zxNCol]
    &                &                                      & \zxN{}
  \end{ZX} =
  \begin{tikzpicture}[
    baseline=(A.base)
    ]
    \node[inner sep=0pt](A){%
      \begin{ZX}
        &[\zxwCol]                              & \zxZ{\pi} \ar[dd,3 vdots] \rar &[\zxwCol] \zxN{} \\[\zxNCol]
        \zxN{} \rar & \zxX-{\alpha} \ar[ru,N'-] \ar[rd,N.-] &                                        \\[\zxNCol]
        &                                       & \zxZ{\pi} \rar & \zxN{}
      \end{ZX}
    };
    \node[inner sep=0pt,below=\zxDefaultColumnSep of A](B){
      \begin{ZX}
        \zxOneOverSqrtTwo{} & \zxX{\alpha} \rar & \zxZ{\pi}
      \end{ZX}
    };
  \end{tikzpicture}
\end{equation*}
\end{codeexample}

\subsection{Further customization}

You can further customize your drawings using any functionality from \tikzname{} and |tikz-cd| (but it is of course at your own risk). For instance, we can change the separation between rows and/or columns for a whole picture (but prefer to use |zx row sep| as it also updates pre-configured column spaces):
\begin{codeexample}[width=0pt]
  \begin{ZX}[row sep=1mm]
                &                                         &                        & & \zxZ{\pi} \\
    \zxN{} \rar & \zxX{} \ar[rd,(.] \ar[urrr,(',H]        &                        & &  & \zxN{} \\
                &                                         & \zxZ{} \ar[rd,s.] \rar &
       \zxFracX{\pi}{2} \ar[uur,('] \ar[rru,<'] \ar[rr] &  & \zxN{} \\
    \zxN{} \rar & \zxFracZ-{\pi}{4} \ar[ru,('] \ar[rr,o.] &            & \zxX{} \ar[rr] &  & \zxN{}
  \end{ZX}
\end{codeexample}
Or we can define our own style to create blocks:
{\catcode`\|=12 % Ensures | is not anymore \verb|...|
\begin{codeexample}[width=0pt]
{ % \usetikzlibrary{shadows}
  \tikzset{
    my bloc/.style={
      anchor=center,
      inner sep=2pt,
      inner xsep=.7em,
      minimum height=3em,
      draw,
      thick,
      fill=blue!10!white,
      double copy shadow={opacity=.5},tape,
    }
  }
  \zx{|[my bloc]| f \rar &[1mm] |[my bloc]| g \rar &[1mm] \zxZ{\alpha} \rar & \zxNone{}}
}
\end{codeexample}
}
We can also use for instance |fit|, |alias|, |execute at end picture| and layers (the user can use |background| for things behind the drawing, |box| for special nodes above the drawings (like multi-column nodes, see below), and |foreground| which is even higher) to do something like that:
{\catcode`\|=12 % Ensures | is not anymore \verb|...|
\begin{codeexample}[width=3cm]
% \usetikzlibrary{fit}
\begin{ZX}[
  execute at end picture={
    \node[
      inner sep=2pt,
      node on layer=background, %% Ensure the node is behind the drawing
      rounded corners,
      draw=blue,
      dashed,
      fill=blue!50!white,
      opacity=.5,
      fit=(cnot1)(cnot2), %% position of the node, thanks fit library
      "\textsc{cnot}" {above, inner sep=1mm} %% Adds label, thanks quote library. Not sure why, inner sep is set to 0 when using tikz "external" library.
    ] {};
  }
  ]
  \zxNone{} \rar & \zxZ[alias=cnot1]{} \dar \rar & \zxNone{}\\
  \zxNone{} \rar & \zxX[alias=cnot2]{} \rar      & \zxNone{}\\
\end{ZX}
\end{codeexample}
Because this code is quite lengthy and useful, we provide in \cref{subsec:decoration} a shorter syntax.
}

\section{Future works}

There is surely many things to improve in this library, and given how young it is there is surely many undiscovered bugs. So feel free to propose ideas or report bugs \mylink{https://github.com/leo-colisson/zx-calculus/issues}{one the github page}. The intersections code is also quite slow, so I would be curious to check if I can optimize it (the first goal was to make it work). I should also work on the compatibility with tikzit (basically just write tikz configuration files that you can just use and document how to use tikzit with it), or even write a dedicated graphical tool (why not based on tikzit itself, or \mylink{https://tikzcd.yichuanshen.de/}{this tool}). I also want to find a nicer way to merge cells (for now I propose to use the |fit| library but it's not very robust against overlays) and to nest ZX diagrams. And of course fix typos in the documentation and write other styles, including notations not specific to ZX-calculus. Feel free to submit Pull Requests to solve that, or to submit bug reports to explain uncovered use-cases!

\section{Acknowledgement}

I'm very grateful of course to everybody that worked on these amazing field which made diagramatic quantum computing possible, and to the many StackExchange users that helped me to understand a bit more \LaTeX{} and \tikzname{} (sorry, I don't want to risk an incomplete list, but many thanks to egreg, David Carlisle, cfr, percusse, Andrew Stacey, Henri Menke, SebGlav, Qrrbrbirlbel\dots{}). I also thank Robert Booth for making me realize how my old style was ugly, and for giving advices on how to improve it. Thanks to John van de Wetering, whose style has also been a source of inspiration~\cite{van20_ZXcalculusWorkingQuantum}.

\section{Changelog}

\begin{itemize}
\item 2023/09/26:
  \begin{itemize}
  \item Many updates in the past months, custom nodes are made easy, multi-gates nodes as well, added |post arrow style if start node| and other similar commands, updated the doc to use the latest version of robust-externalize that is much simpler to use now\dots Some nodes can also now be usable outside tikzcd, but I've not tried (I should provide more options for this).
  \end{itemize}
\item 2022/02/09:
  \begin{itemize}
  \item Added compatibility with external tikz library
  \item More robust handling of |&|: align and macros does not need |\&| anymore.
  \item |\&| in |\zxSaveDiagram| is replaced with |&|. This introduces a small backward incompatibility, but hey, I said it was still subject to changes :-)
  \end{itemize}
\end{itemize}

\section{TODO}
\begin{itemize}
\item I think that some anchors are not properly configured in stuff like |\zxBox| etc
\item I set some global options like |/zx/internals/postRun...| but I should make sure to reset them for every pictures, or I might get weird behaviors (not sure why they don't seem to appear right now?)
\item Make zxNamedBox usable inside cells directly
\item Clean the code since it sta
\end{itemize}

\printindex

\printbibliography[heading=bibintoc]

\end{document}
