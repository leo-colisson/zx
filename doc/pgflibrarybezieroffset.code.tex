%% tikz-nfold.sty
%% Copyright 2023 Jonathan Schulz
%
% This work may be distributed and/or modified under the
% conditions of the LaTeX Project Public License, either version 1.3c
% of this license or (at your option) any later version.
% The latest version of this license is in
% http://www.latex-project.org/lppl.txt
% and version 1.3c or later is part of all distributions of LaTeX
% version 2008-05-04 or later.
%
% This work has the LPPL maintenance status 'maintained'.
%
% The Current Maintainer of this work is Jonathan Schulz.
%
% This work consists of the files pgflibrarybezieroffset.code.tex,
% tikzlibrarynfold.code.tex, tikz-nfold-doc.tex, and tikz-nfold-doc.pdf.


% Don't delete this code in case we need it in the future
%
% Split a Bezier curve (de Casteljau's algorithm)
% #1 = time (between 0 and 1)
% #2-#5: control points
% Outputs the first part into \pgf@splitbezier@i@i, ... , \pgf@splitbezier@i@iv,
% and the second part into \pgf@splitbezier@ii@i, ... , \pgf@splitbezier@ii@iv.
%\def\pgf@splitbezier#1#2#3#4#5{%
%  % based on pgfcorepoints.code.tex, \pgfpointcurveattime
%  \edef\pgf@time@s{#1}%
%  \pgf@x=-\pgf@time@s pt%
%  \advance\pgf@x by 1pt%
%  \edef\pgf@time@t{\pgf@sys@tonumber{\pgf@x}}%
%  % P^0_3
%  \pgfextract@process\pgf@splitbezier@ii@iv{#5}%
%  \pgf@xc=\pgf@x%
%  \pgf@yc=\pgf@y%
%  % P^0_2
%  \pgf@process{#4}%
%  \pgf@xb=\pgf@x%
%  \pgf@yb=\pgf@y%
%  % P^0_1
%  \pgf@process{#3}%
%  \pgf@xa=\pgf@x%
%  \pgf@ya=\pgf@y%
%  % P^0_0
%  \pgfextract@process\pgf@splitbezier@i@i{#2}%
%  % First iteration:
%  % P^1_0
%  \pgf@x=\pgf@time@t\pgf@x\advance\pgf@x by\pgf@time@s\pgf@xa%
%  \pgf@y=\pgf@time@t\pgf@y\advance\pgf@y by\pgf@time@s\pgf@ya%
%  \pgfextract@process\pgf@splitbezier@i@ii{}%
%  % P^1_1
%  \pgf@xa=\pgf@time@t\pgf@xa\advance\pgf@xa by\pgf@time@s\pgf@xb%
%  \pgf@ya=\pgf@time@t\pgf@ya\advance\pgf@ya by\pgf@time@s\pgf@yb%
%  % P^1_2
%  \pgf@xb=\pgf@time@t\pgf@xb\advance\pgf@xb by\pgf@time@s\pgf@xc%
%  \pgf@yb=\pgf@time@t\pgf@yb\advance\pgf@yb by\pgf@time@s\pgf@yc%
%  \edef\pgf@splitbezier@ii@iii{\noexpand\pgfqpoint{\the\pgf@xb}{\the\pgf@yb}}%
%  % P^2_0
%  \pgf@x=\pgf@time@t\pgf@x\advance\pgf@x by\pgf@time@s\pgf@xa%
%  \pgf@y=\pgf@time@t\pgf@y\advance\pgf@y by\pgf@time@s\pgf@ya%
%  \pgfextract@process\pgf@splitbezier@i@iii{}%
%  % P^2_1
%  \pgf@xa=\pgf@time@t\pgf@xa\advance\pgf@xa by\pgf@time@s\pgf@xb%
%  \pgf@ya=\pgf@time@t\pgf@ya\advance\pgf@ya by\pgf@time@s\pgf@yb%
%  \edef\pgf@splitbezier@ii@ii{\noexpand\pgfqpoint{\the\pgf@xa}{\the\pgf@ya}}%
%  % P^3_0
%  \pgf@x=\pgf@time@t\pgf@x\advance\pgf@x by\pgf@time@s\pgf@xa%
%  \pgf@y=\pgf@time@t\pgf@y\advance\pgf@y by\pgf@time@s\pgf@ya%
%  \pgfextract@process\pgf@splitbezier@i@iv{}%
%  \let\pgf@splitbezier@ii@i\pgf@splitbezier@i@iv
%}

% Split the Bezier curve defined by #1-#4 at t=0.5 using de Casteljau's algorithm
\def\pgf@halfsplitbezier#1#2#3#4{%
  % based on pgfcorepoints.code.tex, \pgfpointcurveattime
  % P^0_3
  \pgfextract@process\pgf@splitbezier@ii@iv{#4}%
  \pgf@xc=\pgf@x%
  \pgf@yc=\pgf@y%
  % P^0_2
  \pgf@process{#3}% removing these does not yield a significant speedup
  \pgf@xb=\pgf@x%
  \pgf@yb=\pgf@y%
  % P^0_1
  \pgf@process{#2}%
  \pgf@xa=\pgf@x%
  \pgf@ya=\pgf@y%
  % P^0_0
  \pgfextract@process\pgf@splitbezier@i@i{#1}%
  % First iteration:
  % P^1_0
  \pgf@x=.5\pgf@x\advance\pgf@x by.5\pgf@xa%
  \pgf@y=.5\pgf@y\advance\pgf@y by.5\pgf@ya%
  \edef\pgf@splitbezier@i@ii{\pgf@x\the\pgf@x\pgf@y\the\pgf@y}%
  % P^1_1
  \pgf@xa=.5\pgf@xa\advance\pgf@xa by.5\pgf@xb%
  \pgf@ya=.5\pgf@ya\advance\pgf@ya by.5\pgf@yb%
  % P^1_2
  \pgf@xb=.5\pgf@xb\advance\pgf@xb by.5\pgf@xc%
  \pgf@yb=.5\pgf@yb\advance\pgf@yb by.5\pgf@yc%
  \edef\pgf@splitbezier@ii@iii{\pgf@x\the\pgf@xb\pgf@y\the\pgf@yb}%
  % P^2_0
  \pgf@x=.5\pgf@x\advance\pgf@x by.5\pgf@xa%
  \pgf@y=.5\pgf@y\advance\pgf@y by.5\pgf@ya%
  \edef\pgf@splitbezier@i@iii{\pgf@x\the\pgf@x\pgf@y\the\pgf@y}%
  % P^2_1
  \pgf@xa=.5\pgf@xa\advance\pgf@xa by.5\pgf@xb%
  \pgf@ya=.5\pgf@ya\advance\pgf@ya by.5\pgf@yb%
  \edef\pgf@splitbezier@ii@ii{\pgf@x\the\pgf@xa\pgf@y\the\pgf@ya}%
  % P^3_0
  \pgf@x=.5\pgf@x\advance\pgf@x by.5\pgf@xa%
  \pgf@y=.5\pgf@y\advance\pgf@y by.5\pgf@ya%
  \edef\pgf@splitbezier@i@iv{\pgf@x\the\pgf@x\pgf@y\the\pgf@y}%
  \let\pgf@splitbezier@ii@i\pgf@splitbezier@i@iv
}


% computes the cross product and puts it into \pgfmathresult
\def\pgfmathcrossproduct#1#2{%
  \begingroup
    \pgf@process{#1}%
    \pgf@xa=\pgf@x%
    \pgf@ya=\pgf@y%
    \pgf@process{#2}%
    \pgf@y=\pgf@sys@tonumber\pgf@xa\pgf@y
    \advance\pgf@y by -\pgf@sys@tonumber\pgf@ya\pgf@x
  \pgfmath@returnone\pgf@y
  \endgroup
}

\def\pgfmathdotproduct#1#2{%
  \begingroup
    \pgf@process{#1}%
    \pgf@xa=\pgf@x%
    \pgf@ya=\pgf@y%
    \pgf@process{#2}%
    \pgf@x=\pgf@sys@tonumber\pgf@xa\pgf@x
    \advance\pgf@x by \pgf@sys@tonumber\pgf@ya\pgf@y
  \pgfmath@returnone\pgf@x
  \endgroup
}

\def\pgfmathcrossdot#1#2{%
  \begingroup
    \pgf@process{#1}%
    \pgf@xa=\pgf@x%
    \pgf@ya=\pgf@y%
    \pgf@process{#2}%
    \pgf@xb=\pgf@sys@tonumber\pgf@xa\pgf@x
    \pgf@yb=\pgf@sys@tonumber\pgf@xa\pgf@y
    \advance\pgf@xb by \pgf@sys@tonumber\pgf@ya\pgf@y
    \advance\pgf@yb by -\pgf@sys@tonumber\pgf@ya\pgf@x
    \edef\pgf@temp{%
      \edef\noexpand\pgf@tmp@dot{\pgf@sys@tonumber\pgf@xb}%
      \edef\noexpand\pgf@tmp@cross{\pgf@sys@tonumber\pgf@yb}%
    }%
  \expandafter
  \endgroup\pgf@temp
}

% Calculates abs(\pgf@x) + abs(\pgf@y) in #1
\def\pgfpointtaxicabnorm#1{%
  \ifdim\pgf@x<0pt
    #1=-\pgf@x
  \else
    #1=\pgf@x
  \fi
  \ifdim\pgf@y<0pt
    \advance#1 by -\pgf@y
  \else
    \advance#1 by \pgf@y
  \fi
}

% Computes the normalised tangents of a given Bezier curve and stores them in \pgf@tmp@tang@i and \pgf@tmp@tang@ii.
% Also computes the angles and stores them in  \pgf@tmp@angle@i and \pgf@tmp@angle@ii.
% All degenerate cases are covered. For a triple degenerate curve (all points equal), the vector (1,0) is returned.
\def\pgf@offset@compute@tangents#1#2#3#4{%
  \pgf@process{\pgfpointdiff{#1}{#2}}% unintuitively, this is PTii - PTi
  \pgfpointtaxicabnorm\pgf@xa
  \ifdim\pgf@xa<0.1pt\relax
    % edge case: first point and first control point are equal
    \pgf@process{\pgfpointdiff{#1}{#3}}%
    \pgfpointtaxicabnorm\pgf@xa
    \ifdim\pgf@xa<0.1pt\relax
      % edge case: first three points are equal
      \pgf@process{\pgfpointdiff{#1}{#4}}%
    \fi
  \fi
  \pgfextract@process\pgf@tmp@tang@i{%
    \pgfpointnormalised{}%
    % \pgfpointnormalised stores the angle of the vector in \pgf@tmp
    \global\let\pgf@nfold@tmp\pgf@tmp%
  }%
  \let\pgf@tmp@angle@i\pgf@nfold@tmp%
  \pgf@process{\pgfpointdiff{#3}{#4}}%
  \pgfpointtaxicabnorm\pgf@xa
  \ifdim\pgf@xa<0.1pt\relax
    \pgf@process{\pgfpointdiff{#2}{#4}}%
    \pgfpointtaxicabnorm\pgf@xa
    \ifdim\pgf@xa<0.1pt\relax
      \pgf@process{\pgfpointdiff{#1}{#4}}%
    \fi
  \fi
  \pgfextract@process\pgf@tmp@tang@ii{\pgfpointnormalised{}\global\let\pgf@nfold@tmp\pgf@tmp}%
  \let\pgf@tmp@angle@ii\pgf@nfold@tmp%
}


%%%%%%%%%%%%%%%%%%%%%%%%%%%%%%%
% Offsetting a simple section %
%%%%%%%%%%%%%%%%%%%%%%%%%%%%%%%

\def\pgf@offset@bezier@segment#1#2#3#4#5{%
  % normalise tangents and normals; this avoids overflow issues later, and we need
  % the normal vector to be of length 1 anyway
  \pgf@offset@compute@tangents{#1}{#2}{#3}{#4}%
  \pgf@offset@bezier@segment@{#1}{#2}{#3}{#4}{#5}%
}

% this version assumes that the tangents have been precomputed in \pgf@tmp@tang@i and @ii
\def\pgf@offset@bezier@segment@#1#2#3#4#5{%
  % offset A1
  % compute the normal
  \pgf@tmp@tang@i
  \edef\pgf@tmp@normal@i{\noexpand\pgfqpoint{-\the\pgf@y}{\the\pgf@x}}%
  \pgfextract@process\pgf@bezier@offset@i
    {\pgfpointadd{\pgfpointscale{#5}{\pgf@tmp@normal@i}}{#1}}%
  % offset A4
  \pgf@tmp@tang@ii
  \edef\pgf@tmp@normal@ii{\noexpand\pgfqpoint{-\the\pgf@y}{\the\pgf@x}}%
  \pgfextract@process\pgf@bezier@offset@iv
    {\pgfpointadd{\pgfpointscale{#5}{\pgf@tmp@normal@ii}}{#4}}%
  % now compute A'_2 and A'_3
  \pgf@process{\pgfpointdiff{#1}{#4}}%
  \pgfmathveclen@{\pgf@sys@tonumber\pgf@x}{\pgf@sys@tonumber\pgf@y}%
  \let\pgf@tmp@secantlen\pgfmathresult
  \ifdim\pgf@tmp@secantlen pt<0.1pt\relax
    % Edge case: Either the curve is degenerate to a point or it is not simple.
    % Either way we offset A1 and A4, and preserve the vectors A1A2 and A3A4.
    \pgfutil@packagewarning{tikz-nfold}{first and last point are too close, expect glitches}%
    \pgfextract@process\pgf@bezier@offset@ii
      {\pgfpointadd{\pgf@bezier@offset@i}{\pgfpointdiff{#1}{#2}}}%
    \pgfextract@process\pgf@bezier@offset@iii
      {\pgfpointadd{\pgf@bezier@offset@iv}{\pgfpointdiff{#4}{#3}}}%
  \else
    \pgfextract@process\pgf@tmp@secant{\pgfpointnormalised{}}%
    \pgfmathcrossdot{}{\pgf@tmp@tang@ii}%
     \ifdim\pgf@tmp@dot pt<.5pt\relax%
       % this can only happen in non-simple curves
       \pgfutil@packagewarning{tikz-nfold}{cosine of \pgf@tmp@dot\space clamped to 0.5 in non-simple segment}%
       \def\pgf@tmp@dot{.5}%
    \fi%
    \pgfmathdivide@{\pgf@tmp@cross}{\pgf@tmp@dot}%
    \let\pgf@tmp@tanbeta\pgfmathresult
    \pgfmathcrossdot{\pgf@tmp@secant}{\pgfpointnormalised{\pgfpointdiff{#1}{#2}}}
    % There are cases where we want #5/secantlen to be quite large, so we should not clamp the value here
    % \pgfmathparse{1 + #5/\pgf@tmp@secantlen*(\pgf@tmp@cross - \pgf@tmp@dot*\pgf@tmp@tanbeta)}%
    \pgfmath@offset@calculate@scale{\pgf@tmp@secantlen}{\pgf@tmp@cross}{\pgf@tmp@dot}{\pgf@tmp@tanbeta}{#5}%
    \pgfextract@process\pgf@bezier@offset@ii{%
      \pgfpointadd
        {\pgf@bezier@offset@i}
        {\pgfqpointscale{\pgfmathresult}{\pgfpointdiff{#1}{#2}}}%
    }%
    % third control point
    \pgfmathcrossdot{\pgf@tmp@secant}{\pgf@tmp@tang@i}%
    \ifdim\pgf@tmp@dot pt<.5pt\relax
      \pgfutil@packagewarning{tikz-nfold}{cosine of \pgf@tmp@dot\space clamped to 0.5 in non-simple segment}%
      \def\pgf@tmp@dot{.5}%
    \fi
    \pgfmathdivide@{\pgf@tmp@cross}{\pgf@tmp@dot}%
    \let\pgf@tmp@tanbeta\pgfmathresult
    \pgfmathcrossdot{\pgf@tmp@secant}{\pgfpointnormalised{\pgfpointdiff{#4}{#3}}}%
    % \pgfmathparse{1 + #5/\pgf@tmp@secantlen*(\pgf@tmp@cross - \pgf@tmp@dot*\pgf@tmp@tanbeta)}%
    \pgfmath@offset@calculate@scale{\pgf@tmp@secantlen}{\pgf@tmp@cross}{\pgf@tmp@dot}{\pgf@tmp@tanbeta}{#5}%
    \pgfextract@process\pgf@bezier@offset@iii{%
      \pgfpointadd
        {\pgf@bezier@offset@iv}
        {\pgfqpointscale{\pgfmathresult}{\pgfpointdiff{#4}{#3}}}%
    }%
  \fi
}

% calculates 1+#5/#1*(#2-#3*#4)
% #1 = secantlen
% #2 = cross
% #3 = dot
% #4 = tanbeta
% #5 = #5 (offset)
\def\pgfmath@offset@calculate@scale#1#2#3#4#5{%
  \begingroup
    \pgfmathmultiply@{#3}{#4}%
    \pgfmathsubtract@{#2}{\pgfmathresult}%
    \let\pgfmath@temp\pgfmathresult
    \pgfmathreciprocal@{#1}%
    \pgfmathmultiply@{\pgfmathresult}{\pgfmath@temp}%
    \pgfmathmultiply{\pgfmathresult}{#5}%
    \pgfmathadd@{\pgfmathresult}{1}%
    \pgfmath@smuggleone\pgfmathresult
  \endgroup
}

%%%%%%%%%%%%%%%%%%%%%%%%%%%%%%
% Subdividing and offsetting %
%%%%%%%%%%%%%%%%%%%%%%%%%%%%%%

% Maximum level of recursion. The theoretical limit to the number of subdivisions in the final curve
% is given by 2^\pgf@offset@max@recursion
\newcount\pgf@offset@max@recursion
\pgf@offset@max@recursion=4

%
% Subdivides a Bezier curve into "simple" segments (according to the definition below),
% offsets the segments, and draws them.  Because offsetting also involves relocating
% the starting points, these macros come in two variants: with and without a \pgfpathmoveto{}
% to the new starting point.
%
% Interface:
% #1-#4: control points of the whole Bezier curve
% #5: offset

\def\pgfoffsetcurve#1#2#3#4#5{%
  \pgfoffsetcurvecallback{#1}{#2}{#3}{#4}{#5}{\pgf@nfold@callback@move}%
}
\def\pgfoffsetcurvenomove#1#2#3#4#5{%
  \pgfoffsetcurvecallback{#1}{#2}{#3}{#4}{#5}{\pgf@nfold@callback@nomove}%  
}

% Arguments:
% #1-#4: control points of the segment
% #5: =0 if this is the first segment of the curve, =1 otherwise
%   (checking for #5=0 allows us to draw the curve without interruptions)
\def\pgf@nfold@callback@move#1#2#3#4#5{%
  \if#50\pgfpathmoveto{#1}\fi%
  \pgfpathcurveto{#2}{#3}{#4}%
}
% this version never does a moveto at the start. Useful for drawing a path consisting of
% multiple Bezier curves.
\def\pgf@nfold@callback@nomove#1#2#3#4#5{\pgfpathcurveto{#2}{#3}{#4}}

% This macro subdivides and offsets a curve and executes a given callback on every offset simple segment.
% #1-#4: control points of the segment
% #5: =0 if this is the first segment of the curve, =1 otherwise
% #6: callback (see \pgf@nfold@callback@move) for the parameters
\def\pgfoffsetcurvecallback#1#2#3#4#5#6{%
  \edef\pgf@offset@tmp@callback##1##2##3##4##5{%
    \noexpand\pgf@offset@bezier@segment{##1}{##2}{##3}{##4}{#5}%
    \noexpand#6{\noexpand\pgf@bezier@offset@i}{\noexpand\pgf@bezier@offset@ii}{\noexpand\pgf@bezier@offset@iii}{\noexpand\pgf@bezier@offset@iv}{##5}%
  }
  \pgf@subdividecurve{#1}{#2}{#3}{#4}{\pgf@offset@max@recursion}{0}{\pgf@offset@tmp@callback}%
}


% This macro subdivides (but does not offset) a curve and executes a callback on every simple segment.
% #1-#4: control points
% #5: recursion limit (decreases on every recursive call)
% #6: =0 if this is the start of the curve, =1 otherwise
% #7: callback for output (see above)
\newif\ifpgf@offset@subdivide
\def\pgf@subdividecurve#1#2#3#4#5#6#7{%
  \begingroup%
    \pgfextract@process\pgf@ctrl@i{#1}%
    \pgfextract@process\pgf@ctrl@ii{#2}%
    \pgfextract@process\pgf@ctrl@iii{#3}%
    \pgfextract@process\pgf@ctrl@iv{#4}%
    \pgf@offset@compute@tangents{\pgf@ctrl@i}{\pgf@ctrl@ii}{\pgf@ctrl@iii}{\pgf@ctrl@iv}%
    \pgf@subdividecurve@{#5}{#6}{#7}%
  \endgroup%
}

% This macro assumes that the curve is defined in \pgf@ctrl@i-\pgf@ctrl@iv
% and that its tangents have already been computed.
\def\pgf@subdividecurve@#1#2#3{%
  \pgf@offset@subdividefalse%
  \c@pgf@counta=#1\relax
  \advance\c@pgf@counta by -1
  % Use the non-degenerate tangents for the simplicity check
  \pgfextract@process\pgf@itoiv{\pgfpointdiff{\pgf@ctrl@i}{\pgf@ctrl@iv}}%
  \pgfmathcrossproduct{\pgf@itoiv}{\pgf@tmp@tang@i}%
  \let\firstcross\pgfmathresult
  \pgfmathcrossproduct{\pgf@itoiv}{\pgf@tmp@tang@ii}%
  % First simplicity check: Are A2 and A3 on the same side of the A1-A4 line?
  % -> compute the sign of the cross products, use the sign function to avoid overflows
  % just give it a pass if one of them is zero, hence 2 and 3 at the end
  \ifnum
    \ifdim   \firstcross pt<0pt -1\else\ifdim   \firstcross pt>0pt 1\else 2\fi\fi
   =\ifdim\pgfmathresult pt<0pt -1\else\ifdim\pgfmathresult pt>0pt 1\else 3\fi\fi
  \relax % the \relax is important!
    \pgf@offset@subdividetrue%
  \else%
    % Second simplicity check: How large is the angle between the tangents in A1 and A4?
    \pgfmathdotproduct{\pgf@tmp@tang@i}{\pgf@tmp@tang@ii}%
    \ifdim\pgfmathresult pt<.5pt\relax%
      \pgf@offset@subdividetrue%
    \else
      % Third simplicity check: Put a limit on the lengths of the i-ii and iii-iv vectors combined
      \pgf@itoiv
      \pgfmathveclen@{\pgf@sys@tonumber\pgf@x}{\pgf@sys@tonumber\pgf@y}%
      \pgf@xa=\pgfmathresult pt
      \pgf@process{\pgfpointdiff{\pgf@ctrl@i}{\pgf@ctrl@ii}}%
      \pgfmathveclen@{\pgf@sys@tonumber\pgf@x}{\pgf@sys@tonumber\pgf@y}%
      \pgf@xb=\pgfmathresult pt
      \pgf@process{\pgfpointdiff{\pgf@ctrl@iii}{\pgf@ctrl@iv}}%
      \pgfmathveclen@{\pgf@sys@tonumber\pgf@x}{\pgf@sys@tonumber\pgf@y}%
      \advance\pgf@xb by \pgfmathresult pt
      % veclen(iv-i) < veclen(ii-i) + veclen(iv-iii)
      \ifdim\pgf@xa<\pgf@xb
        \pgf@offset@subdividetrue
      \fi
    \fi%
  \fi%
  \ifpgf@offset@subdivide%
    \ifnum\c@pgf@counta<0%
      % We hit the recursion limit but the segment is not simple
      \pgfutil@packagewarning{tikz-nfold}{Recursion limit reached, glitches may occur. %
        Consider increasing \string\pgf@offset@max@recursion}%
      % Try to offset the curve anyway. The result will not be precise,
      % but the code is sufficiently robust to not crash
      #3{\pgf@ctrl@i}{\pgf@ctrl@ii}{\pgf@ctrl@iii}{\pgf@ctrl@iv}{#2}%
    \else
      % split the non-simple segment and execute recursive calls
      \pgf@halfsplitbezier{\pgf@ctrl@i}{\pgf@ctrl@ii}{\pgf@ctrl@iii}{\pgf@ctrl@iv}%
      % compute the tangent at the spitting point; this vector is only zero if the curve
      % is degenerate to a point, in which case we can't compute the tangent anyway
      \pgfextract@process\pgf@middletangent{%
        \pgfpointnormalised{\pgfpointdiff{\pgf@splitbezier@i@iii}{\pgf@splitbezier@i@iv}}}%
      \begingroup%
        % we need a group to avoid overwriting variables in recursive calls
        \let\pgf@tmp@tang@ii\pgf@middletangent%
        \let\pgf@ctrl@i\pgf@splitbezier@i@i%
        \let\pgf@ctrl@ii\pgf@splitbezier@i@ii%
        \let\pgf@ctrl@iii\pgf@splitbezier@i@iii%
        \let\pgf@ctrl@iv\pgf@splitbezier@i@iv%
        % pass on the "start of the curve flag" only to the first term
        \pgf@subdividecurve@{\c@pgf@counta}{#2}{#3}%
      \endgroup%
      \begingroup%
        \let\pgf@tmp@tang@i\pgf@middletangent%
        \let\pgf@ctrl@i\pgf@splitbezier@ii@i%
        \let\pgf@ctrl@ii\pgf@splitbezier@ii@ii%
        \let\pgf@ctrl@iii\pgf@splitbezier@ii@iii%
        \let\pgf@ctrl@iv\pgf@splitbezier@ii@iv%
        \pgf@subdividecurve@{\c@pgf@counta}{1}{#3}%
      \endgroup%
    \fi%
  \else%
    % curve is simple
    #3{\pgf@ctrl@i}{\pgf@ctrl@ii}{\pgf@ctrl@iii}{\pgf@ctrl@iv}{#2}%
  \fi%
}


%
% Offsetting straight lines
% -------------------------
%
% For convenience we also provide macros that offset straight lines. These also come in two variants
% similar to the macros for curves.
%
\def\pgfoffsetline#1#2#3{%
  \pgfmathparse{#3}%
  \pgfoffsetline@{#1}{#2}{\pgfmathresult}{\pgfpointnormalised{\pgfpointdiff{#1}{#2}}}%
}

% a quicker version in case we already know the tangent and #3 is a number without unit
\def\pgfoffsetline@#1#2#3#4{%
  \pgfqpointscale{#3}{#4}%
  \pgf@xc=-\pgf@y
  \pgf@yc=\pgf@x
  \pgfpathmoveto{\pgfpointadd{#1}{\pgfqpoint{\pgf@xc}{\pgf@yc}}}%
  \pgfpathlineto{\pgfpointadd{#2}{\pgfqpoint{\pgf@xc}{\pgf@yc}}}%
}

\def\pgfoffsetlinenomove#1#2#3{%
  \pgfoffsetlinenomove@{#1}{#2}{#3}{\pgfpointnormalised{\pgfpointdiff{#1}{#2}}}%
}

\def\pgfoffsetlinenomove@#1#2#3#4{%
  \pgfqpointscale{#3}{#4}%
  \pgf@xc=-\pgf@y
  \pgf@yc=\pgf@x
  \pgfpathlineto{\pgfpointadd{#2}{\pgfqpoint{\pgf@xc}{\pgf@yc}}}%
}
